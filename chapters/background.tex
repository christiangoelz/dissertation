\section{Age related reorganization of the brain}
Age related reorganization processes are detectable at the whole body. This is underpinned by multiple interacting biological systems operating on several spatial and temporal scales contributing to the complexity of the phenomenon \cite{Mooney2016}. At the behavioral level these processes are noticeable in changes in cognitive, motor and sensory functioning [QUELLE]. Aging is one of the biggest risk factors for neurodegenerative diseases such as dementia, including Alzheimer's disease, as well as Parkinson's disease making the brain as one of the target systems to study. Patterns of reorganization of the brain are highly individual as they are subject to genetic and environmental influences [QUELLEN]. At the same time, however, overarching, generalizable patterns can be detected [QUELLE]\\
On a structural level aging has been associated with a reduction in gray matter with an onset early in life 

\subsection{Contributing Factors}

\section{A data driven way to study the aging brain}
"If our goal as a field is to use data to solve problems, then we need to move away from exclusive dependence on data models and adopt a more diverse set of tools" [Breimann] 
Besides classical statistical approaches tools from data science, i.e., machine learning, provide new opportunities to understand age-related reorganization and related factors in a naturalistic way [Breimann]. Rather than assuming that data is generated by a given stochastic process an algorithmic view treats data mechanisms as unknown [Breimann]. This view is mainly used in the field of data science, which emerged as an independent scientific discipline at the beginning of the 21st century and has adopted concepts from computer science and statistics, especially in the field of machine learning [NEURAL DATASCIENCE] 

\subsection{Neural Datascience}



The complex interplay of the aforementioned factors leading to the dynamics of brain network changes over the lifetime perspective is not fully understood. However, this is a prerequisite to differentiate healthy from pathological changes and to develop and verify treatments as well as targeted interventions. This requires uncomplicated, easy-to-use, and cost-effective methods and novel analyses to quantify changes in brain organization. Several methods are available to study the brains' structure and function including functional and structural Magnetic Resonance Imaging (s/fMRI) as well as Magnetoencephalography (MEG) and Electroencephalography (EEG). In addition to cost-intensive imaging techniques such as fMRI, EEG, which meets these criteria, is particularly suitable.
Moreover, EEG measures neuronal activity directly with high temporal resolution, which would allow us to gain new insights into the reorganization of brain networks over the lifespan in health and disease. However, it is unclear how the changes described above are reflected in electrophysiological markers. Furthermore, EEG signals are temporally and spatially highly dimensional and have a low signal-to-noise ratio, which makes the detection and visualization of brain networks and their dynamics difficult and requires advanced signal analysis methods. 
Advanced methods such as methods from the field of artificial intelligence with machine learning, as a subbranch, are of special interest in this context. Methods from supervised and unsupervised machine learning are possible candidates. In unsupervised machine learning, the goal is to find structure in the data. This includes methods for dimensionality reduction and clustering. Dimensionality reduction for example allows us to describe the structure of high dimensional data in fewer properties [20]. Two common methods in the analysis of neurophysiological data are the principal component analysis (PCA) and independent component analysis (ICA). These allow the detection of spatial patterns in the data that represent the underlying network characteristics of neurophysiological data [8]. In addition, with dynamic mode decomposition (DMD), Brunton, Johnson, et al. [21] apply for the first time a method to electrophysiological data that allows us to map both the spatial and temporal structure of the network structure of neurophysiological data.
In supervised machine learning, models are created that can predict a certain outcome based on input data. This method is used to detect neuronal representations of the environment or certain behaviors as well as group memberships and to identify relevant markers [22].
In the context of lifespan changes, complex brain network behavior based on EEG data could be extracted and visualized using dimensionality reduction. Supervised machine learning methods could be used to detect representations of the environment and behavior and to draw conclusions about the differentiation of brain networks. Automatic detection of group membership could further provide new predictors of nervous system states.
\subsection{Measuring age related changes}
\subsection{Neural Datascience}
\subsubsection{Dimensionality reduction}
\subsubsection{Machine Learning}
