Chapter ideas: 

- Age related changes occur at different scales and are manifestet at several levels. 
- There is a wide variety in how this changes occur
- Changes are e.g. neural dedifferentiation and compensatory mechanisms (see Reuter Lorenz et al. 2010) and are noticable brain network level and dynamics
- Check what EEG studies said about this...
- The idea is to model these changes with tools from datascience to answer questions in aging neuroscience 
	- Aging and Motor Control 
		- First study is about detecting dedifferentiated and compensatory mechanisms with EEG
		- Tools used are DMD and Machine learning
		- Main idea: Study classification performance as proxy for age related changes in different motor control tasks 
	- Expertise as possible way of builing a reserve: 
		- Higher individuality 
 
	- Dynamics of dedifferentiation and how do they relate to fitness
		- Basic for targeted interventions 
		- How much and what (relate to Julia)

- Background of ML
	- ML as tool 
	- novel insights 
	- Problem: Data is multidimensional and we have often limited data 
	- Solution: Use DMD to reduce Complexity and "model" evolution of signal 
- Dynamic Mode Decompsition 
	- DMD extracts coupled spatio-temporal modes and is able to kind of model the evolution of the signal 
	- Backgrouund + Papers 
	- Mathematical Formulation 


- What can ML tell us? 
    - ML applied in aging Neuroscience
	- Formulating Aims and goals 
	- Formulation expectred outcomes 

