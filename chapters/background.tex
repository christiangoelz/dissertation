\section{Age related reorganization of the brain}
Age related reorganization processes are detectable at the whole body. This is underpinned by multiple interacting biological systems operating on several spatial and temporal scales contributing to the complexity of the phenomenon \cite{Mooney2016}. At the behavioral level these processes are noticeable in changes in cognitive, motor and sensory functioning [QUELLE]. Aging is one of the biggest risk factors for neurodegenerative diseases such as dementia, including Alzheimer's disease, as well as Parkinson's disease making the brain as one of the target systems to study. Patterns of reorganization of the brain are highly individual as they are subject to genetic and environmental influences [QUELLEN]. At the same time, however, overarching, generalizable patterns can be detected [QUELLE]\\
On a structural level aging has been associated with a reduction in gray matter with an onset early in life 

  

\subsection{Contributing Factors}
\section{The brain as a complex Network}
The brain is formed by interconnected neurons and neuronal populations representing a complex network. This network forms the basis of the function of central information processing and thus the basis of perception and action. On a global level, this represents the separation and integration of functionally specialized brain areas that change based on the environmental context.
- During rest several brain networks have been identified:\\
- DMN, SMN, DAN etc. \\
- formed by spatially separated but functionally integrated and connected brain areas
- hirarchy\\
- This network is assumed to be multiscale in nature \\


\section{Methodological Approaches}
\subsection{Network Neuroscience}
\subsubsection{Electrophysiological markers of brain network activity}
\subsection{Neural Datascience}
\subsubsection{Dimensionality reduction}
\subsubsection{Machine Learning}
