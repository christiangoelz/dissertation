\fullcite{Goelz2021a}\\
\\
This study aimed to study the influence of cardiorespiratory fitness on the dedifferentiation of task-related brain network activation patterns by assessing coherent spatio-temporal patterns of EEG in rest as well as tasks representing the sensory, motor, and cognitive domains, respectively.\\
\\
We compared the most dominant \gls{dmd} mode from the \gls{pca} per frequency band between the tasks using permutation t-test to infer the differentiability of task-related \gls{dmd} modes in fit and less fit participants. For a statistical evaluation of group differences, we compared the multivariate distribution of obtained t-values with Cremér tests between the groups. We further compared the singular values associated with the dominant pattern to infer the prominence of that pattern throughout task execution between the groups using ANOVAs.\\
\\
The comparison of t-distributions with Cramér tests between the groups revealed higher differentiability of dominant \gls{dmd} modes in the fit compared to the less fit participants in all frequency bands (all p~$<$~0.05). However, the difference in the differentiability was most pronounced in the theta and high beta frequency bands (see \autoref{fig:results3}). Repeated measures ANOVA revealed that a significantly lower proportion of total variance can be explained by the dominant pattern in the high beta frequency range for the less fit compared to the fit group [F(1,29) = 12.572, p = 0.001, partial $\eta²$ = 0.300] in the motor (fit: M~=~80.5~\% SD~=~0.60~\%, less fit: M~=~79.95~\% SD~=~0.52~\%), the sensory (fit: M~=~80.8~\% SD~=~0.76~\%, less fit: M~=~80.05\% SD~=~0.53~\%) and the cognitive task (fit: M~=~81.07~\% SD~=~0.78~\%, less fit: M~=~80.21~\% SD~=~0.54~\%).
\begin{figure}[h]
    \centering
    \input{figures/paper3_main_results.pdf_tex}
    \caption[Main results of Research Article \uproman{3}]{Main results of Research Article \uproman{3}}
    \label{fig:results3}
\end{figure}

\subsection*{Author Contributions}
C. Goelz, J.K. Stroehlein, F.K. Haase, CR and SV contributed to the study conception and design. CG and FKH set up the experiments. Data collection were performed by CG, JKS and FKH. CG and KM analyzed data. All authors interpreted results, drafted parts of the work, approved the final version of the manuscript, and agreed to be
accountable for all aspects of the work.