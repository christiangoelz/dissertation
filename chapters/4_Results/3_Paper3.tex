\hyperref[pub:paperIII]{\fullcite{Goelz2021b}}\\
\\
This study followed the assumption from the reserve hypothesis that life factors, such as physical fitness, influence age-related reorganization of the brain. Thus, the aim was to investigate the influence of cardiorespiratory fitness on the dedifferentiation of task-related brain network patterns at rest and during tasks representing the sensory, motor, and cognitive domains, respectively.\\
\\
We compared the dominant \gls{dmd} patterns derived by \gls{svd} per frequency band between the tasks using permutation t-tests to infer the differentiability of task-related \gls{dmd} modes in fit and less fit participants. For a statistical evaluation of group differences, we compared the multivariate distribution of obtained t-values with Cramér tests between the groups. We further compared the singular values associated with the dominant pattern to infer the prominence of that pattern throughout task execution between the groups using repeated measures ANOVAs.\\
\\
The comparison of t-distributions with Cramér tests between the groups revealed higher differentiability of dominant \gls{dmd} modes in the fit compared to the less fit participants in all frequency bands (all p~$<$~0.05, see Figure 3 and Table 2 in \hyperref[pub:paperIII]{Published Research Article \uproman{3}} for exact statistical values). However, the difference in the differentiability was most pronounced in the $\theta$ and $\beta_2$ frequency bands (see \autoref{fig:results3}). Furthermore, a significantly lower proportion of total variance could be explained by the dominant pattern in the $\beta_2$ frequency range for the less fit compared to the fit group [F(1,29) = 12.572, p = 0.001, partial $\eta^{2}$ = 0.300] in the motor (fit: M~=~80.5~\%, SD~=~0.60~\%, less fit: M~=~79.95~\%, SD~=~0.52~\%), the sensory (fit: M~=~80.8~\%, SD~=~0.76~\%, less fit: M~=~80.05~\%, SD~=~0.53~\%) and the cognitive task (fit: M~=~81.07~\%, SD~=~0.78~\%, less fit: M~=~80.21~\%, SD~=~0.54~\%).\\

\begin{figure}[ht]
    \centering
    \input{figures/paper3_main_results.pdf_tex}
    \captionsetup{justification=justified}
    \caption[Main results of Published Research Article \uproman{3}]{Main results of Published Research Article \uproman{3}. Statistical t-maps of significant differences of dominant \gls{dmd} mode patterns in the $\theta$ and $\beta_2$ frequency bands between the tasks. Fit participants (blue) showed higher specificity of task-related patterns than less fit participants (turquoise), indicated by higher pronounced differences between the tasks.}
    \label{fig:results3}
\end{figure}

\noindent The higher degree of task differentiability in the fit group compared to the unfit group supports the idea that physical fitness manifests in task-related brain activation patterns consistent with lower dedifferentiation in older adults. The higher proportion of explained variance in the fit participants can be interpreted as higher prominence of the patterns in this group due to lower noise levels, which is consistent with the predictions by the computational model of dedifferentiation. These findings are consistent with the predictions of the reserve hypothesis and support assumptions from the computational model of dedifferentiation.

% \subsection*{Author Contributions}
% \textbf{C. Goelz, J.K. Stroehlein, F.K. Haase, C. Reinsberger, and S. Vieluf} contributed to the study conception and design. \textbf{C. Goelz and F. K. Haase} set up the experiments. Data collection was performed by \textbf{C. Goelz, J. K. Stroehlein and F. K. Haase}. \textbf{C. Goelz and K. Moora} analyzed data. All authors interpreted results, drafted parts of the work, approved the final version of the manuscript, and agreed to be accountable for all aspects of the work.