\fullcite{Goelz2021a}\\
\\
% Classification of motor task conditions from \gls{eeg} data may provide insights into age-related changes such as dedifferentiation and compensatory activation of motor networks. The idea is that less task-specific activation of brain networks is reflected in the classification performance of classifiers trained to discriminate task conditions. So far, only isolated studies have investigated this for one-dimensional task conditions and indeed found losses in classification performance in older adults when classifying right and left handed motor tasks \cite{Chen2019}. The intention was to use classifiers trained to differentiate between different task conditions to detect age-related changes in visuomotor tracking tasks.

In this study we investigated differences of central processing of visuomotor tracking tasks based on the classification of different force control tasks in younger and older adults using machine learning.\\

Statistical Comparison of the classification input revealed significant differences in 






From a practical perspective, this has implications for assistive systems that target the differentiability of task-related brain activity of older users, such as motor BCIs.\\




\subsection*{CRediT: Author Contributions}
C. Goelz: Conceptualization, Software, Formal analysis, Writing - Original Draft.\\
K. Mora: Software, Formal analysis, Writing - Review \& Editing.\\
J. Rudisch: Writing - Review \& Editing.\\
R. Gaidai: Formal analysis, Writing - Review \& Editing.\\ 
E. Reuter: Investigation, Writing - Review \& Editing.\\
B. Godde: Conceptualization, Writing - Review \& Editing.\\ 
C. Reinsberger: Writing - Review \& Editing.\\
C. Voelcker-Rehage: Conceptualization, Writing - Review \& Editing.\\ 
S. Vieluf: Conceptualization, Project administration, Supervision, Writing - Review \& Editing.\\

\begin{figure}[h]
\begin{center}
\input{figures/paper1_main_result.pdf_tex}
\caption[Reuslts]{Results}
\label{fig:DSI_exp1}
\end{center}
\end{figure}