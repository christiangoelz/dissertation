\hyperref[pub:paperI]{Goelz, C. \textit{et al.} Classification of visuomotor tasks based on electroencephalographic data depends on age-related differences in brain activity patterns. \textit{Neural Networks} \textbf{142}, 363--374 (2021)}\\
\\
As described in \chapref{chap:aims_scope}, dedifferentiation was operationalized in this article as the loss of specificity of brain networks affecting the discriminability of task-related network characteristics as shown in studies utilizing \gls{fmri} \cite{Koen2019,Carb2011}. We, therefore, investigated differences in classification performance between younger and older adults in visuomotor tracking tasks to infer age-related dedifferentiation of the motor system.\\
\\
To describe the classifier input, we used classical statistical methods first. Next, we trained a classifier for each participant to output the appropriate task label based on the \gls{eeg} data recorded during a trial. That is, the classifier should be able to identify, given the \gls{eeg} data, whether a participant was tracking a sinusoidal target force with the left or right hand or whether the task was to track a steady target force with the right or left hand. To get a deeper insight into classification performance, we trained classifiers that predict only the hand side (left vs. right) or the target force (sinusoidal vs. steady). We compared the \gls{dmd} derived brain network patterns between the groups and tasks with permutation t-tests and the classification performance between the groups with Mann-Whitney-U tests.\\
\newpage
\noindent We found significant differences in the expression of \gls{dmd} patterns between the tasks and groups focusing mainly on central and posterior electrodes, most pronounced in the $\beta$ frequency bands but also in the $\alpha$ and $\theta$ frequency ranges (see Figures 2 to 4 in \hyperref[pub:paperI]{Published Research Article \uproman{1}} for the statistical values per electrode). In addition, there were group differences in the spatial distribution concerning a more bilateral and frontal expression of the patterns in the group of late middle-aged adults.\\
Overall, the classifiers' performance was above chance level within all participants (accuracy: M~=~0.66, SD~=~0.11; theoretical chance level: 0.25) and yielded distinct misclassification patterns between younger and late middle-aged adults (see \autoref{fig:results1}). The classification of the hand side in the late middle-aged participants performed significantly worse compared to young adults, i.e., the classifier misclassified trials performed with the right hand as left hand trials and vice versa (accuracy late middle-aged adults: M~=~0.70, SD~=~0.08; accuracy young adults: M~=~0.82, SD~=~0.09; U~=~39.5, p~=~0.02, r~=~0.45). On the other hand, the classification of which target force was followed, i.e., steady vs. sinusoidal, worked significantly better in the late middle-aged adults (accuracy late middle-aged adults: M~=~0.86, SD~=~0.09; accuracy young adults: M~=~0.75, SD~=~0.09; U~=~40.00, p~=~0.02, r~=~0.45). Consequently, fewer steady trials were classified as sinusoidal trials in the late middle-aged compared to the younger participants or vice versa.

\begin{figure}[h]
\input{figures/paper1_main_result.pdf_tex}
\captionsetup{justification=justified}
\caption[Main results of Published Research Article \uproman{1}]{Main results of Published Research Article \uproman{1}. The misclassification rate of the hand side (left vs. right)  was higher, and the misclassification rate of the target force level (sinusoidal vs. steady) was lower in late middle-aged compared to young adults.}
\label{fig:results1}
\end{figure}

\noindent The lower classification performance in the classification of the body side, i.e., left vs. right hand task execution in the group of late middle-aged adults, points to a less specific and less segregated brain network activation of the motor system. In contrast, the higher classification performance with respect to the target force, i.e., sinusoidal vs. steady force tracking, might indicate a higher level of compensatory involvement when the task gets more demanding. These findings demonstrate that the reorganization of brain networks is reflected in the classification performance.

