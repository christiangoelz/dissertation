\fullcite{Goelz2023}\\
Following the previous approach, the discriminability of inhibitory and non-inhibitory stimuli within a subject should be investigated. This worked similarly well on a global level in the different age groups, with the exception of the group of children, in whom classification performance was significantly reduced.
To shed light on dynamic reorganization processes, we further investigated which time points after the stimulus response allowed classification. Here we found different trajectories of classification performance with significantly later peaks in the older adults as well as the children. 

\begin{figure}[h]
\begin{center}
\input{figures/paper2_main_results.pdf_tex}
\caption[Reuslts]{Results}
\label{fig:results2}
\end{center}
\end{figure}

\subsection*{CRediT: Author Contributions}
CG: conceptualization, software, formal analysis, writing—original draft.\\
EMR: conceptualization, investigation, data curation, writing—review and editing.\\
SF: writing—review and editing. JR: writing—review and editing.\\ 
BG: conceptualization, writing—review and editing.\\ 
SV: conceptualization, investigation, supervision, writing—review and editing.\\ 
CV-R: conceptualization, investigation, supervision, project administration, writing—review and editing.\\ 
All authors read and approved the final manuscript.\\

