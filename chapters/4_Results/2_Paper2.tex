\fullcite{Goelz2023}\\

Following the previous approach, the differentiability of stimuli that provide inhibitory control and stimuli that do not provide inhibitory control within a subject should be investigated. This worked similarly well at the global level in the different age groups except for the group of children for whom the classification worked significantly worse. Furthermore, different time windows were crucial for the classification. For the children and older adults, it was later time points that allowed maximum differentiability of the stimulus categories. 

\subsection*{CRediT: Author Contributions}
CG: conceptualization, software, formal analysis, writing—original draft. 
EMR: conceptualization, investigation, data curation, writing—review and editing.
SF: writing—review and editing. JR: writing—review and editing. 
BG: conceptualization, writing—review and editing. 
SV: conceptualization, investigation, supervision, writing—review and editing. 
CV-R: conceptualization, investigation, supervision, project administration, writing—review and editing. 
All authors read and approved the final manuscript.

\begin{figure}[h]
\begin{center}
\input{figures/paper2_main_results.pdf_tex}
\caption[Reuslts]{Results}
\label{fig:DSI_exp1}
\end{center}
\end{figure}