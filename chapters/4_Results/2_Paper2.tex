\hyperref[pub:paperII]{\fullcite{Goelz2023}}\\
\\
Following the previous approach, the discriminability of inhibitory and non-inhibitory stimuli within a subject should be investigated using classification techniques. The intention was to study differences in the neural representation of inhibitory control across age groups and extend previous findings on the same data.\\
\\
In this study, we trained a classifier for each participant that could predict which stimulus, i.e., congruent (requires no inhibitory control) or incongruent (requires inhibitory control), was presented based on the \gls{eeg} data (see \chapref{methods:datasets:II:experiment}). We also examined how well classification worked over time after stimulus presentation, extracting the time points at which the most accurate classification was possible and the level of performance at that time. We compared the classification trajectories recorded in this manner between the different age groups using one-way ANOVAs, or Kruskal-Wallis tests, followed by t-tests or Dunn tests for post-hoc comparisons.\\
To further investigate the group structure, we also trained a group-level classifier to predict which age group the performing participant belongs to based on the \gls{eeg} recording of a trial.\\
\\
The maximum classification performance of the model trained to predict which stimulus was presented within a participant was above the chance level in over 95~\% of the participants (AUC: M~=~0.72, SD~=~0.06, analytical chance level: 0.61). The classification performance was further dependent on the group [F(5,206)~=~4.805, p~$<$~0.001], with classification performance lower in the children's group compared to the other groups (p~$<$~0.05). When comparing the trajectories, we found that this also differed between the groups (H(5)~=~35.575, p~$<$~0.001) with later performance peaks in the children and the two oldest age groups (p~$<$~0.05) (see \autoref{fig:results2} A).\\
\\
The classification of group membership was overall above chance level (accuracy: 0.55\%, empirical chance level: 0.17), and a characteristic pattern of misclassifications emerged (see \autoref{fig:results2} B).\\
The classification of children was the most accurate. An increasing number of misclassifications were observed in the other age groups, where the classifier incorrectly assigned trials to a neighboring age group. This resulted in clusters of adjacent age groups within which this misclassification accumulated. The first cluster ranged from young adults to late middle-aged adults, and the second cluster included the two oldest age groups. There were higher misclassification rates within these clusters but fewer misclassifications between clusters, especially between the two oldest groups and the adjacent group of late middle-aged adults.\\
We also studied the time points for which the classification performance of the group model was highest and found a 10~\% performance increase after stimulus onset compared to before, with a peak at 100~ms to 200~ms.

\begin{figure}[h]
\begin{center}
\input{figures/paper2_main_results.pdf_tex}
\caption[Main results of Published Research Article \uproman{2}.]{Main results of Published Research Article \uproman{2}. The evolution of classification performance of models trained to discriminate between congruent and incongruent trials differed between age groups. Mean trajectories per age group are shown here (A). Classifying between these age groups revealed clusters of groups in which misclassifications happened predominately (B). Acc.: accuracy, AUC: area under the receiver operating characteristic curve, ma: middle-aged.}
\label{fig:results2}
\end{center}
\end{figure}
\noindent The results of the task classification suggest that the distinctiveness of the cortical representation of inhibitory control does not differ with older age but that different time windows and, therefore, different processes are important for selective attention at different ages. The higher classification performance during the task than before stimulus onset underscores the added value of task-related EEG. The grouped structure of misclassifications,  especially the comparable fewer misclassifications between the oldest group and the late middle-aged groups, could reflect gross changes, e.g., after retirement. 


\subsection*{Author Contributions}
\textbf{C. Goelz}: Conceptualization, software, formal analysis, writing — original draft.
\textbf{E. Reuter}: Conceptualization, investigation, data curation, writing — review, and editing.
\textbf{S. Froehlich}: Writing — review and editing. 
\textbf{J. Rudisch}: Writing — review and editing.
\textbf{B. Godde}: Conceptualization, writing — review and editing.
\textbf{S. Vieluf}: Conceptualization, investigation, supervision, writing — review, and editing.
\textbf{C. Voelcker-Rehage}: Conceptualization, investigation, supervision, project administration, writing — review, and editing. All authors read and approved the final manuscript.
