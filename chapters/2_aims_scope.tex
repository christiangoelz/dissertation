The main goal of this dissertation is to study age-related brain reorganization, considering both global patterns and individual trajectories, by applying established methods from supervised and unsupervised machine learning to \gls{eeg} signals. More specifically, the focus is investigating age-related phenomena such as dedifferentiation, extending existing studies, and testing the replicability of hypotheses such as reserve.\\
A promising approach to studying dedifferentiation is to examine the differentiability of different task representations within a participant using classification as presented in the previous \chapref{theory:ml:applications_aging}. This approach closely follows the computational model of dedifferentiation presented in \chapref{theory:aging:dedif}, in which dedifferentiation is modeled in terms of specificity in response to different stimuli. However, research on this has so far been limited to \gls{fmri} studies, and it is unclear whether this is also applicable to \gls{eeg}. In this regard, \gls{eeg} offers several advantages in terms of practical availability, ease of use and low cost, but also the possibility of a high temporal resolution that can capture dynamic reorganization processes of the aging brain. In particular, dimensionality reduction methods in combination with the above approach could provide novel insights into the aging brain.  Furthermore, training classifiers to discriminate \gls{eeg} signals from different age groups with different lifestyle backgrounds will help verify factors contributing to healthy aging trajectories and identify corresponding \gls{eeg} markers. This approach not only expands our understanding of dedifferentiation in aging but also provides valuable information on the impact of lifestyle choices on brain function throughout the lifespan.\\
To achieve this four empirical studies use datasets with subjects from different life stages and lifestyles, including work experience and physical fitness. These datasets include experiments covering sensory, motor, and cognitive domains. Results from the analysis are presented in the following research articles that focus on specific sub-questions.\\
\\
\textbf{\hyperref[pub:paperI]{Research Article \uproman{1}}} focused on the question of whether classification methods applied to \gls{eeg} data can effectively reflect the dedifferentiated properties of brain networks in older adults, using the motor system as an example. This work builds on a previous publication that found differences between younger and older adults in \gls{eeg} markers of sensorimotor processing \cite{vieluf2018age}. Based on studies that used classification applied to \gls{fmri} to map dedifferentiation (see \chapref{theory:ml:applications_aging}) the idea was to use this in application to \gls{eeg} data and extend the previous results. We therefore, compared the classification performance, i.e., the discriminability of different visuomotor tasks, between younger and older adults.\\
\\
Continuing this approach, \textbf{\hyperref[pub:paperII]{Research Article \uproman{2}}} aimed to investigate whether the cortical representation of inhibitory control differs across age groups. Again, previously published findings, in which different mechanisms of selective attention in older adults and children were detected by using classical \gls{erp} analyses, should be extended \cite{Reuter2019}. To this end, performance on the classification of two stimulus types of a flanker task, i.e., one with high demands on inhibitory control and one with low demands on inhibitory control, was compared between different age groups. Furthermore, it was investigated whether the age group membership can be predicted based on the \gls{eeg} data. The idea was to identify predictors and to gain insight into the structure of the data set in order to capture overarching patterns of the aging brain. \\
\\
\textbf{\hyperref[pub:paperIII]{Research Article \uproman{3}}} aimed to examine the potential influence of cardiorespiratory fitness, a lifestyle factor, on patterns of dedifferentiation extracted through dimensionality reduction applied to \gls{eeg}. This investigation was motivated by the reserve hypothesis, which postulates that cardiorespiratory fitness could impact age-related brain reorganization and the observed patterns of dedifferentiation. While this has already been shown in \gls{fmri} studies mainly in relation to resting-state brain networks \cite{Stillman2019}, it is not clear whether the differentiability of task-related information processing is affected as well and whether this is reflected in the \gls{eeg}.\\
\\
In addition to cardiorespiratory fitness, another significant lifestyle factor is professional expertise. Therefore, the subsequent \textbf{\hyperref[pub:paperIV]{Research Article \uproman{4}}} aimed to characterize middle-aged experts using supervised and unsupervised machine learning techniques. In doing so, machine learning methods should be applied as a complement to previous studies in which expertise-related differences were investigated by means of classical statistical methods \cite{vieluf2018age, Goelz2018} in order to detect the influence of expertise on the dedifferentiation of fine motor tasks and to better understand the phenomenon of expertise by means of group classifications.\\
\\
The application of machine learning methods, both on individual and group levels, will allow to draw conclusions about markers of reorganization of the brain and will help to identify the individual status as well overreaching trajectories. The information gained from these tools could be used to determine and evaluate intervention programs, on-the-job-trainings, and support diagnosis, and may have applications in the development of assistive technological systems. 