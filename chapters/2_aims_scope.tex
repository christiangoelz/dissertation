The main goal of this dissertation is to study age-related brain reorganization, considering both global patterns and individual trajectories, by applying established methods from supervised and unsupervised machine learning to \gls{eeg} signals.\\
\\
A prominent aspect of age-related reorganization discussed in the literature at the individual level is the loss of specificity of neural representations or dedifferentiation that accompanies the aging process and plays an essential role in the behavioral decline. This aspect is based on a computational model and has been confirmed in animal and human studies, as presented in \chapref{theory:aging:brain} showing more similar activation patterns of brain areas or networks during task execution \cite{Carb2011, Rieck2021, Antonenko2013, Geerligs2014}. According to \citeauthor{Koen2019}, considering \gls{fmri} studies, evidence for that is quite robust when dedifferentiation is operationalized based on the original model of Li and colleagues \cite{Li2001,Li2002} as loss of neuronal selectivity for different stimuli. In other words, the assumption is that there is less difference in brain states between task conditions. In contrast, results based on \gls{eeg} studies are often ambiguous as markers, and clear operationalization of dedifferentiation is often missing. Applying machine learning techniques could add value in testing the discriminability of brain states at the individual level. Although this is already applied in \gls{fmri} studies, as presented in \chapref{theory:ml:applications_aging}, it has only been applied infrequently to \gls{eeg} data.\\
Consequently, the first approach of this work is to use machine learning to test the discriminability of task-related \gls{eeg} signals and draw conclusions about dedifferentiation. Compared with \gls{fmri} studies, this would have the advantage of more directly capturing age-related reorganization and its dynamics at the individual level while offering several advantages in terms of practical availability, low cost, and ease of use.\\
As postulated by the reserve hypothesis, lifestyle factors contribute significantly to the development of individual age trajectories, and it can be assumed that individuals with a higher degree of reserve exhibit less dedifferentiation based on adaptive and flexible resources. As presented in \chapref{theory:aging:brain}, this is based on comparative studies in which individuals with a low expression in a proxy parameter are contrasted with individuals with a high expression. Following this strategy, the second approach of this work is to compare the differentiability of tasks between individuals with low and high expression in known proxies, such as physical fitness or expertise.\\
Furthermore, detecting and better understanding global patterns of reorganization is critical to contextualize individual trajectories. As presented in \ref{theory:ml:applications_aging}, group classifiers and unsupervised methods such as dimensionality reduction techniques could be used. The third approach is, therefore, to use these methods exploratively at the group level to extract patterns and relationships from the complex data sets.\\
\\
Taken together, this work focuses on studying age-related phenomena, such as dedifferentiation, and investigating the replicability of hypotheses, such as reserve. To achieve this, four empirical studies use datasets with participants from different life stages and lifestyles, including work experience and physical fitness. These datasets include experiments covering sensory, motor, and cognitive domains. Results from the analysis are presented in the following research articles that focus on specific sub-questions.\\
\\
In \textbf{\hyperref[pub:paperI]{Research Article \uproman{1}}} we followed the first approach presented above. We investigated the difference in the performance of classifiers trained to discriminate \gls{eeg} derived brain network activation patterns during visuomotor tracking tasks between younger and older participants. The aim here was to draw conclusions about the reorganization of the motor system and extend a previous publication that found differences between younger and older adults in \gls{eeg} markers of sensorimotor processing during visuomotor tracking tasks \cite{Vieluf2018}.\\
\\
Following this approach, \textbf{\hyperref[pub:paperII]{Research Article \uproman{2}}}, aimed to investigate whether the cortical representation of inhibitory control differs across different age groups. Again, previously published findings, in which distinct mechanisms of selective attention in older adults and children were detected using classical \gls{erp} analyses, should be extended \cite{Reuter2019}. To this end, performance on the classification of two stimulus types of a flanker task, i.e., one with high demands on inhibitory control and one with low demands on inhibitory control, was compared between different age groups. Furthermore, following the third explorative approach, we investigated whether we can train a classifier to determine to which age group a participant belongs based on the \gls{eeg} data.\\
\\
\textbf{\hyperref[pub:paperIII]{Research Article \uproman{3}}} aimed to examine the potential influence of cardiorespiratory fitness, a lifestyle factor, on brain network patterns of dedifferentiation extracted through dimensionality reduction applied to \gls{eeg}. This investigation followed the second presented approach and was motivated by the reserve hypothesis, which postulates that cardiorespiratory fitness could impact age-related brain reorganization and the observed patterns of dedifferentiation. While this has already been shown in \gls{fmri} studies mainly concerning resting-state brain networks \cite{Stillman2019}, it is not clear whether the differentiability of task-related information processing is affected as well and whether this is reflected in the \gls{eeg}.\\
\\
In addition to cardiorespiratory fitness, another significant lifestyle factor is professional expertise. Therefore, the subsequent \textbf{\hyperref[pub:paperIV]{Research Article \uproman{4}}} aimed to characterize middle-aged experts using supervised and unsupervised machine learning techniques. In doing so, machine learning methods should be applied as a complement to previous studies in which expertise-related differences were investigated utilizing classical statistical methods \cite{Vieluf2018, Goelz2018} in order to detect the influence of occupational expertise on the dedifferentiation of brain network activation patterns during fine motor tasks and following the third approach to better understand the phenomenon of expertise employing group classifications.\\
\\
In summary, the application of machine learning followed three approaches presented in the research articles with the goal of better understanding individual trajectories and overarching patterns of age-related brain reorganization. The first two approaches followed established hypotheses of age-related reorganization (dedifferentiation, and reserve), while the third approach aimed to provide exploratory insights and novel findings. \autoref{tab:approaches} summarizes the application of these approaches in each study.

\begin{table}[ht]
\captionsetup{justification=raggedright,singlelinecheck=false}
\caption{Summary of the approaches followed.}
\label{tab:approaches}
\begin{tabular}{llll}
\toprule
                     & Approach 1            & Approach 2            & Approach 3            \\ \cmidrule(l){2-4}
                     & Dedifferentiation     & Reserve               & Overreaching patterns \\ \midrule
\hyperref[pub:paperI]{Research Article \uproman{1}}   & \multicolumn{1}{c}{X} &                       &                       \\
\hyperref[pub:paperII]{Research Article \uproman{2}}  & \multicolumn{1}{c}{X} &                       & \multicolumn{1}{c}{X} \\
\hyperref[pub:paperIII]{Research Article \uproman{3}} & \multicolumn{1}{c}{X} & \multicolumn{1}{c}{X} &                       \\
\hyperref[pub:paperIV]{Research Article \uproman{4}}  & \multicolumn{1}{c}{X} & \multicolumn{1}{c}{X} & \multicolumn{1}{c}{X} \\
\bottomrule
\end{tabular}
\end{table}

\noindent Applying machine learning methods on individual and group levels will allow concluding markers of brain reorganization and help identify the individual status and overreaching trajectories. The information gained from these tools could be used to determine and evaluate intervention programs, on-the-job-trainings, and support diagnosis. It may have applications in developing assistive technological systems by providing insights into decoding performance in different age groups and its relation to brain reorganization. 