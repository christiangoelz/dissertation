The main goal of this dissertation is to study age-related brain reorganization, considering both global patterns and individual trajectories, by applying established methods from supervised and unsupervised machine learning to \gls{eeg} signals. The focus is investigating age-related phenomena such as dedifferentiation, extending existing studies, and testing the replicability of hypotheses such as reserve. Four empirical studies use datasets with subjects from different life stages and lifestyles, including work experience and physical fitness. These datasets include experiments covering sensory, motor, and cognitive domains. The following published research articles will be presented.\\
\\
\textbf{\hyperref[paperI]{Research Article \uproman{1}}} focused on whether classification techniques applied to \gls{eeg} data can effectively reflect the dedifferentiated properties of brain networks in older people, using the motor system as an example. This work builds on a previous publication that found differences between younger and older adults in \gls{eeg} markers of sensorimotor processing, and the idea was to extend these results using classification \cite{vieluf2018age}. 
We, therefore, compared the classification performance, i.e., the discriminability of different visuomotor tasks, between younger and older adults.\\
\\
Continuing this approach, \textbf{\hyperref[paperII]{Research Article \uproman{2}}} aimed to investigate whether the cortical representation of inhibitory control differs across age groups. Again, previously published findings, in which different mechanisms of selective attention in older adults and children were detected, should be extended \cite{Reuter2019}. For this purpose, the performance of the classification of different stimulus types of a flanker task was compared between different age groups. Furthermore, it was investigated whether the age group membership can be predicted based on the \gls{eeg} data.\\
\\
\textbf{\hyperref[paperIII]{Research Article \uproman{3}}} aimed to examine the potential influence of cardiorespiratory fitness, a lifestyle factor, on patterns of dedifferentiation extracted through dimensionality reduction. This investigation was motivated by the reserve hypothesis, which postulates that cardiorespiratory fitness could impact age-related brain reorganization and the observed patterns of dedifferentiation.\\
\\
In addition to cardiorespiratory fitness, another significant lifestyle factor is professional expertise. Therefore, the subsequent \textbf{\hyperref[paperIV]{Research Article \uproman{4}}} aimed to characterize middle-aged experts using supervised and unsupervised machine learning techniques.\\
\\
The application of machine learning methods, both on individual and group levels, will allow to draw conclusions about predictors of reorganization of the brain and will help to identify the individual status as well overreaching trajectories. The information gained from these tools could be used to determine and evaluate intervention programs, on-the-job-trainings, and support diagnosis, and may have applications in the development of assistive technological systems. 