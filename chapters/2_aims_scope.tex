The main goal of this dissertation is to study age-related brain reorganization, considering both global patterns and individual trajectories, by applying established methods from supervised and unsupervised machine learning to \gls{eeg} signals.\\
\\
As described in the previous \chapref{theory:aging}, the reorganization of the brain of older adults is described in terms of dedifferentiated activation. According to \citeauthor{Koen2019}, the evidence of dedifferentiation is quite robust if operationalized based on the original model of Li and colleagues \cite{Li2001,Li2002} as loss of neural selectivity to different stimuli or on the network level as decreased segregation and modularity of brain networks. However, the findings are based primarily on \gls{fmri} studies. Often machine learning, more precisely classification techniques, are used in these studies to quantify dedifferentiation (see \chapref{theory:ml:applications_aging}). An application to \gls{eeg} data is pending and would allow a more direct detection of age-related reorganization processes at the individual level while offering several advantages in terms of practical availability, low cost, ease of use, and the possibility to capture the dynamics of reorganization processes of the aging brain. Consequently, the idea is to train classifiers to test the differentiability of task-related \gls{eeg} signals in order to draw conclusions about dedifferentiation. Especially in combination with dimension reduction methods, this approach could provide novel insights into the aging brain. Furthermore, training classifiers at group level to discriminate \gls{eeg} signals from different age groups with different lifestyle backgrounds will help verify factors contributing to healthy aging trajectories and identify corresponding \gls{eeg} markers. This approach not only expands our understanding of dedifferentiation in aging, but also provides valuable information about the effects of lifestyle choices on brain function across the lifespan, contributing to the understanding and testing of the reserve hypothesis.\\
To achieve this four empirical studies use datasets with subjects from different life stages and lifestyles, including work experience and physical fitness. These datasets include experiments covering sensory, motor, and cognitive domains. Results from the analysis are presented in the following research articles that focus on specific sub-questions.\\
\\
\\
In \textbf{\hyperref[pub:paperI]{Research Article \uproman{1}}} we investigated the difference in the classification performance of visuomotor tracking tasks between younger and older participants in order to draw conclusions about the reorganization of the motor system and extend a previous publication that found differences between younger and older adults in \gls{eeg} markers of sensorimotor processing during visuomotor tracking tasks \cite{vieluf2018age}.\\
\\
Continuing this approach, \textbf{\hyperref[pub:paperII]{Research Article \uproman{2}}} aimed to investigate whether the cortical representation of inhibitory control differs across different age groups. Again, previously published findings, in which distinct mechanisms of selective attention in older adults and children were detected by using classical \gls{erp} analyses, should be extended \cite{Reuter2019}. To this end, performance on the classification of two stimulus types of a flanker task, i.e., one with high demands on inhibitory control and one with low demands on inhibitory control, were compared between different age groups.Furthermore, it was investigated whether we can train a classifier that can determine to which age group a participant belongs based on the \gls{eeg} data. The idea was to identify relevant markers and gain insight into the structure of the dataset to capture overarching patterns of the aging brain.\\
\\
\textbf{\hyperref[pub:paperIII]{Research Article \uproman{3}}} aimed to examine the potential influence of cardiorespiratory fitness, a lifestyle factor, on patterns of dedifferentiation extracted through dimensionality reduction applied to \gls{eeg}. This investigation was motivated by the reserve hypothesis, which postulates that cardiorespiratory fitness could impact age-related brain reorganization and the observed patterns of dedifferentiation. While this has already been shown in \gls{fmri} studies mainly in relation to resting-state brain networks \cite{Stillman2019}, it is not clear whether the differentiability of task-related information processing is affected as well and whether this is reflected in the \gls{eeg}.\\
\\
In addition to cardiorespiratory fitness, another significant lifestyle factor is professional expertise. Therefore, the subsequent \textbf{\hyperref[pub:paperIV]{Research Article \uproman{4}}} aimed to characterize middle-aged experts using supervised and unsupervised machine learning techniques. In doing so, machine learning methods should be applied as a complement to previous studies in which expertise-related differences were investigated by means of classical statistical methods \cite{vieluf2018age, Goelz2018} in order to detect the influence of expertise on the dedifferentiation of fine motor tasks and to better understand the phenomenon of expertise by means of group classifications.\\
\\
The application of machine learning methods, both on individual and group levels, will allow to draw conclusions about markers of reorganization of the brain and will help to identify the individual status as well overreaching trajectories. The information gained from these tools could be used to determine and evaluate intervention programs, on-the-job-trainings, and support diagnosis and may have applications in the development of assistive technological systems. 