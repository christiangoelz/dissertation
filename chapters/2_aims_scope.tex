The main goal of this dissertation is to study age-related brain reorganization, considering both global patterns and individual trajectories, by applying established methods from supervised and unsupervised machine learning to \gls{eeg} signals.\\
\\
As described in the previous \chapref{theory:aging:brain}, the loss of specificity of neural representations plays a crucial role in the age-related behavioral decline. According to \citeauthor{Koen2019}, evidence of this so-called dedifferentiation is quite robust when operationalized on the basis of the original model of Li and colleagues \cite{Li2001,Li2002} as loss of neuronal selectivity for different stimuli or at the network level as reduced segregation and modularity of brain networks. However, the results are primarily based on \gls{fmri} studies, in that neural specificity is already being operationalized in part using machine learning techniques (see \chapref{theory:ml:applications_aging}). An application to \gls{eeg} data is pending and would allow more direct detection of age-related reorganization at the individual level while offering several advantages in terms of practical availability, low cost, ease of use, and the possibility to capture the dynamics of reorganization processes of the aging brain. Consequently, the idea is to train classifiers to test the differentiability of task-related \gls{eeg} signals to draw conclusions about dedifferentiation. Especially in combination with dimension reduction methods, this approach could provide novel insights into the aging brain. Furthermore, training classifiers at the group level to discriminate \gls{eeg} signals from different age groups with different lifestyle backgrounds will help verify factors contributing to healthy aging trajectories and identify corresponding \gls{eeg} markers. This approach expands the understanding of dedifferentiation in aging. It provides valuable information about the effects of lifestyle choices on brain function across the lifespan, contributing to the understanding and testing the reserve hypothesis.\\
\\
Taken together the focus of this work is on studying age-related phenomena, such as dedifferentiation, and investigating the replicability of hypotheses, such as reserve. To achieve this, four empirical studies use datasets with participants from different life stages and lifestyles, including work experience and physical fitness. These datasets include experiments covering sensory, motor, and cognitive domains. Results from the analysis are presented in the following research articles that focus on specific sub-questions.\\
\\
In \textbf{\hyperref[pub:paperI]{Research Article \uproman{1}}} we investigated the difference in the classification performance of visuomotor tracking tasks between younger and older participants in order to draw conclusions about the reorganization of the motor system and extend a previous publication that found differences between younger and older adults in \gls{eeg} markers of sensorimotor processing during visuomotor tracking tasks \cite{vieluf2018age}.\\
\\
Continuing this approach, \textbf{\hyperref[pub:paperII]{Research Article \uproman{2}}} aimed to investigate whether the cortical representation of inhibitory control differs across different age groups. Again, previously published findings, in which distinct mechanisms of selective attention in older adults and children were detected using classical \gls{erp} analyses, should be extended \cite{Reuter2019}. To this end, performance on the classification of two stimulus types of a flanker task, i.e., one with high demands on inhibitory control and one with low demands on inhibitory control, was compared between different age groups. Furthermore, it was investigated whether we can train a classifier that can determine to which age group a participant belongs based on the \gls{eeg} data. The idea was to identify relevant markers and gain insight into the dataset's structure to capture overarching patterns of the aging brain.\\
\\
\textbf{\hyperref[pub:paperIII]{Research Article \uproman{3}}} aimed to examine the potential influence of cardiorespiratory fitness, a lifestyle factor, on patterns of dedifferentiation extracted through dimensionality reduction applied to \gls{eeg}. This investigation was motivated by the reserve hypothesis, which postulates that cardiorespiratory fitness could impact age-related brain reorganization and the observed patterns of dedifferentiation. While this has already been shown in \gls{fmri} studies mainly concerning resting-state brain networks \cite{Stillman2019}, it is not clear whether the differentiability of task-related information processing is affected as well and whether this is reflected in the \gls{eeg}.\\
\\
In addition to cardiorespiratory fitness, another significant lifestyle factor is professional expertise. Therefore, the subsequent \textbf{\hyperref[pub:paperIV]{Research Article \uproman{4}}} aimed to characterize middle-aged experts using supervised and unsupervised machine learning techniques. In doing so, machine learning methods should be applied as a complement to previous studies in which expertise-related differences were investigated utilizing classical statistical methods \cite{vieluf2018age, Goelz2018} in order to detect the influence of expertise on the dedifferentiation of fine motor tasks and to better understand the phenomenon of expertise employing group classifications.\\
\\
Applying machine learning methods, both on individual and group levels, will allow concluding markers of brain reorganization and help identify the individual status and overreaching trajectories. The information gained from these tools could be used to determine and evaluate intervention programs, on-the-job-trainings, and support diagnosis. It may have applications in the development of assistive technological systems by providing insights in the performance of decoding behavior in different age groups. 