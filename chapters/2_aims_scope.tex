The primary goal of this dissertation is to investigate age-related brain reorganization, considering both global patterns and individual trajectories by applying established methods from the field of supervised and unsupervised machine learning to \gls{eeg} signals. The focus is on studying age-related phenomena, such as dedifferentiation, and investigating the replicability of hypotheses, such as reserve. In four empirical studies, diverse datasets are utilized with subjects spanning different life stages and lifestyle backgrounds, including occupational expertise and physical fitness. The following published research articles will be presented.\\
\\
\textbf{Research article \uproman{1}} focused on determining whether classification techniques applied to \gls{eeg} data could effectively reflect dedifferentiated brain network characteristics in older people, with the motor system serving as an example. We, therefore, compared the classification performance, i.e., the discriminability of different visuomotor tasks, between younger and older adults.\\
\\
Continuing previous research, \textbf{research article \uproman{2}} aimed to investigate whether the cortical representation of inhibitory control differs across age groups. For this purpose, the performance of the classification of different stimulus types of a flanker task was compared between different age groups. Furthermore, it was investigated whether the age group membership can be predicted based on the \gls{eeg} data.\\
\\
\textbf{Research article \uproman{3}} aimed to examine the potential influence of cardiorespiratory fitness, a lifestyle factor, on patterns of dedifferentiation extracted through dimensionality reduction. This investigation was motivated by the reserve hypothesis, which postulates that cardiorespiratory fitness could impact age-related brain reorganization and the observed patterns of dedifferentiation.\\
\\
In addition to cardiorespiratory fitness, another significant lifestyle factor is professional expertise. Therefore, the subsequent \textbf{research article \uproman{4}} aimed to characterize middle-aged experts using supervised and unsupervised machine learning techniques.\\
\\
The application of machine learning methods, both on individual and group levels, will allow to draw conclusions about predictors of reorganization of the brain and will help to identify the individual status as well overreaching trajectories. The information gained from these tools could be used to determine and evaluate intervention programs, on-the-job-trainings, and support diagnosis, and may have applications in the development of assistive technological systems. 