ETHICS!!!

Erster Schritt gemacht: Wir können dedifferenzierung auf individueller ebene tracken. Außerdem scheinen sich große Veränderungen auch auf Gruppenebene abzuzeichen: Ruhstand... 

Der nächste Schritt wäre es sich langfristige Veränderungen anzuschauen. In großen Datensätzen mit kontinuierlicher Altersstruktur könnte dies außerdem mit BrainAge in Zusammegbracht werden. Außerdem 


Dies geht einher mit theroetischen und viel gefunden Modellen wie reserve: Bei personen mit höherer Fitness zeigt sich dabei ein geringeres Level an dedifferenzierung und neural noise. 

Außerdem konnten neue hypothesen aufgestellt werden. Individualität von Expertise...

Dedifferenzierung unterschiedlicher Systeme scheint nicht uniform einzutreten, während dedifferenzierung des motorischen Systems schon bei late middle aged nachweisbar ist, is visuelle Aufmerksamkeit erst später wahrzunehmen... 


https://elifesciences.org/reviewed-preprints/87297#s4
