Der Umgang mit einer alternden Bevölkerung ist eine der größten Herausforderungen für unsere Gesellschaft. Gezielte Interventionen, die frühzeitige Erkennung ungünstiger Altersverläufe und unterstützende Technologien könnten zu einer Lösung beitragen, erfordern jedoch einfache, leicht anzuwendende und kosteneffiziente Methoden zur Quantifizierung und zum besseren Verständnis der altersbedingten Reorganisation des Gehirns. Der Ansatz dieser Arbeit bestand daher darin, ein solches Verständnis durch die Anwendung von Techniken des maschinellen Lernens auf \gls{eeg} zu gewinnen.\\
Zu diesem Zweck haben wir Datensätze aus Studien verwendet, die typische Paradigmen der Altersforschung verwenden, um die Informationsverarbeitung während der Ausführung motorischer, sensorischer und kognitiver Aufgaben zu untersuchen. Wir analysierten diese Datensätze mit Dimensionsreduktionsmethoden sowie Klassifikationsalgorithmen, um die Dedifferenzierung der funktionellen Organisation, d.h. den Abbau der selektiven Informationsverarbeitung und den Einfluss von Lebensstilfaktoren, zu untersuchen sowie explorative Einblicke in Gruppenmuster der altersbedingten Reorganisation des Gehirns und deren Einflussfaktoren zu gewinnen.\\
Auf diese Weise konnten wir zeigen, dass sich die Dedifferenzierung der Hirnorganisation in der Leistung von Klassifikatoren widerspiegelt, die darauf trainiert wurden, auf der Grundlage von \gls{eeg} Messungen Aufgabenzustände oder -merkmale zu unterscheiden, was Möglichkeiten eröffnet, dies auf individueller Ebene zu quantifizieren. Unsere Ergebnisse deuten darauf hin, dass verschiedene Systeme des Gehirns unterschiedliche Reorganisationsmuster aufweisen, die sich in ihrer Ausprägung und dem Alter ihres Auftretens unterscheiden. Konkret zeigte sich die Dedifferenzierung des motorischen Systems bereits bei Probanden im späten mittleren Alter, während wir für Elemente des Aufmerksamkeitssystems Unterschiede erst bei älteren Erwachsenen fanden, die sich auf die Dynamik der Dedifferenezierrung bezoogen. In diesem Zusammenhang wies die Leistung von Klassifikatoren, die zur Charakterisierung der Gruppe trainiert wurden eine geringe Rate an Fehlklassifikation zwischen Personen im späten mittleren Alter und älteren Erwachsenen hin, was auf grobe Veränderungen nach dem Eintritt in den Ruhestand hindeuten könnte.\\
Darüber hinaus konnten wir den Einfluss von Lebensstilfaktoren wie kardiorespiratorischer Fitness und beruflicher Expertise auf diese Reorganisation des alternden Gehirns charakterisieren und unterschiedliche Auswirkungen der genannten Faktoren identifizieren. Genauer gesagt fanden wir heraus, dass ältere Erwachsene mit höherer kardiorespiratorischer Fitness weniger Indikatoren für Dedifferenzierung und weniger Rauschen in den Aktivierungsmustern des Gehirns aufwiesen, während sich berufliche Expertise eher in einer hochgradig individualisierten Gehirnorganisation zeigte.\\
Insgesamt konnten in dieser Arbeit Methoden des maschinellen Lernens eingesetzt werden, um Hypothesen über altersbedingte Veränderungen zu testen und neue Hypothesen aufzustellen. Die Möglichkeit, altersbedingte Reorganisation auf individueller Ebene zu quantifizieren, könnte direkte praktische Relevanz für die Entwicklung von Markern zur Früherkennung ungünstiger Altersverläufe und für die Entwicklung von Rehabilitations- oder Präventionstherapien haben. Darüber hinaus könnten die nachgewiesenen Wechselwirkungen zwischen altersbedingter Hirnreorganisation und der Leistung von Klassifikatoren wichtig für die Entwicklung von Hilfstechnologien sein, die auf der Nutzung von Hirnsignalen beruhen und für die die Leistung entscheidend ist.