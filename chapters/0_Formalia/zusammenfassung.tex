Der Umgang mit einer alternden Bevölkerung ist eine der größten Herausforderungen für unsere Gesellschaft. Gesundes Altern kann durch gezielte Interventionen, Früherkennung ungünstiger Alterungsverläufe und unterstützende Technologien gefördert werden. Dies setzt jedoch ein umfassendes Verständnis der Dynamik des Alterns voraus. Da der Abbau kognitiver und körperlicher Fähigkeiten mit Veränderungen auf der funktionellen Ebene des Gehirns einhergeht, ist es von besonderem Interesse, einfache und kostengünstige Methoden zur Quantifizierung dieser Veränderungen zu entwickeln, um das Verständnis der Reorganisation auf dieser Ebene zu verbessern. Der Ansatz dieser Arbeit war es daher, ein solches Verständnis durch die Anwendung von Methoden des maschinellen Lernens auf die Elektroenzephalographie (EEG) zu gewinnen.\\
Zu diesem Zweck haben wir Datensätze aus Studien verwendet, die typische Paradigmen der Altersforschung verwenden, um die Informationsverarbeitung während der Ausführung motorischer, sensorischer und kognitiver Aufgaben zu untersuchen. Wir analysierten diese Datensätze mit Methoden der Dimensionsreduktion und Klassifikationsalgorithmen, um die Dedifferenzierung der funktionellen Organisation des Gehirns, d.h. den Abbau der selektiven Informationsverarbeitung und den Einfluss von Lebensstilfaktoren zu untersuchen. Ebenso sollten explorative Einblicke in Gruppenmuste der altersbedingten Hirnreorganisation und deren Einflussfaktoren gewonnen werden.\\
Auf diese Weise konnten wir zeigen, dass sich die Dedifferenzierung der Gehirnorganisation in der Leistung von Klassifikatoren widerspiegelt, die darauf trainiert wurden, auf der Grundlage von EEG Aufgabenzustände oder -merkmale zu unterscheiden. Dies eröffnet die Möglichkeit, die Reorganisation des Gehirns auf individueller Ebene zu quantifizieren. Unsere Ergebnisse deuten darauf hin, dass verschiedene Gehirnsysteme unterschiedliche Reorganisationsmuster aufweisen, die sich in ihrer Ausprägung und dem Alter, in dem sie auftreten, unterscheiden. Im Einzelnen zeigte sich die Dedifferenzierung des motorischen Systems bereits bei Teilnehmern im späten mittleren Alter, während wir für Elemente des Aufmerksamkeitssystems Unterschiede in der Dynamik der Differenzierbarkeit erst bei älteren Erwachsenen fanden. In diesem Zusammenhang war die Vorhersage der Gruppenzugehörigkeit bei Teilnehmern im späten mittleren Alter und älteren Erwachsenen mit geringen Fehlklassifikationen verbunden. Diese Differenzierbarkeit könnte auf Veränderungen nach dem Eintritt in den Ruhestand hindeuten.\\
Darüber hinaus haben wir den Einfluss von Lebensstilfaktoren, wie kardiorespiratorische Fitness und berufliche Expertise, auf die Reorganisation des alternden Gehirns charakterisiert und unterschiedliche Auswirkungen der genannten Faktoren festgestellt. Ältere Erwachsene mit höherer kardiorespiratorischer Fitness wiesen weniger Indikatoren für Dedifferenzierung und weniger Rauschen in den Aktivierungsmustern des Gehirns auf, wohingegen sich berufliche Expertise eher in einer hochgradig individualisierten Gehirnorganisation zeigte.\\
Zusammenfassend wurden Methoden des maschinellen Lernens eingesetzt, um Hypothesen über altersbedingte Veränderungen zu testen und neue Hypothesen zu entwickeln und einzigartige Einblicke zu erhalten. Die Möglichkeit, altersbedingte Reorganisation auf individueller Ebene zu quantifizieren, könnte praktische Bedeutung für die Entwicklung von Markern zur Früherkennung ungünstiger Altersverläufe und für die Entwicklung von Rehabilitations- oder Präventionstherapien haben. Darüber hinaus könnten die nachgewiesenen Wechselwirkungen zwischen altersbedingter Reorganisation des Gehirns und der Leistung von Klassifikatoren wichtig für die Entwicklung von assistiver Technologien sein, die auf der Nutzung von Hirnsignalen basieren und für die die Leistung entscheidend ist.