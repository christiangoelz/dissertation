Der Umgang mit einer alternden Bevölkerung ist eine der größten Herausforderungen für unsere Gesellschaft. Gezielte Interventionen, die frühzeitige Erkennung ungünstiger Altersverläufe und unterstützende Technologien könnten zu einer Lösung beitragen, erfordern jedoch einfache, leicht anzuwendende und kosteneffiziente Methoden zur Quantifizierung und zum besseren Verständnis altersbedingter Reorganisation des Gehirns. Der Ansatz dieser Arbeit war daher, ein solches Verständnis durch die Anwendung von Methoden des maschinellen Lernens auf die Elektroenzephalographie (EEG) zu gewinnen.\\
Zu diesem Zweck haben wir Datensätze aus Studien verwendet, die typische Paradigmen der Altersforschung verwenden, um die Informationsverarbeitung während der Ausführung motorischer, sensorischer und kognitiver Aufgaben zu untersuchen. Wir analysierten diese Datensätze mit Methoden der Dimensionsreduktion und Klassifikationsalgorithmen, um die Dedifferenzierung der funktionellen Organisation, d.h. den Abbau der selektiven Informationsverarbeitung und den Einfluss von Lebensstilfaktoren zu untersuchen. Ebenso sollten explorative Einblicke in Gruppenmuster der altersbedingten Hirnreorganisation und deren Einflussfaktoren gewonnen werden.\\
Auf diese Weise konnten wir zeigen, dass sich die Dedifferenzierung der Gehirnorganisation in der Leistung von Klassifikatoren widerspiegelt, die darauf trainiert wurden, auf der Grundlage von EEG Aufgabenzustände oder -merkmale zu unterscheiden, was Möglichkeiten eröffnet, dies auf individueller Ebene zu quantifizieren. Unsere Ergebnisse deuten darauf hin, dass verschiedene Gehirnsysteme unterschiedliche Reorganisationsmuster aufweisen, die sich in ihrer Ausprägung und dem Alter ihres Auftretens unterscheiden. Im Einzelnen zeigte sich die Dedifferenzierung des motorischen Systems bereits bei Teilnehmern im späten mittleren Alter, während wir für Elemente des Aufmerksamkeitssystems Unterschiede in der Dynamik der Differenzierbarkeit erst bei älteren Erwachsenen fanden. In diesem Zusammenhang zeigte die Leistung von Klassifikatoren, die zur Vorhersage der Gruppenzugehörigkeit trainiert wurden, nur geringe Fehlklassifikationen zwischen Teilnehmern im späten mittleren Alter und älteren Erwachsenen, was auf grobe Veränderungen nach dem Eintritt in den Ruhestand hindeuten könnte.\\
Darüber hinaus haben wir den Einfluss von Lebensstilfaktoren, wie kardiorespiratorische Fitness und berufliche Expertise, auf die Reorganisation des alternden Gehirns charakterisiert und unterschiedliche Auswirkungen der genannten Faktoren festgestellt. Genauer gesagt fanden wir heraus, dass ältere Erwachsene mit höherer kardiorespiratorischer Fitness weniger Indikatoren für Dedifferenzierung und weniger Rauschen in den Aktivierungsmustern des Gehirns aufwiesen, wohingegen sich berufliche Expertise eher in einer hochgradig individualisierten Gehirnorganisation zeigte.\\
Zusammenfassend wurden Methoden des maschinellen Lernens eingesetzt, um Hypothesen über altersbedingte Veränderungen zu testen und neue Hypothesen zu entwickeln und einzigartige Einblicke zu erhalten. Die Möglichkeit, altersbedingte Reorganisation auf individueller Ebene zu quantifizieren, könnte praktische Bedeutung für die Entwicklung von Markern zur Früherkennung ungünstiger Altersverläufe und für die Entwicklung von Rehabilitations- oder Präventionstherapien haben. Darüber hinaus könnten die nachgewiesenen Wechselwirkungen zwischen altersbedingter Reorganisation des Gehirns und der Leistung von Klassifikatoren wichtig für die Entwicklung von assistiver Technologien sein, die auf der Nutzung von Hirnsignalen basieren und für die die Leistung entscheidend ist.

