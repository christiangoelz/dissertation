Ziel dieser Dissertation war die Erweiterung des Verständnisses altersbedingter funktioneller Hirnreorganisation durch Anwendung maschinellen Lernens auf die Elektroenzephalographie (EEG).\\
Basierend auf EEG-Daten, die während sensorischer, kognitiver oder motorischer Aufgaben abgeleitet wurden, wurden Klassifikationsalgorithmen trainiert, die die jeweilige ausgeführte Aufgabe sowie die individuelle Alters- und Lebensstilgruppe vorhersagen. Dimensionsreduktionstechniken wurden eingesetzt, um EEG-Muster zu extrahieren, die den Zusammenhang zwischen altersbedingter Hirnreorganisation und Lebensstilfaktoren aufzeigen.\\
Die Performanz der Klassifikatoren offenbarte aufgabenspezifische Signaturen der Dedifferenzierung, d.h. des Verlusts der Spezialisierung neuraler Systeme. Zudem gaben die Ergebnisse Hinweise auf kognitive Veränderungen in bestimmten Lebensphasen, z.B. nach Renteneintritt, und den Einfluss von Lebensstilfaktoren. Eine hohe kardiorespiratorische Fitness war mit einer geringeren Dedifferenzierung verbunden und berufliche Expertise führte zu einer stärkeren Individualisierung funktioneller Hirnaktivierungsmuster.\\
Maschinelles Lernen ermöglichte die Identifikation und Quantifizierung altersbedingter Veränderungen des Gehirns. Dadurch wurden bestehende Erkenntnisse gestützt und darüber hinaus neue Hypothesen aufgestellt. Diese Ergebnisse könnten zur Entwicklung von Diagnoseinstrumenten, Therapien oder technischen Assistenzsystemen beitragen.