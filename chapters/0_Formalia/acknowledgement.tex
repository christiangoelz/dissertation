Diese Arbeit wäre ohne das hervorragende Umfeld, in dem ich die letzten Jahre verbringen durfte, nicht möglich gewesen.\\
Daher möchte ich mich an erster Stelle ganz herzlich bei Dr. Solveig Vieluf bedanken, die mich bereits als wissenschaftliche Hilfskraft so intensiv gefördert hat und mein Promotionsvorhaben über die Jahre hinweg so exzellent und aufopferungsvoll unterstützt hat. Darüber hinaus möchte ich mich bei Prof. Dr. Dr. Claus Reinsberger bedanken, der mir so viel ermöglicht hat und von dem ich in jedem unserer Gespräche und Treffen unglaublich viel lernen durfte. Ich hätte mir keine besseren Betreuer als diese beiden vorstellen können!\\
Ebenso möchte ich mich bei Prof. Dr. Claudia Voelcker-Rehage bedanken, die einen wesentlichen Beitrag zu den dieser Arbeit zugrundeliegenden Teilprojekten geleistet hat. In diesem Zusammenhang möchte ich mich auch bei den Studienteams der Bremer-Hand-Studie@Jacobs und aller weiteren Studien, die meiner Dissertation zugrunde liegen, bedanken. Insbesondere danke ich Prof. Dr. Ben Godde, Dr. Eva-Maria Reuter, Dr. Julian Rudisch und Stephanie Fröhlich für ihre Unterstützung. Ein ganz besonderer Dank gilt Dr. Karin Mora, die mir mit ihrer mathematischen Expertise stets zur Seite stand und damit einen wesentlichen Beitrag zu dieser Arbeit geleistet hat.
Bedanken möchte ich mich auch bei meinen Kolleginnen und Kollegen am Institut für Sportmedizin. Jeder Einzelne hat zu einem durchweg positiven Arbeitsklima beigetragen. Besonders hervorheben möchte ich Roman Gaidai, Fraziska Hasse, Julia Gowik und Franziska van den Bongard, die mir sowohl fachlich als auch persönlich immer mit Rat und Tat zur Seite standen.\\
Ohne die emotionale Unterstützung und den bedingungslosen Rückhalt meiner lieben Eltern, meiner Geschwister und meiner Freundin Heike wäre diese Arbeit nicht möglich gewesen. Euch allen gilt mein tiefster Dank!