% However, such applications in research and smart technologies are still under development.\\
% Acceptance and targeted use of \gls{ai} based technology requires piloting of application areas in science and practice. 

%%%%%%%%%%%%%%%%%%%%%%%%%%%%%
%%% LOOSE IDEAS FOR LATER %%%
%%%%%%%%%%%%%%%%%%%%%%%%%%%%%
% Besides the early automatic detection of pathological changes, . Other applications include systems to assist older adults in daily living or care assistive technology, such as robotic tools for neurorehabilitation.

% As mentioned at the beginning, the promotion of healthy ageing is one of the most important consequences
% of an aging society. To quantify the urgency of this challenge, the \gls{un} forecasts that the proportion of the world's population over 65 will rise from current 10 percent to over 16 percent by 2050, while life expectancy will continue to increase \cite{united2022world}. To advance research and society-wide efforts, the \gls{un} has therefore proclaimed the \textit{Decade for Healthy Aging} (2021-2030). In this, the \gls{who} defines healthy aging in a holistic way as the development and preservation of functional ability reflecting physical and mental capacities as well as environmental characteristics \cite{who_decade_ha2020}


% Tools from \gls{ai}, especially machine learning, could be used to describe individual trajectories of aging, helping to identify biomarkers that could be used to verify treatments as well as targeted interventions

% Identfying risk factors, developing plans to support healthy aging 
% The practical acceptance and use of AI requires an evaluation of application areas in science and practice. 
% \begin{itemize}
%  \item Age related changes occur at different scales and are manifestet at several levels.
%  \item There is a wide variety in how this changes occur
%  \item Changes are e.g. neural dedifferentiation and compensatory mechanisms (see Reuter Lorenz et al. 2010) and are noticable brain network level and dynamics
%  \item NOTE: Check what EEG studies said about this...
%  \item The idea is to model these changes with tools from datascience to answer questions in aging neuroscience
%  \item First study is about detecting dedifferentiated and compensatory mechanisms with EEG
%  \item Tools used are DMD and Machine Learning
%  \item Main idea: Study classification performance as proxy for age related changes in different motor control tasks
%  \item Expertise as possible way of building a reserve:
%  \item Higher individuality 
%  \item Dynamics of dedifferentiation and how do they relate to fitness
%  \item Basic for targeted interventions 
%  \item How much and what (relate to Julia)
%  \item Background of ML
%  \item ML as tool 
%  \item novel insights s
%  \item Problem: Data is multidimensional and we have often limited data 
%  \item Solution: Use DMD to reduce Complexity and "model" evolution of signal 
%  \item Dynamic Mode Decompsition
%  \item DMD extracts coupled spatio-temporal modes and is able to kind of model the evolution of the signal 
%  \item Backgrouund + Papers 
%  \item Mathematical Formulation
%  \item What can ML tell us?
%  \item ML applied in aging Neuroscience
%  \item Formulating Aims and goals 
%  \item Formulation expected outcomes
% \end{itemize}


Struktur: 

- Das Alternde Gehirn im Kontext von individuellen Veränderungen und übergreifenden Mustern zu verstehen ist ausschlaggebend für gezielte Interventionen die ein gesundes Altern fördern, Therapien die einer ungünstigen Trajektorie möglichst früh entgegenwirken und um gessundes Altern von ungesundem altern differenzierren zu können. 
- Dies setzt eine quantifiezierung des Alternden Gehirns voraus 
- Sowie sensitive Marker die Altern anzeigen
Was brauchen wir dafür? 
- Einfache Möglichkeiten zur Erfassung und quantifizierung 
- Insbesondere Machine learning kann ein Game Changer sein 

Ziel dieser Arbei war es folglich Altersbezogene reorganisationprozessen zu verstehene in dem ML eingesetzt wird.  
- Ergebnissse Studie 1:
    - Wir haben Dedifferenzierung auf EEG Ebene detektiert indem wir Klasssifikatoren eingesetzt haben 
    - Wir haben Kompensation detektiert 
    --> Ergebnisse tragen zur Diskussion zu Dedifferenzierung und Kompensation bei. 







The overreaching objective of this thesis was to gain insights into the age-related reorganization of the brain by applying supervised and unsupervised machine learning to electrophysiological data. 

Erster Schritt gemacht: Wir können dedifferenzierung auf individueller ebene tracken. Außerdem scheinen sich große Veränderungen auch auf Gruppenebene abzuzeichen: Ruhstand... 

Der nächste Schritt wäre es sich langfristige Veränderungen anzuschauen. In großen Datensätzen mit kontinuierlicher Altersstruktur könnte dies außerdem mit BrainAge in Zusammegbracht werden.  


Dies geht einher mit theroetischen und viel gefunden Modellen wie reserve: Bei personen mit höherer Fitness zeigt sich dabei ein geringeres Level an dedifferenzierung und neural noise. 

Außerdem konnten neue hypothesen aufgestellt werden. Individualität von Expertise...

Dedifferenzierung unterschiedlicher Systeme scheint nicht uniform einzutreten, während dedifferenzierung des motorischen Systems schon bei late middle aged nachweisbar ist, is visuelle Aufmerksamkeit erst später wahrzunehmen... 

Paper2 
Außerdem scheinen nicht alle reaktionen auf Stimuli von Dedifferenzierung betroffen zu sein (aufmerksamkeit) 

https://elifesciences.org/reviewed-preprints/87297#s4

Implikationen für die Praxis 
- Entwicklung von BCI 
- Veränderungen zeigen sich in der Dekodierungssleistung von Classifieren. 
- Classifier werden standardmäßig in Systemen eingesetzt die...
- Dekodierung von Expertise ist gerade im Hinblick auf adaptive systeme interessant die 

Wie trägt diese Arbeit zum Ziel der WHO adding life to years bei?

ETHICS!!!