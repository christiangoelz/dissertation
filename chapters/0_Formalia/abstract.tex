Dealing with an aging population is one of our society's most significant challenges. Targeted interventions, early detection of unfavorable aging trajectories, and assistive technologies could all help to provide a solution but require simple, easy-to-use, and cost-effective methods to quantify and better understand age-related reorganization of the brain. The approach of this work was, therefore, to gain such an understanding by applying machine learning techniques to \gls{eeg}.\\
To this end, we used datasets obtained from studies using paradigms typical in aging research to investigate information processing during the performance of motor, sensory, and cognitive tasks. We analyzed these datasets using dimensionality reduction methods and classification algorithms to investigate the dedifferentiation of functional organization, i.e., the deterioration of selective information processing and the influence of lifestyle factors, as well as to gain exploratory insights into group patterns of age-related brain reorganization and their influencing factors.\\
In this way, we demonstrated that the dedifferentiation of the brain's organization is reflected in the performance of classifiers trained to differentiate different task states or characteristics on the basis of \gls{eeg} measurements, which opens up possibilities to quantify this at the individual level. Following this, our results suggest that different brain systems show different patterns of reorganization that differ in their expression and age of onset. More specifically, dedifferentiation of the motor system was already evident in late middle-aged subjects, whereas for elements of the attentional system, we found differences in the dynamics of differentiability only in older adults. In this context, the performance of classifiers trained to characterize the group suggests little misclassification between late middle-aged and older adults, which could indicate gross changes after retirement.\\
Furthermore, we could characterize the influence of lifestyle factors, such as cardiorespiratory fitness and occupational expertise, on this reorganization of the aging brain and could identify differential effects of the mentioned factors. More specifically, we found that older adults who exhibited higher cardiorespiratory fitness showed fewer indicators of dedifferentiation and less noise in brain activation patterns, whereas professional expertise was more likely to show up in highly individualized brain organization.\\
Altogether, machine learning methods could be used in this work to test hypotheses about age-related changes and to generate new hypotheses. The possibility to quantify age-related reorganization on an individual level could have direct practical relevance for the development of markers for the early detection of unfavorable age trajectories and for the development of rehabilitation or prevention therapies. Furthermore, the demonstrated interactions between age-related brain reorganization and classifier performance could be important for the development of assistive technology that is based on the use of brain signals and for which performance is crucial.

