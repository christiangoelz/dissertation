Coping with an aging society is one of the greatest challenges facing our society. Targeted interventions, early detection of unfavorable developments and assistive technologies can contribute to this and require simple, easy-to-use and cost-effective methods that allow quantifying and better understanding age-related changes in the brain.  The approach of this work was, therefore, to gain such an understanding by applying machine learning techniques. Four research articles were presented in which classification and dimensionality reduction methods were used to detect and quantify dedifferentiated mechanisms, influencing factors and get exploratory insights. 

