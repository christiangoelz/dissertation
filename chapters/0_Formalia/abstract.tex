Dealing with an aging population is one of the major challenges facing our society. Promoting healthy aging through targeted interventions, early detection of unfavorable aging trajectories, and assistive technologies could help to address this challenge but require a deep understanding of the dynamics of aging. Given that cognitive and physical decline are related to changes at the functional level of the brain, it is of particular interest to develop simple and cost-effective ways to quantify these changes, thereby improving our understanding of reorganization at this level. Therefore, this dissertation's approach was to extend the understanding by applying machine learning techniques to \gls{eeg}, providing novel evidence for theoretical constructs and unique insights.\\
To this end, we used datasets obtained from studies using paradigms typical in aging research to investigate information processing during the performance of motor, sensory, and cognitive tasks. We analyzed these datasets using dimensionality reduction methods and classification algorithms to investigate the dedifferentiation of functional organization, i.e., the deterioration of selective information processing and the influence of lifestyle factors. Likewise, to gain unique exploratory insights into group patterns of age-related brain reorganization and their influencing factors.\\
In this way, we demonstrated that the dedifferentiation of the brain's organization is reflected in the performance of classifiers trained to differentiate different task states or characteristics on the basis of \gls{eeg} measurements, which opens up possibilities to quantify brain reorganization at the individual level. Following this, our results suggest that different brain systems show different patterns of reorganization that differ in their expression and age of onset. More specifically, dedifferentiation of the motor system was already evident in late middle-aged participants, whereas for elements of the attentional system, we found differences in the dynamics of differentiability only in older adults. In this context, the performance of classifiers trained to predict the group membership showed little misclassification between late middle-aged and older adults, which could indicate gross changes after retirement.\\
Furthermore, we characterized the influence of lifestyle factors, such as cardiorespiratory fitness and occupational expertise, on the reorganization of the aging brain and identified differential effects of the mentioned factors. Older adults who exhibited higher cardiorespiratory fitness showed fewer indicators of dedifferentiation and less noise in brain activation patterns, whereas professional expertise was more likely to show up in highly individualized brain organization.\\
Altogether, machine learning methods were used to verify hypotheses about age-related changes and generate novel hypotheses and insights. The possibility of quantifying age-related reorganization on an individual level may have practical relevance for the development of markers for the early detection of unfavorable age trajectories and for the development of rehabilitation or prevention therapies. Darüber hinaus könnten die aufgezeigten Wechselwirkungen zwischen der altersbedingten Umstrukturierung des Gehirns und der Leistung des Klassifizierers wichtig für die Entwicklung von Hilfstechnologien sein, die sich auf Gehirnsignale stützen und bei denen die Leistung ein entscheidender Faktor ist.

