Das übergeoordnete Ziel dieser Arbeit war es die reorganisations des allternden Gehrins durch die Anwendung maschineller Lernverfaheren besser zu verstehen. Literaturbasiert wurden dazu in dieser Arbeit drei Ansatzpunkte hergeleitet, die in vier Veröffentlichungen aufgegriffen wurden. Im folgenden sollen diese einteln Diskutiert werden. 

\section{Zur detektion dedifferenzierter Aktivität}
Der erste Ansatzpunkt leitete sich aus der Dedifferenzierungshypothese ab, die wie im 


\section{Charakterisierung von Einflussfaktoren}


\section{Explorative Ansätze}

\section{Konsequenzen für die Praxis}
Einer der größten Heruasforderungen unserer Gesellschaft ist es mit einer alternden Gesellscaft klarzukommen. Dies setzt gezielte Interventionen, die früher Detektion von ungünstigen trajektorien und assisstive technologie voraus. Um dies zu erreich ist ein Verstännis individueller trajektorien alltersbedingter Abbauprozesse unabdingbar. Der Ansatz dieser Arbeit war es ein solches Verständnis durch die Anwwendung maschineller Lernverfahren zu erlangen. 
 

"CRUNCH proposes that during task performance, as task difficulty (or load) increases, more cortical regions will be activated. Older adults reach their load capacity sooner than younger adults, so at easy and intermediate levels of task difficulty, they will recruit more neural resources than younger adults – the classic ‘compensation’ effect."


However it has been observed that wen adults with high levells of reserve. as indicated by one or more reserve proxies, do eventually display cognitive decline...