
Einer der größten Heruasforderungen unserer Gesellschaft ist es mit einer alternden Gesellscaft klarzukommen. Dies setzt gezielte Interventionen, die früher Detektion von ungünstigen trajektorien und assisstive technologie voraus. Um dies zu erreich ist ein Verstännis individueller trajektorien alltersbedingter Abbauprozesse unabdingbar. Der Ansatz dieser Arbeit war es ein solches Verständnis durch die Anwwendung maschineller Lernverfahren zu erlangen. 

Erkenntnissen zuvolge ist der Abbau von kognition und Verhalten zumindest teilweise auf die Dedifferenzierung der neuralenm reprsentation zurückzuführen. Eine direkte quantifizierung ist dabei bislang nur in fMRT beschrieben. Die Idee die in den ersten beiden Studien Anwendung fand, war es folglich Klassifikatoren zu nutzen, um die Differnzierrbarkeit von Aufgabenrepresentationen innerhalb von Probanden zu testen.  
 

"CRUNCH proposes that during task performance, as task difficulty (or load) increases, more cortical regions will be activated. Older adults reach their load capacity sooner than younger adults, so at easy and intermediate levels of task difficulty, they will recruit more neural resources than younger adults – the classic ‘compensation’ effect."