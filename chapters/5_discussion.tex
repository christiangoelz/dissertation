This work aimed to better understand the reorganization of the aging brain by applying machine learning techniques to \gls{eeg} data. Three approaches were derived from the literature and used in four publications.
The first approach was based on a reorganization of the aging brain often described in the literature, namely dedifferentiation, i.e., less specific activation of brain networks expressed in a lower differentiability of brain states. Based on a computational model, this could be due to the loss of selectivity of neuronal responsiveness, which has been confirmed in numerous animal and human studies. Building upon this, we tested the differentiability of brain states, e.g., in the form of activation patterns of brain networks, and thus the selectivity of neuronal representations using machine learning methods. This approach provided insights into age-related dedifferentiation and allowed the formulation of new hypotheses, e.g., regarding this phenomenon's temporal aspect.\\
The second approach of this work addressed the reserve hypothesis. We examined the possible positive effects of certain lifestyle factors, such as physical fitness or occupational expertise, on brain reorganization during aging. Using machine learning techniques, we compared individuals with different lifestyle backgrounds and the effects on age-related brain reorganization. This analysis provided new perspectives and helped confirm certain predictions of the reserve hypothesis.\\
In the third approach, dimensionality reduction and group-level classification algorithms were used to identify overarching patterns in brain reorganization. This approach enabled the generation of new insights and hypotheses by these methods on high-dimensional data.\\
In the following, the main results of these approaches are discussed. However, for a detailed discussion of each specific result, please refer to the corresponding published journal articles.\\

\section{On the detection of dedifferentiated brain organization}
Motivated by applications from research using \gls{fmri}, in which dedifferentiation is quantified based on the performance of classification algorithms, we applied this to \gls{eeg} data to investigate the differentiability of signals in motor, cognitive and sensory tasks in participants of different age groups.\\
In \hyperref[results:paperI]{Research Article \uproman{1}}, we found differences in the classification performance of fine motor tasks based on \gls{eeg} data. In particular, performance in classifying which hand was used to perform the task was reduced in the late middle-aged compared with younger participants. Studying the classifier input, i.e., the spatiotemporal coherent activation patterns extracted with \gls{dmd}, we found that this corresponds to differences in the electrophysiological network activation patterns pointing to more bilateral and widespread activation in late middle-aged adults. These results correspond to a dedifferentiated motor network as reported in \gls{fmri} studies \cite{Carb2011, Cassedy2020} and are also consistent with other \gls{eeg} findings that have compared classification between age groups finding reduced performance in the classification of the body side of task execution in older adults \cite{Chen2019, Zich2015}. On the other hand, we found an increased classification performance in the late middle-aged participants when predicting whether the participant tracked a steady or sinusoidal target force. When considering the classifier input again, we found a stronger frontal involvement in the network activation patterns of the late middle-aged participants and thus suspected that compensation processes play a role here. Compensation could be present,  especially in the more complex sinusoidal force tracking, and thus positively influence the differentiability of sinusoidal and steady force tracking. This result is consistent with the \gls{crunch} hypothesis, which postulated that during task performance, as task difficulty (or load) increases, more cortical resources will be activated \cite{Festini2018}. Combining both results, we could demonstrate that age-related reorganization of the aging brain is reflected in the performance of classifiers based on EEG measurements, thus providing new insights into the aging brain.\\
In \hyperref[results:paperII]{Research Article \uproman{2}}, we again utilized classification algorithms to investigate whether the cortical representation of selective attention differed between six different age groups ranging from children to older adults. Here we found that the discriminability between stimuli requiring and stimuli not requiring inhibitory control was reduced only in the children's group. This reduced discriminability in the group of children goes along with the assumption of inhibitory control in children that is not yet differentiated at the neural level \cite{Waszak2010, Reuter2019}. However, the results of similar good performance of the classifiability of the two stimulus categories contradict the assumption of a general dedifferentiation of all cortical systems. Similar results are also known in \gls{fmri} studies, which, for example, found no dedifferentiation in the activation of the visual system in response to certain stimulus categories \cite{Voss2008}. By using \gls{eeg}, however, we were able to investigate the temporal response to the stimulus in addition. For this reason, the classifier input did not represent coherent activation patterns as in the previously described analysis, but rather single-trial \gls{eeg} traces filtered by xDAWN in response to the stimuli, through which it was possible to examine the performance of the classifier over time and thus gain insight into the temporal dynamics of information processing. In this way, we found that the differentiability of the stimuli was delayed in older adults.\\
Taken together, dedifferentiation might also have a dynamic component related to the rise of neural noise, as postulated in the computational model of \citeauthor{Li2001} \cite{Li2001, Li2000}. However, this only occurred in the two oldest age groups and not, as in the previous study, in the late middle-aged, so this effect might occur later in the aging process.

\section{The role of lifestyle factors}
An important aspect to better understand the individuality of age-related alterations is to investigate how influencing factors affect the reorganization of the aging brain. From the literature, physical fitness can be deduced as a significant influencing factor that affects the brain's ability to resist degradation processes and maintain function, thus contributing to reserve. In this context, \citeauthor{Stillman2019} \cite{Stillman2019} summarizes findings of \gls{fmri} studies proving that the network structure of the brain at rest of physically fit people show fewer signs of dedifferentiation. Therefore, in \hyperref[results:paperIII]{Research Article \uproman{3}} we investigated to what extent cardiorespiratory fitness affects the dedifferentiation of task-related network activation patterns based on \gls{eeg}. To do so, we extracted dominant spatiotemporal network patterns and their prominence over time during the performance of motor, sensory and cognitive tasks. In this way, we were able to estimate task-related brain network activation patterns in the main domains in which age-related changes are reported and compare this between a fit and less fit group \cite{Baltes1997, Sala-Llonch2015, Park2009}. This comparison revealed greater differences between the task-related activation patterns in the fit compared to the less fit group, suggesting lower levels of dedifferentiation in the less fit group. We further found that the extracted dominant modes of the fit group resolved more variance compared to the less fit group, indicating higher stability or prominence of the extracted patterns in the fit group. These differences were frequency dependent mainly to the $\theta$ and $\beta_2$ bands which could be carefully attributed to processes reflecting cognitive involvement and information integration \cite{Siegel2012}. From the perspective of the computation model of dedifferentiation, this could indicate less neural noise and a lower rate of neural variability, which points to a higher efficiency of information processing. Overall, these results follow the prediction of the reserve hypothesis and provide evidence that supports the assumptions of the computational model of dedifferentiation.\\
Another frequently studied factor of influence in the literature is professional expertise. Thus, in \hyperref[results:paperIV]{Research Article \uproman{4}}, we compared fine motor experts and novices with respect to the classification performance of visuomotor tasks. We found no group differences in the performance of a classifier trained to classify the same task characteristics as in \hyperref[results:paperI]{Research Article \uproman{1}}. Consequently, we could not detect any differences in the differentiation of the neural representation of the task between novices and experts. At first glance, these results contradict previous studies from the Bremen-Hand-Study@Jacobs, which showed statistical differences in the expression of network characteristics of experts compared to novices in terms of neural efficiency \cite{Goelz2018, Vieluf2018}. However, our study tested a particular assumption of age-related reorganization, namely a reduced differentiation, and did not test neural efficiency as in the previous work. Also, in contrast to \hyperref[results:paperIII]{Research Article \uproman{3}}, the results seem to contradict the reserve concept, based on which a lower level of dedifferentiation would have been expected. Therefore, taking a more differentiated view of the reserve concept concerning individual factors is necessary. We have only explored one aspect in considering dedifferentiation in the aging brain. As a result, other characteristics of brain reorganization may differ, such as the efficiency of information processing or compensational network involvement. These factors were not accounted for in our analyses.

\section{Exploratory insights into age-related reorganization}
While the previous results refer to the application at the participant level, we also used the machine learning methods at the group level to gain explorative insights into the overreaching group patterns.\\
In \hyperref[results:paperII]{research article \uproman{2}}, we found a grouped structure of misclassifications when using a classifier that was trained to determine the group membership in one of the six age groups based on \gls{eeg} data. The classifier tended to misclassify only between certain age groups. This was especially noticeable in the two oldest age groups, between which the misclassification rate was highest. Less frequently, however, older participants were classified into the group of late middle-aged adults and vice versa. Considering the age structure of the older adults (minimum age of 65 years) and the middle-aged adult (maximum age of 64 years) groups, as well as a retirement age of around 65, this could indicate major cognitive changes after retirement. Deterioration of cognitive abilities and cerebral blood flow after retirement has been reported in the literature \cite{Celidoni2017, Rohwedder2010, Rogers1990}. However, we could not further test this hypothesis because of a lack of data on the retirement age of the participants. In addition, no misclassification occurred between children and the oldest groups. Although differences based on \gls{erp} markers are often described as u-shaped \cite{Mueller2008, Reuter2019}, this result suggests different mechanisms, i.e., differentiation in children vs. dedifferentiation in older adults. Overall, we found an increase in classification performance shortly after stimulus onset of about 10~\%, highlighting the added value of task-related \gls{eeg} recordings and suggesting early processing of the stimulus to be affected by age-related reorganization.\\
Also, in \hyperref[results:paperIV]{Research Article \uproman{4}}, a classifier should be trained based on the \gls{eeg} data to determine the membership to the group of experts or novices. In contrast to the previous article, however, the performance was below the random level. This finding contrasts with studies in which the classification of the level of expertise was already successful \cite{Hosp2021, Winkler-Schwartz2019, Shourie2016}. In contrast to these studies, the occupational background of the fine motor experts studied in this study, such as opticians and watchmakers, was very diverse, and the visuomotor tracking tasks selected for analysis offered only a general approximation of the context of their expertise. Based on this, we visualized the underlying brain network characteristics using a nonlinear dimensionality reduction in two-dimensional space. Differences in the data structure between experts and novices emerged. Compared to each other, the brain network patterns of the experts showed a higher degree of a clustered structure in which each cluster could be clearly assigned to a participant. This finding can be interpreted as a higher individuality of the brain network patterns in the experts. A high degree of individuality is also known in the movement patterns of flute players and the muscle synergies of expert powerlifters \cite{Albrecht2014, Caramiaux2018, Kristiansen2015}. Our results suggest that individuality, which was gained in the context of professional expertise, is as well present at the network level of the brain.\\
\\
Overall, the results presented in the four research articles showed that machine learning techniques, including dimensionality reduction and classification of \gls{eeg} signals, successfully detected age-related brain reorganization. These results provided new starting points for understanding the reorganization process of the aging brain, particularly in relation to dedifferentiation, and partially supported the predictions of the reserve hypothesis.

\section{Methodological considerations}
Before discussing specific methodological aspects regarding the characteristics of the datasets as well as the application of machine learning procedures, it should be noted that we have not presented changes but \textit{differences} as the study design of all studies from which the datasets were derived were cross-sectional. We also focused only on some well-documented specific aspects of the reorganization of the aging brain and explored overarching patterns in an exploratory manner. The results should be evaluated in light of the following methodological considerations. 

\subsection{Datasets}
With regard to the individual datasets and the underlying study protocols, there are some considerations that should be taken into account when evaluating the results presented.\\
Overall, different age groups were present in the datasets underlying the individual research articles. The older (late middle-aged) participants in the first dataset, which were examined in \hyperref[results:paperI]{Research Article \uproman{1}}, were comparatively young so for these results it is to be expected that possible age effects have been underestimated. In the second dataset, the minimum age difference between the six age groups varied from ten years (between children and young adults) to one year (within the two oldest groups). This may have significantly conditioned the classification results of the second research article. However, the minimum age difference between the oldest groups and the next younger group (the middle-aged adults) is also only two years, so the cutoff found between these groups is less likely to be explained by the age structure of the dataset. In regard to the age structure, it would be advantageous to have a continuous age structure for a fine-grained analysis of age-related reorganization.\\
Regarding the proxies of the reserve construct, it should be noted that the construct of the reserve hypothesis is more diverse than the dimesions depicted in \hyperref[results:paperI]{Research Articles \uproman{3}} and \hyperref[results:paperI]{\uproman{4}}. In our considerations, we followed the common scientific practice of comparing individuals with low expression with individuals with higher expression of a given proxy, i.e. cardiorespiratory fitness or expertise \cite{Koen2019}. Cardiorespiratory fitness was indirectly measured by a 6 min walk test, which is dependent on the motivation of the participants. However, a high correlation to the more objective VO$_2$max is reported in the literature \cite{Zhang2017}. The selection of the fine motor experts was based on the professional group of precision mechanics, whose daily routines and hand use may differ. On the other hand, it cannot be excluded that the novices also had a certain level of fine motor skill, which they acquired in their everyday life, so expertise effects may have been underestimated. By the criterion of a minimum of ten years of professional experience, however, the deliberate practice approach was followed, which was evaluated as a suitable criterion for professional expertise \cite{Ericsson1991, Voelcker-Rehage2013}. Furthermore, previous studies have shown that the selected fine motor task appropriately reflects the expertise context \cite{Vieluf2018, Goelz2018, Vieluf2012, Vieluf2013}. In general, a comparison to young subjects could have tested further assumptions of the reserve hypothesis. For example, whether different networks are activated and have higher task specificity or whether we can assume that networks are maintained. However, this was not possible during participant recruitment due to the selected criterion, i.e., at least ten years of work experience, in Dataset \uproman{1} or was not considered in the original intervention study on which Dataset \uproman{3} is based on.\\
In addition to the age structure and the proxies of the reserve concept, it should be noted that the tasks evaluated comprised very specific laboratory tasks, whose transferability to everyday life may be limited. Nevertheless, the tasks should map across relevant domains in which age-related changes are reported, i.e., sensory, cognitive, and motor domains. The task types were all based on well-tested and common paradigms in aging research and had the advantage that the results are comparable to other studies. One aspect that we cannot exclude is the possible effect of fatigue and motivation, which could have played a role, especially during longer sessions. Other aspects relate to the specific characteristics of the task designs. For example, the force control task in Dataset \uproman{1} was designed in a block design and the participants already attended several similar experiments in the context of the Bremen-Hand-Study@Jacobs, so that practice effects cannot be excluded. This could also have been advantageous since age differences in short-term adaptation effects could thus be eliminated. The flanker task in Dataset \uproman{2} deferred slightly between the individual studies on which this dataset is based on. This was present in the length of stimulus presentation, number of trials, and gender. However, because the subjects of the individual studies were present in several groups and we did not find increased rates of misclassification between these groups, we excluded that this had a major influence on the results. In addition, the influence of gender and the number of trials in previous or comparable studies was estimated to be low \cite{Reuter2019, Vahid2020}. In Dataset \uproman{3}, the task was quite long, which may have contributed to the aforementioned fatigue effects, but had the advantage of allowing the temporal dynamics to be better mapped using the machine learning algorithms of brain activation using the machine learning algorithms 

\subsection{Machine learning approaches}
The application of machine learning algorithms is strongly dependent on the dataset and utilized algorithm.\\
Especially different noise levels in the individual \gls{eeg} recordings could have influenced the results of the group comparisons. However, studies using comparable methods show that this approach is valid, provided that the noise level is taken into account. This is also confirmed by our results, which are consistent with common hypotheses from the \gls{fmri} literature. In the research articles, we used dimension reduction methods on the first level to improve the signal-to-noise ratio and to extract relevant features from the pre-processed \gls{eeg} data.\\
In \hyperref[results:paperI]{Research Article \uproman{1}}, \hyperref[results:paperIII]{\uproman{3}}, and \hyperref[results:paperIV]{\uproman{4}} we decided to use \gls{dmd}, which allows to extract coherent spatiotemporal patterns from the \gls{eeg} data and thus network characteristics. This approach allowed us to draw inferences about the dynamic properties of complex networks from coherent activation patterns \cite{Brunton2016} but differs from analyses of bivariate connectivity of voxel or (source) signals followed by graph analytic approaches to describe brain networks derived in this way. The decision to use \gls{dmd} in this study was driven by its ability to capture the complex dynamics of networks that include factors such as frequencies, growth, and decay. As a result, \gls{dmd} provides a physiologically plausible and low-dimensional representation of \gls{eeg} dynamics. Moreover, the effectiveness of \gls{dmd} in combination with machine learning has been demonstrated in previous studies. In particular, \gls{dmd} has been shown to be very successful in classifying fine motor tasks \cite{Shiraishi2020}. Additionally, we have already used \gls{dmd} in our own research to extract task-related sensorimotor network dynamics associated with age and expertise \cite{Vieluf2018,Goelz2018}. Also \gls{dmd} in this work focused on task-related network characteristics derived from \gls{eeg} signals and not on resting state networks. Thus, we cannot draw conclusions about the relationship to reorganization, i.e., less separated and modular resting networks. Although it is generally unclear how task-related dedifferentiation and the reorganization of resting networks are related, a recent study shows that these two levels are related and predict performance \cite{Cassedy2020netw_distinct}. Also, the analyses were performed in the signal space and the signals were not projected into the anatomical source space, so the anatomical interpretation was not possible with respect to known resting state networks. This decision was justified by the small number of electrodes in Dataset \uproman{1} and \uproman{2}. Studies show that a valid source reconstruction depends strongly on the number of electrodes \cite{Song2015,Lantz2003}. Only for Dataset \uproman{3} a sufficient number of electrodes was given and we visualized the found patterns in the source space but did not perform \gls{dmd} in the source space because validation studies for such an approach have not been performed so far and we only had individual \glspl{mri} for some of the subjects so that no individual head models were available.\\
To align our results with previous \gls{erp} analyses and enhance the signal-to-noise ratio, we employed the xDAWN algorithm in \hyperref[results:paperIV]{Research Article \uproman{4}}. This design choice allowed us to explore the temporal dynamics in response to stimuli, rather than inferring network characteristics as done in the other research articles. While calculating dynamic functional connectivity over time would have been an alternative approach, it is less extensively validated at this stage. Hence, we relied on the established xDAWN algorithm for our investigation.\\
The selection of classification algorithms and secondary dimensionality reduction algorithms may also have affected the results obtained. The process of picking suitable algorithms followed standard procedures in the field of machine learning \cite{Shalev2014}. That is the choice of the dimensionality reduction algorithms was based on the respective objectives of the analyses and of the classifiers was based on the performance of the respective algorithm within the training portions. In addition, we replicated the results obtained by the classifiers with different algorithms and provided them as supplementary material in the respective published research articles. The results of the classification algorithms were carefully evaluated using various metrics such as accuracy, precision, recall, and F1 scores. We used classical algorithms that fit better to the amount of data available, i.e. tens to hundreds of samples. Additionally, we used cross-validation procedures and hyperparameter tuning, e.g. to tune the regularization parameters, limit this, and estimate the generalization performance.\\
At this point, it should be mentioned that in \hyperref[results:paperIII]{Research Article \uproman{3}}, we did not use supervised classification to map the dedifferentiation, but instead extracted the patterns unsupervised via PCA and inferred the dedifferentiation using classical statistical methods. This was due to the relative amount of data available and the continuous design of the task. For this reason, it is possible that the effects of dedifferentiation were underestimated in both groups studied.\\
Although the classification results do not allow conclusions about the direction of the effects, these analyses should be seen as a supplement. We related the results to previous research and examined the input to the classification with methods that allow conclusions about directional effects. For this purpose, we partly used a combination of inferential statistical methods and dimension reduction methods. Overall, it was possible to bridge the gap between traditional and modern machine learning analysis.

\section{Outlook and practical implications}
In the research articles presented in this thesis, we used machine learning methods to gain valuable insights into the reorganization of the aging brain from \gls{eeg} data. Our results have provided new insights at the individual and group levels and offer perspectives for future research efforts.\\
To begin with, the analyses in \hyperref[results:paperI]{Research Article \uproman{1}} have shown that effects of dedifferentiation of the motor system can be detected at the \gls{eeg} level already in late middle-aged participants. It would therefore be exciting to investigate this in continuous age groups and whether there are signatures of different systems as reported e.g. in \gls{mri} findings \cite{Raz2006}. We found initial indications for this in \hyperref[results:paperII]{Research Article \uproman{2}}. In contrast to the motor system, we found signatures of dedifferentiation, here related to dynamics, only in the two oldest groups and not already in the late-middle-aged group. Moreover, especially with respect to temporal dynamics, dedifferentiation has not been explored, so this might represent a new dimension of dedifferentiation that could be investigated in further studies. Aiming at this dynamic, we also found in \hyperref[results:paperIII]{Research Article \uproman{3}} signs that the dynamic cortical representations could be a crucial point. Inspired by the computational model of \citeauthor{Li2002} \cite{Li2002,Li2000}, we proposed neural noise as an explanation for our findings. Although initial findings suggest that neural noise may indeed explain dedifferentiation at the \gls{eeg} level, this requires further investigation \cite{Pichot2022}.\\
Moreover, we only tested two dimensions of the reserve concept and tested varying results for dedifferentiation. It would be interesting to test other dimensions such as education. To understand the concept in relation to the reorganization of the brain, the development of a gold standard tool that captures all or as many as possible of the known domains of reserve would be necessary \cite{Nogueira2022}.\\
\indent Exploratory analysis at the group level additionally generated two hypotheses that could be tested in future research. With regard to expertise, the exploratory analysis in \hyperref[results:paperIII]{Research Article \uproman{3}} of the group structure allowed us to show that the factor of individuality, which was previously known only at the muscular or behavioral level, also exists at the neuronal level. Furthermore, \hyperref[results:paperII]{Research Article \uproman{2}} suggests significant alterations in the functional organization of the brain following retirement, as indicated by the classification of age groups.\\
\\
In addition to these points, there are several implications for practice. We could show with our analyses that the reorganization of the aging brain, more precisely dedifferentiation, can be mapped on an individual level. This could be useful for the development of potential markers that can be used for the development of preventive interventions. 
Although the clinical relevance of dedifferentiation is not fully understood, \citeauthor{Fornito2015} \cite{Fornito2015} presents it as a maladaptive mechanism of brain networks, so our results could also be useful for the early detection of unfavorable age trajectories and eventually for the development and evaluation of therapeutic concepts. The results of \hyperref[results:paperII]{Research Article \uproman{2}} showed that up to 10 \% more performance is possible in predicting the age group during the task, which underlines the added value of task-related \gls{eeg} data so that it could also be used in clinical practice to predict the cognitive status of patients. This could be of particular interest for \gls{eeg}-based detection of mild cognitive impairment, where resting state \gls{eeg} alone has limited informative value \cite{Froehlich2021, Farina2020}. An exciting starting point would be to integrate the methods used in this work into the framework of brain age prediction (see \chapref{theory:ml:applications_aging}). Perhaps this would help to overcome some limitations of this framework, such as the limited ability to predict cognitive decline \cite{Tetereva2023}. Interestingly, studies suggest that a dedifferentiated activation pattern of brain networks may be related to poorer behavioral performance independent of age, so this may have applications in other areas independent of aging research.\\
The decoding approaches we have used are applied in technical systems such as \glspl{bci} or systems used in therapy such as neurofeedback systems. Our results imply that age and reorganization of the brain are significant factors to be considered in the development of assistive systems.  Direct implications for the development of such systems include, for example, thje usage of age-appropriate features, paradigms and algorithms in the development of assistive systems for the elderly. This arises from the differences in decoding performance between age groups and could affect the selection and placement of electrodes and a possible delay in decoding attempts. Likewise, our findings suggest that fitness or expertise level could have an influence and thus have implications for the development of adaptive systems, i.e. systems that automatically adapt to the user. 

\section{Conclusion}
One of the greatest challenges of our society is to cope with an aging society. This requires targeted interventions, early detection of unfavorable trajectories, and assistive technology. To achieve this, an understanding of individual trajectories of age-related decline is essential. The approach of this work was to gain such an understanding by applying machine learning techniques. 



This thesis approached age-related reorganization with data-driven methods from the field of machine learning. Four resrearch articles were presented in which classification and dimensionality reduction methods were used to provide additional insights. 




Specifically in the firesst reserach article we used these methods and detecetd a 

% Study 1
% Based on age-related changes of brain networks, such as ad-
% ditional recruitment of bilateral motor areas (Carp et al., 2011;
% Ward & Frackowiak, 2003) and attentional resources (Berghuis
% et al., 2019), we aimed to study differences in the classification
% of EEG data recorded in active visuomotor tracking tasks.
% In summary we found electrophysiological patterns associated
% with an altered sensorimotor network in OA. Lower task speci-
% ficity in combination with changes in symmetry of brain activity
% point to bilateral and dedifferentiated, i.e., less task specific, brain
% activity of the motor network and activation and interrelation of
% several networks with age.
% Most importantly these electrophysiological brain activity pat-
% terns resulted in lower classification performance in the clas-
% sification of body side of task execution in OA, indicating less
% segregated brain network activation of the motor system. In
% contrast, OA showed higher classification performance with re-
% spect to the task characteristic. The study of the classifier input
% indicates the relevance of markers of information integration for
% classification performance in OA.
% The current results confirm previous findings on age-related
% reorganization of task-related brain networks and expand them
% with reference to the characteristics of the task. Furthermore,
% the findings may have practical implications for areas of applied
% research such as BCI applications. Age-related differences should
% be taken into account in the development of BCI and neurofeed-
% back systems if they are designed for this target group. This could
% include the selection of the appropriate positioning of electrodes,
% e.g., the use of frontal and occipital electrodes, as well as the
% choice of suitable features and algorithms.

% Study 2
% In summary, we were able to extend previous results
% using machine learning techniques to detect age and
% task differences in cognitive processing on a single-trial
% level. This is especially crucial for a step behind classical
% ERP components and a more direct link between behav-
% ior and neural dynamics [55]. The data-driven approach
% used in this research particularly highlights early atten-
% tional processes in the classification of age groups and
% suggests the benefit of task-related EEG data in the clas-
% sification of different age groups, which could be used in
% clinical contexts. With respect to information process-
% ing in selective attention our analyses could confirm the
% relevance of time windows corresponding to N1 and N2
% components reported in ERP studies. Furthermore, the
% time windows relevant for inhibitory control differed
% between groups, i.e., later time windows were relevant
% in older adults, suggesting that different processes are
% important for selective attention at different ages. Over-
% all, we showed that using machine learning compared
% to a priori selected electrodes and timepoints, we were
% able to obtain assumption-free insights into differences
% in inhibitory control over the lifespan. Machine learning
% thus represents an extension to classical methods that
% can be used to test existing theories but also to extend
% them.

% Study 3 
% In applying DMD to continuous EEG recordings during
% rest and three different tasks, we considered both topo-
% logical properties and the temporal dynamics of task-re-
% lated
% brain
% networks.
% Thus, we
% identified electrophysiological signatures of age-related brain reor-
% ganization processes in fit and less fit older adults. Fit
% participants showed higher task specificity, i.e., more dif-
% ferentiated brain activation patterns, as well as higher
% prominence of these patterns, indicating less neural noise
% throughout task execution. Our findings support the idea
% that physical fitness manifests in task-related brain network
% activation patterns that are in line with reduced dediffer-
% entiation in older adults.

% Study 4
% In this study, we used machine learning to gain data-driven insights
% into expressions of expertise. We found that classification of group
% membership based on tasks that roughly reflect the expertise context is
% not possible with high accuracy. In contrast to positive reports from the
% literature, we assume a high importance of the expertise context in
% classification. Furthermore, the classification of tasks is possible with
% high accuracy within novices and experts at individual level. By exam­
% ining a low-dimensional representation of the feature space, we found a
% more pronounced individuality of EEG patterns in experts, suggesting
% more specialized neural mechanisms in fine motor experts during task
% performance. In addition to providing data-driven insights, these results
% could be relevant to the application of machine learning in the context of
% expertise classification as well as the generalizability of such algorithms
% in the context of BCI research.

By using machine learning techniques, more precisely dimensionality reduction and the classification of tasks-related \gls{eeg} signals, it was consequently possible to detect age-related differences and confirm hypotheses about the aging brain such as dedifferentiation and compensation directly based on \gls{eeg} signals. In addition, new hypotheses such as the temporal component of dedifferentiation could be established.  Since classifiers are trained for each participant individually, the level of dedifferentiated and compensatory activity could thus be quantified on an individual level.