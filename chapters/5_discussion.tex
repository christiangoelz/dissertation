BLA BLA in Bezug zur Einleitung\\
The overall goal of this thesis was to better understand the reorganization of the aging brain by applying machine learning techniques. Three approaches were derived from literature, which were taken up in four publications.
The first approach was based on a widely reported reorganization of the aging brain in the literature, namely dedifferentiation, which is based on a computational model with broad evidence from animal models and human studies. In this work, machine learning was used to quantify the selectivity of neural representations at the individual level.\\
The comparison of individuals with different lifestyle backgrounds was the second approach of this work and followed the reserve hypothesis, which is the ability of the brain to hold a reserve against degradation processes based on its plastic properties.\\
In general, it is possible to use machine learning to gain insight into high-dimensional data and to detect novel relationships. In this way, new hypotheses can be generated. Thus, in the third approach of this thesis, dimensionality reduction as well as classification algorithms should be used at the group level to generate insights into overreaching general patterns.\\
The results of these approaches are discussed in the following sections in relation to the overarching question. a detailed discussion of the specific results can be found in the attached \hyperref[pub:papers]{published reserach articles}.

\section{On the detection of dedifferentiated brain organization}
Motivated by applications from \gls{fmri} research, in which dedifferentiation, operationalized as loss of neural specificity, is quantified based on the performance of classification algorithms, we applied this to \gls{eeg} data to investigate the differentiability of signals in motor, cognitive, and sensory tasks in participants of different age groups.\\
In \hyperref[results:paperI]{Research Article \uproman{1}}, we found differences in the classification performance of fine motor tasks based on \gls{eeg} data. In particular, performance in classifying which hand was used to perform each task was reduced in older compared with younger participants. Studying the classifier input, i.e. the spatiotemporal coherent activation patterns extracted with \gls{dmd}, we found that this corresponds to differences in the electrophysiological network activation patterns pointing to more bilateral and widespread activation in older adults. Taken together these results correspond to a dedifferentiated motor network as reported in \gls{fmri} studies \cite{Carb2011,Cassedy2020} and are also consistent with other \gls{eeg} findings that have compared classification between age groups finding reduced performance in the classification of the body side of task execution \cite{Chen2019, Zich2015}. On the other hand, we found an increased classification performance in older participants when predicting whether a steady or sinusoidal target force was tracked by the participant. When considering the classifier input again, we found a stronger involvement of frontal electrodes in the network activation patterns of the older participants and thus suspected that compensation processes play a role here. This could be present,  especially in the more complex sinusoidal force tracking and thus positively influence the differentiability of the two task classes and is consistent with the \gls{crunch} hypothesis which  postulated that during task performance, as task difficulty (or load) increases, more cortical resources will be activated \cite{Festini2018}.\\ 
In \hyperref[results:paperII]{Research Article \uproman{2}} we again utilized classification algorithms to investigate whether the cortical representation of selective attention differed between six different age groups ranging from children to older adults. Here we found that the discriminability between stimuli requiring and stimuli not requiring inhibitory control was reduced only in the children's group. This reduced discriminability in the group of children goes along with the assumption of inhibitory control in children that is not yet differentiated at the neural level \cite{Waszak2010,Reuter2019}. However, the results of similar good performance of the classifiability of the two stimulus categories contradict the assumption of a general dedifferentiation of all cortical systems. This is also known in \gls{fmri} studies, which, for example, found no dedifferentiation in the activation of the visual system in response to specific stimulus categories \cite{Voss2008}. In this article, we were interested in the temporal response to the stimulus and so the classifier input did not represent coherent activation patterns as in the previously described analysis, but rather single-trial eeg traces filtered by xDAWN in response to the stimulus, through which it was possible to examine the performance of the classifier over time and thus gain insight into the temporal dynamics of information processing. In this way we found that the differentiability of the stimuli was delayed in older adults. Thus, dedifferentiation might also have a dynamic component related to the rise of neural noise, as postulated in the model of Li and colleagues \cite{Li2001,Li2000}.\\
\\
In the first two research articles, it was shown that machine learning, more precisely dimensionality reduction and the classification of tasks-related \gls{eeg} signals successfully can be used to detect age-related reorganization in terms of dedifferentiation and provide novel entry points for better understanding brain reorganization. Building on this, the studies discussed in the following section aimed to examine influencing factors that are prominently discussed in the reserve hypothesis. 


\section{The role of lifestyle factors}
An important aspect to better understand the individuality of age-related alterations is to investigate how influencing factors affect the reorganization of the aging brain. From the literature, physical fitness can be deduced as a significant influencing factor that affects the brain's ability to resist degradation processes and maintain function, thus contributing to reserve. In this context \citeauthor{Stillman2019} \cite{Stillman2019} summarizes findings of \gls{fmri} studies proving that the network structure of the brain at rest of pyhiscal fit people show less signs of dedifferentieton. Based on this, in \hyperref[results:paperIII]{Research Article \uproman{3}} we investigated to what extent cardioresoiratory fitness affects the dedifferentiation of task-related network characteristics based on \gls{eeg}. To do so, we used \gls{dmd} and extracted the dominant spatiotemporal \gls{eeg} patterns and their prominence over time of task performance in a motor, sensory and cognitive tasks. In this way, we were able to extimate task-related brain network dynamic activation patterns in the main domains in which age-related changes are reported \cite{Baltes1997, Sala-Llonch2015, Park2009}. Comparison between a fit and less fit group revealed greater differences in activation patterns in the fit compared to the unfit group, suggesting lower levels of dedifferentiation. We further found that the extracted dominant modes of the fit group resolved more variance compared to the less fit group, indicating a higher stability or prrominence of the extracted patterns in the fit group. In this way, it was possible to investigate the dynamics of age-related changes described in the literature. From the perspective of the dedifferentiation model, this could indicate less neural noise and a lower rate of neural variability predicted by this model which points to higher efficiency of information processing. These differences were frequency dependent mainly to the $\theta$ and $\beta_2$ bands which could be carefully attributed to processes reflecting cognitive involvement and information integration \cite{Siegel2012}.\\
Another frequently studied factor of influence in the literature is that of professional expertise. Thus, we compared fine motor experts and novices \hyperref[results:paperIV]{Research Article \uproman{4}} analogous to \hyperref[results:paperI]{Research Article \uproman{1}} with respect to the classification performance of the visuomotor tasks and found no differences in classifier performance. Consequently, we could not detect any differences in the differentiation of the neural representation of the task. At first glance, these results contradict previous studies from the Bremen Hand Study@Jacobs, which showed statistical differences in the expression of network characteristics of experts compared to novices in terms of neural efficiency \cite{Goelz2018, vieluf2018age}. However, in our study we tested a very specific assumption of the age related reorganization, namely a reduced differentiation, and did not test neural efficiency as in the previous work. Also, the results seem to contradict the reserve concept, on the basis of which a lower level of dedifferentiation would have been expected. However, this is only one specific point and the concept of reserve is more multifaceted than this specific assumption including structural as well as functional aspect such as compensatory involvement and efficiency of brain networks \cite{Cabeza2018}. Therefore, our results do not necessarily contradict the reserve concept.\\
\\
In summary, we were able to expand upon the findings from the first two research articles and utilize machine learning methods to gain insights into the factors influencing age-related brain reorganization. As a result, we were partially able to confirm specific predictions of the reserve hypothesis. 

\section{Exploratory Insights into Age-related Reorganization}
Während sich die bisherigen Ergbenisse auf der Anwendung auf probandenebene beziehen, nutzen wir die machine learning verfahren ebenso auf Gruppenebene, um explorative Einblicke in die Strukktur der untersuchten Gruppen zu erhalten. 

\section{Methodological considerations}
- Unterschiedliche Altersgruppern
- DMD als Mehtode
- Klassifikation als Methode 
- Unterschiedliche Rauschlevel 
- Proxy Messsungen von reserve spiegeln nur bestimmte Aspekte wieder
....


\section{Consequences for Practice and Future Investigations}
Einer der größten Heruasforderungen unserer Gesellschaft ist es mit einer alternden Gesellscaft klarzukommen. Dies setzt gezielte Interventionen, die früher Detektion von ungünstigen trajektorien und assisstive technologie voraus. Um dies zu erreich ist ein Verstännis individueller trajektorien alltersbedingter Abbauprozesse unabdingbar. Der Ansatz dieser Arbeit war es ein solches Verständnis durch die Anwwendung maschineller Lernverfahren zu erlangen. 
 
However it has been observed that wen adults with high leves of reserve. as indicated by one or more reserve proxies, do eventually display cognitive decline...


\section{Conclusion}

By using machine learning techniques, more precisely dimensionality reduction and the classification of tasks-related \gls{eeg} signals, it was consequently possible to detect age-related differences and confirm hypotheses about the aging brain such as dedifferentiation and compensation directly based on \gls{eeg} signals. In addition, new hypotheses such as the temporal component of dedifferentiation could be established.  Since classifiers are trained for each participant individually, the level of dedifferentiated and compensatory activity could thus be quantified on an individual level. 