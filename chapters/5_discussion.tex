This work aimed to better understand the reorganization of the aging brain by applying machine learning techniques to \gls{eeg} data. Three approaches were derived from the literature and used in four publications.
The first approach was based on a reorganization of the aging brain often described in the literature, namely dedifferentiation, i.e., less specific activation of brain networks expressed in a lower differentiability of brain states. Based on the computational model by \citeauthor{Li2001} \cite{Li2001, Li2000}, this is attributed to a loss of selectivity of neuronal responsiveness, which has been confirmed in numerous animal and human studies. Building upon this, we tested the differentiability of brain states, e.g., in the form of activation patterns of brain networks, and thus the selectivity of neuronal representations using machine learning methods. This approach provided insights into age-related dedifferentiation and allowed the formulation of new hypotheses, e.g., regarding this phenomenon's temporal aspect.\\
The second approach of this work addressed the reserve hypothesis. We examined the possible positive effects of certain lifestyle factors, such as physical fitness or professional expertise, on age-related brain reorganization. Using machine learning techniques, we compared individuals with different lifestyle backgrounds and the effects on age-related brain reorganization. This analysis provided new perspectives and helped to test the assumption that individuals with certain lifestyle factors exhibit less dedifferentiation.\\
In the third approach, dimensionality reduction and group-level classification algorithms were used to identify overarching patterns in brain reorganization. This approach enabled the generation of new insights and hypotheses.\\
In the following, the main results of these approaches are discussed. However, for a detailed discussion of each specific result, please refer to the corresponding published journal articles.\\

\section{On the Detection of Dedifferentiated Brain Organization}
Motivated by applications from research using \gls{fmri}, in which dedifferentiation is quantified based on the performance of classification algorithms, we applied this to \gls{eeg} data to investigate the differentiability of signals in motor, cognitive and sensory tasks in participants of different age groups.\\
In \hyperref[results:paperI]{Research Article \uproman{1}}, we found differences in the classification performance of fine motor tasks based on \gls{eeg} data. In particular, performance in classifying which hand was used to perform the task was reduced in late middle-aged compared to younger participants. Studying the classifier input, i.e., the network activation patterns extracted with \gls{dmd}, we found signs of a more bilateral and widespread activation pattern in late middle-aged adults which might explain the difference in the classifier performance. These results correspond to a dedifferentiated motor network as reported in \gls{fmri} studies and are also consistent with other \gls{eeg} findings that have compared classification between age groups finding reduced performance in the classification of the body side of task execution in older adults \cite{Chen2019, Zich2015, Carb2011, Cassedy2020}. In contrast, we found an increased classification performance in the late middle-aged participants when predicting whether the participant tracked a steady or sinusoidal target force. When considering the classifier input again, we found a stronger frontal involvement in the network activation patterns of the late middle-aged participants and thus suspected that compensation processes play a role here. Compensation could be present, especially in the more complex sinusoidal force tracking, and thus positively influence the differentiability of sinusoidal and steady force tracking. This result is consistent with the \gls{crunch} hypothesis, which postulated that during task performance, as task difficulty (or load) increases, more cortical resources will be activated \cite{Festini2018}. Combining both results, we could demonstrate that age-related reorganization of the aging brain is reflected in the performance of classifiers based on EEG measurements, thus providing new insights into the aging brain.\\
In \hyperref[results:paperII]{Research Article \uproman{2}}, we again utilized classification algorithms to investigate whether the cortical representation of selective attention differed between six different age groups ranging from children to older adults. Here we found that the discriminability between stimuli requiring and stimuli not requiring inhibitory control was reduced only in the children's group. This reduced discriminability in the group of children goes along with the assumption of inhibitory control in children that is not yet differentiated at the neural level \cite{Waszak2010, Reuter2019}. However, the results of similar good performance in the classification of the two stimulus categories contradict the assumption of a general dedifferentiation of all cortical systems. Similar results are also known in \gls{fmri} studies, which, for example, found no dedifferentiation in the activation of the visual system in response to certain stimulus categories \cite{Voss2008}. By using \gls{eeg}, however, we were able to investigate the temporal response to the stimulus in addition. For this reason, the classifier input did not represent coherent activation patterns as in the previously described analysis, but rather single-trial \gls{eeg} traces filtered by xDAWN in response to the stimuli, through which it was possible to examine the performance of the classifiers over time and thus gain insights into the temporal dynamics of information processing. In this way, we found that the differentiability of the stimuli was delayed in older adults. Thus, dedifferentiation might also have a dynamic component related to the rise of neural noise, as postulated in the computational model of \citeauthor{Li2001} \cite{Li2001, Li2000}. However, this only occurred in the two oldest age groups and not, as in the previous study, in the late middle-aged, so this effect might occur later in the aging process.

\section{The Role of Lifestyle Factors}
An important aspect to better understand the individuality of age-related changes is to investigate how influencing factors affect the reorganization of the aging brain. From the literature, physical fitness can be deduced as a significant influencing factor that affects the brain's ability to resist degradation processes and maintain function, thus contributing to reserve. In this context, \citeauthor{Stillman2019} \cite{Stillman2019} summarized findings of \gls{fmri} studies proving that the network structure of the brain at rest of physically fit people show fewer signs of dedifferentiation. Therefore, in \hyperref[results:paperIII]{Research Article \uproman{3}} we investigated to what extent cardiorespiratory fitness affects the dedifferentiation of task-related network activation patterns based on \gls{eeg}. To do so, we extracted dominant spatiotemporal network patterns and their prominence over time during the performance of motor, sensory and cognitive tasks. In this way, we were able to estimate task-related brain network activation patterns in the main domains in which age-related changes are reported and compare this between a fit and less fit group \cite{Baltes1997, Sala-Llonch2015, Park2009}. This comparison revealed greater differences between the task-related activation patterns in the fit compared to the less fit group, suggesting lower levels of dedifferentiation in the less fit group. We further found that the extracted dominant patterns of the fit group resolved more variance compared to the less fit group, indicating higher stability or prominence of the patterns in the fit group. These differences were frequency dependent mainly to the $\theta$ and $\beta_2$ bands which could be carefully attributed to processes reflecting cognitive involvement and information integration \cite{Siegel2012}. From the perspective of the computational model of dedifferentiation, the higher stability could indicate less neural noise and a lower rate of neural variability, which points to a higher efficiency of information processing. Overall, these results follow the prediction of the reserve hypothesis and provide evidence that supports the assumptions of the computational model of dedifferentiation.\\
Another frequently studied factor of influence in the literature is professional expertise. Thus, in \hyperref[results:paperIV]{Research Article \uproman{4}}, we compared fine motor experts and novices with respect to the classification performance of visuomotor tasks. We found no group differences in the performance of a classifier trained to classify the same task characteristics as in \hyperref[results:paperI]{Research Article \uproman{1}}. Consequently, we could not detect any differences in the differentiation of the neural representation of the task between novices and experts. At first glance, these results contradict previous studies from the Bremen-Hand-Study@Jacobs, which showed statistical differences in the expression of network characteristics of experts compared to novices in terms of neural efficiency \cite{Goelz2018, Vieluf2018}. However, our study tested a particular assumption of age-related reorganization, namely dedifferentiation, and did not test neural efficiency as in the previous work. Also, in contrast to \hyperref[results:paperIII]{Research Article \uproman{3}}, the results seem to contradict the reserve concept, based on which a lower level of dedifferentiation would have been expected. Therefore, taking a more differentiated view of the reserve concept concerning individual factors could be necessary to refine this concept. We have only explored one aspect in considering dedifferentiation in the aging brain. As a result, other characteristics of brain reorganization may differ, such as the efficiency of information processing or compensational network involvement. These factors were not accounted for in our analyses.

\section{Exploratory Insights into Age-Related Reorganization}
While the previous results refer to the application at the participant level, we also used the machine learning methods at the group level to gain explorative insights into overreaching group characteristics.\\
In \hyperref[results:paperII]{Research Article \uproman{2}}, we found a grouped structure of misclassifications when using a classifier that was trained to determine the group membership in one of the six age groups based on \gls{eeg} data. The classifier tended to misclassify only between certain age groups. This was especially noticeable in the two oldest age groups, between which the misclassification rate was highest. Less frequently, however, older participants were classified into the group of late middle-aged adults and vice versa. Considering the age structure of the older adults (minimum age of 65 years) and the middle-aged adults (maximum age of 64 years), as well as a retirement age of around 65, this could indicate major cognitive changes after retirement. Deterioration of cognitive abilities and cerebral blood flow after retirement has been reported in the literature \cite{Celidoni2017, Rohwedder2010, Rogers1990}. However, we could not further test this hypothesis because of a lack of data on the exact retirement age of the participants. In addition, no misclassification occurred between children and the oldest groups. Although differences based on \gls{erp} markers are often described as u-shaped \cite{Mueller2008, Reuter2019}, this result suggests different mechanisms, i.e., differentiation in children vs. dedifferentiation in older adults. Overall, we found an increase in classification performance shortly after stimulus onset of about 10~\%, highlighting the added value of task-related \gls{eeg} recordings and suggesting early processing of the stimulus to be affected by age-related reorganization.\\
Also, in \hyperref[results:paperIV]{Research Article \uproman{4}}, a classifier was trained based on the \gls{eeg} data to determine the membership to the group of experts or novices. In contrast to the previous article, however, the performance was below the chance level. This finding contrasts with studies in which the classification of the level of expertise was already successful \cite{Hosp2021, Winkler-Schwartz2019, Shourie2016}. In contrast to these studies, the professional background of the fine motor experts examined in this study, such as opticians and watchmakers, was very diverse, and the visuomotor tracking tasks selected for analysis offered only a general approximation of the context of their expertise. Based on this, we visualized the underlying brain network characteristics using a nonlinear dimensionality reduction in two-dimensional space. Differences in the data structure between experts and novices emerged. Compared to each other, the brain network patterns of the experts showed a higher degree of a clustered structure in which each cluster could be clearly assigned to a participant. This finding can be interpreted as a higher individuality of the brain network patterns in the experts. A high degree of individuality is also known in the movement patterns of flute players and the muscle synergies of expert powerlifters \cite{Albrecht2014, Caramiaux2018, Kristiansen2015}. Our results suggest that individuality, which was gained in the context of professional expertise, is as well present at the network level of the brain.\\
\\
Overall, the results presented in the four research articles showed that machine learning techniques, including dimensionality reduction and classification of \gls{eeg} signals, successfully detected age-related brain reorganization. These results provided new starting points for understanding the reorganization process of the aging brain, particularly in relation to dedifferentiation, and partially supported the predictions of the reserve hypothesis.

\section{Methodological Considerations}
Before discussing specific methodological aspects regarding the characteristics of the datasets as well as the application of machine learning procedures, it should be noted that we have not presented \textit{changes} but \textit{differences} as the study design of all studies from which the datasets were derived were cross-sectional. We also focused only on some well-documented specific aspects of the reorganization of the aging brain and explored overarching patterns in an exploratory manner. In addition, the results should be evaluated in light of the following methodological considerations. 

\subsection{Datasets}
Concerning the individual datasets and the underlying study protocols, some considerations should be taken into account when evaluating the results presented.\\
Different age groups were present in the datasets underlying the individual research articles. The older (late middle-aged) participants in \hyperref[methods:datasets:I]{Dataset \uproman{1}} examined in \hyperref[results:paperI]{Research Article \uproman{1}} were comparatively young, so for these results, it is to be expected that possible age effects have been underestimated. In \hyperref[methods:datasets:II]{Dataset \uproman{2}}, the minimum age difference between the six age groups varied from ten years (between children and young adults) to one year (within the two oldest groups). This age structure may have significantly conditioned the classification results in \hyperref[results:paperII]{Research Article \uproman{2}}. However, the minimum age difference between the oldest and the next younger groups (late middle-aged adults) is also only two years. Hence, the finding of misclassification cutoff between these groups is less likely to be explained by the age structure of the dataset. In general, regarding the age structure, it would be advantageous to have a continuous age range for a fine-grained analysis of age-related reorganization.\\
Concerning the proxies of the reserve construct, it should be noted that the construct of the reserve hypothesis is more diverse than the dimensions studied in \hyperref[results:paperI]{Research Articles \uproman{3}} and \hyperref[results:paperI]{\uproman{4}}. In our considerations, we followed the common practice of comparing individuals with low expression with individuals with higher expression of a given proxy, i.e., cardiorespiratory fitness or expertise \cite{Koen2019}. Cardiorespiratory fitness was indirectly measured by a 6 min walk test, which depends on the participants' motivation. However, a high correlation to the more objective VO$_2$max is reported in the literature \cite{Zhang2017}. The selection of the fine motor experts was based on the professional group of precision mechanics, whose daily routines and hand use may differ. Conversely, it cannot be excluded that the novices also had a certain level of fine motor skill, which they acquired in their everyday life, so that expertise effects may have been underestimated. However, by the criterion of a minimum of ten years of professional experience, the deliberate practice approach was followed, which was evaluated as a suitable criterion for professional expertise \cite{Ericsson1991, Voelcker-Rehage2013}. Furthermore, previous studies have shown that the selected fine motor task appropriately reflects the expertise context \cite{Vieluf2018, Goelz2018, Vieluf2012, Vieluf2013}. In general, a comparison to young participants could have tested further assumptions of the reserve hypothesis. For example, whether different networks are activated and have higher task specificity or whether we can assume that networks are maintained. However, this was not possible during participant recruitment due to the selected criterion, i.e., at least ten years of work experience in \hyperref[methods:datasets:I]{Dataset \uproman{1}} or was not considered in the original intervention study on which \hyperref[methods:datasets:III]{Dataset \uproman{3}} is based on.\\
In addition to the age structure and the proxies of the reserve concept, it should be noted that the tasks evaluated comprised particular laboratory tasks, whose transferability to everyday life may be limited. Nevertheless, the tasks should map across relevant domains in which age-related changes are reported, i.e., sensory, cognitive, and motor domains. The task types were all based on well-tested and common paradigms in aging research and had the advantage that the results were comparable to other studies. One aspect we cannot exclude is the possible effect of fatigue and motivation, which could have played a role, especially during longer sessions. Other aspects relate to the specific characteristics of the task designs. For example, the force control task in \hyperref[methods:datasets:I]{Dataset \uproman{1}} was designed in a block design, and the participants already attended several similar experiments in the context of the Bremen-Hand-Study@Jacobs, so practice effects cannot be excluded. These effects could also have been advantageous since age differences in short-term adaptation could thus be eliminated. The flanker task in \hyperref[methods:datasets:II]{Dataset \uproman{2}} differed slightly between the individual studies on which this dataset is based. This was present in the length of stimulus presentation, number of trials, and gender. However, because the participants of the individual studies were present in several groups and we did not find increased misclassification rates between these groups, we excluded that this had a significant influence on the results. In addition, the influence of gender and the number of trials in previous or comparable studies was estimated to be low \cite{Reuter2019, Vahid2020}. In \hyperref[methods:datasets:III]{Dataset \uproman{3}}, the task was quite long, which may have contributed to the fatigue effects mentioned above. Still, it allowed the temporal dynamics to be better mapped using machine learning algorithms.

\subsection{Machine Learning Approaches}
Applying machine learning algorithms depends on the dataset and utilized algorithm. Especially different noise levels in the individual \gls{eeg} recordings could have influenced the results of the group comparisons. However, studies using comparable methods show that this approach is valid, provided that the noise level is accounted for \cite{Bae2020, Vahid2020}. We used dimensionality reduction methods on the first level to improve the signal-to-noise ratio and to extract relevant features from the preprocessed \gls{eeg} data.\\
In \hyperref[results:paperI]{Research Article \uproman{1}}, \hyperref[results:paperIII]{\uproman{3}}, and \hyperref[results:paperIV]{\uproman{4}} we used \gls{dmd}, which allows to extract spatiotemporal coherent patterns from the \gls{eeg} data \cite{Brunton2016}. This approach allowed us to draw inferences about the dynamic properties of complex networks but differs from analyses of bivariate connectivity of voxel or (source) signals followed by graph analytic approaches to describe the brain networks derived in this way. The decision to use \gls{dmd} in this study was driven by its ability to capture the complex dynamics of networks that include factors such as frequency, growth, and decay. As a result, \gls{dmd} provides a physiologically plausible and low-dimensional representation of \gls{eeg} dynamics. Moreover, the effectiveness of \gls{dmd} in combination with machine learning has been demonstrated in previous studies \cite{Brunton2016, Kunert-Graf2019, Shiraishi2020}. In particular, \gls{dmd} has been shown to be very successful in classifying fine motor tasks \cite{Shiraishi2020}. Additionally, our research has already used \gls{dmd} to extract task-related sensorimotor network dynamics associated with age and expertise \cite{Vieluf2018, Goelz2018}. Also, \gls{dmd} in this work focused on task-related network characteristics derived from \gls{eeg} signals and not on resting state networks. Thus, we cannot conclude the relationship to the reorganization of large-scale resting state brain networks known in \gls{fmri} research, i.e., less separated and modular resting networks. Although it is generally unclear how task-related dedifferentiation and the reorganization of resting networks are related, a recent study shows that these two levels are related and predict performance \cite{Cassedy2020netw_distinct}. Another aspect to consider here is that our analyses were performed in the signal space and were not projected into the anatomical source space, so the anatomical interpretation was not possible with respect to the known resting state networks. This decision was justified by the small number of electrodes in \hyperref[methods:datasets:I]{Dataset \uproman{1}} and \hyperref[methods:datasets:II]{\uproman{2}}. Studies show that a valid source reconstruction depends strongly on the number of electrodes \cite{Song2015, Lantz2003}. Only for \hyperref[methods:datasets:III]{Dataset \uproman{3}} a sufficient number of electrodes was given. We, therefore, visualized the found patterns in the source space (see Figure 2 in \hyperref[pub:paperIII]{Published Research Article \uproman{3}}) but did not perform \gls{dmd} in the source space because validation studies for such an approach have not been performed so far. We only had individual \glspl{mri} for some participants, so no individual head models were available.\\
To align our results with previous \gls{erp} analyses and enhance the signal-to-noise ratio, we employed the xDAWN algorithm in \hyperref[results:paperIV]{Research Article \uproman{4}}. This choice allowed us to explore the temporal dynamics in response to stimuli rather than inferring network characteristics, as done in the other research articles. While calculating dynamic functional connectivity over time would have been an alternative approach, it is less extensively validated at this stage. Hence, we relied on the established xDAWN algorithm for our investigation.\\
The selection of classification algorithms and secondary dimensionality reduction algorithms may also have affected the results obtained. The process of picking suitable algorithms followed standard procedures in the field of machine learning \cite{Shalev2014}. That is, the choice of the dimensionality reduction algorithms was based on the respective objectives of the analyses, and the selection of the classifiers was based on the performance of the respective algorithm within the training portions. In addition, we replicated the results obtained by the classifiers with different algorithms and provided them as supplementary material in the respective published research articles. The results of the classification algorithms were carefully evaluated using various metrics such as accuracy, precision, recall, and F1 scores. We used classical algorithms that fit better to the amount of data available, i.e., tens to hundreds of samples. Additionally, we used cross-validation procedures and hyperparameter tuning, e.g., to tune the regularization parameters, limit overfitting, and estimate the generalization performance.\\
At this point, it should be mentioned that in \hyperref[results:paperIII]{Research Article \uproman{3}}, we did not use supervised classification to map dedifferentiation but instead extracted the patterns unsupervised via PCA and tested the discriminability using classical statistical methods. This decision was due to the relative amount of data available and the continuous design of the task. For this reason, it is possible that the effects of dedifferentiation were underestimated in both groups studied.\\
Although the classification results do not allow conclusions about the direction of the effects, these analyses should be seen as a supplement. In all research articles, we, therefore, related the results to previous research and examined the input to the classification with methods that allow conclusions about directional effects. For this purpose, we partly used inferential statistical methods and dimensionality reduction methods. As such, bridging the gap between traditional statistical and modern machine learning analysis was possible.

\section{Outlook and Practical Implications}
The results of this work have provided valuable insights at both the individual and group levels and have implications for future research and practical applications. The analyses conducted in \hyperref[results:paperI]{Research Article \uproman{1}} have shown that dedifferentiation of the motor system can be detected using \gls{eeg} measures even in late middle-aged individuals. Therefore, it would be exciting to further explore dedifferentiation in continuous age groups and to investigate whether different brain systems exhibit specific trajectories, such as those reported in \gls{mri} findings \cite{Raz2006}. We found initial evidence for this in \hyperref[results:paperII]{Research Article \uproman{2}}, in which we discovered reorganization of attentional control only in the two oldest groups and not, as with respect to the motor system, already in the late middle-aged participants.\\
Moreover, the temporal dynamics of dedifferentiation has not been studied so far and provides a new starting point for future studies. Supporting this idea, the results of \hyperref[results:paperIII]{Research Article \uproman{3}} suggest that the stability of cortical representations could be a crucial point. Drawing on the computational model by \citeauthor{Li2001} \cite{Li2001, Li2000}, we propose neural noise to explain our observation. Although initial findings suggest that neural noise may explain dedifferentiation at the \gls{eeg} level, this requires further investigation \cite{Pichot2022}.\\
Regarding the reserve concept, we examined only two dimensions and found varying results, so a systematic investigation of further lifestyle factors, such as education, would be of interest. To understand the concept concerning the reorganization of the brain, the development of a gold standard tool that captures all or as many as possible of the known domains of reserve would be necessary \cite{Nogueira2022}.\\
Exploratory group-level analysis also generated two hypotheses that could be tested in future research. Concerning expertise, the exploratory analysis of the group structure in \hyperref[results:paperIII]{Research Article \uproman{3}} allowed us to show that the factor of individuality, which was previously known only at the muscular or behavioral level, might also exist at the neuronal level. Furthermore, \hyperref[results:paperII]{Research Article \uproman{2}} suggests significant changes in the functional organization of the brain following retirement, as indicated by the classification of age groups.\\
\\
In addition to these points, the findings of this work have several practical implications. First, our analyses have shown that the reorganization of the aging brain, particularly the dedifferentiation, can be mapped at the individual level. Although the clinical relevance of dedifferentiation is not fully understood, \citeauthor{Fornito2015} \cite{Fornito2015} have described it as a maladaptive mechanism in brain networks. Therefore, our results could be valuable for the early detection of unfavorable age-related changes and contribute to developing and evaluating therapeutic concepts and interventions.\\
Additionally, in \hyperref[results:paperII]{Research Article \uproman{2}}, our results demonstrated that task-related EEG data improved age group prediction performance by up to 10 \%. This highlights the potential of using task-based EEG in clinical practice to predict the cognitive status of patients and may be particularly relevant for EEG-based detection of mild cognitive impairment, for which resting state \gls{eeg} alone was shown to be of limited value \cite{Froehlich2021, Farina2020}. An exciting starting point would be to integrate the methods used in this work into the framework of brain age prediction (see \chapref{theory:ml:applications_aging}). Perhaps this would help to overcome some limitations of the current framework, such as the limited ability to predict cognitive decline \cite{Tetereva2023}. Interestingly, studies suggest that a dedifferentiated activation pattern of brain networks may be related to poorer behavioral performance independent of age \cite{Koen2019}, so this may have applications in other areas independent of aging research.\\
The results of this work may also be relevant for the development of assistive technology for the rehabilitation and prevention of declining cognitive and motor abilities of older adults. In this context, the results could be particularly important for developing adaptive systems, i.e., systems that automatically adapt to the user. Such technologies could be based on individual EEG patterns and adapt the training content based on the signatures of dedifferentiation.\\
The development of such systems, especially BCIs, often relies on the differentiability of EEG features, e.g., based on the laterality of certain frequency features. However, our results suggest that this may lead to lower classifier performance, and rather age-specific features, such as higher frontality, could be exploited here. Consequently, our results could help address a possible age bias in developing assistive technology.

\section{Conclusion}
By using machine learning methods, this work has contributed to understanding age-related reorganization of the brain. More specifically, using classification and dimensionality reduction methods, this work quantified dedifferentiation based on \gls{eeg} measurements at the individual level. The results indicate that the expression and characteristics of dedifferentiation could be very different in different brain systems. For example, we found dedifferentiation of the motor system already in late middle-age. In contrast, differences in elements of the attentional system only became apparent in older age and are related to temporal aspects of information processing. It was also confirmed that reorganization depends on lifestyle factors such as cardiorespiratory fitness or work experience. However, different factors could also lead to different adaptation mechanisms and contribute to individuality in old age, underlining the need for a more differentiated view of the reserve concept.\\
In addition, two hypotheses emerged from exploratory machine learning analyses at the group level. The first is that professional expertise leads to an individualized neural representation of domain-specific tasks, and the second is that significant changes in the brain's functional organization occur after retirement.\\
Taken together, predictions of the computational model of dedifferentiation and the reserve hypothesis could be tested and confirmed. In addition, new hypotheses regarding the dynamics of brain reorganization and lifestyle factors were generated. The findings of this work may contribute to the development of markers of age-related changes and, thus, to the detection of unfavorable trajectories. Furthermore, the findings could contribute to the development of age-appropriate assistive systems. Thus, the findings are relevant to promoting healthy aging and developing technical systems that enable older adults to participate in society.




