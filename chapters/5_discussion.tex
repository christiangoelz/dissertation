This work aimed to better understand the reorganization of the aging brain by applying machine learning techniques to \gls{eeg} data. To ensure the proper integration of machine learning methods into the conventional science system, four publications used three theory-based approaches derived from literature. The focus was on specific characteristics of brain reorganization related to aging. This yielded unique insights and findings that would not have been possible through traditional analyses. Overall, the approaches formed a valuable starting point for tackling the complex nature of age-related brain reorganization.\\
The first approach was based on dedifferentiation, i.e., less specific activation of brain networks expressed in a lower differentiability of brain states. Based on the computational model by \citeauthor{Li2001} \cite{Li2001, Li2000}, this is attributed to a loss of selectivity of neuronal responsiveness, which has been confirmed in numerous animal and human studies. Building upon this, we tested the differentiability of brain states, e.g., in the form of activation patterns of brain networks, and thus the selectivity of neuronal representations using machine learning methods. More specifically, we trained classifiers based on \gls{eeg} data to predict specific task states. A better performance indicated that the signals or signal patterns were more similar between the task states, i.e., less differentiable. This approach provided direct insights into age-related dedifferentiation and allowed the formulation of novel hypotheses, e.g., regarding this phenomenon's temporal aspect. The second approach dealt with the reserve hypothesis. We investigated the possible positive effects of certain lifestyle factors, such as physical fitness or occupational expertise, on age-related brain reorganization. We compared the results of the machine learning analyses, i.e., the extracted brain network patterns and their differentiability, between participants with different lifestyles backgrounds. These analyses revealed unique perspectives and helped to test the assumption that individuals with certain lifestyle backgrounds show less dedifferentiation. In the third approach, dimensionality reduction and group-level classification algorithms were used to identify overarching patterns in brain reorganization. That is, the classifiers were trained to predict which age or lifestyle group a participant belonged to based on the \gls{eeg} data. Furthermore, dimensionality reduction methods were applied to the data points of all participants to extract dominant patterns and interactions. This approach enabled unique insights and the generation of novel hypotheses.\\
In the following, the main results of these approaches are discussed and evaluated in light of the current scientific discourse. However, for a detailed discussion of each specific result, please refer to the corresponding \hyperref[pub:papers]{published journal articles}, which are attached at the end of this thesis.

\section{Detecting the Dedifferentiation of the Aging Brain}
Motivated by applications from research using \gls{fmri}, in which dedifferentiation is quantified based on the performance of classification algorithms, we trained classifiers to predict a given task state based on \gls{eeg} data to investigate the differentiability of signals in motor, cognitive, and sensory tasks in participants of different age groups.\\
In \hyperref[results:paperI]{Research Article \uproman{1}}, we found differences in the classification performance of fine motor tasks based on \gls{eeg} data. In particular, performance in classifying which hand was used to perform the task was reduced in the late middle-aged (the oldest group in this dataset) compared to younger participants. That is, the differentiability of brain network patterns of left hand execution and right hand execution was reduced in the late middle-aged adults. Studying the classifier input, i.e., the brain network activation patterns, we found signs of a more bilateral and widespread activation pattern in late middle-aged adults, which might explain the difference in the classifier performance. These results are consistent with a dedifferentiated motor network as reported in \gls{fmri} studies \cite{Carb2011, Cassedy2020}. They are also consistent with other \gls{eeg} results that found it was more difficult in older adults to predict which hand was used to perform a motor task \cite{Chen2019, Zich2015}. In addition, we could more easily predict whether a participant followed a steady or sinusoidal target force in late middle-aged adults compared with younger adults. When considering the classifier input again, we found a stronger frontal involvement in the network activation patterns of the late middle-aged participants and thus suspected that compensation processes play a role here. Compensation could be present, especially in the more complex sinusoidal force tracking. Here, late middle-aged adults might use additional (frontal) brain resources, which increases the differentiability of the activation patterns and might explain the better classifier performance in the late middle-aged participants. This interpretation is consistent with the \gls{crunch} hypothesis, which postulated that during task performance, as task difficulty (or load) increases, more cortical resources will be activated \cite{Festini2018}.\\
In \hyperref[results:paperII]{Research Article \uproman{2}}, we again used classification algorithms to investigate whether the cortical representation of selective attention differs between six different age groups, from children to older adults. To do so, we trained classifiers that predict which stimulus type, i.e., stimuli requiring and stimuli not requiring inhibitory control, was presented and compared the classification performance between the groups. Here we found that only in children it was not possible to predict which stimulus was presented, i.e., the differentiability between stimuli requiring inhibitory control and stimuli not requiring inhibitory control was reduced in this group. This reduced differentiability is linked to the assumption of inhibitory control in children that is not yet differentiated at the neural level \cite{Waszak2010, Reuter2019}. Contrary to the findings for the motor system, here we found no differences in classification performance in the older participants compared to the younger age groups. These results of similar good performance in classifying the two stimulus categories contradict the assumption of a general dedifferentiation of all cortical systems. Similar results are also known in \gls{fmri} studies, which, for example, found no dedifferentiation in the activation of the visual system in response to certain stimulus categories \cite{Voss2008}. However, by using \gls{eeg}, we were able to examine the temporal response to the stimulus additionally. Thus, it was possible to examine the classifier performance over time and to determine which time point after the presentation of the stimulus is crucial for predicting the stimulus category. In this way, we found that later time points were crucial for prediction in older adults, i.e., delayed differentiability occurred. Thus, dedifferentiation might also have a dynamic component related to the rise of neural noise, as postulated in the computational model of \citeauthor{Li2001} \cite{Li2001, Li2000}. However, this only occurred in the two oldest age groups and not, as in the previous study, in the late middle-aged, so this effect might arise later in the aging process.\\
\\
Taken together, it was shown that age-related reorganization of the brain is reflected in the performance of classifiers based on \gls{eeg} measures. Compared to traditional analysis methods, we were able to test theories of age-related reorganization more directly using machine learning algorithms and to generate new hypotheses, e.g., the dynamic component of dedifferentiation, by providing unique insights into central information processing in older adults.\\

\section{The Impact of Lifestyle Factors on Age-Related Brain Reorganization}
An important aspect of better understanding the individuality of age-related reorganization is to investigate the impact of influencing factors. From the literature, physical fitness can be deduced as a significant candidate that affects the brain's ability to resist degradation processes and maintain function, thus contributing to reserve. In this context, \citeauthor{Stillman2019} \cite{Stillman2019} summarized findings of \gls{fmri} studies proving that the network structure of the brain at rest of physically fit people show fewer signs of dedifferentiation. Therefore, in \hyperref[results:paperIII]{Research Article \uproman{3}} we investigated to what extent cardiorespiratory fitness affects the dedifferentiation of task-related network activation patterns based on \gls{eeg}. To do so, we extracted dominant spatiotemporal network patterns and their prominence over time during the performance of motor, sensory and cognitive tasks. In this way, we were able to estimate task-related brain network activation patterns in the main domains in which age-related changes are reported and compare this between a fit and less fit group \cite{Baltes1997, Sala-Llonch2015, Park2009}. This comparison revealed greater differences between the task-related activation patterns in the fit compared to the less fit group, suggesting lower levels of dedifferentiation in the less fit group. We further found that the extracted dominant patterns of the fit group resolved more variance compared to the less fit group, indicating higher stability or prominence of the patterns in the fit group. These differences were frequency dependent mainly to the $\theta$ and $\beta_2$ bands which could be carefully attributed to processes reflecting cognitive involvement and information integration \cite{Siegel2012}. From the perspective of the computational model of dedifferentiation, the higher stability could indicate less neural noise and a lower rate of neural variability, which points to a higher efficiency of information processing.\\
Another frequently studied factor of influence in the literature is professional expertise. Thus, in \hyperref[results:paperIV]{Research Article \uproman{4}}, we compared fine motor experts and novices with respect to the classification performance of visuomotor tasks. We found no group differences in the performance of a classifier trained to classify the same task characteristics as in \hyperref[results:paperI]{Research Article \uproman{1}}. Consequently, we could not detect any differences in the differentiation of the neural representation of the task between novices and experts. At first glance, these results contradict previous studies from the Bremen-Hand-Study@Jacobs, which showed statistical differences in the expression of network characteristics of experts compared to novices in terms of neural efficiency \cite{Goelz2018, Vieluf2018}. However, our study tested a particular assumption of age-related reorganization, namely dedifferentiation, and did not test neural efficiency as in the previous work. Also, in contrast to \hyperref[results:paperIII]{Research Article \uproman{3}}, the results seem to contradict the reserve concept, based on which a lower level of dedifferentiation would have been expected. Therefore, taking a more differentiated view of the reserve concept concerning individual factors could be necessary to refine this concept. We have only explored one aspect in considering dedifferentiation in the aging brain. As a result, other characteristics of brain reorganization may differ, such as the efficiency of information processing or compensational network involvement. These factors were not accounted for in our analyses.\\
\\
Overall, these results are partially consistent with the predictions of the reserve hypothesis and provide novel evidence supporting the assumptions of the computational model of dedifferentiation. Nevertheless, the results also suggest that a more nuanced view of the reserve concept is needed with respect to various influencing factors.

\section{Exploratory Insights into Age-Related Brain Reorganization}
While the previous results refer to the application at the participant level, we also used the machine learning methods at the group level to gain explorative insights into overreaching patterns and group characteristics.\\
In \hyperref[results:paperII]{Research Article \uproman{2}}, we found a grouped structure of misclassifications when using a classifier that was trained to predict the group membership in one of the six age groups based on \gls{eeg} data. Misclassification into a younger group would mean that the associated brain activity pattern would be more like that of the younger group and vice versa. The classifier tended to misclassify primarily between certain age groups. This was particularly noticeable in the two oldest age groups, between which misclassification was more frequent. That means it was more difficult to predict whether a participant belonged to the oldest or the second oldest group. Less frequently, however, late-middle-aged participants were classified as part of one of the two oldest groups of adults. That is, the \gls{eeg} of the late middle-aged were more clearly assignable to the correct group or, vice versa, the participants of the two older groups seemed to show rather different activation patterns compared to the late middle-aged group. Since the late middle-aged participants are still of working age (up to 65 years in Germany) and the participants of the two older groups have already exceeded the retirement age, a possible explanation could be that significant cognitive changes occur after reaching retirement age. Deterioration of cognitive abilities and cerebral blood flow after retirement has been reported in the literature \cite{Celidoni2017, Rohwedder2010, Rogers1990}. However, we could not further test this hypothesis because of a lack of data on the exact retirement age of the participants. In addition, no misclassification occurred between children and the oldest groups. Although differences based on \gls{erp} markers are often described as u-shaped \cite{Mueller2008, Reuter2019}, this result suggests different mechanisms, i.e., differentiation in children vs. dedifferentiation in older adults. Overall, we found an increase in classification performance shortly after stimulus onset of about 10~\%, highlighting the added value of task-related \gls{eeg} recordings and suggesting early processing of the stimulus to be affected by age-related reorganization.\\
Also, in \hyperref[results:paperIV]{Research Article \uproman{4}}, a classifier was trained based on the \gls{eeg} data to determine the membership to the group of experts or novices. In contrast to the previous article, however, the performance was below the chance level, so it can be assumed that the brain activation between experts and novices is similar. This finding contrasts with studies in which the classification of the level of expertise was already successful \cite{Hosp2021, Winkler-Schwartz2019, Shourie2016}. In contrast to these studies, the professional background of the fine motor experts examined in this study, such as opticians and watchmakers, was very diverse, and the visuomotor tracking tasks selected for analysis offered only a general approximation of the context of their expertise. Based on this, we visualized the underlying brain network characteristics using a nonlinear dimensionality reduction in two-dimensional space. Unique differences in the data structure between experts and novices emerged. Compared to each other, the brain network patterns of the experts showed a higher degree of a clustered structure in which each cluster could be clearly assigned to one participant. This finding can be interpreted as a higher individuality of the brain network patterns in the experts. A high degree of individuality is also known in the movement patterns of flute players and the muscle synergies of expert powerlifters \cite{Albrecht2014, Caramiaux2018, Kristiansen2015}. Our results suggest that individuality, which was gained in the context of professional expertise, is as well present at the network level of the brain.\\
\\
In summary by using machine learning, it was possible to identify and visualize relevant changes and group characteristics. This led to new hypotheses. One is the hypothesis that major cognitive changes occur after retirement, which is reflected in the organization of the brain. Another is that the formation of fine motor expertise could be reflected in individualized central information processing.

\section{Methodological Considerations}
Before discussing specific methodological aspects regarding the characteristics of the datasets as well as the application of machine learning procedures, it should be noted that we have not presented \textit{changes} but \textit{differences} as the study design of all studies from which the datasets were derived were cross-sectional. We also focused only on some well-documented specific aspects of the reorganization of the aging brain and explored overarching patterns in an exploratory manner. In addition, the results should be evaluated in light of the following methodological considerations. 

\subsection{Datasets}
Concerning the individual datasets and the underlying study protocols, some considerations should be taken into account when evaluating the results presented.\\
Different age groups were present in the datasets underlying the individual research articles. The older (late middle-aged) participants in \hyperref[methods:datasets:I]{Dataset \uproman{1}} examined in \hyperref[results:paperI]{Research Article \uproman{1}} were comparatively young, so for these results, it is to be expected that possible age effects have been underestimated. In \hyperref[methods:datasets:II]{Dataset \uproman{2}}, the minimum age difference between the six age groups varied from ten years (between children and young adults) to one year (within the two oldest groups). This age structure may have significantly conditioned the classification results in \hyperref[results:paperII]{Research Article \uproman{2}}. Nevertheless, the minimum age difference between the oldest and the next younger groups (late middle-aged adults) is also only two years. Hence, the finding of misclassification cutoff between these age groups is less likely to be explained by the age structure of the dataset. In general, regarding the age structure, it would be advantageous to have a continuous age range for a fine-grained analysis of age-related reorganization. Nevertheless, all datasets taken together represent a comparatively large number of age groups and participants. In addition, the datasets have the advantage of having been recorded consistently under controlled measurement conditions.\\
Concerning the proxies of the reserve construct, it should be noted that the construct of the reserve hypothesis is more diverse than the dimensions studied in \hyperref[results:paperI]{Research Articles \uproman{3}} and \hyperref[results:paperI]{\uproman{4}}. In our considerations, we followed the common practice of comparing individuals with low expression with individuals with higher expression of a given proxy, i.e., cardiorespiratory fitness or expertise \cite{Koen2019}. Cardiorespiratory fitness was indirectly measured by a 6 min walk test, which depends on the participants' motivation and daily condition. However, a high correlation to the more objective VO$_2$max is reported in the literature \cite{Zhang2017}. The selection of fine motor experts was based on the professional group of precision mechanics, whose daily routines and hand use may differ. Conversely, it cannot be excluded that the novices also had a certain level of fine motor skill, which they acquired in their everyday life, so that expertise effects may have been underestimated. However, by the criterion of a minimum of ten years of professional experience, the deliberate practice approach was followed, which was evaluated as a suitable criterion for professional expertise \cite{Ericsson1991, Voelcker-Rehage2013}. Furthermore, the study context of a laboratory task did not exactly reflect the experts' domain of expertise. Regardless, previous studies of the Bremen-Hand study have shown that the selected fine motor task adequately reflects the given context of fine motor expertise \cite{Vieluf2018, Goelz2018, Vieluf2012, Vieluf2013}. In general, a comparison to young participants could have tested further assumptions of the reserve hypothesis. For example, whether different networks are activated and have higher task specificity or whether we can assume that networks are maintained in older age. However, this was not possible during participant recruitment due to the selected criterion, i.e., at least ten years of work experience in \hyperref[methods:datasets:I]{Dataset \uproman{1}} or was not considered in the original intervention study on which \hyperref[methods:datasets:III]{Dataset \uproman{3}} is based. Regarding recruitment in general, it should be noted that due to voluntary recruitment in all underlying studies, selection bias cannot be ruled out. Voluntary participation in studies is often associated with higher socioeconomic status, better health, physical condition, or a generally greater interest in the study topic \cite{Ganguli1998, Peters-Davis2001, Dodge2014}. This may have led to underestimating age effects and fitness or expertise as influencing factors.\\
With regard to the underlying study paradigms, the tasks considered were specific laboratory tasks, the applicability of which to daily life may be limited. Only selected task paradigms were considered in this work. In the motor domain, we limited our analyses to visuomotor force control tasks and left out other paradigms frequently used in aging research, such as finger tapping or gross motor tasks. However, we consider the task highly relevant to everyday life, as a major component is hand-eye coordination. Similarly, the cognitive domain was restricted to selective attention tested by a flanker paradigm and working memory tested by a n-back task. Of course, other domains could extend the results presented here. However, these are task paradigms that are widely used in aging research and for which we could draw on existing results or use a broad research base. The same applies to the sensory stimulation task used. Taken together, it can be stated that the tasks covered all relevant domains in which age-related changes are reported, i.e., sensory, cognitive, and motor domains. The task types were all based on well-tested and common paradigms in aging research and had the advantage that the results were comparable to other studies. One aspect we cannot exclude is the possible effect of fatigue and motivation, which could have played a role, especially during longer sessions. Furthermore, a dependency of the results on the examination personnel and the original examination objective cannot be excluded, especially for \hyperref[methods:datasets:II]{Dataset \uproman{2}}. Here, the datasets originate from different data collections with different study objectives and study personnel. However, the same paradigms and measurement methods were used, resulting in a comparatively large dataset with six different age groups and consistent measurement protocol.\\
Other aspects relate to the specific characteristics of the task designs. For example, the force control task in \hyperref[methods:datasets:I]{Dataset \uproman{1}} was designed in a block design, and the participants already attended several similar experiments in the context of the Bremen-Hand-Study@Jacobs, so practice effects cannot be excluded. These effects could also have been advantageous since age differences in short-term adaptation could thus be eliminated. The flanker task in \hyperref[methods:datasets:II]{Dataset \uproman{2}} was a colored flanker task that might have been too easy, especially for young and middle-aged adults, but has the advantage of being feasible for older adults and children as well \cite{Reuter2019}. The task differed slightly between the individual studies on which this dataset is based. This was present in the length of stimulus presentation, number of trials, and sex. However, because the participants of the individual studies were present in several groups and we did not find increased misclassification rates between these groups, we excluded that this had a significant influence on the results. In addition, the influence of sex and the number of trials in previous or comparable studies was estimated to be low \cite{Reuter2019, Vahid2020}. In \hyperref[methods:datasets:III]{Dataset \uproman{3}}, the tasks were quite long, so fatigue effects are likely. Still, it allowed the temporal dynamics to be better mapped using machine learning algorithms. Also, the sensory task was passive, so participants' engagement could not be supervised during task execution. Nonetheless, this reduced the participants' active load and fatigue effects.\\
During all tasks in the datasets used, brain functioning was recorded using \gls{eeg} only. Other methods, such as \gls{fnirs}, \gls{fmri} or \gls{meg}, would undoubtedly have been feasible for the types of tasks used, but as presented in \chapref{theory:aging:EEG} \gls{eeg} had the advantage of the ease of use, cost effectiveness and direct measurement of neuronal activity in the millisecond range. Machine learning techniques were employed to address the challenges associated with low signal-to-noise ratio and complex interpretation, effectively overcoming these limitations in all presented research articles.

\subsection{Machine Learning Approaches}
Applying machine learning algorithms depends on the dataset and utilized algorithm. Especially different noise levels in the individual \gls{eeg} recordings could have influenced the results of the group comparisons. However, studies using comparable methods show that this approach is valid, provided that the noise level is accounted for \cite{Bae2020, Vahid2020}. We used dimensionality reduction methods on the first level to improve the signal-to-noise ratio and to extract relevant features from the preprocessed \gls{eeg} data.\\
In \hyperref[results:paperI]{Research Article \uproman{1}}, \hyperref[results:paperIII]{\uproman{3}}, and \hyperref[results:paperIV]{\uproman{4}} we used \gls{dmd}, which allows to extract spatiotemporal coherent patterns from the \gls{eeg} data \cite{Brunton2016}. This approach allowed us to draw inferences about the dynamic properties of complex networks but differs from analyses of bivariate connectivity of voxel or (source) signals followed by graph analytic approaches to describe the brain networks derived in this way. The decision to use \gls{dmd} in this study was driven by its ability to capture the complex dynamics of networks that include factors such as frequency, growth, and decay. As a result, \gls{dmd} provides a physiologically plausible and low-dimensional representation of \gls{eeg} dynamics. Moreover, the effectiveness of \gls{dmd} in combination with machine learning has been demonstrated in previous studies \cite{Brunton2016, Kunert-Graf2019, Shiraishi2020}. In particular, \gls{dmd} has been shown to be very successful in classifying fine motor tasks \cite{Shiraishi2020}. Additionally, our research has already used \gls{dmd} to extract task-related sensorimotor network dynamics associated with age and expertise \cite{Vieluf2018, Goelz2018}. Also, \gls{dmd} in this work focused on task-related network characteristics derived from \gls{eeg} signals and not on resting state networks. Thus, we cannot conclude the relationship to the reorganization of large-scale resting state brain networks known in \gls{fmri} research, i.e., less separated and modular resting networks. Although it is generally unclear how task-related dedifferentiation and the reorganization of resting networks are related, a recent study shows that these two levels are related and predict performance \cite{Cassedy2020netw_distinct}. Another aspect to consider here is that our analyses were performed in the signal space and were not projected into the anatomical source space, so the anatomical interpretation was not possible with respect to the known resting state networks. This decision was justified by the small number of electrodes in \hyperref[methods:datasets:I]{Dataset \uproman{1}} and \hyperref[methods:datasets:II]{\uproman{2}}. Studies show that a valid source reconstruction depends strongly on the number of electrodes \cite{Song2015, Lantz2003}. Only for \hyperref[methods:datasets:III]{Dataset \uproman{3}} a sufficient number of electrodes was given. We, therefore, visualized the found patterns in the source space (see Figure 2 in \hyperref[pub:paperIII]{Published Research Article \uproman{3}}) but did not perform \gls{dmd} in the source space because validation studies for such an approach have not been performed so far. We only had individual \glspl{mri} for some participants, so no individual head models were available.\\
To align our results with previous \gls{erp} analyses and enhance the signal-to-noise ratio, we employed the xDAWN algorithm in \hyperref[results:paperII]{Research Article \uproman{2}}. This choice allowed us to explore the temporal dynamics in response to stimuli rather than inferring network characteristics, as done in the other research articles. While calculating dynamic functional connectivity over time would have been an alternative approach, it is less extensively validated at this stage. Hence, we relied on the established xDAWN algorithm for our investigation.\\
The selection of classification algorithms and secondary dimensionality reduction algorithms may also have affected the results obtained. The process of picking suitable algorithms followed standard procedures in the field of machine learning \cite{Shalev2014}. That is, the choice of the dimensionality reduction algorithms was based on the respective objectives of the analyses, and the selection of the classifiers was based on the performance of the respective algorithm within the training portions. In addition, we replicated the results obtained by the classifiers with different algorithms and provided them as supplementary material in the respective published research articles. The results of the classification algorithms were carefully evaluated using various metrics such as accuracy, precision, recall, and F1 scores. It should also be noted that machine learning is developing quickly as a field, but new algorithms mostly focus on big data applications. In contrast, we have used classical algorithms that are more adaptable to the amount of data available, i.e., tens to hundreds of samples. In addition, we used cross-validation and hyperparameter tuning techniques, e.g., to tune the regularization parameters, limit overfitting, and estimate generalization performance.\\
At this point, it should be mentioned that in \hyperref[results:paperIII]{Research Article \uproman{3}}, we did not use supervised classification to map dedifferentiation but instead extracted the patterns unsupervised via PCA and tested the discriminability using classical statistical methods. This decision was due to the continuous design of the task. For this reason, it is possible that the effects of dedifferentiation were underestimated in both groups studied.\\
Although the classification results do not allow conclusions about the direction of the effects, these analyses should be seen as a supplement. In all research articles, we, therefore, related the results to previous research and examined the input to the classification with methods that allow conclusions about directional effects. For this purpose, we partly used inferential statistical methods and dimensionality reduction methods. As such, bridging the gap between traditional statistical and modern machine learning analysis was possible.

\section{Outlook and Practical Implications}
The results of this dissertation have provided valuable and novel insights at both the individual and group levels and have implications for future research and practical applications. The analyses conducted in \hyperref[results:paperI]{Research Article \uproman{1}} have shown that dedifferentiation of the motor system can be detected using \gls{eeg} measures even in late middle-aged individuals. Therefore, it would be exciting to further explore dedifferentiation in continuous age groups and to investigate whether different brain systems exhibit specific trajectories, such as those reported in \gls{mri} findings \cite{Raz2006}. We found initial evidence for this in \hyperref[results:paperII]{Research Article \uproman{2}}, in which we discovered reorganization of attentional control only in the two oldest groups and not, as with respect to the motor system, already in the late middle-aged participants.\\
Moreover, the temporal dynamics of dedifferentiation has not been studied so far, providing a new starting point for future studies. Supporting this idea, the results of \hyperref[results:paperIII]{Research Article \uproman{3}} suggest that the stability of cortical representations could be a crucial point. Drawing on the computational model by \citeauthor{Li2001} \cite{Li2001, Li2000}, we propose neural noise to explain our observation. Although initial findings suggest that neural noise may explain dedifferentiation at the \gls{eeg} level, this requires further investigation \cite{Pichot2022}.\\
Regarding the reserve concept, we examined only two dimensions and found varying results, so a systematic investigation of further lifestyle factors, such as education, would be of interest. To understand the concept concerning the reorganization of the brain, the development of a gold standard tool that captures all or as many as possible of the known domains of reserve would be necessary \cite{Nogueira2022}.\\
Exploratory group-level analysis also generated two novel hypotheses that could be tested in future research. Concerning expertise, the exploratory analysis of the group structure in \hyperref[results:paperIII]{Research Article \uproman{3}} allowed us to show that the factor of individuality, which was previously known only at the muscular or behavioral level, might also exist at the neuronal level. Furthermore, \hyperref[results:paperII]{Research Article \uproman{2}} suggests significant changes in the functional organization of the brain following retirement, as indicated by the classification of age groups.\\
\\
In addition to these points, the findings of this work have several practical implications. First, our analyses have shown that the reorganization of the aging brain, particularly the dedifferentiation, can be mapped at the individual level using \gls{eeg} measurements. Although the clinical relevance of dedifferentiation is not fully understood, \citeauthor{Fornito2015} \cite{Fornito2015} have described it as a maladaptive mechanism in brain networks. Therefore, our results could be valuable for the early detection of unfavorable age-related changes and contribute to developing and evaluating therapeutic concepts and interventions.\\
Additionally, in \hyperref[results:paperII]{Research Article \uproman{2}}, our results demonstrated that task-related \gls{eeg} data improved age group prediction performance by up to 10 \%. This highlights the potential of using task-based \gls{eeg} in clinical practice to predict the cognitive status of patients and may be particularly relevant for EEG-based detection of mild cognitive impairment, for which resting state \gls{eeg} alone was shown to be of limited value \cite{Froehlich2021, Farina2020}. An exciting starting point would be to integrate the methods used in this work into the framework of brain age prediction (see \chapref{theory:ml:applications_aging}). Perhaps this would help to overcome some limitations of the current framework, such as the limited ability to predict cognitive decline \cite{Tetereva2023}. Interestingly, studies suggest that a dedifferentiated activation pattern of brain networks may be related to poorer behavioral performance independent of age \cite{Koen2019}, so this may have applications in other areas independent of aging research.\\
The results of this work may also be relevant for the development of assistive technology for the rehabilitation and prevention of declining cognitive and motor abilities of older adults. In this context, the results could be significant for developing adaptive systems, i.e., systems that automatically adapt to the user. Such technologies could be based on individual \gls{eeg} patterns and adapt the training content based on the signatures of dedifferentiation.\\
Developing such systems, especially BCIs, often relies on the differentiability of \gls{eeg} features, e.g., based on the laterality of certain frequency features. However, our results suggest that this may lead to lower classifier performance, and rather age-specific features, such as higher frontality, could be exploited here. Consequently, our results could help address a possible age bias in developing assistive technology.

\section{Conclusion}
This thesis used machine learning techniques to better understand how the brain reorganizes as we age. The research presented shows that analyzing \gls{eeg} data with machine learning can uniquely identify age-related brain reorganization and provide novel insights. The approaches followed in this work allowed to test theoretical models of brain aging resulting in previously unexplored discoveries that would have remained hidden with traditional analyses. The results provide new starting points for understanding the reorganization of the aging brain, especially with regard to the loss of neural specialization, i.e., dedifferentiation, and how lifestyle factors might contribute to building a reserve to age-related deterioration.\\
More specifically, using classification and dimensionality reduction methods, this work uniquely quantified dedifferentiation based on \gls{eeg} measurements at the individual level. The results indicate that the expression and characteristics of dedifferentiation could be very different in different brain systems. For example, we found dedifferentiation of the motor system already in late middle age. In contrast, differences in elements of the attentional system only became apparent in older age and are related to temporal aspects of information processing. It was also confirmed that reorganization depends on lifestyle factors such as cardiorespiratory fitness or work experience. However, different factors could also lead to different adaptation mechanisms and contribute to individuality in old age, underlining the need for a more differentiated view of the reserve concept.\\
In addition, two novel hypotheses emerged from exploratory machine learning analyses at the group level. The first is that professional expertise leads to an individualized neural representation of domain-specific tasks, and the second is that significant changes in the brain's functional organization occur after retirement.\\
Taken together, predictions of the computational model of dedifferentiation and the reserve hypothesis could be tested and confirmed. In addition, new hypotheses regarding the dynamics of brain reorganization and lifestyle factors were generated. The findings of this work may contribute to the development of markers of age-related reorganization and, thus, to the detection of unfavorable trajectories. Furthermore, the findings could contribute to the development of age-appropriate assistive systems. Thus, the findings are relevant to promoting healthy aging and developing technical systems that enable older adults to participate in society.




