Driven by ever-increasing amounts of data and advancing computer infrastructure, the field of artificial intelligence (AI) is becoming increasingly influential in socially important areas. These include public transportation, e.g., autonomous vehicles \cite{Leonard2020}, the medical sector, e.g, diagnostic imaging \cite{Liu2020} or social interaction, e.g., conversational AI or so-called chatbots \cite{Adamopoulou2020}. Likewise, progress in science is increasingly characterized by the application of methods from AI, which makes it possible to analyze the increasingly large and complex amounts of data systematically \cite{Brunton2019}. This has led to proclamations of an "AI revolution in science" \cite{Appenzeller2017} or promoting science has entered the fourth paradigm characterized by \textit{data-intensive computing} \cite{Hey2009}.\\
Thus, AI as a key technology becomes a hope for solving societal challenges. One of the greatest challenges in industrialized countries is the demographic shift toward an older population. This poses enormous challenges for society as a whole raising issues for the healthcare system, infrastructure, family policy and the occupational sector. 
To address these challenges, it is crucial to develop an understanding of individual aging trajectories. This is relevant to distinguish age-related diseases from healthy aging at an early stage, to develop and plan targeted interventions, or to develop technologies adapted to an aging population. 
- hot topic: Sensorimotorische Veränderungen, kognitive Veränderungen und die neuronale Grundlage, da für alltag relevant 
- AI kann helfen, krankheitsbedingte Veränderungen zu detektiern aber generelle Altersbedingte Veränderungen zu charakterisiere 

Besides the early automatic detection of pathological changes, tools from AI can be used to describe individual trajectories of aging helping to identfy biomarkers that could be used to verify treatments as well as targeted interventions.



Identfying risk factors, developing plans to support healthy aging 

 Other applications include systems to assist older adults in daily living or care assistive technology, such as tools for neurorehabilitation. However, such applications and technical systems are still under development. The practical acceptance and use of AI requires an evaluation of application areas in science and practice. 
Was ist genau unklar? 


- Lassen sich altersbedingte Veränbderungen mit KI abbilden und was lässt sich daraus für die Anwenung in KI basierten Systemen ableiten? 
- Ergeben sich neue Hypothesen aus der Anwendung? 


sensomotorische Kontrolle und Altern: 
- Lassen sich Veränderungen des sensomotorischen Systems abbilden 

kognitive Verarbeitung: 
- Lassen sich 

Handlunsgfeld Einflussfaktoren 




\begin{itemize}
    \item Age related changes occur at different scales and are manifestet at several levels.
    \item There is a wide variety in how this changes occur
    \item Changes are e.g. neural dedifferentiation and compensatory mechanisms (see Reuter Lorenz et al. 2010) and are noticable brain network level and dynamics
    \item NOTE: Check what EEG studies said about this...
    \item The idea is to model these changes with tools from datascience to answer questions in aging neuroscience
    \item First study is about detecting dedifferentiated and compensatory mechanisms with EEG
    \item Tools used are DMD and Machine learning
    \item Main idea: Study classification performance as proxy for age related changes in different motor control tasks
    \item Expertise as possible way of builing a reserve:
    \item Higher individuality 
    \item Dynamics of dedifferentiation and how do they relate to fitness
    \item Basic for targeted interventions 
    \item How much and what (relate to Julia)
    \item Background of ML
    \item ML as tool 
    \item novel insights 
    \item Problem: Data is multidimensional and we have often limited data 
    \item Solution: Use DMD to reduce Complexity and "model" evolution of signal 
    \item Dynamic Mode Decompsition
    \item DMD extracts coupled spatio-temporal modes and is able to kind of model the evolution of the signal 
    \item Backgrouund + Papers 
    \item Mathematical Formulation
    \item What can ML tell us?
    \item ML applied in aging Neuroscience
    \item Formulating Aims and goals 
    \item Formulation expected outcomes
\end{itemize}
