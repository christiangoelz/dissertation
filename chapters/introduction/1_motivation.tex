Driven by ever-increasing amounts of data and advancing computer infrastructure, the field of artificial intelligence (AI) in particular is becoming increasingly influential in socially important areas. These areas include public transportation, e.g., autonomous vehicles \cite{Leonard2020}, the medical sector, e.g, diagnostic imaging \cite{Liu2020} or social interaction, e.g., conversational AI or so-called chatbots \cite{Adamopoulou2020}. Likewise, progress in science is increasingly characterized by the application of methods from AI, which makes it possible to analyze the increasingly large and complex amounts of data systematically \cite{Brunton2019}. This has led, among other things, to proclamations of an "AI revolution in science" \cite{Appenzeller2017} or promoting science has entered the fourth paradigm characterized by \textit{data-intensive computing} \cite{Hastie2009}.\\
One of the major challenges of industrialized countries is a demographic change towards an older population. 
- Include care assistive technology in so called human machine interfacing 
- Describing and understnding age related changes to guide to differentiate healthy from pathological changes and to develop and verify treatments as well as targeted interventions

Understanding age related changes of the brain is fundamental to differentiate healthy from pathological changes and to develop and verify treatments as well as targeted interventions. This implies a deeper understanding of complex multidimensional and multimodal data \cite{Brunton2019}. In addition to classical statistical approaches, tools from data science, particularly machine learning, offer new ways to understand multidimensional data in a naturalistic way \cite{Breiman2001}. 

basic science, e.g. understanding protein structure \cite{Jumper2021}
\begin{itemize}
    \item ML as the next frontier in science \
    \item Open questions in aging neuroscience 
    \item What can ML tell us? 
\end{itemize}

\begin{itemize}
    \item Age related changes occur at different scales and are manifestet at several levels.
    \item There is a wide variety in how this changes occur
    \item Changes are e.g. neural dedifferentiation and compensatory mechanisms (see Reuter Lorenz et al. 2010) and are noticable brain network level and dynamics
    \item NOTE: Check what EEG studies said about this...
    \item The idea is to model these changes with tools from datascience to answer questions in aging neuroscience
    \item First study is about detecting dedifferentiated and compensatory mechanisms with EEG
    \item Tools used are DMD and Machine learning
    \item Main idea: Study classification performance as proxy for age related changes in different motor control tasks
    \item Expertise as possible way of builing a reserve:
    \item Higher individuality 
    \item Dynamics of dedifferentiation and how do they relate to fitness
    \item Basic for targeted interventions 
    \item How much and what (relate to Julia)
    \item Background of ML
    \item ML as tool 
    \item novel insights 
    \item Problem: Data is multidimensional and we have often limited data 
    \item Solution: Use DMD to reduce Complexity and "model" evolution of signal 
    \item Dynamic Mode Decompsition
    \item DMD extracts coupled spatio-temporal modes and is able to kind of model the evolution of the signal 
    \item Backgrouund + Papers 
    \item Mathematical Formulation
    \item What can ML tell us?
    \item ML applied in aging Neuroscience
    \item Formulating Aims and goals 
    \item Formulation expected outcomes
\end{itemize}
