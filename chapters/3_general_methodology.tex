\section{Datasets}
The data sets were selected from experiments in published projects in which different study paradigms were used to investigate age-related differences between age groups and groups with different lifestyle backgrounds.

\subsection{Dataset \uproman{1}}
\label{methods:datasets:I}
Dataset \uproman{1} was collected as part of the Bremen Hand Study@Jacobs, which investigated the influence of age and expertise on hand dexterity over the working life \cite{Voelcker-Rehage2013}. The study was approved by the Ethics Committee of the German Psychological Society and adhered to the principles of the Declaration of Helsinki. The following descriptions are adapted from two research papers underlying this dissertation \cite{Gaidai2022, Goelz2021a}.

\subsubsection{Participants}
\label{methods:datasets:I:participants}
The dataset contains recordings from 59 participants, who were recruited in the context of the Bremen Hand Study@Jacobs via flyers, newspaper articles, and phone calls. Prior to inclusion, all individuals gave their informed consent to participate and completed a questionnaire, in which they reported good health, no neurological disorders, and normal or corrected-to-normal hearing and vision. All participants were identified as right-handed using the Edinburgh Handedness Inventory \cite{Oldfield1971} and were paid \euro{8} per hour for compensation.\\
Based on their age and occupation, participants were labeled as young novice (N=xY, age=XY), middle-aged novice (N=xY, age=XY), old novice (N=xY, age=XY), middle-aged expert (N=xY, age=XY) or old expert (N=xY, age=XY). Novices were defined as occupational profiles whose daily routine does not require fine motor control of the hands, such as service personnel, insurance agents, office workers, and students. Experts, on the other hand, referred to persons with more than 10 years of professional experience in a job with pronounced fine motor requirements for hand control such as opticians, goldsmiths, dentists, dental technicians, or hearing aid technicians. This criterion was selected in accordance with \cite{Ericsson1991}. 

\subsubsection{Experimental Procedures}
\label{methods:datasets:I:experiment}
The experiment conducted was a force-tracking experiment conducted blockwise (see \autoref{fig:DSI_exp1}. Participants sat about 80 cm in front of a computer screen (19’’, frame rate 60 Hz). Their arms rested on arm pads. Thumb and index finger grasped a force transducer (model Mini-40, ATI Industrial Automation, Garner, NC, United States) in a pinch grip. Using the right- or left hand, the task was to follow a target force level for five seconds by applying the required amount of force to the force transducer. The target line was either presented as a constant level (steady), i.e., a straight line, corresponding to 2 N, or as a sinusoidal curve (sine) ranging from 2 N to 12 N with a frequency of 1 Hz on the y-axis whereas time was presented on the x-axis. The time axis (x-axis) covered 5 s, allowing participants to see one second of the upcoming target line and 4 s of the preceding target line and the applied force. A total of 160 trials were carried out, each trial lasting 5 s with an individual break of 5 s to 7 s during which participants were instructed to focus on a fixation cross on the screen in front of them. Initially, 80 trials were performed with the right hand. The first 40 trials involved the steady force level and the following 40 trials the sine force level. The sequence was then repeated with the left hand. The participants had an individual break between each task block. Prior to the experiment, the \gls{mvc} was recorded with three maximum precision grip trials. Each grip lasted 5 s with approximately 2 min break in between. The experiments and grip force acquisition were realized using customized LabVIEW (National Instruments, Austin) software.\\
Grip force was recorded with 120 Hz sampling rate and amplitude resolution of 0.06 N via the force transducer.
\gls{eeg} was recorded with 32 active electrodes (ActiveTwo, BioSemi, Amsterdam, Netherlands) placed on the scalp according to the international 10–20 System. Ocular artifacts as well as mastoid potential were recorded with additionally placed electrodes. Common Mode Sense (CMS) and Driven Right Leg (DRL) electrodes were placed next to Cz. All \gls{eeg} signals were recorded with a sampling rate of 2048 Hz applying an online filter between 0.16 and 100 Hz. Before the experiments resting \glspl{eeg} with eyes open and eyes closed were recorded for 30 s each while participants sat comfortably on a chair.

\begin{figure}[h]
\begin{center}
\input{figures/paradigma_1.pdf_tex}
\caption[Schematic presentation of the force-tracking task conducted in dataset \uproman{1}.]{Schematic presentation of the force-tracking task conducted in dataset \uproman{1}. The task was to apply the correct force to a force transducer using the right or left hand to track a target force level (green line) as precisely as possible. Participants received feedback, i.e., they saw their applied force (yellow line).}
\label{fig:DSI_exp1}
\end{center}
\end{figure}

\subsection{Dataset \uproman{2}}
\label{methods:datasets:II}
The \uproman{2} dataset contains recordings from three experimental studies, each focusing on a different age group and referred to below as Study 1, Study 2, and Study 3. The following descriptions are adapted from one of the research papers underlying this dissertation \cite{Goelz2023}.\\
Study 1 is the Bremen-Hand-Study@Jacobs presented above. Study 2 is the Re-LOAD project, which investigated the relationship between motor learning and cognitive function in older adults \cite{HUBNER2018104, Hübner2018}. Study 3 is the CEBRIS project, in which the influence of physical training on the cognitive functions of children was investigated \cite{Koutsandreou2016}. For Study 1 and Study 3 the German Psychological Society and for Study 2 the Ethics Committee of the Faculty of Humanities of the Saarland University, Germany, granted ethical approval. The dataset is described detailed in \cite{Reuter2019}. 

\subsubsection{Participants}
\label{methods:datasets:II:participants}
The full dataset contains recordings from a total of 222 participants including 92 participants recorded in Study 1, 81 participants recorded in Study 2, and 49 participants recorded in Study 3. The data were first analyzed in a comprehensive manner by \citeauthor{Reuter2019} \cite{Reuter2019}. All adult participants gave their written informed consent. For children, guardians gave their written informed consent, and children agreed to participate. Participants older than 65 scored higher than 27 in the, \cite{Folstein1975}) or at least 23 in the \gls{moca} \cite{Julayanont2017, Nasreddine2005}. Participants are separated into the following age categories \cite{Reuter2019}: children (8 to 10 years), young adults (20 to 29 years), early middle-aged adults (36 to 48 years), late middle-aged adults (55 to 64), old adults $<$75 (66 to 75 years), very old adults $>$75 (76 to 83 years). Eight participants were excluded as they had less than 35 correct trials in one of the conditions. Due to poor \gls{eeg} data quality, we further excluded five participants. Group characteristics included in the final dataset are displayed in Table XYZ.

\subsubsection{Experimental Procedures}
\label{methods:datasets:II:experiment}
All participants performed a modified version of the Flanker task previously reported in \cite{Reuter2017, Winneke2012, Winneke2019}, and summarized \cite{Reuter2019} (see \autoref{fig:DSII_exp2}). The stimuli consisted of four circles surrounding a target circle in the middle. The target circle was either set to red or green and the task was to press the corresponding button with the index or middle finger of the right hand as fast as possible. The surrounding (flanking) targets were either set to blue (neutral condition) to the same color as the target (congruent condition) or the opposite color, i.e., green target and red flanker, and vice versa (incongruent condition). The experimental procedures were identical between all studies except for trial number and stimulus duration. In Study 1 and Study 3, participants performed 300 trials (approx. 100 trials per condition), whereas in Study 2, they performed 150 trials (approx. 50 trials per condition) in randomized order. Stimuli were presented for 200 ms in Study 1 and Study 3, whereas in Study 2, stimuli were presented for 500 ms. Each trial started with a white fixation cross (300 ms), next a blank screen (200 ms) was presented followed by the presentation of the stimulus and a variable inter-trial interval of about 950 ms (i.e., 800 ms to 1100 ms). Participants did a minimum of 20 practice trials and were asked to respond as fast and precisely as possible. Only congruent (no inhibitory control) and incongruent (inhibitory control) conditions as well as correct trials, i.e., trials with a correct response between 100 ms and 1200 ms after stimulus onset, were considered in the following analyses.\\
\Gls{eeg} data acquisition was performed with the same system throughout all studies and is already described above (see \autoref{methods:datasets:I:experiment}). 

\begin{figure}[h]
\begin{center}
\input{figures/paradigma_2.pdf_tex}
\caption[Schematic presentation of the flanker task conducted in dataset \uproman{2}.]{Schematic presentation of the flanker task conducted in dataset \uproman{2}.}
\label{fig:DSII_exp2}
\end{center}
\end{figure}

\subsection{Dataset \uproman{3}}
\label{methods:datasets:III}
Dataset \uproman{3} was collected during an intervention study at Paderborn University. The study was registered at the German Clinical Trials Register (DRKS00014921) and approved by the ethics committee of the University of Muenster. The dataset is described in \cite{Goelz2021b} on which the following descriptions are based on.

\subsubsection{Participants}
\label{methods:datasets:III:participants}
The dataset contains recording from 41 participants. Participants were recruited via local newspaper and social media advertisements as well as by personal contact with organizations providing leisure activities for seniors. Participants were included if they (1) were above 60 years old, (2) free of diagnosed neurological or mental diseases and (3) right-handed. Half of the participants participated in a golf training and half continued their normal activities prior to the recording.\footnote{In the context of this thesis golf was considered as part of the daily activities.} All included participants reported subjective memory complaints in daily life but had no diagnosed form of dementia or its preclinical manifestation mild cognitive impairment, and were therefore considered healthy. All participants scored below 13 on a neuropsychological test battery (see Alzheimer’s Disease Assessment Scale-Cognitive Subscale (ADAS-cog)). All participants had normal or corrected to normal vision and hearing.\\

\subsubsection{Experimental Procedures}
\label{methods:datasets:II:experiment}
Participants performed sensory, motor, and cognitive tasks lasting 90 s. Each task was presented 2 times. During the tasks, participants sat 80 cm in front of a screen (23’’, 60 Hz). Their right arm rested comfortably on an armrest grasping a force transducer with index finger and thumb in a precision grip (1022-C3-20 kg, SOEMER, Lennestadt-Elspe, Germany).
Participants placed their left index finger on a braille device (P11, Metec Ag, Stuttgart Germany). In addition, speakers were placed approximately 50 cm behind the participants as well as a footswitch (StealthSwitch SS1R4 Pro USB, StealthSwitch, Highwood, IL, United States) next to their right foot.\\
In preceding appointments, participants’ cardiorespiratory fitness was assessed with a 6 min walking test (Enright 2003). Participants had to walk 6 min as far as possible at a fast and constant pace. The distance was assessed as a marker of cardiorespiratory fitness. Prior to the experiments \gls{mvc} was assesed with three 5 s lasting maximum precision grip trials with 60 s rest in between. See \cite{Goelz2021b} for additional assesments.

\begin{figure}[h]
\begin{center}
\input{figures/paradigma_3.pdf_tex}
\caption[Schematic presentation of the motor, sensory and cognitive tasks conducted in dataset \uproman{3}.]{Schematic presentation of the motor, sensory and cognitive tasks conducted in dataset \uproman{3}.}
\label{fig:DSII_exp3}
\end{center}
\end{figure}

\paragraph{Motor task}
The motor task corresponded to a force tracking experiment as described in \autoref{methods:datasets:I:experiment}. Here, the visual target was a line that moved from the right to the left on the screen for 90 s and changed level every 3 s in randomized order between heights that represented 10\%, 20\% and 30\% of participants individual \gls{mvc} (see Fig. XYZ). The \gls{mvc} was assessed prior to the experiment started and consisted of three 5 s lasting maximum precision grip trials.

\paragraph{Cognitive task}
Participants were asked to listen to a sequence of letters presented via two speakers behind them and press the foot
switch with the right foot if a letter appeared again two letters later (2-back; see Fig. 1c). The sequences consisted of 60 letters in predefined randomized selections with 20\% matching rate (Bopp and Verhaeghen 2018; Gajewski and Falkenstein 2014).

\paragraph{Sensory task}
The braille device presented two stimuli (see Fig. 1a and 1b) to the participants with a randomized inter stimulus
interval of 0.8–1.2 s (mean 1 s) and a duration of 0.5 s each. In this passive oddball design, no response was
required from the participants. Pattern (a) was set to 80\% occurrence and pattern (b) to 20\% (Reuter et al. 2012, 2013, 2014).\\
\\
All \gls{eeg} measurements were recorded with an actiCap electrode cap and BrainAmp standard amplifiers (Brain Products, Munich, Germany). We recorded brain activity at 128 electrodes with a sampling rate of 500 Hz. Ground and reference electrodes were fixed at FPz and FCz, respectively. Impedance was kept below 15 k$\Omega$. Prior to the tasks, \gls{eeg} was recorded for four minutes in a rest condition in a supine position with eyes closed

\subsection{Machine learning procedures} 
We aimed at uncovering brain reorganization by using methods from the field of machine learning. Following state of the art approaches (see \autoref{theory:ml:applications_aging}) we used a combination of classification as well as dimensionality reduction methods. Classification procedures were used both at the individual level as well as at the group level. The former means that one model per subject was trained, which represents the cortical representation of an experimental condition, e.g., a task, and finally allows conclusions, e.g., about the dedifferentiation of cortical processes at the individual level. The latter means that a model was trained for the whole group to detect general overlapping patterns in the group structure. Following state of the art approaches, different dimensionality reduction methods have been applied, which on the one hand allow a suitable representation of the \gls{eeg} signals, i.e. as feature extraction, for the following classification, and on the other hand, allow the visualization of hidden patterns. The selection of a suitable method was based on the data set, i.e. data structure and experimental conditions. For the decoding of the continuous force control experiments, a combination of DMD and Common Spatial Patterns was used, while the decoding of ERP responses was performed using the xDAWN algorithm. The overall analysis approach and the selected methods are summarized in Figure XY. 

In the following, the methods will be briefly described. The exact pipelines are described in the appended research papers. This general approach is represented in Figure \autoref{fig:ML_appreoach}
\begin{figure}
  \dummyfig{Machine learning approach} 
  \caption{Machine learning approach}
  \label{fig:ML_appreoach}
\end{figure}

\subsubsection{Approaches based on dynamic mode decomposition}
To extract a relevant representation of the continuous \gls{eeg} activity, we chose \gls{dmd} because it is able to decompose the signals into spatial activation patterns that are dynamically coherent, reflecting the network nature of the underlying brain activity \cite{Brunton2016}. \Gls{dmd} was developed in the field of fluid mechanics and was applied to various fields to model time-varying dynamical systems including neuroscience proving to extract physiological valid signal patterns \cite{Brunton2016, Kunert-Graf2019}. The activation patterns (modes) extracted from \gls{dmd} analysis can reveal key features such as the spatial distribution of coherent dynamics in relation to oscillation frequencies and mechanisms of growth or decay. This provides a deeper understanding of the functional reorganization of the brain and can serve as a starting point for further analysis \cite{Brunton2016}.\\
For the computation of the DMD, the \textit{exact \gls{dmd}} algorithm introduced by \citeauthor{Tu2014} \cite{Tu2014} and described in \cite{Brunton2016} as applied to electrophysiological data. The analysis was based on the windowed preprocessed \gls{eeg} data considered \gls{dmd} modes associated with the frequency ranges $\theta$ (4 to $<$ 7 Hz), $\alpha$ (7 to $<$ 12 Hz), $\beta_1$ (12 to $<$ 16 Hz), and $\beta_1$ (16 to $<$ 30 Hz). We calculated the \gls{dmd} mode magnitude (absolute value) to obtain the influence of each electrode in a \gls{dmd} mode representing spatially coherent activation \cite{Brunton2016}.

\paragraph{DMD mean modes}
The \gls{dmd} mean mode represents the arithmetic mean over all windows and modes per frequency band, task, and participant. These were used for group classification as well as the statistical description of the input features and as a starting point for knowledge discovery.

\paragraph{DMD and CSP}
To extract the information from the \gls{dmd} modes that would allow the best possible differentiation between the tasks we leveraged supervised dimensionality reduction (see \autoref{theory:ml:applications_eeg}). This approach is based on \gls{fbcsp} a widely used algorithm for the classification of continuous tasks that extracts a weighting for each \gls{eeg} channel that maximizes the class discriminative energy for selected frequency bands. By multiplying these weights with the channel values, meaningful features are generated. These weightings were calculated based on \gls{dmd} magnitudes in each frequency band (see research paper \uproman{1} \cite{Goelz2021a} for details on the implementation).\\
Essentially this algorithm solves the generalized eigenvalue problem 

\subsubsection{xDAWN}
To process the event-related \gls{eeg} data we relied on the xDAWN algorithm. Similar to \gls{csp} by applying the xDAWN algorithm it is possible to obtain a set of weights that emphasize the relevant \gls{eeg} activity while suppressing noise and artifacts, leading to improved signal quality for further analysis or classification tasks. In this way, it is possible to induce spatial patterns, i.e. neural responses to external stimuli at the level of individual trials with non-phase-locked dynamics and is therefore advantageous compared to averaging across trials.\\

\subsection{Classification}
Approaches 
DATA
To approximate the performance of a classification model a dataset is typically divided into a training and testing set. The training set is used for learning a model whereas the testing set is used to estimate the generalization performance to new unseen data, i.e. data which was not used during the process of training. The training data can further be divided into a training and validation portion in order to compare different model types or user defined settings of learning algorithms, so-called hyperparameters. However, this three time division may drastically reduce the data size usable for training and my result in flawed generalization evaluation due to the randomness of the split. Therefore several procedures can be applied. In simple k-fold cross-validation, for example, the training data is divided k-times. Thus each time a different subset of the data is used for validation while the rest is used for training. Usually this is repeated for a range of models and subsequent hyperparameters and the model and hyperparameter performing best on average are selected for final testing. A more advanced method denoted nested cross-validation adds a second k-fold cross-validation loop for the final model evaluation (see Figure \ref{fig1:CV} for a visual representation of the procedures).    

% % \begin{figure*}[h]
% %   \dummyfig{Cross-validation procedures} 
% %   \caption{Cross-validation procedures}
% %   \label{fig1:CV}
% % \end{figure*}
\subsection{Extraction of feature space characteristics}
PCA and UMAP



% Given \gls{eeg} data containing measurements $x_k$ sampled every $\Delta t$ by $n$ electrodes over $m$ time points, two matrices $X$ and $X'$ can be constructed, where $X'$ are the data points of $X$ shifted by one $\Delta t$, i.e,
% \begin{equation}
% X = \begin{bmatrix}
%     | & | &  & |\\
%     x_1 & x_2 & \dots & x_{n-1}\\
%     | & | &  & |
% \end{bmatrix}
% \label{eqn:X}
% \end{equation}
% and
% \begin{equation}
% X' = \begin{bmatrix}
%     | & | &  & |\\
%     x_2 & x_3 & \dots & x_n\\
%     | & | &  & |
% \end{bmatrix},
% \label{eqn:X'}
% \end{equation}
% The relationship between the matrices $X$ and $X'$ can be represented by a linear operator $A$ which characterizes the underlying dynamics in terms of the relationship between $x_k$ and $x_{k+1}$, such that
% \begin{equation}
% \label{eq:dmd_model}
% X' = AX.
% \end{equation}
% The eigendecomposition of $A$ is referred to as \gls{dmd} and yields the spatial patterns, i.e. \gls{dmd} modes, and associated eigenvalues. In this thesis, the \textit{exact \gls{dmd}} algorithm as reported by \cite{Brunton2016} was applied to compute these quantities. Based on this an approximation $\hat{X}$ of the observed measurements $X$ by defining the dynamical model
% \begin{equation}
%     \hat{X}=\Phi exp(\Omega t) z
% \end{equation}
% Every column of $\Phi$ is a DMD mode $\phi_i$ and the matrix $\Omega$, expressed by $\Omega=\log(\lambda)/\Delta t$, reveals the dynamics of the system, where the diagonal matrix $\Lambda$ contains the DMD eigenvalues, i.e., eigenvalues of $A$, on its diagonal. The variable $t$ represents time and $z$ is a constant vector that can be determined from the first time point in each channel, i.e., $x_i = \Phi z$. Each value $\omega_i$ in $\Omega$ corresponds to $\phi_i$ and describes its dynamic behavior in terms of growth, decay, and oscillation. Thus, the oscillation frequency in cycles per second (Hz) of each mode can be determined by $f_i = imag(\omega_i)/2\pi$.

% \subsubsection{Machine learning pipelines}
% Training testing 
% Evaluation

% % To approximate the performance of a classification model a dataset is typically divided into a training and testing set. The training set is used for learning a model whereas the testing set is used to estimate the generalization performance to new unseen data, i.e. data which was not used during the process of training. The training data can further be divided into a training and validation portion in order to compare different model types or user defined settings of a learning algorithms, so called hyperparameters. However, this three time division may drastically reduce the data size usable for training and my result in flawed generalization evaluation due to the randomness of the split. Therefore several procedures can be applied. In a simple k-fold cross-validation, for example, the training data is divided k-times. Thus each time a different subset of the data is used for validation while the rest is used for training. Usually this is repeated for a range of models and subsequent hyperparamters and the model and hyperparameter performing best on average are selected for final testing. A more advanced method denoted nested cross-validation adds a second k-fold cross-validation loop for the final model evaluation (see Figure \ref{fig1:CV} for a visual representation of the procedures).    

% % \begin{figure*}[h]
% %   \dummyfig{Cross-validation procedures} 
% %   \caption{Cross-validation procedures}
% %   \label{fig1:CV}
% % \end{figure*}
