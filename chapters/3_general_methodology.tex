As stated previously diverse datasets containing \gls{eeg} data from individuals spanning different life stages and lifestyle backgrounds were analyzed by using methods from the field of supervised and unsupervised machine learning. Unsupervised learning was used to generate a low-dimensional representation of the EEG that can be used to gain insight and as input to supervised learning. Different age groups and groups with different lifestyle factors were considered, and tested with different paradigms so that both task-related and resting EEG were used in the analyses. Supervised learning consequently took place at the individual level as well as at the group level. The former means that one model was trained per subject, which for each subject individually the cortical representation of the task for each subject individually and finally allows conclusions about e.g. dedifferentiation. The latter means that one model was trained for the whole group to detect general overlapping patterns in the group structure. This approach is visualized in Figure XY

\section{Datasets}
The data sets were selected from experiments in published projects in which different study paradigms were used to investigate age-related differences between age groups and groups with different lifestyle backgrounds.

\subsection{Dataset \uproman{1}}
\label{methods:datasets:I}
Dataset \uproman{1} was collected as part of the Bremen Hand Study@Jacobs, which investigated the influence of age and expertise on hand dexterity over the working life \cite{Voelcker-Rehage2013}. The following descriptions are adapted from two research papers underlying this dissertation \cite{Gaidai2022, Goelz2021a}.

\subsubsection{Participants}
\label{methods:datasets:I:participants}
The dataset considered here contains recordings from 59 participants, who took part in an experiment that assessed fine motor control using force transducers in conjunction with \gls{eeg}. The participants were recruited in the context of the Bremen Hand Study@Jacobs via flyers, newspaper articles, and phone calls. Prior to inclusion, all individuals gave their informed consent to participate and completed a questionnaire, in which they reported good health, no neurological disorders, and normal or corrected-to-normal hearing and vision. All participants were identified as right-handed using the Edinburgh Handedness Inventory \cite{Oldfield1971} and were paid \euro{8} per hour for compensation. The study was approved by the Ethics Committee of the German Psychological Society and adhered to the principles of the Declaration of Helsinki.\\
Based on their age and occupation, participants were labeled as young novice (N=xY, age=XY), middle-aged novice (N=xY, age=XY), old novice (N=xY, age=XY), middle-aged expert (N=xY, age=XY) or old expert (N=xY, age=XY). Novices were defined as occupational profiles whose daily routine does not require fine motor control of the hands, such as service personnel, insurance agents, office workers, and students. Experts, on the other hand, referred to persons with more than 10 years of professional experience in a job with pronounced fine motor requirements for hand control such as opticians, goldsmiths, dentists, dental technicians, or hearing aid technicians. This criterion was selected in accordance with \cite{Ericsson1991}. 

\subsubsection{Experimental Procedures}
\label{methods:datasets:I:experiment}
The visuomotor force-control experiment was designed as a force-tracking experiment conducted clockwise (see \autoref{fig:DSI_exp1}. Participants sat about 80 cm in front of a computer screen (19’’, frame rate 60 Hz). Their arms rested on arm pads. Thumb and index finger grasped a force transducer (model Mini-40, ATI Industrial Automation, Garner, NC, United States) in a pinch grip. Using the right- or left hand, the task was to follow a target force level for five seconds by applying the required amount of force to the force transducer. The target line was either presented as a constant level (steady), i.e., a straight line, corresponding to 2 N, or as a sinusoidal curve (sine) ranging from 2 N to 12 N with a frequency of 1 Hz on the y-axis whereas time was presented on the x-axis. The time axis (x-axis) covered 5 s, allowing participants to see one second of the upcoming target line and 4 s of the preceding target line and the applied force. A total of 160 trials were carried out, each trial lasting 5 s with an individual break of 5 s to 7 s during which participants were instructed to focus on a fixation cross on the screen in front of them. Initially, 80 trials were performed with the right hand. The first 40 trials involved the steady force level and the following 40 trials the sine force level. The sequence was then repeated with the left hand. The participants had an individual break between each task block. Prior to the experiment, the maximum voluntary contraction (MVC) was recorded with three maximum precision grip trials. Each grip lasted 5 s with approximately 2 min break in between. The experiments and grip force acquisition were realized using customized LabVIEW (National Instruments, Austin) software.\\
Grip force was recorded with 120 Hz sampling rate and amplitude resolution of 0.06 N via the force transducer.
EEG was recorded with 32 active electrodes (ActiveTwo, BioSemi, Amsterdam, Netherlands) placed on the scalp according to the international 10–20 System. Ocular artifacts as well as mastoid potential were recorded with additionally placed electrodes. Common Mode Sense (CMS) and Driven Right Leg (DRL) electrodes were placed next to Cz. All EEG signals were recorded with a sampling rate of 2048 Hz applying an online filter between 0.16 and 100 Hz. Prior to the experiments resting \glspl{eeg} with eyes open and eyes closed were recorded for 30 s each while participants sat comfortably on a chair.

\begin{figure}[h]
\begin{center}
\input{figures/paradigma_1.pdf_tex}
\caption[Experimental setup in dataset \uproman{1}.]{Experimental setup in dataset \uproman{1}. Participants sat in front of a computer screen and held a force transducer with the thumb and index finger of the right and left hand, respectively, i.e., in a pinch grip. The task was to apply the correct force to the transducer using the right or left hand to track a target force level (green line) as precisely as possible. Participants received feedback, i.e., they saw their applied force (yellow line).}
\label{fig:DSI_exp1}
\end{center}
\end{figure}

\subsection{Dataset \uproman{2}}
\label{methods:datasets:II}
The \uproman{2} dataset contains recordings from three experimental studies, each focusing on a different age group and referred to below as Study 1, Study 2, and Study 3. Study 1 is the Bremen-Hand-Study@Jacobs presented above. Study 2 is the Re-LOAD project, which investigated the relationship between motor learning and cognitive function in older adults \cite{HUBNER2018104, Hübner2018}. Study 3 is the CEBRIS project, in which the influence of physical training on the cognitive functions of children was investigated \cite{Koutsandreou2016}. The dataset is described detailed in \cite{Reuter2019}. The following descriptions are adapted from one of the research papers underlying this dissertation \cite{Goelz2023}.

\subsubsection{Participants}
\label{methods:datasets:II:participants}
The full dataset includes recordings of 92 participants recorded in Study 1, 81 participants recorded in Study 2, and 49 participants recorded in Study 3. The data were first analyzed in a comprehensive manner by \citeauthor{Reuter2019} \cite{Reuter2019} including all 222 participants. All adult participants gave their written informed consent. For children, guardians gave their written informed consent, and children agreed to participate. For Study 1 and Study 3 the German Psychological Society and for Study 2 the Ethics Committee of the Faculty of Humanities of the Saarland University, Germany, granted ethical approval. Participants older than 65 scored higher than 27 in the Mini-Mental State Examination (MMSE, \cite{Folstein1975}) or at least 23 in the Montreal Cognitive Assessment (MoCA, \cite{Julayanont2017, Nasreddine2005}). Participants are separated into the following age categories \cite{Reuter2019}: children (8 to 10 years), young adults (20 to 29 years), early middle-aged adults (36 to 48 years), late middle-aged adults (55 to 64), old adults $<$75 (66 to 75 years), very old adults $>$75 (76 to 83 years). Eight participants were excluded as they had less than 35 correct trials in one of the conditions. Due to poor EEG data quality, we further excluded five participants. Group characteristics included in the final dataset are displayed in Table XYZ.

\subsubsection{Experimental Procedures}
All participants performed a modified version of the Flanker task previously reported in \cite{Reuter2017, Winneke2012, Winneke2019}, and summarized \cite{Reuter2019} (see \autoref{fig:DSII_exp2}). The stimuli consisted of four circles surrounding a target circle in the middle. The target circle was either set to red or green and the task was to press the corresponding button with the index or middle finger of the right hand as fast as possible. The surrounding (flanking) targets were either set to blue (neutral condition) to the same color as the target (congruent condition) or the opposite color, i.e., green target and red flanker, and vice versa (incongruent condition). The experimental procedures were identical between all studies except for trial number and stimulus duration. In Study 1 and Study 3, participants performed 300 trials (approx. 100 trials per condition), whereas in Study 2, they performed 150 trials (approx. 50 trials per condition) in randomized order. Stimuli were presented for 200 ms in Study 1 and Study 3, whereas in Study 2, stimuli were presented for 500 ms. Each trial started with a white fixation cross (300 ms), next a blank screen (200 ms) was presented followed by the presentation of the stimulus and a variable intertrial interval of about 950 ms (i.e., 800 ms to 1100 ms). Participants did a minimum of 20 practice trials and were asked to respond as fast and precisely as possible. Only congruent (no inhibitory control) and incongruent (inhibitory control) conditions as well as correct trials, i.e., trials with a correct response between 100 ms and 1200 ms after stimulus onset, were considered in the following analyses.\\
\Gls{eeg} data acquisition was performed with the same system throughout all studies and is already described above (see \autoref{methods:datasets:I:experiment}). 

\begin{figure}[h]
\begin{center}
\input{figures/paradigma_2.pdf_tex}
\caption[Experimental setup in dataset \uproman{2}.]{Experimental setup in dataset \uproman{2}. Participants task was to press the corresponding button .}
\label{fig:DSII_exp2}
\end{center}
\end{figure}

% To approximate the performance of a predictive model a dataset is typically divided into a training and testing set. The training set is used for learning a model whereas the testing set is used to estimate the generalization performance to new unseen data, i.e. data which was not used during the process of training. The training data can further be divided into a training and validation portion in order to compare different model types or user defined settings of a learning algorithms, so called hyperparameters. However, this three time division may drastically reduce the data size usable for training and my result in flawed generalization evaluation due to the randomness of the split. Therefore several procedures can be applied. In a simple k-fold cross-validation, for example, the training data is divided k-times. Thus each time a different subset of the data is used for validation while the rest is used for training. Usually this is repeated for a range of models and subsequent hyperparamters and the model and hyperparameter performing best on average are selected for final testing. A more advanced method denoted nested cross-validation adds a second k-fold cross-validation loop for the final model evaluation (see Figure \ref{fig1:CV} for a visual representation of the procedures).    

% \begin{figure*}[h]
%   \dummyfig{Cross-validation procedures} 
%   \caption{Cross-validation procedures}
%   \label{fig1:CV}
% \end{figure*}
