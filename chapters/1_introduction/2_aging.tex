Biologically aging is "the time-dependent functional decline that affects most living organisms" \cite{López-Otín2013}. It can be observed in the reorganization of multiple interacting physiological systems operating at different spatial and temporal scales \cite{Mooney2016}. The underlying patterns of reorganization within and between these systems are highly individual, as they are subject to internal (e.g., genetic, cellular, molecular) as well as external (e.g., environmental, and lifestyle) influences \cite{Smith2020, Mooney2016, Cohen2022}. At the same time, however, overarching, generalizable patterns can be identified \cite{Salthouse2019}. The most recognizable consequences of aging are alterations in cognitive, sensory, and motor abilities that challenge the daily lives of older adults \cite{Li2002}. However, not all abilities are equally affected by declines, and the alterations are highly individual. While sensory, motor, and cognitive abilities, such as memory and processing speed, are generally declining, abilities in the context of acquired knowledge, such as verbal abilities, tend to be stable or even improve with age \cite{Park2009}. One factor that plays a crucial role in these alterations is reorganization at the level of the brain \cite{Reuter-Lorenz2010}. A profound understanding is, therefore, of particular interest to research efforts as this is a prerequisite to identifying unfavorable trajectories and developing prevention and therapy concepts. It is important to note that the reorganization of the brain can be viewed from many perspectives, so in the following, only the aspects and concepts essential for the understanding of this work will be presented. 

\subsection{Age-related Reorganization of the Brain}
\label{theory:aging:brain}
Reorganization in the brain's structure includes, among others, atrophy of the gray and white matter and enlargement of cerebral ventricles \cite{Fjell2010}. The efficiency of neuromodulation declines mainly driven by the loss of dopaminergic receptors indicative of a reorganization of neurotransmitter systems \cite{Li2001}. Besides this, the study of the functional properties of the brain and their relationship to behavioral changes is of great interest. In neuroimaging studies, both under-activation and over-activation of brain areas have been reported in older adults compared to younger adults during the performance in various tasks with sensory, cognitive as well as motor demands \cite{Reuter-Lorenz2010, Sala-Llonch2015}. Regarding activation dynamics, brain activity in response to a stimulus is often slower or delayed. Moreover, the frequency distribution of oscillatory neural activity changes to a slowing of the primary rhythms and altered temporal dynamics, which is interpreted as changes in neural communication \cite{Courtney2021}.\\ 
By emphasizing neural communication and information flow, rather than viewing the brain as functionally separate, it can be conceptualized as a complex system whose functional units, i.e., neurons, areas, and subsystems, are interconnected structurally and functionally \cite{Friston2011,Deery2023}. In this concept, functional connectivity reflects coherent activation patterns within and between these units. Several distinct but interconnected functional networks were identified \cite{Uddin2019}. The dynamic interplay between and within these networks is characterized by segregation and integration at different levels, indicating the flow of information in the brain \cite{Sporns2013}. Older adults' information flow tends to be less efficient and is characterized by lower within-network connectivity and higher between-network connectivity associated with a less segregated, less modular, and more integrated brain network organization \cite{Sala-Llonch2015,Deery2023, Betzel2014}. However, studies on sensorimotor and visual networks seem very heterogeneous, which could indicate individual reorganization patterns \cite{Deery2023}.

\subsubsection{Dedifferentiation}
\label{theory:aging:dedif}
The functional reorganization patterns described in the previous section have been attributed to dedifferentiation \cite{Grady2012}. Dedifferentiation refers to the loss of neural specialization or reduced distinctiveness of neural responses resulting in diffuse, nonspecific recruitment of brain resources \cite{Koen2019}. Historically, the term originates from behavioral research in which an increased correlation of performance between sensory, cognitive, and sensorimotor domains was reported in older adults \cite{Baltes1997,Li2002}. To explain this behavioral dedifferentiation \citeauthor{Li2000}\cite{Li2000, Li2001} provided a computational model. According to this model, deficient neurotransmitter modulation observed in older adults may affect the responsiveness of cortical neurons, leading to higher levels of neuronal noise and ultimately to less differentiated, more diffuse neuronal activation patterns in response to different stimuli \cite{Li2000,Li2001} (see Figure \autoref{fig:dedifferentiation} for an overview on the computational model). In several computational simulations, the authors demonstrated that the proposed model could explain behavioral co-variation and several other phenomena, such as decreased average behavioral performance or increased behavioral intra- and inter-person variability \cite{Li2000,Li2002}. Furthermore, decreased stimulus selectivity of cells, for example, in the primary visual, auditory, or somatosensory cortices with increasing age, has been confirmed in animal models at the single-cell level \cite{Koen2019, Schmolesky2000, Costa2016, Spengler1995}. Neuroimaging studies in humans have also demonstrated that the activation patterns to various visual, cognitive, and motor stimuli are less specific in older compared to young adults \cite{Tucker2019, Koen2019,Carb2011}. Importantly, dedifferentiation was also attributed to functional networks, as considered in the computational model presented in \autoref{fig:dedifferentiation}. As such, the aforementioned reorganization of functional networks, characterized by less segmented and modular and less specialized network organization in older adults, has also been referred to as dedifferentiation \cite{Deery2023, Koen2019, Sala-Llonch2015}. In addition, the activation of functional networks during task performance was also found to be less specific \cite{Rieck2021,Geerligs2014,Antonenko2013}. Additionally, research has indicated that the degree of dedifferentiation might predict behavioral performance in cognitive tasks \cite{Koen2019}. Based on this, it can be assumed that age-related decline is at least partially due to the dedifferentiation of cortical representations \cite{Koen2019}.

\begin{figure}[h]
\def\svgwidth{\columnwidth}
\input{figures/dedifferentiation.pdf_tex}
\caption[The computational model of dedifferentiation.]{The computational model of dedifferentiation. Li and colleagues \cite{Li2001,Li2002} used a feedforward backpropagation neural network model with logistic activation function $f(z)$ and simulated altered neuromodulation by varying the gain parameter $g$ in $f(z)$ of each neuron (A). Lower $g$ values represent deficient neuromodulation and responsiveness due to aging, resulting in a dampened neuron activation (B). Simulations showed that the activation pattern of simulated neurons differs less for different stimuli, i.e., the network's hidden layer shows a less distinctive representation of the stimulus (C). The activation of a single neuron is more variable in networks with lower $g$ value, i.e., older networks, for multiple stimulations with the same stimulus (D). Adapted from \citeauthor{Li2001} \cite{Li2001} with permission.}
\label{fig:dedifferentiation}
\end{figure}

\noindent  \citeauthor{Fornito2015} \cite{Fornito2015} describe dedifferentiation as a fundamental maladaptive mechanism of brain networks that requires compensation. This view is consistent with the argument that dedifferentiation and compensation are complementary mechanisms \cite{Reuter-Lorenz2010}. However, dedifferentiation could also represent a compensatory response, in that the brain attempts to maintain function in the face of deterioration \cite{Stern2009}. By definition, compensation refers to the ability to recruit additional brain resources to compensate for decline and functional loss to maintain cognitive or behavioral functioning \cite{Reuter-Lorenz2010, Grady2012}. Here, the \gls{crunch} hypothesizes that compensatory activity changes as a function of task demands \cite{Festini2018}. Moreover, compensation often occurs in a specific pattern of under-activation of posterior areas and prefrontal over-activation, known as \gls{pasa} \cite{Davis2007}. Another frequently reported pattern is the more bilateral recruitment and loss of hemispheric specialization, known as \gls{harold} \cite{Cabeza2002}.

\subsubsection{Reserve}
\label{theory:aging:reserve}
It is important to note that age-related alterations of the brain and behavior are highly individual and dynamic \cite{Smith2020,Koen2019,Douw2014}. In this context, the reserve hypothesis defines \textit{reserve} as the accumulated capacity of neural resources over the lifespan that can withstand decline or pathology \cite{Cabeza2018, Stern2009}. Although the concept was initially based on observations that the degree of pathological changes in the brain does not necessarily mean clinical manifestation, it has also been applied to explain the individuality of non-pathological aging \cite{Esiri2001,Cabeza2018,Stern2009}.\\
Reserve can be both anatomically quantifiable, referred to as brain reserve, and more functional in nature, referred to as cognitive reserve \cite{Stern2009}. At the functional level, compensatory activation, as well as more efficient utilization (less activation of neural resources), and increased capacity (increased availability of neural resources) were described as key mechanisms of cognitive reserve \cite{Stern2004,Stern2009}. Brain and cognitive reserve influence each other, and \citeauthor{Cabeza2018} \cite{Cabeza2018} argue against a strict separation of brain reserve and cognitive reserve.\
One aspect that explicitly determines the definition of reserve is the lifelong ability of the brain to adapt its structure and function to internal and external requirements. It is known from the animal model that environments rich in cognitive, social, sensory, and motor stimuli contribute to positive plastic changes \cite{Fabel2009}. As a result, reserve is influenced by an interplay between genetic and environmental factors, including lifestyle factors \cite{Cabeza2018}. Essential elements for increasing reserve have been identified in education, occupation as well as physical activity, with cognitive training, physical fitness, and professional expertise having a considerable impact on the brain's functional organization \cite{Vieluf2018,Voss2016,Soldan2021}. Since reserve is not directly measurable, proxies are often used to compare individuals with high or low scores for a particular proxy value (e.g., physical fitness, education, or professional expertise) \cite{Cabeza2018}.\\
Other complementary concepts, such as the maintenance or the \gls{stac} model, highlight these influencing factors additionally. The concept of maintenance emphasizes the ability of the brain to repair. \Gls{stac} postulates that lifelong positive and negative plasticity defines a framework that enables compensation and shapes the individual trajectory of aging \cite{Reuter-Lorenz2014}. 

\subsection{Studying Brain Aging by Electroencephalography}
\label{theory:aging:EEG}
The complex interplay of the factors mentioned above leading to the dynamics of age-related reorganization of the brain is highly complex. Understanding these dynamics regarding individual trajectories and overarching patterns is a prerequisite to differentiating healthy from pathological changes and developing and verifying treatments and targeted interventions. This requires uncomplicated, easy-to-use, and cost-effective methods and novel analyses to quantify changes in brain organization. Several noninvasive methods are available to study the brain's structure and function. \Gls{mri} is the most widely used method in science to image the structure or, using \gls{fmri}, the function of the brain, which is the dominant method in the study of the functional reorganization described in the previous sections \cite{Reuter-Lorenz2010}. However, its use in the public health system is mainly limited to cases with a clear indication, making early detection of unfavorable aging trajectories challenging. In addition, limited availability substantially restricts the development of preventive and rehabilitative interventions and therapies and excludes areas and sites with low levels of equipment and expertise. Here, \gls{eeg} could represent a real added value since it is characterized by simple use, mobility, and relative cost-effectiveness. Although it has a lower spatial resolution than \gls{mri} based methods, \gls{eeg} measures neuronal activity directly with a high temporal resolution which allows for the detection of age-related changes in the temporal dynamics of brain activity and networks, which could be of particular interest to understand age-related changes of the brain and their relation to behavior \cite{Courtney2021}.

\begin{tcolorbox}[breakable, enhanced]
    \subsubsection{Excursus: A Brief Overview on Electroencephalography}
    \Gls{eeg} measures time-varying electrical fields on the surface of the head by using several sensors placed in a standardized position \cite{Jackson2014}. The measured signals reflect synchronously active populations of neurons. Electrical activity can only accumulate and be detected on the surface of the head if spatially similar neurons, aligned perpendicular to the surface, are synchronously activated. Based on the conductive properties of the brain, the signal can travel through the different layers to the surface due to volume and capacitive conduction. For this reason, and due to the orientation of neural cell assemblies, the signal in each sensor reflects a summed signal of different neuron patches. The signal expressions are in the range of a few micro-volts and are much lower than other biological and non-biological electrical generators, e.g., muscular activity or line noise, so the EEG signal is often affected by a low signal-to-noise ratio \cite{CohenX2017}.\\
    One of the EEG's most striking signal characteristics is the rhythmic voltage fluctuations that define the signal and are summarized under the term oscillation. Commonly, the EEG signal is analyzed based on the frequency composition of oscillatory activity in loosely defined frequency ranges, i.e., $\delta$ ($<$4~Hz), $\theta$ ($\sim$4-8~Hz), $\alpha$ ($\sim$8-12~Hz), $\beta$ ($\sim$12-30~Hz) and $\gamma$ ($>$30 Hz), which have been demonstrated to be related to perceptual, cognitive, motor and emotional processes \cite{CohenX2017}. Furthermore, the analysis of frequency-dependent synchrony or functional connectivity in terms of statistical dependence of the signals, e.g., by coherence or the phase synchrony of the signal, can provide information about the network characteristics of the brain \cite{Siegel2012}. Finally, the analysis of event-related activation, so-called \glspl{erp}, can provide information on the direct processing of stimuli. The analysis of \glspl{erp} involves time-locking the EEG data to the onset of a specific stimulus and averaging the EEG signal across hundreds of trials to extract a reliable signal related to the processing of the stimulus.
\end{tcolorbox}

\subsubsection{Electroencephalographic Signatures of Age-related Reorganization}
Age-related \gls{eeg} characteristics have been extensively studied. Specifically, it has been reported that aging is associated with changes in the frequency composition of the EEG signal, regardless of any specific task involvement. These changes include a decrease in amplitude within the $\alpha$ frequency band, a shift in the $\alpha$ peak frequency towards lower frequencies, an increase in amplitude within the $\beta$ frequency band, and varying results regarding changes in the amplitude of the $\theta$ and $\delta$ bands \cite{Rossini2007, Ishii2017, Courtney2021}. Moreover, age-related changes have also been reported in terms of reduced \gls{eeg} synchrony and a more random, less segregated organization of \gls{eeg} derived network topology \cite{Smit2012, Samogin2022}. Altogether, these changes are believed to reflect alterations in brain function and connectivity associated with healthy aging, and variations in these have shown potential utility in diagnosing pathological conditions such as Alzheimer's disease \cite{Babiloni2021}. In contrast, assessing preclinical or mild stages such as mild cognitive impairment poses additional challenges, and researchers have proposed the potential benefits of incorporating task-related \gls{eeg} measures for a more effective evaluation \cite{Froehlich2021, Farina2020}. However, this requires a deep understanding of the changes in task-related information processing and reorganization in healthy aging.\\
Age-related \gls{eeg} alterations in relation to tasks are highly dependent on the task context or domain studied. For example, unilateral motor tasks may display lower frequency specificity and more bilateral spatial expression of $\alpha$ and $\beta$ frequency power modulations \cite{Quandt2016}. In contrast, attention tasks may demonstrate enhanced frontal network involvement and power in the $\theta$ frequency band \cite{Hong2016}. In addition, the neural response to stimuli may exhibit a temporal slowing and altered spatial expression. These alterations can be seen, for example, in a delay of early \gls{erp} components as well as a more frontal expression of later \gls{erp} components in visual attention tasks \cite{Li2013, Reuter2017}.\\
Often these characteristics are discussed in terms of dedifferentiation and compensation described above and have been shown to be modulated by lifetime experience such as occupational expertise \cite{Vieluf2018} or physical fitness \cite{Douw2014}. However, the relationship between \gls{eeg} parameters and these mechanisms is often ambiguous. Other \gls{eeg} findings, for example, point in the opposite direction than described above. \citeauthor{Hübner2018a} \cite{Hübner2018a}, for instance, found no age effects in central lateralization in the $\beta$ frequency band in a complex fine motor control task, which again highlights the dependency on the task context considered. Age-related changes in decreased \gls{erp} latency and lower or increased functional connectivity of the examined networks depending on the task context are also reported \cite{Courtney2021}. Moreover, the interpretation of dedifferentiation is often based on \gls{fmri} findings that report over-activation and loss of segregation of brain networks, although the relationship between frequency-specific \gls{eeg} and \gls{fmri} findings acting on different spatial and temporal scales and measurement principles might be unclear. \citeauthor{Koen2019} \cite{Koen2019} further points out that over-activation should be interpreted cautiously and does not necessarily imply loss of neural specificity, as predicted in the original model of \citeauthor{Li2000} \cite{Li2000,Li2001}. The authors, therefore, propose to operationalize dedifferentiation clearly in terms of the selectivity of the neural response between two or more task modulations. While in this operationalization, the evidence regarding dedifferentiation in \gls{fmri} studies is quite clear, this has not been explored in \gls{eeg} studies so far \cite{Koen2019}.\\
\\
Altogether the \gls{eeg} represents an easy-to-use, low-cost method that can provide valuable insights into age-related changes. However, the link to age-related changes reported consistently in the \gls{fmri}, such as dedifferentiation, is often challenging and needs to be clarified. \Gls{eeg} signals are temporally and spatially highly dimensional, i.e., large amounts of data points contain intricate patterns of electrical activity. Additionally, the signals often have a low signal-to-noise ratio, making it difficult to detect and visualize age-related brain reorganization and its dynamics. As such, analysis of \gls{eeg} signals requires advanced methods. In this context, methods from the field of machine learning could be of particular interest. By leveraging machine learning techniques, it is possible to extract meaningful patterns from the high-dimensional \gls{eeg} data and uncover subtle age-related changes that may not be evident through traditional analysis methods.
