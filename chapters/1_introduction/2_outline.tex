This thesis is separated in six main chapters. In this chapter (\autoref{chap:intro}) the framework of this thesis is described. This includes a theoretical and methodological introduction to machine learning and aging with a focus on neuroscience aspects (\autoref{sec:theory}).  General terminology as well as a literature-based overview of the use of machine learning methods in neuroscience and especially in the neuroscientific research on aging will form the basis for the deduction of research aim and scope of this thesis in the following \autoref{sec:aims_scope}. Furthermore, this shall serve as a background for considerations and the description of the methodological approach of this work (\autoref{sec:methods}).\\
This forms the basis for subsequent chapters 2 to 5, in which the published subprojects underlying this thesis are presented. This includes the following publications.
\begin{itemize}
    \item Chapter 2: Christian Goelz, Karin Mora, Julian Rudisch, Roman Gaidai, Eva-Maria Reuter, Ben Godde, Claus Reinsberger, Claudia Voelcker-Rehage, and Solveig Vieluf. Classification of visuomotor tasks based on electroencephalographic data depends on age-related differences in brain activity patterns. Neural Networks, 142, 05 2021.
    \item Chapter 3: Christian Goelz, Eva-Maria Reuter, Stefanie Fröhlich, Julian Rudisch, Solveig Vieluf, and Claudia Voelcker-Rehage. Using machine learning tocharacterize electrophysiological correlates of selective attention across the lifespan. unpublished, 2022. 
    \item Chapter 4: Christian Goelz, Karin Mora, Julia Stroehlein, Franziska Haase, Michael Dellnitz, Claus Reinsberger, and Solveig Vieluf. Electrophysiological signatures of dedifferentiation differ between fit and less fit older adults. Cognitive Neurodynamics, 15:1–13, 10 2021.
    \item Chapter 5: Roman Gaidai, Christian Goelz, Karin Mora, Julian Rudisch, Eva-Maria Reuter, Ben Godde, Claus Reinsberger, Claudia Voelcker-Rehage, and Solveig Vieluf. Classification characteristics of fine motor experts based on electroencephalographic and force tracking data. Brain Research, 1792:148001, 07 2022.
\end{itemize}
