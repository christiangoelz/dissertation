This thesis is separated in six main chapters. In this chapter (\autoref{chap:intro}) the framework of this thesis is described. This includes a theoretical and methodological introduction to aging and machine learning with a focus on neuroscience aspects (\autoref{sec:theory}). General terminology as well as a literature-based overview of the use of machine learning methods in neuroscience and especially in the neuroscientific research on aging will form the basis for the deduction of the research aim and scope of this thesis in the following \autoref{sec:aims_scope}. Furthermore, this shall serve as a background for considerations and the description of the methodological approach of this work (\autoref{sec:methods}). In the subsequent chapters 2 to 5 the following published sub-projects underlying this thesis will be presented:

\begin{itemize}
\item \autoref{chap:paper1}: \fullcite{Goelz2021a}
\item \autoref{chap:paper2}: \fullcite{Goelz2022}
\item \autoref{chap:paper3}: \fullcite{Goelz2021b}
\item \autoref{chap:paper4}: \fullcite{Gaidai2022}
\end{itemize}

The thesis concludes with an overreaching discussion highlighting consequences and future research topics (\autoref{chap:discussion}).
