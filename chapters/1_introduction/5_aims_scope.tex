The overall goal of my dissertation is to apply previously presente methods from supervised and unsupervised machine learning to \gls{eeg} signals in order to capture the age-related reorganization of the brain both in terms of overreaching patterns as well as individual trajectories. This is done by using data sets that include subjects at different life stages including influencing factors such as professional expertise and physical fitness. In four empirical studies it will be investigated how the reorganization of the brain can be detected based on \gls{eeg} data and how factors contributing to reserve such as professional expertise or physical activity can be characterized using supervised and unsupervised machine learning. In detail the following research questions will be answered.\\
\\
\textit{How can the reorganization of the brain be detected based on \gls{eeg} data using supervised and unsupervised machine learning algorithms?}\\
\\
\textit{How do professional expertise and physical activity affect the reorganization of the brain, and can these factors be characterized based on \gls{eeg} data using machine learning techniques?}\\
\\
\textit{Can machine learning algorithms accurately differentiate between individuals with different levels of professional expertise based on \gls{eeg} data, and what are the key features of the EEG data that contribute to this differentiation?}\\
\\
\textit{How does physical fitness affect the brain's reorganization over time, and can this be detected based on changes in EEG data using machine learning techniques?}\\
\\
The application of supervised machine learning methods both on individual as well as group level, as well as dimensionality reduction methods, will allow to draw conclusions about predictors of reorganization of brain networks and will help to identify the individual status as well overreaching trajectories. The information gained from these tools could be usedto determine and evaluate intervention programs, on-the-job-trainings, support diagnosis, and may have applications in the development on assistive technological systems. 