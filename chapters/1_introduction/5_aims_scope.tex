The overall goal of my dissertation is to apply methods from supervised and unsupervised machine learning to \gls{eeg} signals in order to capture the age-related reorganization of the brain both in terms of overreaching patterns as well as individual trajectories. This is done by using data sets that include subjects at different life stages including influencing factors such as professional expertise and physical fitness.\\
In four subprojects the aim is to learn about how lifestyle and health determinants, such as selecting the profession of a precision mechanic or physical fitness, influence brain activity patterns, networks, and the development of brain and behavior. Thereby, it will be investigated how the reorganization of the brain can be detected based on electroencephalographic (EEG) data and how influencing factors such as professional expertise or physical activity can be characterized using supervised machine learning and dimensionality reduction. This 


AS Well test to what extend hypothesis can be confirmed and new hypothesis can be formed...\\
- Dedifferentiation hypothesis i



Classification of visuomotor tasks based on electroencephalographic data depends on age-related differences in brain activity patterns 
In this project we explore to what extent compensatory and dedifferentiated network characteristics can be detected in subjects of older age, using the motor system as an example


Classification characteristics of attentional networks over the lifespan
Continuing this finding, the goal of an ongoing project is to use machine learning to assess the discriminability of visual attention in different age groups, ranging from children to the very old. Furthermore, by generating a model for age prediction, conclusions will be drawn about the change of brain networks over age groups and the relation to their cognitive status. In this project, the data analyses are currently taking place.

Electrophysiological signatures of dedifferentiation differ between fit and less fit older adults
Based on this, the aim of the following projects will be to characterize influencing factors of brain network reorganization. Thus, the influence of cardiorespiratory fitness on the dedifferentiation of brain networks in elderly subjects will be investigated. In an already published paper, we were able to show a positive effect of cardiorespiratory fitness on the differentiation and dynamics of brain networks [24]. 

Classification characteristics of fine motor experts based on electroencephalographic and force tracking data
Another major influencing factor is professional expertise. Thus, the goal of further analyses is to characterize middle-aged experts with the help of machine learning methods. The associated paper is currently under review and shows no differences in brain network differentiability between experts and middle-aged novices nor differentiability of groups using machine learning. Using dimensionality reduction, however, a high individuality of brain network mechanisms could be demonstrated. 

the differentiation of brain activity over the lifetime perspective will be reflected in changes in the classification performance of different tasks based on \gls{eeg} data. 



The development and application of group classifiers, as well as dimensionality reduction methods, will allow to draw conclusions about predictors of reorganization of brain network organization and treatment of pathological alterations. The information gained from classifications can be used to determine intervention programs, on-the-job-trainings, support diagnosis, or treatment plan development. 