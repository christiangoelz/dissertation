The overall goal of this dissertation is to apply established methods from the field of supervised and unsupervised machine learning to analyze \gls{eeg} signals with the primary objective of capturing the age-related reorganization of the brain both in terms of overreaching patterns as well as individual trajectories. The focus is on studying age-related phenomena, such as dedifferentiation and investigate the replicability of hypotheses such as reserve. In four empirical studies diverse datasets are utilized with subjects spanning different life stages and lifestyle backgrounds, including occupational expertise and physical fitness. The following empirical studies will be presented.
\begin{quote}
\fullcite{Goelz2021a}
\end{quote}
In this empirical study, the focus was on determining whether classification techniques applied to \gls{eeg} data could effectively reflect dedifferentiated brain network characteristics in older people with the the motor system serving as an example. We therefore compared the classification performance, i.e. the discriminability of different visuomotor task, between younger and older adults.

\begin{quote}
\fullcite{Goelz2023}
\end{quote}
Continuing previous research, this study aimed to investigate whether the cortical representation of inhibitory control differs across age groups. For this purpose, the performance of the classification of different stimulus types of a flanker task was compared between different age groups. Furthermore, it was investigated whether the age group membership can be predicted based on the EEG data.

\begin{quote}
\fullcite{Goelz2021b}
\end{quote}
The objective of this study was to examine the potential influence of cardiorespiratory fitness, a lifestyle factor, on patterns of dedifferentiation extracted through dimensionality reduction. This investigation was motivated by the reserve hypothesis, which postulates that cardiorespiratory fitness could impact age related brain reorganization and therefore the observed patterns of dedifferentiation.

\begin{quote}
\fullcite{Gaidai2022}
\end{quote}
In addition to cardiorespiratory fitness, another significant lifestyle factor is professional expertise. Therefore, our subsequent analyses aimed to characterize middle-aged experts using supervised and unsupervised machine learning techniques.\\
\\
The application of machine learning methods both on individual and on group level, will allow to draw conclusions about predictors of reorganization of the brain and will help to identify the individual status as well overreaching trajectories. The information gained from these tools could be used to determine and evaluate intervention programs, on-the-job-trainings, support diagnosis, and may have applications in the development on assistive technological systems. 