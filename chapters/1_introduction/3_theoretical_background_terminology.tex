The following chapters include an introduction to machine learning along with basic formalism and terminology. Next an overview of how machine learning is used in the field of neuroscience will be given by highlighting relevant literature. Subsequently, the changes in the aging brain that are relevant for this thesis will be presented as well as the application of methods used in this field. The goal is to highlight application areas and methodologies that are relevant for the following research.  

\subsection{Machine learning}
\label{subsec:ML}
Machine Learning emerged in the 1950s as a subbranch of \gls{ai} to enable computers to learn without being explicitly programmed \cite{Samual1959}. It is defined by algorithms that automatically extract patterns and trends or \textit{learn} from data \cite{Hastie2009}. The notion of \textit{learning} therein describes the process of acquiring the ability to generalize these trends and patterns to unknown data not used during this process. The goal of machine learning is therefore to extract generalizable patterns based on examples or so-called training data that allow new data to be classified or predictions to be made. The data may contain few or multiple properties, so-called features, and may be multidimensional with variable sources for example sensor recordings or pixel values.\\ 
Besides solving computational problems algorithms from machine learning offer additional value for scientific inquiry including the possibility for the automatic analysis of complex multidimensional data \cite{Brunton2019,Breiman2001}. In this approach, rather than assuming that data is generated by an underlying stochastic model as in classical statistical modelling the mechanisms are treated as unknown which may overcome inaccuracies in the analysis \cite{Breiman2001}. Furthermore, by extracting generalizable principles from the complex interaction of features it offers additional values to traditional hypothesis-driven approaches \cite{Vu1601,Bzdok2017}. \medskip\\
Machine learning can be subdivided into three categories, supervised, unsupervised and reinforcement learning. In supervised machine learning the goal is to learn a function representing the relationship between data and associated information or descriptions, so-called labels or targets.  This function also called model can be thought of a mathematical description of the real world \cite{Brunton2019}, which can than be used to predict the label of new data that not have been used during the process of learning. If the labels are categorical, it is called classification; for continuous labels, the term is regression. The goal of unsupervised machine learning is to find hidden structure in data without taking into account associated labels. This could be grouping similar data points, i.e. clustering, or uncovering a meaningful low dimensional representation of the data, i.e. dimensionality reduction. 
A formal description of supervised and unsupervised learning can be found in \cite{Brunton_kutz_2019} and \cite{Murphy2012}:\\
Given an open bounded set \(\mathcal{D}\) of dimension \(n\) so that
\begin{equation}
\mathcal{D}\subset\mathbb{R}^{n},
\end{equation}
as well as the subset \(\mathcal{D}^{'}\) 
\begin{equation}
    \mathcal{D}^{'}\subset\mathcal{D}^{n}
\end{equation}
the goal is to build a model from data \(\mathcal{D}^{'}\) that can generalize to \(\mathcal{D}\). As written above, in supervised learning this is to learn a mapping function \(f\) from inputs \(x\) (features) to outputs \(y\) (labels), i.e. \(y = f(x)\), based on the subset, or training data
\begin{equation}
    \mathcal{D}^{'}=\{(x_j, y_j), j \in Z := \{1,2,...m\}\}. 
\end{equation}
For classification \(y_j\) is a categorical finite set \(y_j \in \{1,2...C\}\) and for regression a real valued scalar \(y_j \in \mathbb R^{n}\). Each element \(x_j \in \mathbb R^{n}\) is a description of a sample and called feature vector.
In the unsupervised case, however, only inputs are given and 
\begin{equation}
    \mathcal{D}^{'}=\{x_j, j \in Z := \{1,2,...m\}\}. 
\end{equation}
Rather than a specific label in reinforcement learning the goal is to learn optimal actions to solve a certain problem by maximizing the reward linked to that action. However this type of machine learning is less commonly used in scientific enquiry and therefore not further described here.

\subsubsection{Applications in Neuroscience}
Both fields, research on \gls{ai} and Neuroscience are strongly interconnected as information processing in the brain serves as role model for the ultimate goal in \gls{ai} research, creating an artificial general intelligence system \cite{Macpherson2021}. This led, for example, to the development of artificial neural networks, a class of machine learning algorithms that underlie modern advances in the field of \gls{ai} \cite{Cox2014}. Other examples are so called \glspl{cnn}, which are used in computer vision and inspired by the architecture of the visual ventral stream or \glspl{rnn} mimicking the functioning of working memory \cite{Macpherson2021, Fukushima1982, Yin2020}. On the other side, \glspl{cnn} and \glspl{rnn} are used to understand functional and organizational properties of the visual system \cite{Yamins2014} or working memory \cite{Kim2021} which highlights the added value of machine learning for neuroscience.     
General usage of Neural Networks as model of the brain which can be used to test and generate new theories.  

On the other side machine learning is an increasingly powerful tool within Neuroscience. The importance and application of machine learning in neuroscience continues to grow and encompasses a variety of applications and goals including all categories mentioned above, i.e. supervised and unsupervised method. Supervised: Engineering applications, within fMRI research. Unsupervised: Clustering and grouping as well as dimensionality reduction is a big part.  

% In g

% - Clinical: identify disease, develop bio-markers, characterize patients, epilepsy detection/prediction
% - Basic: Understand working principle of processing, e.g. visual system, working memory
% - Cognitive: Identify brain states and study brain behavior interaction

% Summary: 
% - Solving engineering problems as well as understanding brain processing 
% - Investigate high dimensional representations with classification/regression/model selection 
% - Uncover underlying processes with dimensionality reduction


% \subsection{Age related reorganization of the brain}
% \label{subsec:Aging}
% In general aging is an ongoing process that can be detected at multiple interacting biological systems operating on several spatial and temporal scales contributing to the complexity of the phenomenon \cite{Mooney2016}. The most prominent consequences of this are declines in cognitive and sensorimotor abilities challenging the daily life of older adults.
% Age related reorganization processes are detectable at the whole body. This is underpinned by multiple interacting biological systems operating on several spatial and temporal scales contributing to the complexity of the phenomenon \cite{Mooney2016}. At the behavioral level these processes are noticeable in changes in cognitive, motor and sensory functioning [QUELLE]. Aging is one of the biggest risk factors for neurodegenerative diseases such as dementia, including Alzheimer's disease, as well as Parkinson's disease making the brain as one of the target systems to study. Patterns of reorganization of the brain are highly individual as they are subject to genetic and environmental influences [QUELLEN]. At the same time, however, overarching, generalizable patterns can be detected [QUELLE]\\
% On a structural level aging has been associated with a reduction in gray matter with an onset early in life 

\subsubsection{Machine Learning usages in aging neuroscience}
TEXT