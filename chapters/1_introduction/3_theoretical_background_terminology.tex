The following chapters include an introduction to Machine Learning along with basic formalism and terminology. Next an overview of how Machine Learning is used in the field of Neuroscience will be given by highlighting relevant literature. Subsequently, the changes in the aging brain that are relevant for this thesis will be presented as well as the application of Machine Learning methods used in this field. The goal is to highlight application areas and methodologies that are relevant for the following research.  

\subsection{Machine learning}
\label{subsec:ML}
Machine learning emerged in the 1950s as a subbranch of artificial intelligence to enable computers to learn without being explicitly programmed \cite{Samual1959}. It is defined by algorithms that automatically extract patterns and trends or \textit{learn} from data \cite{Hastie2009}. The notion of \textit{learning} therein describes the process of acquiring the ability to generalize these trends and patterns to unknown data not used during the process. The goal of Machine Learning is therefore to extract generalizable patterns based on examples or so-called training data that allow new data to be classified or predictions to be made. The data may contain few or multiple properties, so-called features, and may be multidimensional with variable sources for example sensor recordings or pixel values.\\ 
Besides solving computational problems algorithms from Machine Learning offer additional value for scientific inquiry including the possibility for the automatic analysis of complex multidimensional data \cite{Brunton2019,Breiman2001}. In this approach, rather than assuming that data are generated by an underlying stochastic model as in classical statistical modelling the data mechanisms are treated as unknown which may overcome inaccuracies in the analysis \cite{Breiman2001}. Furthermore, by extracting generalizable principles from the complex interaction of features it offers additional values to traditional hypothesis-driven approaches \cite{Vu1601,Bzdok2017}. \medskip\\
Machine learning can be subdivided into three categories, supervised, unsupervised and reinforcement learning. In supervised machine learning the goal is to learn a function representing the relationship between data and associated information or description, a so-called label or target.  This function also called model can be thought of a mathematical description of the real world \cite{Brunton2019}. This model can than be used to predict the label of new data that not have been used during the process of learning. If the labels are categorical, it is called classification; for continuous labels, the term is regression. The goal of unsupervised machine learning is to find hidden structure in data without taking into account associated labels. This could be grouping similar data points, i.e. clustering, or uncovering a meaningful low dimensional representation of the data, i.e. dimensionality reduction. A mathematical description of supervised and unsupervised learning can be found in \cite{Brunton_kutz_2019} and \cite{Murphy2012}:\\
Given an open bounded set \(\mathcal{D}\) of dimension \(n\) so that
\begin{equation}
\mathcal{D}\subset\mathbb{R}^{n},
\end{equation}
as well as the subset \(\mathcal{D}^{'}\) 
\begin{equation}
    \mathcal{D}^{'}\subset\mathcal{D}^{n}
\end{equation}
the goal is to build a model from data \(\mathcal{D}^{'}\) that can generalize to \(\mathcal{D}\). As written above, in supervised learning this is to learn a mapping from inputs \(x\), i.e. features, to outputs \(y\), i.e. labels based on the subset, or training data
\begin{equation}
    \mathcal{D}^{'}=\{(x_j, y_j), j \in Z := \{1,2,...m\}\}. 
\end{equation}
\(y_j\) can either be a categorical or nominal finite set \(y_j \in \{1,2...C\}\) and \(x_j \in \mathbb R^{n}\) is a feature vector describing each example.\\
In the unsupervised case, however, only inputs are given such as 
\begin{equation}
    \mathcal{D}^{'}=\{x_j, j \in Z := \{1,2,...m\}\}. 
\end{equation}
and the goal is to find hidden structure. 











Rather than a specific label in reinforcement learning the goal is to learn optimal actions to solve a certain problem by maximizing the reward linked to that action.

\subsubsection{Applications on Neuroscience}
Aiming at building an artificial general intelligence machine learning algorithms and technology were inspired by the working principle of the brain leading to advances of neural networks \cite{Macpherson2021} and neuromorphic computing \cite{Choi2022}. As already introduced machine learning on the other side offers great tools for a wide variety of problems addressed areas of neuroscience. 

- Clinical: identify disease, develop biomarkers, characterize patients, epilepsy detection/prediction
- Basic: Understand working principle of processing, e.g. visual system, working memory
- Cognitive: Identify brain states and study brain behavior interaction

Summary: 
- Solving engineering problems as well as understanding brain processing 
- Investigate high dimensional representations with classification/regression/model selection 
- Uncover underlying processes with dimensionality reduction


\subsection{Age related reorganization of the brain}
\label{subsec:Aging}

Age related reorganization processes are detectable at the whole body. This is underpinned by multiple interacting biological systems operating on several spatial and temporal scales contributing to the complexity of the phenomenon \cite{Mooney2016}. At the behavioral level these processes are noticeable in changes in cognitive, motor and sensory functioning [QUELLE]. Aging is one of the biggest risk factors for neurodegenerative diseases such as dementia, including Alzheimer's disease, as well as Parkinson's disease making the brain as one of the target systems to study. Patterns of reorganization of the brain are highly individual as they are subject to genetic and environmental influences [QUELLEN]. At the same time, however, overarching, generalizable patterns can be detected [QUELLE]\\
On a structural level aging has been associated with a reduction in gray matter with an onset early in life 

\subsubsection{Machine Learning usages in aging neuroscience}
