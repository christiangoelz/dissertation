[Hier fehlt noch einleitendes BLABLA]


\subsection{Aging - a definition}
Biologically aging can be broadly characterized as "the time-dependent functional decline that affects most living organisms" \cite{López-Otín2013}. It can be observed in the reorganization of multiple interacting physiological systems operating at different spatial and temporal scales \cite{Mooney2016}. The underlying patterns of reorganization within and between these systems are highly individual, as they are subject to internal (e.g., genetic, cellular, molecular) as well as external (e.g., environmental, and lifestyle) influences \cite{Smith2020, Mooney2016, Cohen2022}. At the same time, however, overarching, generalizable patterns can be identified \cite{Salthouse2019}. The most recognizable consequences of aging are alterations in cognitive, sensory, and motor abilities that challenge the daily lives of older adults \cite{Li2002}. However, not all abilities are equally affected by declines and the alterations are highly individual. While sensory and motor abilities and cognitive abilities, such as memory and processing speed, are described as generally declining, abilities in the context of acquired knowledge, such as verbal abilities, tend to be stable or even improve with age \cite{Park2009}. Understanding brain reorganization is of particular interest because of its interrelation with these alterations. Moreover, aging is one of the major risk factors for neurodegenerative diseases \cite{Hou2019}.

\subsection{Age related reorganization of the brain}
\label{theory:aging:brain}
Reorganization in the structure of the brain include, among others, atrophy of the gray and white matter as well as enlargement of cerebral ventricles \cite{Fjell2010}. The efficiency of neuromodulation declines mainly driven by the loss of dopaminergic receptors indicative of a reorganization of neurotransmittter systems \cite{Li2001}. Besides this, the study of the functional properties of the brain and their relationship to behavioural changes is of great interest. In neuroimaging studies, both under-activation and over-activation of brain areas have been reported in older adults compared to younger adults during performance in various tasks with sensory, cognitive as well as motor demands \cite{Reuter-Lorenz2010, Sala-Llonch2015}. In terms of activation dynamics, brain activity in response to a stimulus is often slower or delayed. Moreover, the frequency distribution of neural oscillatory activity changes with respect to a slowing of the main rhythms and altered temporal dynamics which is interpreted as changes in neural communication \cite{Courtney2021}.\\ 
By emphasising neural communication and information flow, rather than viewing the brain as functionally separate, it can be conceptualised as a complex system whose functional units, i.e. neurons, areas and subsystems, are interconnected both structurally and functionally \cite{Friston2011,Deery2023}. In this concept, functional connectivity reflects coherent patterns of activation within and between these units. Several distinct but interconnected functional networks\footnote{These networks mainly include the sensorimotor, visual, attention, control, salience, and \gls{dmn}, which have been named based on the functional systems they are thought to support \cite{Uddin2019}.} were identified. The dynamic interplay between and within these networks is characterized by segregation and integration at different levels, characterizing the flow of information in the brain \cite{Sporns2013}. Older adults information flow tend be less efficient and is characterized by lower within network connectivity and higher between network connectivity associated with a less segregated, less modular and more integrated brain network organization \cite{Sala-Llonch2015,Deery2023, Betzel2014}. However, studies on sensorimotor and visual networks seem to be very heterogeneous, which could indicate very individual reorganization patterns \cite{Deery2023}.

\subsubsection{Dedifferentiation and compensation}
\label{theory:aging:dedif}
Age related activation patterns and functional network reorganization, can be attributed to dedifferentiation and compensation \cite{Grady2012}. Dedifferentiation refers to the loss of neural specialisation or reduced distinctiveness of neural responses resulting in a diffuse, non specific recruitment of brain resources \cite{Koen2019}. Historically, the term originates from behavioral research in which an increased correlation of performance between sensory and cognitive and sensorimotor domains was reported in older adults was reported \cite{Baltes1997,Li2002}. In order to explain this behavioral dedifferentiation Li and colleagues \cite{Li2001, Li2002} provided a computational model. According to this model, deficient dopaminergic modulation observed in older adults may affect the responsiveness of cortical neurons, leading to higher levels of neuronal noise and ultimately to less differentiated, more diffuse neuronal activation patterns in response to different stimuli \cite{Li2001,Li2002}. In several simulations, the authors demonstrated that the proposed model can explain not only behavioral co-variation, but also several other phenomena, such as the decrease in average performance or the increase in behavioural intra- and inter-person variability \cite{Li2000,Li2002}. In addition, the proposition of a less distinctive, less specific neuronal activation in response to stimuli could be confirmed in neuroimaging studies showing that the neural responses to various visual, cognitive and motor stimuli are less specific in older compared to young adults \cite{Tucker2019, Koen2019,Carb2011}. Recently the reorganization of functional networks as described above, i.e. less segmented and modular, and less specialised organization in older adults was framed in terms of dedifferentiation \cite{Deery2023, Koen2019, Sala-Llonch2015}.\\
\citeauthor{Fornito2015}\cite{Fornito2015} describe dedifferentiation as a fundamental maladaptive mechanism of brain networks that requires compensation. This is consistent with the argument that dedifferentiation and compensation are complementary mechanisms \cite{Reuter-Lorenz2010}. However, dedifferentiation could also itself represent a compensatory response, in that the brain attempts to maintain function in the face of deterioration \cite{Stern2009}. By definition compensation refers to the ability to recruit additional brain resources to compensate for decline and functional loss in order to maintain cognitive and behavioural functioning \cite{Reuter-Lorenz2010, Grady2012}. Here, the \gls{crunch} hypothesizes that compensatory activity changes as a function of task demands. Moreover, compensation often occurs in a specific pattern of under-activation of posterior areas and prefrontal over-activation, known as \gls{pasa} \cite{Davis2007}. Another often reported pattern is the more bilateral recruitment and loss of hemispheric specialisation, known as \gls{harold} \cite{Cabeza2002}.

\subsubsection{Reserve}
\label{theory:aging:reserve}
It is important to note that age related alterations of the brain and behavior are highly individual and dynamic \cite{Smith2020,Koen2019,Douw2014}. In this context, the concept of reserve was defined as the accumulated capacity of neural resources over the lifespan that can withstand decline or pathology \cite{Cabeza2018, Stern2009}. Although the concept was originally based on observations that the degree of pathological changes in the brain do not necessarily mean clinical manifestation, it has also been applied to non-pathological aging \cite{Esiri2001,Cabeza2018,Stern2009}. As such, the concept postulates that inter-individual differences in brain structure and function allow some individuals to cope better with negative brain changes than others \cite{Stern2009}.\\
Reserve can be both anatomically quantifiable, which is referred to as brain reserve, and more functional in nature, which is referred to as cognitive reserve \cite{Stern2009}. At the functional level, compensatory activation as well as more efficient utilisation (less activation of neural resources), increased capacity (increased availability of neural resources) of brain networks were described as key mechanisms of cognitive reserve \cite{Stern2004,Stern2009}. However, brain and cognitive reserve influence each other and \citeauthor{Cabeza2018} \cite{Cabeza2018} argue against a strict separation of brain reserve and cognitive reserve.\\
One aspect that explicitly determines the definition of reserve is the lifelong ability of the brain to adapt its structure and function to internal and external requirements, i.e. lifetime neuroplasticity. Thus, reserve is influenced by an interplay between genetic and environmental including lifestyle factors \cite{Cabeza2018}. Important factors for increasing reserve have been identified in education, occupation as well as physical activity \cite{Cabeza2018,Stern2009}. The importance of these factors is also supported by other complementary concepts such as the maintenance or the \gls{stac}. The concept of maintenance emphasizes the ability to repair. \Gls{stac} postulates that lifelong positive and negative plasticity define a framework that enables compensation and shapes an individual trajectory of aging \cite{Reuter-Lorenz2014}.\\
\\
The \textit{Decade of Healthy Aging} initiative highlights the significance of intrinsic capacity, which encompasses all the mental and physical capabilities of an individual, as a key determinant of healthy aging \cite{who_decade_ha2020}. Reserve at the level of the brain can be understood as an essential component, which is by no means static but dynamic throughout life and represents a basis for preventive interventions and an important component of healthy aging. 

% \subsection{Aging and Electroencephalography}
% An essential part of promoting healthy aging is to build an understanding of age-related reorganization and to have easy-to-use methods which is crucial to identify entry points for preventive interventions. Several noninvasive methods are available to study the brains' structure and function. \Gls{fmri} measures brain activation indirectly via changes in blood oxygenation, is the dominant neuroimaging method in the study of the functional reorganization described in previous section \ref{theory:aging:brain} \cite{Reuter-Lorenz2010}. The high spatial resolution, however, comes at the expense of temporal resolution. \citeauthor{Courtney2021} \cite{Courtney2021} highlight the role of dynamics of neural activity to understand age related changes of the brain and their relation to behavior. \Gls{eeg} measures neuronal activity directly with high temporal resolution which allows for the detection of age-related changes in the temporal dynamics of brain activity and networks \cite{Courtney2021}. Although it has a lower temporal resolution than fMRI, it is characterized by its ease of use, mobility, and relative cost effectiveness. --> EEG als wichtiger Bestandteil da es easy to use und kostengünstig also in allen teilen der Welt eingesetzt werden kann. Allerdings hat es nachteile und auch wenn generlle marker bekannt sind weiß man nicht so richtig wie sich das was oben gestellt wird in EEG markern wiederspiegelt. Eine Möglichkeit dabei zu unterstützen bieten Machine learnig Verfahren.
