Biologically aging can be broadly characterized as "the time-dependent functional decline that affects most living organisms" \cite{López-Otín2013}. It can be observed in the reorganization of multiple interacting physiological systems operating at different spatial and temporal scales \cite{Mooney2016}. The underlying patterns of reorganization within and between these systems are highly individual, as they are subject to internal (e.g., genetic, cellular, molecular) as well as external (e.g., environmental, and lifestyle) influences \cite{Smith2020, Mooney2016, Cohen2022}. At the same time, however, overarching, generalizable patterns can be identified \cite{Salthouse2019}. The most recognizable consequences of aging are alterations in cognitive, sensory, and motor abilities that challenge the daily lives of older adults \cite{Li2002}. However, not all abilities are equally affected by declines and the alterations are highly individual. While sensory and motor abilities and cognitive abilities, such as memory and processing speed, are described as generally declining, abilities in the context of acquired knowledge, such as verbal abilities, tend to be stable or even improve with age \cite{Park2009}. Understanding brain reorganization is of particular interest because of its interrelation with these alterations.

\subsection{Brain aging}
\label{theory:aging:brain}
Reorganization in the structure of the brain include, among others, atrophy of the gray and white matter as well as enlargement of cerebral ventricles \cite{Fjell2010}. The efficiency of neuromodulation declines mainly driven by the loss of dopaminergic receptors indicative of a reorganization of neurotransmittter systems \cite{Li2001}. Besides this, the study of the functional properties of the brain and their relationship to behavioural changes is of great interest. In neuroimaging studies, both under-activation and over-activation of brain areas have been reported in older adults compared to younger adults during performance in various tasks with sensory, cognitive as well as motor demands \cite{Reuter-Lorenz2010, Sala-Llonch2015}. In terms of activation dynamics, brain activity in response to a stimulus is often slower or delayed. Moreover, the frequency distribution of neural oscillatory activity changes with respect to a slowing of the main rhythms and altered temporal dynamics which is interpreted as changes in neural communication \cite{Courtney2021}.\\ 
By emphasising neural communication and information flow, rather than viewing the brain as functionally separate, it can be conceptualised as a complex system whose functional units, i.e. neurons, areas and subsystems, are interconnected both structurally and functionally \cite{Friston2011,Deery2023}. In this concept, functional connectivity reflects coherent patterns of activation within and between these units. Several distinct but interconnected functional networks\footnote{These networks mainly include the sensorimotor, visual, attention, control, salience, and \gls{dmn}, which have been named based on the functional systems they are thought to support \cite{Uddin2019}.} were identified. The dynamic interplay between and within these networks is characterized by segregation and integration at different levels, characterizing the flow of information in the brain \cite{Sporns2013}. Older adults information flow tend be less efficient and is characterized by lower within network connectivity and higher between network connectivity associated with a less segregated, less modular and more integrated brain network organization \cite{Sala-Llonch2015,Deery2023, Betzel2014}. However, studies on sensorimotor and visual networks seem to be very heterogeneous, which could indicate very individual reorganization patterns \cite{Deery2023}.

\subsubsection{Dedifferentiation and compensation}
\label{theory:aging:dedif}
Age related activation patterns and functional network reorganization, can be attributed to dedifferentiation and compensation \cite{Grady2012}. Dedifferentiation refers to the loss of neural specialisation or reduced distinctiveness of neural responses resulting in a diffuse, non specific recruitment of brain resources \cite{Koen2019}. Historically, the term originates from behavioral research in which an increased correlation of performance between sensory and cognitive and sensorimotor domains was reported in older adults was reported \cite{Baltes1997,Li2002}. In order to explain this behavioral dedifferentiation Li and colleagues \cite{Li2001, Li2002} provided a computational model. According to this model, deficient dopaminergic modulation observed in older adults may affect the responsiveness of cortical neurons, leading to higher levels of neuronal noise and ultimately to less differentiated, more diffuse neuronal activation patterns in response to different stimuli \cite{Li2001,Li2002}. In several simulations, the authors demonstrated that the proposed model can explain not only behavioral co-variation, but also several other phenomena, such as the decrease in average behavioural performance or the increase in behavioral intra- and inter-person variability \cite{Li2000,Li2002}. In addition, the proposition of a less distinctive, less specific neuronal activation in response to stimuli could be confirmed in neuroimaging studies showing that the neural responses to various visual, cognitive and motor stimuli are less specific in older compared to young adults \cite{Tucker2019, Koen2019,Carb2011}. Recently the reorganization of functional networks as described above, i.e. less segmented and modular, and less specialised organization in older adults was framed in terms of dedifferentiation \cite{Deery2023, Koen2019, Sala-Llonch2015}.\\
\citeauthor{Fornito2015}\cite{Fornito2015} describe dedifferentiation as a fundamental maladaptive mechanism of brain networks that requires compensation. This is consistent with the argument that dedifferentiation and compensation are complementary mechanisms \cite{Reuter-Lorenz2010}. However, dedifferentiation could also itself represent a compensatory response, in that the brain attempts to maintain function in the face of deterioration \cite{Stern2009}. By definition compensation refers to the ability to recruit additional brain resources to compensate for decline and functional loss in order to maintain cognitive and behavioural functioning \cite{Reuter-Lorenz2010, Grady2012}. Here, the \gls{crunch} hypothesizes that compensatory activity changes as a function of task demands. Moreover, compensation often occurs in a specific pattern of under-activation of posterior areas and prefrontal over-activation, known as \gls{pasa} \cite{Davis2007}. Another often reported pattern is the more bilateral recruitment and loss of hemispheric specialisation, known as \gls{harold} \cite{Cabeza2002}.

\subsubsection{Reserve}
\label{theory:aging:reserve}
It is important to note that age related alterations of the brain and behavior are highly individual and dynamic \cite{Smith2020,Koen2019,Douw2014}. In this context, the concept of reserve was defined as the accumulated capacity of neural resources over the lifespan that can withstand decline or pathology \cite{Cabeza2018, Stern2009}. Although the concept was originally based on observations that the degree of pathological changes in the brain do not necessarily mean clinical manifestation, it has also been applied to non-pathological aging \cite{Esiri2001,Cabeza2018,Stern2009}.\\
Reserve can be both anatomically quantifiable, which is referred to as brain reserve, and more functional in nature, which is referred to as cognitive reserve \cite{Stern2009}. At the functional level, compensatory activation as well as more efficient utilisation (less activation of neural resources), increased capacity (increased availability of neural resources) of brain networks were described as key mechanisms of cognitive reserve \cite{Stern2004,Stern2009}. However, brain and cognitive reserve influence each other and \citeauthor{Cabeza2018} \cite{Cabeza2018} argue against a strict separation of brain reserve and cognitive reserve.\\
One aspect that explicitly determines the definition of reserve is the lifelong ability of the brain to adapt its structure and function to internal and external requirements. Thus, reserve is influenced by an interplay between genetic and environmental factors including lifestyle factors \cite{Cabeza2018}. Important factors for increasing reserve have been identified in education, occupation as well as physical activity \cite{Cabeza2018,Stern2009}. This is also supported by other complementary concepts such as the maintenance or the \gls{stac}. The concept of maintenance emphasizes the ability to repair. \Gls{stac} postulates that lifelong positive and negative plasticity define a framework that enables compensation and shapes the individual trajectory of aging \cite{Reuter-Lorenz2014}.\\        

\subsection{Studying brain aging by electroencephalography}
An essential part of promoting healthy aging is to build an understanding of age-related reorganization, to have easy-to-use methods which is crucial to identify entry points for preventive interventions and allow an early detection of unfavorable conditions. Several noninvasive methods are available to study the brains' structure and function. \Gls{mri} is the most widely used method in science to image the structure or, using \gls{fmri}, the function of the brain, which is the dominant method in the study of the functional reorganization described in the previous section \cite{Reuter-Lorenz2010}. However, this method is very costly and requires expertise that is not widely available. As a result, its use in the public health system is mostly limited to cases with a clear indication, so that early detection of unfavourable aging trajectories is rather difficult. In addition, limited availability substantially restricts the development of preventive and rehabilitative interventions and therapies and excludes areas and sites with low levels of equipment and expertise. Here, \gls{eeg} could represent a real added value, since it is characterized by a simple use, mobility, and relative cost effectiveness.  Although it has a lower spatial resolution than \gls{mri} based methods, \gls{eeg} measures neuronal activity directly with high temporal resolution which allows for the detection of age-related changes in the temporal dynamics of brain activity and networks, which could be of special interest to understand age related changes of the brain and their relation to behavior \cite{Courtney2021}.\\
\\
\Gls{eeg} measures time varying electrical fields on the surface of the head by using several sensors placed in a standardized position. The measured signals reflect synchronously active populations of  neurons. Electrical activity can only accumulate and be detected on the surface of the head if spatially similar neurons, aligned perpendicular to the surface, are synchronously activated. Based on the conductive properties of the brain the signal can travel through the different layers to the surface due to volume conduction. For this reason, and due to the orientation of the neural cell assemblies, the signal in each sensor then reflects a summed signal of different neuron patches. Based on the conductive properties of the different tissue layers, the signal can then travel to the surface due to capacitive and volumetric conduction. For this reason, and due to the orientation of the neuron patches, the signal at each sensor reflects a summed signal over several cell assemblies. The signal expression is in the range of a few microvolts and is much lower than other biological and non-biological electrical generators, e.g. muscular activity or line noise, so that the EEG signal is often affected by a low signal-to-noise ratio.\\
\\
One of the oldest methods of analyzing the EEG is to average the signals generated in response to a brief external event or stimulus over a large number of times and analyze the time course in terms of the timing and amplitude of specific components. In so-called ERP analyses

BLA BLA BLA zu aging. 
\\ 
Darüber hinaus stellt 



Typischerweise kann das EEG in bezug auf ein event, in sogenannten event related potentials, oder auch in Bezug auf die oszillative aktiviät währendd   


Wege wie EEG in der Literatur beschrieben und analysiert wird... 
- ERP
- Oszillationen (delta, theta, alpha, beta, gamma)  

# ERP, Oscillations
- Most frequent reported is global slowing interpreted as changes neurotransmission and conduction velocity: 
    - ERP markers in different tasks: 
        - N2 and P3 (more to tell here?) 
    - oscillatory activity M/EEG both during task and rest: 
        - relative preservation of the frequency distribution 
        - shift in the frequency shape towards lower frequencies, lower alpha peak frequency (higher in AD patients)
        - modulations due to task demands seem to be more different accross brain regions

- Event related responses and oscillatory activity to specific events:
    - visuospatial attention: early markers N1, P1 show increased latency but comparable behaviour: Compensation
    - P3 showed increased latency in attention tasks: fast determination that an event is new and the goal-directed allocation of attention is critically altered during both helaty and pathological aging 
    - alpha and beta oscilatory modulation in older adults is delayed (but not reducing amplitude)

- But earlier latencies are also reported: 
    - alpha modulation earlier as well as earlier P2: Compensatory?

--> Overall slowing of brain activity, oscilatory structure is in total not altered but relative more power is present in lower frequencies and less in higher frequencies, changes in latencies and amplitude of ERP components are specific altered specific to the task at hand, often showing relations to attention and cognitive control(?) processing steps might be altered 

# FC 
- decline in fronto-central PLV in theta
- MCI patients showed reduced alpha and gamma synchrony.
- increased synchrony also reported and can link to compensation or reserve 
--> loss of synchrony as a result of aging, higher synchrony highly variable as linked to compensation and 

- Auserßerdem hochdimensionales Signal mit geringer Signal Rausch Verhältnis 
- ML könnte helfen einen direkten Link zwischen Verhalten und EEG herzustellen sowie neue Einblicke gewähren 
- Klären des Verhältnis zwischen EEG und anderen modalitäten... 
- Theoretical concepts of aging such as dedifferentiation, compensation, and reserve
- Einblicke in die temporale dynamic 
- ML könnte helfen das hochdimensionale Siganl zu analysieren und Informationen zu extrahierne. 
- Auserßerdem einen direkten Link zwischen Verhalten und EEG herzustellen sowie neue Einblicke gewähren 




\citeauthor{Courtney2021} \cite{Courtney2021} highlight the role of dynamics of neural activity to understand age related changes of the brain and their relation to behavior. \Gls{eeg} measures neuronal activity directly with high temporal resolution which allows for the detection of age-related changes in the temporal dynamics of brain activity and networks \cite{Courtney2021}. Although it has a lower spatial resolution than fMRI, it is characterized by its ease of use, mobility, and relative cost effectiveness. \\



- EEG could be of added value in understanding dedifferentiation
- Machine learning could provide a clear link between behavior and EEG and therfore may add value to understanding dedifferentiation 
 Furthermore, \gls{eeg} signals are temporally and spatially highly dimensional and have a low signal-to-noise ratio, which makes the detection and visualization of brain networks and their dynamics difficult and requires advanced signal analysis methods. Advanced methods such as methods from the field of machine learning, as a subbranch, could be of special interest in this context.\\


% --> EEG als wichtiger Bestandteil da es easy to use und kostengünstig also in allen teilen der Welt eingesetzt werden kann. Allerdings hat es nachteile und auch wenn generlle marker bekannt sind weiß man nicht so richtig wie sich das was oben gestellt wird in EEG markern wiederspiegelt. Eine Möglichkeit dabei zu unterstützen bieten Machine learnig Verfahren.
% As mentioned at the beginning, the promotion of healthy ageing is one of the most important consequences of an aging society. To quantify the urgency of this challenge, the \gls{un} forecasts that the proportion of the world's population over 65 will rise from current 10 percent to over 16 percent by 2050, while life expectancy will continue to increase \cite{united2022world}. To advance research and society-wide efforts, the \gls{un} has therefore proclaimed the \textit{Decade for Healthy Aging} (2021-2030). In this, the \gls{who} defines healthy aging in a holistic way as the development and preservation of functional ability reflecting physical and mental capacities as well as environmental characteristics \cite{who_decade_ha2020}. Einer der Eckpunkte ist es dabei   