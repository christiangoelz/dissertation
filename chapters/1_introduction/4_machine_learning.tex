\subsection{Forms of Machine learning}
Machine Learning emerged in the 1950s to enable computers to learn without being explicitly programmed \cite{Samual1959}. It is defined by computational methods combining fundamental concepts from computer science, statistics, probability and optimization that automatically extract patterns and trends, i.e., \textit{learn} from data \cite{Hastie2009}. The notion of \textit{learning} therein describes the automated inference of general rules based on the observation of examples using algorithms with the goal to solve a certain task or problem \cite{Von_luxburg2011}. Often, in its basic form these tasks involve making predictions based on learned relationships or the extraction of information based on automatically detected patterns and structures from data. The three main forms of machine learning are supervised, unsupervised, and reinforcement learning which are defined by the type of feedback a machine learning algorithm has access to during learning \cite{Shalev2014}.\\
\\
In supervised machine learning the goal is to learn a generalizable relationship between data and associated information, so-called labels or target. This can then be used to predict the label of new data that not have been used during the process of learning. If the labels are categorical, the task of learning called classification; for continuous labels, the term is regression.\\
The goal of unsupervised machine learning is to find hidden structure in data without taking into account associated labels. This could be grouping similar data points, i.e. clustering, or uncovering a meaningful low dimensional representation of the data, i.e. dimensionality reduction. This type of learning is also referred to as \textit{knowledge discovery}\cite{Murphy2012}.\\
Reinforcement learning describes the task to to learn optimal actions to solve a certain problem by maximizing the reward linked to that action.\\
\begin{figure*}[h]
  \dummyfig{Categories of ML} 
  \caption{Categories of ML}
  \label{fig1:ml_types}
\end{figure*}
In practice however a clear separation is often not possible. In semi-supervised learning, for example, the goal is the same as for supervised learning. However the data set used to learn the relationship contains both, labeled and unlabeled examples and the hope is to build a stronger representation by providing more information in form of data \cite{Burkov2019}. In the context of this thesis the tasks considered involve the processing of a large amount of information with the goal to learn meaningful patterns and relationships in data. The most commonly used forms are supervised and unsupervised learning. For this reason these forms will be the focus in the following chapters. 

\subsection{Machine learning applied to electroencephalography}
Especially in areas where high-dimensional data is prevalent, such as in neuroscience, the use of supervised as well as unsupervised machine learning offers insight by extracting complex patterns purely data driven \cite{Bzdok2017}. In terms of \gls{eeg} this means that machine learning can help identify subtle patterns and relationships from the multidimensional complex structure of \gls{eeg} data, allowing for more accurate and efficient analysis of brain activity. As such various research approaches use machine learning to extract information from the highly complex data and to relate it to experimental conditions or phenomena or to detect brain states.\\
\\
Traditionally, machine learning has been the core building block for the development of intelligent systems that can automate tasks or enhance and assist humans in performing their tasks. In the medical field the hope is to develop intelligent medical systems to inform clinical theory and support clinical decision making, i.e. assist in diagnosis, risk assessment by predicting health status or forecasting of treatment responses \cite{Woo2017}. In this context supervised learning is frequently use to identify biomarkers from \gls{eeg} by identifying signal characteristics that are predictive of a certain disease or health condition \cite{Babiloni_AlzCons2021,Mei2021}. An exciting application in the context of this work is the estimation of biological age based on regression models trained on the basis on neural data, e.g. \gls{eeg}, recorded in large population studies \cite{Engemann2022}. Using data of an individual person, a model can predict the age of that person. If the brain appears older than it would chronologically, this can be an early indication of an unfavorable state of health \cite{Gonneaud2021}. Another in the context of healthy aging highly relevant application is the development of devices with the goal to assist, augment or enhance humans capabilities such as \glspl{bci}. In \glspl{bci}, neural activity is decoded, using classification to generate control commands for various external devices such as computers or prosthetic limbs \cite{Saha2021, Anumanchipalli2019}.\\
Decoding hereby refers to learning a classification or regression model that is capable of predicting behavioral outcomes or cognitive states characteristics based on neural data. Beyond the application in \glspl{bci}, decoding techniques are widely used in neuroscientific research to gain insights into the neural mechanisms underlying perception, cognition, and behavior. This type of analysis is often referred to as \gls{mvpa} because its goal is to detect multivariate patterns, e.g., a set of voxels in \gls{fmri} or an electrical pattern at a given time in \gls{eeg}, associated with an experimental condition based on goodness of classification or regression \cite{Holdgraf2017}. This can also be applied to study the reorganization of the brain of aging individuals. \Citeauthor{Carb2011}\cite{Carb2011} for example applied machine learning to \gls{fmri} to decode brain activation during a motor task and detected age-related changes such as dedifferentiation of the motor system. Furthermore, the classification of group membership or group level regression can provide information about interesting relationships and their generalizability. Particularly for \gls{eeg} markers representing functional network characteristics, this can reveal insightful findings about the relationship to age-related changes \cite{Petti2016}.\\
Unsupervised methods are often used as a preprocessing step to reduce the complexity of the underlying data or provide a framework for statistical source imaging, but can also provide interesting insights into the structure of data sets. Clustering is used in aging research to identify clinical subgroups or phenotype of healthy aging \cite{Marron2019}. Furthermore, dimensionality reduction algorithms can provide insights into high-dimensional data and, for example, into the temporal structure of \gls{eeg} signals of Alzheimer's patients \cite{Smailovic2019}.\\
In summary, the use of machine learning is very diverse and ranges from engineering applications to scientific knowledge discovery. Especially in the latter case, it offers the advantage of automated extraction of patterns from highly complex data that can contribute to the study of age-related changes. 


% \subsection{Methodological considerations}
% Methods from supervised and unsupervised machine learning are possible candidates. In unsupervised machine learning, the goal is to find structure in the data. This includes methods for dimensionality reduction and clustering. Dimensionality reduction for example allows us to describe the structure of high dimensional data in fewer properties [20]. Two common methods in the analysis of neurophysiological data are the principal component analysis (PCA) and independent component analysis (ICA). These allow the detection of spatial patterns in the data that represent the underlying network characteristics of neurophysiological data [8]. In addition, with dynamic mode decomposition (DMD), Brunton, Johnson, et al. [21] apply for the first time a method to electrophysiological data that allows us to map both the spatial and temporal structure of the network structure of neurophysiological data.
% In supervised machine learning, models are created that can predict a certain outcome based on input data. This method is used to detect neuronal representations of the environment or certain behaviors as well as group memberships and to identify relevant markers [22].
% In the context of lifespan changes, complex brain network behavior based on EEG data could be extracted and visualized using dimensionality reduction. Supervised machine learning methods could be used to detect representations of the environment and behavior and to draw conclusions about the differentiation of brain networks. Automatic detection of group membership could further provide new predictors of nervous system states.

