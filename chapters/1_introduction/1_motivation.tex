\setlength{\epigraphwidth}{0.6\textwidth}
\epigraph{\centering "Humans now live longer than at any time in history. But adding more years to life can be a mixed blessing if it is not accompanied by adding more life to years."} {Dr. Tedros Adhanom Ghebreyesus, WHO Director-General, 2020}

\noindent One of Western society's most significant societal challenges is the demographic shift towards an older population, which poses enormous demands for society, raising issues for the healthcare system, infrastructure, family policy, and the occupational sector \cite{who_aging2023}. To avoid overburdening social structures, one of the main goals is to promote healthy, independent aging and improve the quality of life in old age. As part of efforts to promote these goals, the \gls{who} launched the \textit{Decade of Healthy Aging (2021-2030)}, which aims to encourage global action to improve the lives of older people, their families, and the communities in which they live with the ultimate goal of \textit{adding life to years} \cite{who_aging2023}.\\
An essential part of promoting healthy aging and enabling participation in society includes  the early identification and treatment of pathological conditions, developing and evaluating targeted interventions for prevention and therapy, or designing assistive technologies for older adults. These efforts require a deep understanding of the dynamics of aging in the context of individual trajectories and general patterns. Since many of the mechanisms leading to cognitive and physical decline are related to changes in the brain, it is of great interest to understand and quantify the aging process at this level.\\
Not only is aging a highly complex phenomenon, but also the brain is a complex system that is nonlinear, dynamic, and multi-scale in space and time \cite{Betzel2017}. Machine learning offers valuable data-driven methods to unravel this complexity and gain insights by uncovering complex relationships and identifying predictive markers related to the aging process and associated health status.\\
In general, progress in science is more and more characterized by the application of methods from \gls{ai}, including machine learning algorithms, which make it possible to systematically analyze large and complex amounts of data \cite{Brunton2019}. This development has led to proclamations of an "\gls{ai} revolution in science" \cite{Appenzeller2017} or promoting science has entered a new area characterized by \textit{data-intensive computing} \cite{Hey2009}. Moreover, these methods serve as the foundations for solving various practical problems, as demonstrated by applications in many socially relevant areas, such as public transport, e.g., autonomous or self-driving vehicles \cite{Leonard2020}, the medical sector, e.g., diagnostic imaging \cite{Liu2020}, or social interaction, e.g., tools for communicative interaction \cite{Adamopoulou2020}, and are thus one of the basic building blocks for assistive technology facilitating the participation of older people with disabilities in society. \Gls{ai} and machine learning as a key technology have become a hope for solving societal challenges including the shift towards an older population.\\
However, the implementation of machine learning approaches in aging research is still at an early stage compared to the rapid development in the commercial sector, and the most effective applications and integration into the traditional scientific system have yet to be evaluated, despite the potential to better understand the aging brain.\\
This is the starting point of this work which aims to investigate brain aging using machine learning techniques. The focus is on using these methods to gain a better understanding of the neurophysiological factors that contribute to age-related sensory, motor, and cognitive alterations. To this end, existing hypotheses about the aging brain will be tested and validated while new hypotheses will be generated. The results may inform the development of targeted interventions and assistive technologies to counteract age-related decline.

\section{Outline}
This thesis is separated into five main chapters. This \autoref{chap:intro} describes the thesis's theoretical framework. A description of aging at the level of the brain focuses on the most relevant concepts for the context of this work and forms the starting point for introducing the added value of applying machine learning in the context of studying the reorganization of the aging brain. Next, machine learning is introduced to provide the methodological framework. The general terminology and a literature-based overview of the use of machine learning methods in neuroscience and especially in the neuroscientific research on aging will form the basis for the deduction of the research aim and scope of this thesis in the following \autoref{chap:aims_scope}. The following \autoref{chap:methods} includes a description of the methodological approach of this work. In the subsequent \autoref{chap:results}, the main results of the published research articles underlying this thesis will be presented. These include:

\begin{itemize}
\item \hyperref[paperI]{Research Article \uproman{1}}:\\ \fullcite{Goelz2021a}
\item \hyperref[paperII]{Research Article \uproman{2}}:\\ \fullcite{Goelz2023}
\item \hyperref[paperIII]{Research Article \uproman{3}}:\\ \fullcite{Goelz2021b}
\item \hyperref[paperIV]{Research Article \uproman{4}}:\\ \twocite{Gaidai2022}
\end{itemize}
\noindent The thesis concludes with an overreaching discussion in which the results are evaluated in the light of the current scientific discourse highlighting consequences and future research topics (\autoref{chap:discussion}). 
