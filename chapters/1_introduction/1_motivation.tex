% \setlength{\epigraphwidth}{0.6\textwidth}
% \epigraph{\centering "Artificial intelligence has the power to transform scientific research and discovery by providing new tools and methods for analyzing and interpreting data, leading to unprecedented insights and breakthroughs in various fields."} {ChatGPT 2022 \cite{Chatgpt_openai_web}:\\
% Prompt: "Create a one sentence quote about the impact if artificial intelligence on science"}
 
%Driven by ever-increasing amounts of data and advancing computer infrastructure, the field of \gls{ai} is becoming increasingly influential in social important areas. 
Driven by the development and commercialization of computer systems capable of generating realistic images or communicative interaction as well as their impact on society \cite{lin2023} the field if \gls{ai} has been receiving a lot of attention lately. Applications of this already exist in many socially relevant areas including public transportation, e.g., autonomous or self driving vehicles \cite{Leonard2020}, the medical sector, e.g, diagnostic imaging \cite{Liu2020} or social interaction, e.g. tools for communicative interaction \cite{Adamopoulou2020}. Yet, progress in science is more and more characterized by the application of complex methods from \gls{ai} such as machine learning algorithms, which make it possible to systematically analyze large and complex amounts of data \cite{Brunton2019}. This has led to proclamations of an "\gls{ai} revolution in science" \cite{Appenzeller2017} or promoting science has entered a new area characterized by \textit{data-intensive computing} \cite{Hey2009}. \Gls{ai} and machine learning as a key technology becomes a hope for solving societal challenges.\\
One of the greatest challenges is the demographic shift towards an older population which poses enormous demands for society as a whole raising issues for the healthcare system, infrastructure, family policy, and the occupational sector \cite{who_aging2023}. To avoid overburdening social structures, one of the main goals is to promote healthy, independent aging and improve the quality of life in old age. This includes the development and evaluation of targeted interventions, the design of support systems for older adults, or the early differentiation between age-related diseases and healthy aging. However, an understanding of the dynamics of age-related changes in the context of individual trajectories and general patterns is required. Understanding and quantifying this at the level of the brain is of particular interest, as the brain can be seen as the basis for age-related changes in behavior and cognition. Not only the phenomenon of aging is highly complex but also the brain can be understood as a complex system that is nonlinear, dynamic and multi-scale in space and time \cite{Betzel2017}.\\ 
Modern data-driven methods from the field of machine learning offer a lot to cope with this complexity. In comparison to the rapid development mentioned earlier, the application of machine learning in research outside the mathematical and computer science area is in its infancy and the most effective applications and integration into the traditional science systems remain to be evaluated \cite{Bzdok2019}. Therefore, the aim of this thesis is to identify and apply machine learning techniques to address age-related changes of the brain such as the study of neurophysiological underpinnings and influencing factors of sensory, motor and cognitive changes. Using machine learning the intention is to test to what extent hypotheses about age related changes of the brain can be confirmed, new hypotheses can be formed and derivations for the development of targeted interventions and assistive technologies can be made.

% However, such applications in research and smart technologies are still under development.\\
% Acceptance and targeted use of \gls{ai} based technology requires piloting of application areas in science and practice. 

%%%%%%%%%%%%%%%%%%%%%%%%%%%%%
%%% LOOSE IDEAS FOR LATER %%%
%%%%%%%%%%%%%%%%%%%%%%%%%%%%%
% Besides the early automatic detection of pathological changes, . Other applications include systems to assist older adults in daily living or care assistive technology, such as robotic tools for neurorehabilitation.




% Tools from \gls{ai}, especially machine learning, could be used to describe individual trajectories of aging, helping to identify biomarkers that could be used to verify treatments as well as targeted interventions

% Identfying risk factors, developing plans to support healthy aging 
% The practical acceptance and use of AI requires an evaluation of application areas in science and practice. 
% \begin{itemize}
%  \item Age related changes occur at different scales and are manifestet at several levels.
%  \item There is a wide variety in how this changes occur
%  \item Changes are e.g. neural dedifferentiation and compensatory mechanisms (see Reuter Lorenz et al. 2010) and are noticable brain network level and dynamics
%  \item NOTE: Check what EEG studies said about this...
%  \item The idea is to model these changes with tools from datascience to answer questions in aging neuroscience
%  \item First study is about detecting dedifferentiated and compensatory mechanisms with EEG
%  \item Tools used are DMD and Machine Learning
%  \item Main idea: Study classification performance as proxy for age related changes in different motor control tasks
%  \item Expertise as possible way of building a reserve:
%  \item Higher individuality 
%  \item Dynamics of dedifferentiation and how do they relate to fitness
%  \item Basic for targeted interventions 
%  \item How much and what (relate to Julia)
%  \item Background of ML
%  \item ML as tool 
%  \item novel insights s
%  \item Problem: Data is multidimensional and we have often limited data 
%  \item Solution: Use DMD to reduce Complexity and "model" evolution of signal 
%  \item Dynamic Mode Decompsition
%  \item DMD extracts coupled spatio-temporal modes and is able to kind of model the evolution of the signal 
%  \item Backgrouund + Papers 
%  \item Mathematical Formulation
%  \item What can ML tell us?
%  \item ML applied in aging Neuroscience
%  \item Formulating Aims and goals 
%  \item Formulation expected outcomes
% \end{itemize}