Driven by ever-increasing amounts of data and advancing computer infrastructure, the field of artificial intelligence (AI) is becoming increasingly influential in socially important areas. These include public transportation, e.g., autonomous vehicles \cite{Leonard2020}, the medical sector, e.g, diagnostic imaging \cite{Liu2020} or social interaction, e.g., conversational AI or so-called chatbots \cite{Adamopoulou2020}. Likewise, progress in science is more and more characterized by the application of methods from AI such as algorithms from Machine Learning, which makes it possible to analyze large and complex amounts of data systematically \cite{Brunton2019}. This has led to proclamations of an "AI revolution in science" \cite{Appenzeller2017} or promoting science has entered the fourth paradigm characterized by \textit{data-intensive computing} \cite{Hey2009}.\\
Thus, AI as a key technology becomes a hope for solving societal challenges. One of the greatest challenges in industrialized countries is the demographic shift toward an older population. This poses enormous challenges for society as a whole raising issues for the healthcare system, infrastructure, family policy, and the occupational sector. To avoid overloading these social structures, one of the main goals is the promotion of healthy, independent aging and reduction of the impact on the healthcare system by use of smart technologies for assistance in daily life, rehabilitation or care.\\ 
In general aging is an ongoing process that can be detected at multiple interacting biological systems operating on several spatial and temporal scales contributing to the complexity of the phenomenon \cite{Mooney2016}. The most prominent consequences of this are declines in cognitive and sensorimotor abilities challenging the daily life of older adults. Understanding and describing the neurophysiological underpinnings is crucial to distinguish age-related diseases from healthy aging at an early stage, to develop and plan targeted interventions, or to develop smart technologies adapted to an aging population. Besides the early automatic detection of pathological changes, tools from AI and especially Machine Learning can be used to describe individual trajectories of aging helping to identify biomarkers that could be used to verify treatments as well as targeted interventions. Other applications include systems to assist older adults in daily living or care assistive technology, such as robotic tools for neurorehabilitation. However, such applications and smart technologies are still under development.\\
Acceptance and targeted use of AI based technology requires piloting of application areas in science and practice. In particular, it is unclear to what extent age-related changes can be modelled by algorithms from the field of Machine Learning and, conversely, how smart technologies and age-related changes may interfere. Therefore, the aim of this thesis is to identify and apply Machine Learning techniques to address age-related changes such as the neurophysiological underpinnings and influencing factors of sensorimotor and cognitive decline. The intention is to test to what extent hypotheses can be confirmed, new hypotheses can be formed and derivations for the development of targeted interventions and AI-based smart technology can be made.

%%%%%%%%%%%%%%%%%%%%%%%%%%%%%
%%% LOOSE IDEAS FOR LATER %%%
%%%%%%%%%%%%%%%%%%%%%%%%%%%%%

% Identfying risk factors, developing plans to support healthy aging 
% The practical acceptance and use of AI requires an evaluation of application areas in science and practice. 
% \begin{itemize}
%  \item Age related changes occur at different scales and are manifestet at several levels.
%  \item There is a wide variety in how this changes occur
%  \item Changes are e.g. neural dedifferentiation and compensatory mechanisms (see Reuter Lorenz et al. 2010) and are noticable brain network level and dynamics
%  \item NOTE: Check what EEG studies said about this...
%  \item The idea is to model these changes with tools from datascience to answer questions in aging neuroscience
%  \item First study is about detecting dedifferentiated and compensatory mechanisms with EEG
%  \item Tools used are DMD and Machine Learning
%  \item Main idea: Study classification performance as proxy for age related changes in different motor control tasks
%  \item Expertise as possible way of builing a reserve:
%  \item Higher individuality 
%  \item Dynamics of dedifferentiation and how do they relate to fitness
%  \item Basic for targeted interventions 
%  \item How much and what (relate to Julia)
%  \item Background of ML
%  \item ML as tool 
%  \item novel insights s
%  \item Problem: Data is multidimensional and we have often limited data 
%  \item Solution: Use DMD to reduce Complexity and "model" evolution of signal 
%  \item Dynamic Mode Decompsition
%  \item DMD extracts coupled spatio-temporal modes and is able to kind of model the evolution of the signal 
%  \item Backgrouund + Papers 
%  \item Mathematical Formulation
%  \item What can ML tell us?
%  \item ML applied in aging Neuroscience
%  \item Formulating Aims and goals 
%  \item Formulation expected outcomes
% \end{itemize}