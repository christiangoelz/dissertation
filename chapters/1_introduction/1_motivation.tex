\setlength{\epigraphwidth}{0.6\textwidth}
\epigraph{\centering “Humans now live longer than at any time in history. But adding more years to life can be a mixed blessing if it is not accompanied by adding more life to years,”} {Dr Tedros Adhanom Ghebreyesus, WHO Director-General"}

One of the greatest societal challenges is the demographic shift towards an older population which poses enormous demands for society as a whole raising issues for the healthcare system, infrastructure, family policy, and the occupational sector \cite{who_aging2023}. To avoid overburdening social structures, one of the main goals is to promote healthy, independent aging and improve the quality of life in old age. This includes the development and evaluation of targeted interventions, the design of assistive technologies, for older adults, or the early differentiation between age-related diseases and healthy aging. To accomplish this an understanding of the dynamics of aging in the context of individual trajectories and general patterns is required. Understanding and quantifying this at the level of the brain is of particular interest, as changes in the brain are highly interrelated to declines in behavior and cognition in older age. Not only the phenomenon of aging is highly complex but also the brain can be understood as a complex system that is nonlinear, dynamic and multi-scale in space and time \cite{Betzel2017}.\\ 
Modern data-driven methods from the field of machine learning offer a lot to cope with this complexity. Moreover, these methods form the basis for solving various practical problems, as demonstrated by applications in many socially relevant areas, such as public transport, e.g. autonomous or self-driving vehicles \cite{Leonard2020}, the medical sector, e.g. diagnostic imaging \cite{Liu2020}, or social interaction, e.g. tools for communicative interaction \cite{Adamopoulou2020}, and are thus one of the basic building blocks for assistive technology. Yet, progress in science is more and more characterized by the application of methods from \gls{ai} such as machine learning algorithms, which make it possible to systematically analyze large and complex amounts of data \cite{Brunton2019}. This has led to proclamations of an "\gls{ai} revolution in science" \cite{Appenzeller2017} or promoting science has entered a new area characterized by \textit{data-intensive computing} \cite{Hey2009}. \Gls{ai} and machine learning as a key technology become a hope for solving societal challenges.\\ 
In comparison to the rapid development in the commercial sector, the implementation of machine learning approaches in the study of healthy aging is still in its early stages and the most effective applications and integration into the traditional science system remain to be evaluated \cite{Bzdok2019}.\\
The aim of this thesis is to identify and apply machine learning techniques to address brain aging such as the study of neurophysiological underpinnings and influencing factors of sensory, motor and cognitive alterations. Using machine learning the intention is to test to what extent hypotheses about the aging brain can be confirmed, new hypotheses can be formed and derivations for the development of targeted interventions and assistive technologies can be made.

\section{Outline}
This thesis is separated in six main chapters. In this chapter the theoretical framework of this thesis is described. A description of aging at the level of the brain in chapter (\autoref{sec:theory:aging}) forms the starting point for clarifying the added value of applying machine learning in the context of this work. Building on this, basic concepts of machine learning are introduced to provide the methodological framework in chapter (\autoref{sec:theory:ml}). The general terminology as well as a literature-based overview of the use of machine learning methods in neuroscience and especially in the neuroscientific research on aging will form the basis for the deduction of the research aim and scope of this thesis in the following \autoref{sec:aims_scope}. This shall serve as a background for considerations and the description of the methodological approach of this work (\autoref{sec:methods}). In the subsequent chapters 2 to 5 the following published sub-projects underlying this thesis will be presented:
\begin{itemize}
\item \Chapref{chap:paper1}:\\ \fullcite{Goelz2021a}
\item \Chapref{chap:paper2}:\\ \fullcite{Goelz2022}
\item \Chapref{chap:paper3}:\\ \fullcite{Goelz2021b}
\item \Chapref{chap:paper4}:\\ \fullcite{Gaidai2022}
\end{itemize}
\noindent The thesis concludes with an overreaching discussion highlighting consequences and future research topics (\autoref{chap:discussion}). 

































% However, such applications in research and smart technologies are still under development.\\
% Acceptance and targeted use of \gls{ai} based technology requires piloting of application areas in science and practice. 

%%%%%%%%%%%%%%%%%%%%%%%%%%%%%
%%% LOOSE IDEAS FOR LATER %%%
%%%%%%%%%%%%%%%%%%%%%%%%%%%%%
% Besides the early automatic detection of pathological changes, . Other applications include systems to assist older adults in daily living or care assistive technology, such as robotic tools for neurorehabilitation.

% As mentioned at the beginning, the promotion of healthy ageing is one of the most important consequences
% of an aging society. To quantify the urgency of this challenge, the \gls{un} forecasts that the proportion of the world's population over 65 will rise from current 10 percent to over 16 percent by 2050, while life expectancy will continue to increase \cite{united2022world}. To advance research and society-wide efforts, the \gls{un} has therefore proclaimed the \textit{Decade for Healthy Aging} (2021-2030). In this, the \gls{who} defines healthy aging in a holistic way as the development and preservation of functional ability reflecting physical and mental capacities as well as environmental characteristics \cite{who_decade_ha2020}


% Tools from \gls{ai}, especially machine learning, could be used to describe individual trajectories of aging, helping to identify biomarkers that could be used to verify treatments as well as targeted interventions

% Identfying risk factors, developing plans to support healthy aging 
% The practical acceptance and use of AI requires an evaluation of application areas in science and practice. 
% \begin{itemize}
%  \item Age related changes occur at different scales and are manifestet at several levels.
%  \item There is a wide variety in how this changes occur
%  \item Changes are e.g. neural dedifferentiation and compensatory mechanisms (see Reuter Lorenz et al. 2010) and are noticable brain network level and dynamics
%  \item NOTE: Check what EEG studies said about this...
%  \item The idea is to model these changes with tools from datascience to answer questions in aging neuroscience
%  \item First study is about detecting dedifferentiated and compensatory mechanisms with EEG
%  \item Tools used are DMD and Machine Learning
%  \item Main idea: Study classification performance as proxy for age related changes in different motor control tasks
%  \item Expertise as possible way of building a reserve:
%  \item Higher individuality 
%  \item Dynamics of dedifferentiation and how do they relate to fitness
%  \item Basic for targeted interventions 
%  \item How much and what (relate to Julia)
%  \item Background of ML
%  \item ML as tool 
%  \item novel insights s
%  \item Problem: Data is multidimensional and we have often limited data 
%  \item Solution: Use DMD to reduce Complexity and "model" evolution of signal 
%  \item Dynamic Mode Decompsition
%  \item DMD extracts coupled spatio-temporal modes and is able to kind of model the evolution of the signal 
%  \item Backgrouund + Papers 
%  \item Mathematical Formulation
%  \item What can ML tell us?
%  \item ML applied in aging Neuroscience
%  \item Formulating Aims and goals 
%  \item Formulation expected outcomes
% \end{itemize}