\documentclass[oneside]{book}

%------------------------------------ Packages ------------------------------------
%\usepackage[utf8]{inputenc}
\usepackage{graphicx}
\usepackage{transparent}
\usepackage{import}
\usepackage{color} 
\usepackage{nomencl}
\usepackage{amssymb}
\usepackage{amsmath}
\usepackage[english]{babel}
\usepackage[ddmmyyyy]{datetime}
\usepackage{titlesec}
\usepackage{hyperref}
\usepackage[style=nature,backend=biber,maxbibnames=99]{biblatex}
\usepackage{csquotes}
\usepackage[a4paper,lmargin={3.5cm},rmargin={2.5cm},tmargin={2.5cm},bmargin={2.5cm}]{geometry}
\usepackage{epigraph}
\usepackage[acronym]{glossaries}
% \usepackage[nopostdot,nogroupskip,style=super,nonumberlist,toc,automake,toc]{glossaries}
\usepackage[most]{tcolorbox}
\usepackage[font=small,labelfont=bf]{caption}
\usepackage{titlesec}
\usepackage{tikz}
\usetikzlibrary{trees,fit,shapes}
\usepackage[official]{eurosym}
\usepackage{algorithm}
\usepackage{algpseudocode}
\usepackage{multirow}
\usepackage{booktabs}
\usepackage{array}
\usepackage[flushleft]{threeparttable}
\usepackage{pdfpages}
\usepackage{pdflscape}
\usepackage[T1]{fontenc}

%------------------------------------ Settings ------------------------------------
% CITATION
    % ADD TWO AUTHOR CITE
        \newrobustcmd{\twocite}{\AtNextCite{\defcounter{maxnames}{2}\defcounter{minnames}{2}}\fullcite}

    % ADD BOLD AUTHOR NAME
        \makeatletter
        % auxiliary bibfile
        \def\hlblx@bibfile@name{\jobname -boldnames.bib}
        \newwrite\hlblx@bibfile
        \immediate\openout\hlblx@bibfile=\hlblx@bibfile@name
        
        \newcounter{hlblx@name}
        \setcounter{hlblx@name}{0}
        
        \newcommand*{\hlblx@writenametobib}[1]{%
          \stepcounter{hlblx@name}%
          \edef\hlblx@tmp@nocite{%
            \noexpand\AfterPreamble{%
              \noexpand\setbox0\noexpand\vbox{%
                \noexpand\hlblx@getmethehash{hlblx@name@\the\value{hlblx@name}}}}%
          }%
          \hlblx@tmp@nocite
          \immediate\write\hlblx@bibfile{%
            @misc{hlblx@name@\the\value{hlblx@name}, author = {\unexpanded{#1}}, %
                  options = {dataonly=true},}%
          }%
        }
        
        \AtEndDocument{%
          \closeout\hlblx@bibfile}
        
        \addbibresource{\hlblx@bibfile@name}
        
        \newcommand*{\hlbxl@boldhashes}{}
        \DeclareNameFormat{hlblx@hashextract}{%
          \xifinlist{\thefield{hash}}{\hlbxl@boldhashes}
            {}
            {\listxadd{\hlbxl@boldhashes}{\thefield{fullhash}}}}
        
        \DeclareCiteCommand{\hlblx@getmethehash}
          {}
          {\printnames[hlblx@hashextract][1-999]{author}}
          {}
          {}
        
        % user-level macros
        \newcommand*{\addboldnames}{\forcsvlist\hlblx@writenametobib}
        \newcommand*{\resetboldnames}{\def\hlbxl@boldhashes{}}
        
        
        \newcommand*{\mkboldifhashinlist}[1]{%
          \xifinlist{\thefield{hash}}{\hlbxl@boldhashes}
            {\mkbibbold{#1}}
            {#1}}
        \makeatother
        
        \DeclareNameWrapperFormat{given-family:hash:bold}{%
          \renewcommand*{\mkbibcompletename}{\mkboldifhashinlist}%
          #1}
        
        \DeclareCiteCommand{\fullcitebold}
          {\usebibmacro{prenote}}
          {\usedriver
             {\DeclareNameAlias{sortname}{default}%
              \DeclareNameWrapperAlias{default}{given-family:hash:bold}%
              \defcounter{maxnames}{99}%
              \defcounter{minnames}{99}}
             {\thefield{entrytype}}}
          {\multicitedelim}
          {\usebibmacro{postnote}}
        
        \addboldnames{{Gölz, Christian},{Goelz, Christian}} 
    
% REFERENCES
    \addbibresource{literature.bib} 
    \newcommand{\Chapref}[1]{\hyperref[#1]{Chapter~\ref*{#1}}} % capitalize link to chapter
    \newcommand{\chapref}[1]{\hyperref[#1]{chapter~\ref*{#1}}}

% FIGURES
    \graphicspath{{figures/}}
    \captionsetup[figure]{font=small,justification=raggedright,singlelinecheck=false}
    \newcommand{\dummyfig}[1]{
      \centering
      \fbox{
        \begin{minipage}[c][0.33\textheight][c]{0.5\textwidth}
          \centering{#1}
        \end{minipage}
      }
    }

% SECTION NUMBERS
    \setcounter{tocdepth}{2}
    \setcounter{secnumdepth}{2}
    

% COLOR LINKS
    \hypersetup{colorlinks=true,
    linkcolor=black,
    urlcolor=balck
    }
    
% LINE SPACING
    \renewcommand{\baselinestretch}{1.5}

% FOOTNOTES
    
    \renewcommand{\footnotesize}{\fontsize{10}{12}\selectfont}
    \setlength{\footnotesep}{1.5em}
    \setlength{\skip\footins}{1.5em}
    
    \newcommand{\symbfootnote}[1]{%
    \renewcommand{\thefootnote}{\fnsymbol{footnote}}%
    \footnote{#1}%
    \renewcommand{\thefootnote}{\arabic{footnote}}%
    }


% NUMBERING
    \newcommand{\uproman}[1]{\uppercase\expandafter{\romannumeral#1}}
    \newcommand{\lowroman}[1]{\romannumeral#1\relax}

% !!!!! ADDED DUE TO UNICODE ERROR: TO DO - SEE EFFECT 
\DeclareUnicodeCharacter{2212}{\textendash} 
%------------------------------------ Abbreviations -------------------------------
\makeglossaries

%Here we define a set of example acronyms

%%%%%%%%%%%%%%%%%%%%%%%%%% A %%%%%%%%%%%%%%%%%%%%%%%%%%
\newacronym{ai}{AI}{artificial intelligence}
\newacronym{acc}{Acc.}{accuracy}
\newacronym{auc}{AUC}{area under the receiver operating characteristic curve}
%%%%%%%%%%%%%%%%%%%%%%%%%% B %%%%%%%%%%%%%%%%%%%%%%%%%%
\newacronym{bci}{BCI}{brain computer interface}

%%%%%%%%%%%%%%%%%%%%%%%%%% C %%%%%%%%%%%%%%%%%%%%%%%%%%
\newacronym{cnn}{CNN}{convolutional neural network}
\newacronym{credit}{CRediT}{contributor roles taxonomy}
\newacronym{crunch}{CRUNCH}{compensation-related utilization of neural circuits hypothesis}
\newacronym{csp}{CSP}{common spatial patterns}
%%%%%%%%%%%%%%%%%%%%%%%%%% D %%%%%%%%%%%%%%%%%%%%%%%%%%
\newacronym{dmd}{DMD}{dynamic mode decomposition}
\newacronym{dmn}{DMN}{default mode network}

%%%%%%%%%%%%%%%%%%%%%%%%%% E %%%%%%%%%%%%%%%%%%%%%%%%%%
\newacronym{ecog}{ECoG}{electrocorticography}
\newacronym{eeg}{EEG}{electroencephalography}
\newacronym{erm}{ERM}{empirical risk minimization}
\newacronym{erp}{ERP}{event related potential} 
%%%%%%%%%%%%%%%%%%%%%%%%%% F %%%%%%%%%%%%%%%%%%%%%%%%%%
\newacronym{fmri}{fMRI}{functional magnetic resonance imaging}
\newacronym{fbcsp}{FBCSP}{filter based common spatial patterns}
\newacronym{fp}{FP}{false positive}
\newacronym{fn}{FN}{false negative}
\newacronym{fnirs}{fNIRS}{functional near-infrared spectroscopy}

%%%%%%%%%%%%%%%%%%%%%%%%%% G %%%%%%%%%%%%%%%%%%%%%%%%%%
%%%%%%%%%%%%%%%%%%%%%%%%%% H %%%%%%%%%%%%%%%%%%%%%%%%%%
\newacronym{harold}{HAROLD}{hemispheric asymmetry reduction in older adults}
%%%%%%%%%%%%%%%%%%%%%%%%%% I %%%%%%%%%%%%%%%%%%%%%%%%%%
\newacronym{ica}{ICA}{independent component analysis}
%%%%%%%%%%%%%%%%%%%%%%%%%% J %%%%%%%%%%%%%%%%%%%%%%%%%%
%%%%%%%%%%%%%%%%%%%%%%%%%% K %%%%%%%%%%%%%%%%%%%%%%%%%%
%%%%%%%%%%%%%%%%%%%%%%%%%% L %%%%%%%%%%%%%%%%%%%%%%%%%%
\newacronym{lda}{LDA}{linear discriminant analysis}
%%%%%%%%%%%%%%%%%%%%%%%%%% M %%%%%%%%%%%%%%%%%%%%%%%%%%
\newacronym{meg}{MEG}{magnetoencephalography}
\newacronym{mvpa}{MVPA}{multivariate pattern analysis}
\newacronym{mle}{MLE}{maximum likelihood estimation}
\newacronym{mri}{MRI}{magnet resonance imaging}
\newacronym{moca}{MoCA}{montreal cognitive assessment}
\newacronym{mvc}{MVC}{maximum voluntary contraction}
%%%%%%%%%%%%%%%%%%%%%%%%%% N %%%%%%%%%%%%%%%%%%%%%%%%%%
%%%%%%%%%%%%%%%%%%%%%%%%%% O %%%%%%%%%%%%%%%%%%%%%%%%%%
%%%%%%%%%%%%%%%%%%%%%%%%%% P %%%%%%%%%%%%%%%%%%%%%%%%%%
\newacronym{pasa}{PASA}{posterior–anterior shift in aging}
\newacronym{pca}{PCA}{principal component analysis}
\newacronym{pet}{PET}{positron emission tomography}
%%%%%%%%%%%%%%%%%%%%%%%%%% Q %%%%%%%%%%%%%%%%%%%%%%%%%%
%%%%%%%%%%%%%%%%%%%%%%%%%% R %%%%%%%%%%%%%%%%%%%%%%%%%%
\newacronym{rf}{RF}{random forest}
%%%%%%%%%%%%%%%%%%%%%%%%%% S %%%%%%%%%%%%%%%%%%%%%%%%%%
\newacronym{stac}{STAC}{scaffolding theory of cognitive aging}
\newacronym{svm}{SVM}{support vector machine}
\newacronym{svd}{SVD}{singular value decomposition}
%%%%%%%%%%%%%%%%%%%%%%%%%% T %%%%%%%%%%%%%%%%%%%%%%%%%%
\newacronym{tsne}{t-SNE}{t-distributed stochastic neighbor embedding}
\newacronym{tp}{TP}{true positive}
\newacronym{tn}{TN}{true negative}

%%%%%%%%%%%%%%%%%%%%%%%%%% U %%%%%%%%%%%%%%%%%%%%%%%%%%
\newacronym{umap}{UMAP}{uniform manifold approximation and projection for dimension reduction}
\newacronym{un}{UN}{United Nations}
%%%%%%%%%%%%%%%%%%%%%%%%%% V %%%%%%%%%%%%%%%%%%%%%%%%%%
%%%%%%%%%%%%%%%%%%%%%%%%%% W %%%%%%%%%%%%%%%%%%%%%%%%%%
\newacronym{who}{WHO}{World Health Organization}

%%%%%%%%%%%%%%%%%%%%%%%%%% X %%%%%%%%%%%%%%%%%%%%%%%%%%
%%%%%%%%%%%%%%%%%%%%%%%%%% Y %%%%%%%%%%%%%%%%%%%%%%%%%%
%%%%%%%%%%%%%%%%%%%%%%%%%% Z %%%%%%%%%%%%%%%%%%%%%%%%%%
%------------------------------------ Document ------------------------------------
\begin{document}
%%%%%%%%%%%%%%%%%% Formal stuff %%%%%%%%%%%%%%%%%%
    % Title page
    %---------------------------------- Title Page -----------------------------------
\begin{titlepage}
    \begin{center}
        \vspace*{1cm}
        \LARGE
        \textbf{Decoding the functional reorganization of the aging brain}\\
        \vspace{1.5cm}
        \Large
        By\\
        \vspace{0.5cm}
        Christian Johannes Gölz\\
        \vspace{0.5cm}       
        A thesis presented for the degree of\\
        Doctor rerum naturalium\\
        (Dr. rer. nat)
        \vfill
        \Large
        Paderborn University\\
        Faculty of natural sciences\\
        2022
            
    \end{center}
\end{titlepage}


    
    \pagenumbering{roman}
    \tableofcontents

    % Acknowledgement
    \chapter*{Acknowledgement}
    \addcontentsline{toc}{chapter}{Acknowledgement}
    
    \vspace*{\fill}
\begin{center}
Danke.
\end{center}
\vspace*{\fill}

    % Abstract
    \chapter*{Abstract}
    \setcounter{page}{2}
    \addcontentsline{toc}{chapter}{Abstract}
    This dissertation aimed to extend the understanding of age-related functional brain reorganization by applying machine learning to electroencephalography (EEG) data.\\
Based on EEG data derived during sensory, cognitive, or motor tasks, classification algorithms were trained to predict the task performed during recording and the individual's age and lifestyle group. Dimensionality reduction techniques were used to extract EEG patterns that reveal the relationship between age-related brain reorganization and lifestyle factors.\\
The performance of the classifiers revealed task-specific signatures of dedifferentiation, i.e., loss of specialization of neural systems. In addition, the results provided evidence for cognitive changes at specific life stages, e.g., after retirement, and the influence of lifestyle factors. High cardiorespiratory fitness was associated with less dedifferentiation, and occupational expertise led to greater individualization of functional brain activity patterns.\\
Machine learning enabled the identification and quantification of age-related brain reorganization. This supported existing findings and, in addition, generated new hypotheses. These results could contribute to developing diagnostic tools, therapies, or assistive technologies.
    
    % List of figures
    \listoffigures
    \addcontentsline{toc}{chapter}{List of Figures}
    
    % List of tables
    \listoftables
    \addcontentsline{toc}{chapter}{List of Tables}

    % List of acronyms
    \printglossary[nonumberlist, type=\acronymtype, title={List of Abbreviations}]
    \addcontentsline{toc}{chapter}{List of Abbreviations}
    
    % Contributions
    \chapter*{List of scientific contributions}
    \addcontentsline{toc}{chapter}{List of scientific contributions}
    \subsection*{Peer-reviewed journal articles}
Gaidai, R., \textbf{Goelz, C.}, Mora, K., Rudisch, J., Reuter, E.-M., Godde, B., Reinsberger, C., Voelcker-Rehage, C. \& Vieluf, S. Classification characteristics of fine motor experts based on electroencephalographic and force tracking data. \textit{Brain Research} \textbf{1792}, 148001 (2022)\symbfootnote{Shared first authorship between R. Gaidai and C. Goelz}\setcounter{footnote}{0}\footnote{Publication considered for thesis}\\
\\
\textbf{Goelz, C.}, Mora, K., Rudisch, J., Gaidai, R., Reuter, E.-M., Godde, B., Reinsberger, C., Voelcker-Rehage, C. \& Vieluf, S. Classification of visuomotor tasks based on electroencephalographic data depends on age-related differences in brain activity patterns. \textit{Neural Networks} \textbf{142}, 363--374 (2021)\footnotemark[1]\\
\\
\textbf{Goelz, C.}, Mora, K., Stroehlein, J., Haase, F., Dellnitz, M., Reinsberger, C. \& Vieluf, S. Electrophysiological signatures of dedifferentiation differ between fit and less fit older adults. \textit{Cognitive Neurodynamics} \textbf{15}, 1--13 (2021)\footnotemark[1]\\
\\
\textbf{Goelz, C.}, Reuter, E.-M., Fröhlich, S., Rudisch, J., Godde, B., Vieluf, S. \& Voelcker-Rehage, C. Classification of age groups and task conditions provides additional evidence for differences in electrophysiological correlates of inhibitory control across the lifespan. \textit{Brain Informatics} \textbf{10}, 11 (2023)\footnotemark[1]\\
\\
Gowik, J. K., \textbf{Goelz, C.}, Vieluf, S., van den Bongard, F. \& Reinsberger, C. Source connectivity patterns in the default mode network differ between elderly golf-novices and non-golfers. \textit{Scientific Reports} \textbf{13}, 6215 (2023)\\
\newpage
\noindent\textbf{Gölz, C.}, Voelcker-Rehage, C., Mora, K., Reuter, E.-M., Godde, B., Dellnitz, M., Reinsberger, C. \& Vieluf, S. Improved Neural Control of Movements Manifests in Expertise-Related Differences in Force Output and Brain Network Dynamics. \textit{Frontiers in Physiology} \textbf{9} (2018)\\
\\
Niess, A., Widmann, M., Gaidai, R., \textbf{Gölz, C.}, Schubert, I., Castillo, K., Sachs, J. P., Bizjak, D., Vollrath, S., Wimbauer, F., Vogel, A., Keller, K., Burgstahler, C., Quermann, A., Kerling, A., Schneider, G., Zacher, J., Diebold, K., Grummt, M., Beckendorf, C., Buitenhuis, J., Egger, F., Venhorst, a., Morath, O., Barsch, F., Mellwig, K.-P., Oesterschlink, J., Wüstenfeld, J., Predel, H.-G., Deibert, P., Friedmann-Bette, B., Mayer, F., Hirschmüller, A., Halle, M., Steinacker, J. M., Wolfarth, B., Meyer, T., Böttinger, E., Flechtner-Mors, M., Bloch, W., Haller, B., Roecker, K. \& Reinsberger, C. COVID-19 in German Competitive Sports: Protocol for a Prospective Multicenter Cohort Study (CoSmo-S). \textit{International Journal of Public Health} \textbf{67}, 1604414 (2022)\\
\\
Stroehlein, J. K., Vieluf, S., Zimmer, P., Schenk, A., Oberste, M., \textbf{Goelz, C.}, van den Bongard, F. \& Reinsberger, C. Learning to play golf for elderly people with subjective memory complaints: feasibility of a single-blinded randomized pilot trial. BMC Neurology \textbf{21}, 200 (2021)\\
\\
Strote, C., \textbf{Gölz, C.}, Stroehlein, J. K., Haase, F. K., Koester, D., Reinsberger, C. \& Vieluf, S. Effects of force level and task difficulty on force control performance in elderly people. \textit{Experimental Brain Research} \textbf{238}, 2179--2188 (2020)\\
\\
Ströhlein, J. K., Vieluf, S., van den Bongard, Franziska, \textbf{Gölz, C.} \& Reinsberger, C. Golf spielen gegen die Vergesslichkeit: Effekte des Erlernens der Sportart auf das Default Mode Netzwerk des Gehirns. \textit{B\&G Bewegungstherapie und Gesundheitssport} \textbf{36}, 65--72 (2020)\\
\\
Vieluf, S., Mora, K., \textbf{Gölz, C.}, Reuter, E.-M., Godde, B., Dellnitz, M., Reinsberger, C. \& Voelcker-Rehage, C. Age-and expertise-related differences of sensorimotor network dynamics during force control. \textit{Neuroscience} \textbf{388}, 203--213 (2018)\\

\subsection*{Contributions to Conferences}
\subsubsection*{Talks\footnote{First author was always the presenting author}}
Gaidai, R., \textbf{Gölz, C.}, Widmann, M., Niess, A. M. \& Reinsberger, C. \textit{Herausforderungen und Lösungen im Aufbau eines Datenbanksystems im Verbundprojekt „Covid-19 im Spitzensport - Eine multizentrische Kohortenstudie“ (CoSmo-S).} Sports Medicine and Health Summit. Hamburg, Germany, June 2023\\
\\
\textbf{Goelz, C.}, Gaidai, R., Mora, K., Rudisch, J., Reuter, E. M., Godde, B., Reinsberger, C., Voelcker-Rehage, C. \& Vieluf, S. \textit{Classification characteristics of fine motor experts based on electroencephalographic and force tracking data}. 54th Annual Conference of the German Society for Sport Psychology. Muenster, Germany, June 2022\\
\\
Ströhlein, J., \textbf{Gölz, C.}, Vieluf, S., van den Bongard, F. \& Reinsberger, C.\textit{ Funktionelle Netzwerkcharakteristika des Default Mode Netzwerks bei älteren Golf-Novizen.} Sports Medicine and Health Summit. Virtual Conference, Apr. 2021\\
\\
Vieluf, S., Reinsberger, C., \textbf{Gölz, C.} \& Voelcker-Rehage, C. \textit{Age- and expertise related differences of sensorimotor network dynamics during force control.} 23rd University Day of the German Society of Sport Science. Munich, Germany, Sept. 2017\\

\subsubsection*{Poster\footnotemark[2]}
Gaidai, R., \textbf{Gölz, C.}, Widmann, M., Niess, A. M. \& Reinsberger, C. \textit{Herausforderungen und Lösungen im Aufbau eines Datenbanksystems im Verbundprojekt „Covid-19 im Spitzensport - Eine multizentrische Kohortenstudie“ (CoSmo-S). Nachwuchsssymposium Deutsche Gesellschaft für Sportmedizin und Prävention.} Tuebingen, Germany, Sept. 2022\\
\\
\textbf{Goelz, C.}, Mora, K., Stroehlein, J., Reinsberger, C. \& Vieluf, S. \textit{Electrophysiological signatures of brain network dynamics in elderly.} 26th Annual Meeting of the Organization of Human Brain Mapping. Virtual Conference, June 2020\\
\\
Vieluf, S., Voelcker-Rehage, C., \textbf{Goelz, C.}, Mora, K. \& Reinsberger, C. \textit{Age- and expertise-related differences in force control manifest in variability.} 27th Annual Meeting of the Society of the Neural Control of Movements. Dublin, Ireland, May 2017


%%%%%%%%%%%%%%%%%% Content %%%%%%%%%%%%%%%%%%

    % Introduction
    \chapter{General Introduction}\label{chap:intro}
    \pagenumbering{arabic}
    
        \section{Motivation}
        \label{motivation}
        % \setlength{\epigraphwidth}{0.6\textwidth}
% \epigraph{\centering "Artificial intelligence has the power to transform scientific research and discovery by providing new tools and methods for analyzing and interpreting data, leading to unprecedented insights and breakthroughs in various fields."} {ChatGPT 2022 \cite{Chatgpt_openai_web}:\\
% Prompt: "Create a one sentence quote about the impact if artificial intelligence on science"}
 
%Driven by ever-increasing amounts of data and advancing computer infrastructure, the field of \gls{ai} is becoming increasingly influential in social important areas. 
\Gls{ai} is a field of computer science aiming at building computer systems capable of doing things that require human intelligence. The field has been receiving a lot of attention lately. This is driven by the development and commercialization of generative systems capable of generating realistic images or communicative interaction as well as their impact on society \cite{lin2023}. Other applications already exist in many socially relevant areas including public transportation, e.g., autonomous or self driving vehicles \cite{Leonard2020}, the medical sector, e.g, diagnostic imaging \cite{Liu2020} or social interaction \cite{Adamopoulou2020}, e.g. tools for communicative interaction \cite{Chatgpt_openai_web}. Yet, progress in science is more and more characterized by the application of complex methods from \gls{ai} such as algorithms from machine learning, which make it possible to analyze large and complex amounts of data systematically \cite{Brunton2019}. This has led to proclamations of an "\gls{ai} revolution in science" \cite{Appenzeller2017} or promoting science has entered the fourth paradigm characterized by \textit{data-intensive computing} \cite{Hey2009}. However, in comparison to the rapid development in commercial aspects, the application of machine learning in research outside the mathematical and computer science area is in its infancy and the most effective applications and integration into the traditional science systems remain to be evaluated \cite{Bzdok2019}.  Nevertheless, \gls{ai} and machine learning as a key technology becomes a hope for solving societal challenges.\\
One of the greatest challenges is the demographic shift towards an older population which poses enormous demands for society as a whole raising issues for the healthcare system, infrastructure, family policy, and the occupational sector \cite{who_aging2023}. To avoid overloading social structures, one of the main goals is the promotion of healthy, independent aging improving quality of live in old age. This is not only due to distinguishing age-related diseases from healthy aging at an early stage but as well to develop and evaluate targeted intervention or design adaptive systems. To achieve this goal, it is necessary to understand the dynamics of age-related changes in the context of individual trajectories and overall patterns. An understanding of these changes at the level of the brain is of particular interest, as this can be considered the basis of age-related changes in behavior and cognition \cite{}. To avoid overburdening social structures, one of the main goals is to promote healthy, independent aging and improve the quality of life in old age. This involves not only distinguishing age-related diseases from healthy aging at an early stage, but also developing and evaluating targeted interventions or designing assistive systems. To achieve this, it is necessary to understand the dynamics of age-related changes in the context of individual trajectories and overall patterns. Understanding and quantifying these changes at the level of the brain is of particular interest, as this can be seen as the basis for age-related changes in behavior and cognition. 




Ziel ist es diese zu erproben inwieweit diese Dynamik auf verschiedenen Ebenen mithilfe maschineller Lernverfahren abgebildet werden kann und was daraus für die Praxis im Hinblick auf die Entwickling adaptiver Systeme sowie 

s. 




% However, such applications in research and smart technologies are still under development.\\
% Acceptance and targeted use of \gls{ai} based technology requires piloting of application areas in science and practice. Therefore, the aim of this thesis is to identify and apply machine learning techniques to address age-related changes of the brain such as the study of neurophysiological underpinnings and influencing factors of sensorimotor and cognitive changes. Using machine learning the intention is to test to what extent hypotheses about age related changes of the brain can be confirmed, new hypotheses can be formed and derivations for the development of targeted interventions and smart technologies can be made.

%%%%%%%%%%%%%%%%%%%%%%%%%%%%%
%%% LOOSE IDEAS FOR LATER %%%
%%%%%%%%%%%%%%%%%%%%%%%%%%%%%
% Besides the early automatic detection of pathological changes, . Other applications include systems to assist older adults in daily living or care assistive technology, such as robotic tools for neurorehabilitation.




% Tools from \gls{ai}, especially machine learning, could be used to describe individual trajectories of aging, helping to identify biomarkers that could be used to verify treatments as well as targeted interventions

% Identfying risk factors, developing plans to support healthy aging 
% The practical acceptance and use of AI requires an evaluation of application areas in science and practice. 
% \begin{itemize}
%  \item Age related changes occur at different scales and are manifestet at several levels.
%  \item There is a wide variety in how this changes occur
%  \item Changes are e.g. neural dedifferentiation and compensatory mechanisms (see Reuter Lorenz et al. 2010) and are noticable brain network level and dynamics
%  \item NOTE: Check what EEG studies said about this...
%  \item The idea is to model these changes with tools from datascience to answer questions in aging neuroscience
%  \item First study is about detecting dedifferentiated and compensatory mechanisms with EEG
%  \item Tools used are DMD and Machine Learning
%  \item Main idea: Study classification performance as proxy for age related changes in different motor control tasks
%  \item Expertise as possible way of building a reserve:
%  \item Higher individuality 
%  \item Dynamics of dedifferentiation and how do they relate to fitness
%  \item Basic for targeted interventions 
%  \item How much and what (relate to Julia)
%  \item Background of ML
%  \item ML as tool 
%  \item novel insights s
%  \item Problem: Data is multidimensional and we have often limited data 
%  \item Solution: Use DMD to reduce Complexity and "model" evolution of signal 
%  \item Dynamic Mode Decompsition
%  \item DMD extracts coupled spatio-temporal modes and is able to kind of model the evolution of the signal 
%  \item Backgrouund + Papers 
%  \item Mathematical Formulation
%  \item What can ML tell us?
%  \item ML applied in aging Neuroscience
%  \item Formulating Aims and goals 
%  \item Formulation expected outcomes
% \end{itemize}
        
    
        \section{Aging}
        \label{theory:aging}
        Biologically aging is "the time-dependent functional decline that affects most living organisms" \cite{López-Otín2013}. It can be observed in the reorganization of multiple interacting physiological systems operating at different spatial and temporal scales \cite{Mooney2016}. The underlying patterns of reorganization within and between these systems are highly individual, as they are subject to internal (e.g., genetic, cellular, molecular) as well as external (e.g., environmental, and lifestyle) influences \cite{Smith2020, Mooney2016, Cohen2022}. At the same time, however, overarching, generalizable patterns can be identified \cite{Salthouse2019}. The most recognizable consequences of aging are alterations in cognitive, sensory, and motor abilities that challenge the daily lives of older adults \cite{Li2002}. However, not all abilities are equally affected by declines, and the alterations are highly individual. While sensory, motor, and cognitive abilities, such as memory and processing speed, are generally declining, abilities in the context of acquired knowledge, such as verbal abilities, tend to be stable or even improve with age \cite{Park2009}. One factor that plays a crucial role in these alterations is reorganization at the level of the brain \cite{Reuter-Lorenz2010}. A profound understanding is, therefore, of particular interest to research efforts as this is a prerequisite to identifying unfavorable trajectories and developing prevention and therapy concepts. It is important to note that the reorganization of the brain can be viewed from many perspectives, so in the following, only the aspects and concepts essential for the understanding of this work will be presented. 

\subsection{Age-related Reorganization of the Brain}
\label{theory:aging:brain}
Reorganization in the brain's structure includes, among others, atrophy of the gray and white matter and enlargement of cerebral ventricles \cite{Fjell2010}. The efficiency of neuromodulation declines mainly driven by the loss of dopaminergic receptors indicative of a reorganization of neurotransmitter systems \cite{Li2001}. Besides this, the study of the functional properties of the brain and their relationship to behavioral changes is of great interest. In neuroimaging studies, both under-activation and over-activation of brain areas have been reported in older adults compared to younger adults during the performance in various tasks with sensory, cognitive as well as motor demands \cite{Reuter-Lorenz2010, Sala-Llonch2015}. Regarding activation dynamics, brain activity in response to a stimulus is often slower or delayed. Moreover, the frequency distribution of oscillatory neural activity changes to a slowing of the primary rhythms and altered temporal dynamics, which is interpreted as changes in neural communication \cite{Courtney2021}.\\ 
By emphasizing neural communication and information flow, rather than viewing the brain as functionally separate, it can be conceptualized as a complex system whose functional units, i.e., neurons, areas, and subsystems, are interconnected structurally and functionally \cite{Friston2011,Deery2023}. In this concept, functional connectivity reflects coherent activation patterns within and between these units. Several distinct but interconnected functional networks were identified \cite{Uddin2019}. The dynamic interplay between and within these networks is characterized by segregation and integration at different levels, indicating the flow of information in the brain \cite{Sporns2013}. Older adults' information flow tends to be less efficient and is characterized by lower within-network connectivity and higher between-network connectivity associated with a less segregated, less modular, and more integrated brain network organization \cite{Sala-Llonch2015,Deery2023, Betzel2014}. However, studies on sensorimotor and visual networks seem very heterogeneous, which could indicate individual reorganization patterns \cite{Deery2023}.

\subsubsection{Dedifferentiation}
\label{theory:aging:dedif}
The functional reorganization patterns described in the previous section have been attributed to dedifferentiation \cite{Grady2012}. Dedifferentiation refers to the loss of neural specialization or reduced distinctiveness of neural responses resulting in diffuse, nonspecific recruitment of brain resources \cite{Koen2019}. Historically, the term originates from behavioral research in which an increased correlation of performance between sensory, cognitive, and sensorimotor domains was reported in older adults \cite{Baltes1997,Li2002}. To explain this behavioral dedifferentiation Li and colleagues \cite{Li2001, Li2002} provided a computational model. According to this model, deficient neurotransmitter modulation observed in older adults may affect the responsiveness of cortical neurons, leading to higher levels of neuronal noise and ultimately to less differentiated, more diffuse neuronal activation patterns in response to different stimuli \cite{Li2001,Li2002} (see Figure \ref{fig:dedifferentiation} for an overview on the computational model). In several computational simulations, the authors demonstrated that the proposed model could explain behavioral co-variation and several other phenomena, such as decreased average behavioral performance or increased behavioral intra- and inter-person variability \cite{Li2000,Li2002}. In addition, the proposition of a less distinctive, less specific neuronal activation in response to stimuli could be confirmed in neuroimaging studies showing that the neural responses to various visual, cognitive, and motor stimuli are less specific in older compared to young adults \cite{Tucker2019, Koen2019,Carb2011}. 

\begin{figure}[h]
\def\svgwidth{\columnwidth}
\input{figures/dedifferentiation.pdf_tex}
\caption[The computational model proposed by Li and colleagues \cite{Li2001,Li2002}.]{The authors used a feedforward backpropagation neural network model with logistic activation function $f(z)$ and simulated altered neuromodulation by varying the gain parameter $g$ in $f(z)$ of each neuron (A). Lower $g$ values represent deficient neuromodulation and responsiveness due to aging, resulting in a dampened neuron activation (B). Simulations showed that the activation pattern of simulated neurons differs less for different stimuli, i.e., the network's hidden layer shows a less distinctive representation of the stimulus (C). The activation of a single neuron is more variable in networks with lower $g$ value, i.e., older networks, for multiple stimulations with the same stimulus (D).}
\label{fig:dedifferentiation}
\end{figure}

The reorganization of functional networks described above, i.e., a less segmented and modular and less specialized organization in older adults, has also been referred to as dedifferentiation \cite{Deery2023, Koen2019, Sala-Llonch2015}. \citeauthor{Fornito2015} \cite{Fornito2015} describe dedifferentiation as a fundamental maladaptive mechanism of brain networks that requires compensation. This view is consistent with the argument that dedifferentiation and compensation are complementary mechanisms \cite{Reuter-Lorenz2010}. However, dedifferentiation could also represent a compensatory response, in that the brain attempts to maintain function in the face of deterioration \cite{Stern2009}. By definition, compensation refers to the ability to recruit additional brain resources to compensate for decline and functional loss to maintain cognitive or behavioral functioning \cite{Reuter-Lorenz2010, Grady2012}. Here, the \gls{crunch} hypothesizes that compensatory activity changes as a function of task demands. Moreover, compensation often occurs in a specific pattern of under-activation of posterior areas and prefrontal over-activation, known as \gls{pasa} \cite{Davis2007}. Another frequently reported pattern is the more bilateral recruitment and loss of hemispheric specialization, known as \gls{harold} \cite{Cabeza2002}.

\subsubsection{Reserve}
\label{theory:aging:reserve}
It is important to note that age-related alterations of the brain and behavior are highly individual and dynamic \cite{Smith2020,Koen2019,Douw2014}. In this context, \textit{reserve} was defined as the accumulated capacity of neural resources over the lifespan that can withstand decline or pathology \cite{Cabeza2018, Stern2009}. Although the concept was initially based on observations that the degree of pathological changes in the brain does not necessarily mean clinical manifestation, it has also been applied to explain the individuality of non-pathological aging \cite{Esiri2001,Cabeza2018,Stern2009}.\\
Reserve can be both anatomically quantifiable, referred to as brain reserve, and more functional in nature, referred to as cognitive reserve \cite{Stern2009}. At the functional level, compensatory activation, as well as more efficient utilization (less activation of neural resources), and increased capacity (increased availability of neural resources) were described as key mechanisms of cognitive reserve \cite{Stern2004,Stern2009}. Brain and cognitive reserve influence each other, and \citeauthor{Cabeza2018} \cite{Cabeza2018} argue against a strict separation of brain reserve and cognitive reserve.\\
One aspect that explicitly determines the definition of reserve is the lifelong ability of the brain to adapt its structure and function to internal and external requirements. It is known from the animal model that environments rich in cognitive, social, sensory, and motor stimuli contribute to positive plastic changes \cite{Fabel2009}. As a result, reserve is influenced by an interplay between genetic and environmental factors, including lifestyle factors \cite{Cabeza2018}. Essential elements for increasing reserve have been identified in education, occupation as well as physical activity, with cognitive training, physical fitness, and professional expertise having a considerable impact on the brain's functional organization \cite{vieluf2018age,VOSS2016113,Soldan2021}.\\
Other complementary concepts, such as the maintenance or the \gls{stac} model, highlight these influencing factors additionally. The concept of maintenance emphasizes the ability of the brain to repair. \Gls{stac} postulates that lifelong positive and negative plasticity defines a framework that enables compensation and shapes the individual trajectory of aging \cite{Reuter-Lorenz2014}. 

\subsection{Studying Brain Aging by Electroencephalography}
\label{theory:aging:EEG}
The complex interplay of the factors mentioned above leading to the dynamics of age-related reorganization of the brain is highly complex. Understanding these dynamics regarding individual trajectories and overarching patterns is a prerequisite to differentiating healthy from pathological changes and developing and verifying treatments and targeted interventions. This requires uncomplicated, easy-to-use, and cost-effective methods and novel analyses to quantify changes in brain organization. Several noninvasive methods are available to study the brain's structure and function. \Gls{mri} is the most widely used method in science to image the structure or, using \gls{fmri}, the function of the brain, which is the dominant method in the study of the functional reorganization described in the previous sections \cite{Reuter-Lorenz2010}. However, its use in the public health system is mainly limited to cases with a clear indication, making early detection of unfavorable aging trajectories challenging. In addition, limited availability substantially restricts the development of preventive and rehabilitative interventions and therapies and excludes areas and sites with low levels of equipment and expertise. Here, \gls{eeg} could represent a real added value since it is characterized by simple use, mobility, and relative cost-effectiveness. Although it has a lower spatial resolution than \gls{mri} based methods, \gls{eeg} measures neuronal activity directly with a high temporal resolution which allows for the detection of age-related changes in the temporal dynamics of brain activity and networks, which could be of particular interest to understand age-related changes of the brain and their relation to behavior \cite{Courtney2021}.

\begin{tcolorbox}[breakable, enhanced]
    \subsubsection{Excursus: A Brief Overview on Electroencephalography}
    \Gls{eeg} measures time-varying electrical fields on the surface of the head by using several sensors placed in a standardized position \cite{Jackson2014}. The measured signals reflect synchronously active populations of neurons. Electrical activity can only accumulate and be detected on the surface of the head if spatially similar neurons, aligned perpendicular to the surface, are synchronously activated. Based on the conductive properties of the brain, the signal can travel through the different layers to the surface due to volume capacitive conduction. For this reason, and due to the orientation of neural cell assemblies, the signal in each sensor reflects a summed signal of different neuron patches. The signal expressions are in the range of a few micro-volts and are much lower than other biological and non-biological electrical generators, e.g., muscular activity or line noise, so the EEG signal is often affected by a low signal-to-noise ratio \cite{CohenX2017}.\\
    \\
    \noindent One of the EEG's most striking signal characteristics is the rhythmic voltage fluctuations that define the signal and are summarized under the term oscillation. Commonly, the EEG signal is analyzed based on the frequency composition of oscillatory activity in loosely defined frequency ranges, i.e., $\delta$ ($<$4 Hz), $\theta$ ($~$4-8 Hz), $\alpha$ ($~$8-12 Hz), $\beta$ ($~$12-30 Hz) and $\gamma$ ($>$30 Hz), which have been demonstrated to be related to perceptual, cognitive, motor and emotional processes \cite{CohenX2017}. Furthermore, the analysis of frequency-dependent synchrony or functional connectivity in terms of statistical dependence of the signals, e.g., by coherence or the phase synchrony of the signal, can provide information about the network characteristics of the brain \cite{Siegel2012}. Finally, the analysis of event-related activation, so-called \glspl{erp}, can provide information on the direct processing of stimuli. The analysis of \glspl{erp} involves time-locking the EEG data to the onset of a specific stimulus and averaging the EEG signal across hundreds of trials to extract a reliable signal related to the processing of the stimulus.
\end{tcolorbox}

\subsubsection{Electroencephalographic Signatures of Age-related Reorganization}
Age-related changes in \gls{eeg} characteristics have been extensively studied. Specifically, it has been reported that aging is associated with changes in the frequency composition of the EEG signal, regardless of any specific task involvement. These changes include a decrease in amplitude within the $\alpha$ frequency band, a shift in the $\alpha$ peak frequency towards lower frequencies, an increase in amplitude within the $\beta$ frequency band, and varying results regarding changes in the amplitude of the $\theta$ and $\delta$ bands \cite{ROSSINI2007375, Ishii2017, Courtney2021}. Moreover, age-related changes have also been reported in terms of reduced \gls{eeg} synchrony and a more random, less segregated organization of \gls{eeg} derived network topology \cite{Smit2012, Samogin2022}.\\
\Gls{eeg} changes in relation to tasks are highly dependent on the task context or domain studied. For example, unilateral motor tasks may display lower frequency specificity and more bilateral spatial expression of $\alpha$ and $\beta$ frequency power modulations \cite{Quandt2016}. In contrast, attention tasks may demonstrate enhanced frontal network involvement and power in the $\theta$ frequency band \cite{Hong2016}. In addition, the neural response to stimuli may exhibit a temporal slowing and altered spatial expression. These alterations can be seen, for example, in a delay of early \gls{erp} components as well as a more frontal expression of later \gls{erp} components in visual attention tasks \cite{LI2013477, Reuter2017}.\\
Often these changes are discussed concerning the mechanisms of dedifferentiation and compensation described above. These have been shown to be modulated by lifetime experience such as occupational expertise \cite{vieluf2018age} or physical fitness \cite{Douw2014}. However, the relationship between \gls{eeg} parameters and these mechanisms often needs to be clarified. As such, other \gls{eeg} findings may point in the opposite direction than described above. \citeauthor{HUBNER2018104} \cite{HUBNER2018104}, for instance, found no age effects in central lateralization in the $\beta$ frequency band in a complex fine motor control task, which again highlights the dependency on the task context considered. Age-related changes in decreased \gls{erp} latency and lower or increased functional connectivity of the examined networks depending on the task context are also reported \cite{Courtney2021}. Moreover, the interpretation of dedifferentiation is often based on \gls{fmri} findings that report over-activation and loss of segregation of brain networks. However, the relationship between frequency-specific \gls{eeg} and \gls{fmri} findings acting on different spatial and temporal scales and measurement principles might be unclear. \citeauthor{Koen2019} \cite{Koen2019} further points out that over-activation should be interpreted cautiously and does not necessarily imply loss of neural specificity, as predicted in the original model of \citeauthor{Li2000} \cite{Li2000}. He, therefore, proposes to operationalize dedifferentiation clearly in terms of the selectivity of the neural response between two or more task modulations. While in this operationalization, the evidence regarding dedifferentiation in \gls{fmri} studies is quite clear, this has not been explored in \gls{eeg} studies so far \cite{Koen2019}.\\
\\
Altogether the \gls{eeg} represents an easy-to-use, low-cost method that can provide valuable insights into age-related changes. However, the link to age-related changes reported consistently in the \gls{fmri}, such as dedifferentiation, is often challenging and needs to be clarified. \Gls{eeg} signals are temporally and spatially highly dimensional, i.e., large amounts of data points contain intricate patterns of electrical activity. However, the signals often have a low signal-to-noise ratio, making it difficult to detect and visualize age-related brain reorganization and its dynamics. As such, analysis of \gls{eeg} signals requires advanced signal analysis methods. In this context, methods from the field of machine learning could be of particular interest. By leveraging machine learning techniques, it is possible to extract meaningful patterns from the high-dimensional EEG data and uncover subtle age-related changes that may not be evident through traditional analysis methods.
    
        
        \section{Machine learning}
        \label{theory:ml}
        Machine learning emerged in the 1950s to enable computers to learn without being explicitly programmed \cite{Samual1959}. It is defined by computational methods combining fundamental concepts from computer science, statistics, probability, and optimization that automatically extract patterns and trends, i.e., \textit{learn} from data \cite{Hastie2009}. The notion of \textit{learning} therein describes the automated inference of general rules based on the observation of examples using algorithms to solve a specific task or problem \cite{Von_luxburg2011}. In its basic form, these tasks often involve making predictions based on learned relationships or extracting information based on automatically detected patterns and structures from data. Many real-world problems can be tackled by using machine learning. A rise in the methods started in the 1990s to 2000s with the availability of computing resources, data, and the development of algorithms, which have found their way into everyday life not only since the current advancements in generative \gls{ai} systems. Examples can be found in numerous areas, such as predicting stock prices, personalized advertising, or autonomous driving \cite{Rudin2014}.\\
In science, machine learning is increasingly used as a complementary method to classical statistical analyses because of the ability to make predictions and deal with the multidimensional structure and non-linearity in real-world datasets for drawing inference \cite{Bzdok2018}. Especially in areas where high-dimensional data is prevalent, such as in neuroscience, machine learning methods offer insight by extracting complex patterns in a data-driven way \cite{Brunton2019}. In terms of \gls{eeg}, machine learning can help identify subtle patterns and nonlinear relationships from the complex multidimensional structure of the data, allowing for more accurate and efficient analysis of brain recordings. Various methods are available for this purpose, which can be roughly characterized based on certain properties. 

\subsection{Forms of Machine Learning}
\label{theory:ml:forms}
The three main forms of machine learning are supervised, unsupervised, and reinforcement learning. These forms are defined by the type of feedback a machine learning algorithm has access to during learning \cite{Shalev2014}.\\
Supervised machine learning aims to learn a generalizable relationship between data and associated information, so-called labels or targets. The learned model can then be used to predict the label of new data that was not used during the learning process. If the labels are categorical, the prediction task is called classification; for continuous labels, the term is regression. Unsupervised machine learning aims to find hidden structures in data without considering associated labels. This could be grouping similar data points, i.e., clustering, or uncovering a meaningful low dimensional representation of high dimensional data, i.e., dimensionality reduction. This type of learning is also referred to as \textit{knowledge discovery} \cite{Murphy2012}. Reinforcement learning describes the task of learning optimal actions to solve a particular problem by maximizing the reward linked to that action. See \autoref{fig:ml_forms} for an overview of these three main forms.\\
\\

\begin{figure}[h]
\begin{center}
\begin{tikzpicture}[grow cyclic,
	level 2/.append style={level distance=2cm}]
\node[draw,ellipse,align=center,font=\bfseries\large]{Machine\\[-1ex]learning}
child[grow=145,level distance=4cm] { node[draw,ellipse,align=center] {Reinforcement\\[-1ex]learning}
}
child[grow=35, level distance=4cm] {  node[draw,ellipse,align=center] {Supervised\\[-1ex]learning}
	child[grow=-30, level distance=3.95cm] { node {Classification}}
	child[grow=-90]{ node {Regression}}
}
child[grow=-90, level distance=2.5cm] {node[draw,ellipse,align=center] {Unsupervised\\[-1ex]learning}
	child[grow=-160 ,level distance=3cm] { node[align=center] {Dimensionality reduction}}
	child[grow=-20,level distance=3cm] { node {Clustering}}
}
node at (-3.5,-0.2)[scale=0.8]{\input{figures/reinforcement.pdf_tex}}
node at (2.9,-4.65)[scale=0.2]{\input{figures/clustering.pdf_tex}}
node (node1) at (-4.3,-4.65)[scale=0.2]{\input{figures/dim_red3D.pdf_tex}}
node (node2) at (-1.75,-4.65)[scale=0.2]{\input{figures/dim_red.pdf_tex}}
node at (3.25,-0.75)[scale=0.2]{\input{figures/regression.pdf_tex}}
node at (6.5,-0.75)[scale=0.2]{\input{figures/classification.pdf_tex}}
;

\draw[->] (node1) -- (node2);

\end{tikzpicture}
\end{center}
\captionsetup{justification=justified}
\caption[The three main forms of machine learning.]{The three main forms of machine learning.}
\label{fig:ml_forms}
\end{figure}

\noindent In practice, however, a clear separation is often impossible. As such, dimensionality reduction can also be supervised, i.e., labels are provided to learn a new representation of the data \cite{Mcinnes2018}. Besides, in semi-supervised learning, the goal is the same as in supervised learning. However, the data set used to learn the relationship contains labeled and unlabeled examples. The hope is to build a stronger representation by providing more information in the form of data \cite{Burkov2019}. \\
In addition, traditional machine learning is often contrasted with deep learning methods involving artificial neural networks, which are composed of many layers of interconnected nodes often used in an end-to-end fashion in which the input data is used without any form of preprocessing. Usually, they require a vast amount of data and computational power. In the context of this thesis, the tasks considered involve the processing of \gls{eeg} from experiments with mid to small sample sizes to learn meaningful patterns and relationships in data. For this reason, a more detailed introduction to deep learning methods will not be given at this point. The following sections present state-of-the-art approaches for applications of traditional machine learning on \gls{eeg} data.

\begin{tcolorbox}[breakable, enhanced]
    \subsection*{Excursus: How does a machine learn?}
    "A computer program is said to learn from experience $E$ with respect to some task $T$ and some performance measure $P$, if its performance on $T$, as measured by $P$, improves with experience $E$" \cite{Mitchell1997}. In other words, learning in the context of machine learning typically involves solving a specific task by using algorithms that improve their performance by using example data. There are numerous algorithms designed to solve the problems outlined above. Some basic building blocks can be defined, which can be used to describe computational learning formally. In the following description, the view of statistical learning theory is considered, and notation is adapted from \citeauthor{Shalev2014} \cite{Shalev2014} and from \citeauthor{Von_luxburg2011} \cite{Von_luxburg2011}.\\
    \\
    Learning is always based on data, i.e., measurable information about some phenomenon, consisting of attributes of the phenomenon, so-called features, and an associated label in supervised learning. It is mathematically defined as an open bounded set $\mathcal{Z}\subset\mathbb{R}^n$ of dimension $n$. Typically there is only a set of examples or training data $S=\{z_i,...,z_m\}\subset{\mathcal{Z}}^m$ available, where $i = 1,\dots,m$, and each $z_i$ is sampled independently from $\mathcal{Z}$ according to an underlying probability distribution $\mathcal{D}$. Thus the only assumption is that the example data are independent and identically distributed. No assumption on $D$ is made.\\
    In supervised learning, $\mathcal{Z}$ comprises the space of input data $\mathcal{X}$ and the space of labels or output $\mathcal{Y}$. The example data $S$ consists of labeled input-output pairs $z_i=x_i,y_i\in(\mathcal{X}\times\mathcal{Y})^m$, where $x_i$ is an input data vector and $y_i$ is its corresponding output label. The pairs are sampled by some unknown joint probability distribution $\mathcal{D}$ on the space $\mathcal{X}\times\mathcal{Y}$.\\
    The space $\mathcal{Z}$ in unsupervised learning comprises the input data space $\mathcal{X}$ only and the example set $S$ consists of unlabelled examples $z_i=x_i\in\mathcal{X}^m$, sampled according to some unknown probability distribution $\mathcal{D}$ on the space $\mathcal{X}$.\\
    Learning ultimately can be thought of as approximating an underlying ground truth function $f$, also called model, that represents the relationship between input and output in supervised learning, i.e., 
    \begin{equation}
    f:\mathcal{X}\rightarrow\mathcal{Y},
    \end{equation}
    or the mapping to a space of hidden patterns or structure $\mathcal{W}\subset\mathbb{R}^p$, where $p$ can be equal or smaller than $n$, i.e.,
    \begin{equation}
    f:\mathcal{X}\rightarrow\mathcal{W}.
    \end{equation}
    A learning task can be conceptualized as searching through the space of all possible solution functions. As this is not feasible, a finite class of functions, so-called hypotheses, is typically selected a priory. Thus, learning can be thought of as selecting a hypothesis $h$ from a space of potential solutions $\mathcal{H}$ with $\mathcal{H}=\{h:\mathcal{X}\rightarrow\mathcal{Y}\}$ in supervised learning and $\mathcal{H}=\{h:\mathcal{X}\rightarrow\mathcal{W}\}$ in unsupervised learning. \\
    A learner or learning algorithm is the means of selecting the best element from $\mathcal{H}$.
    The cost of a false prediction or an inaccurate representation of the data is quantified using a loss function, $\ell:\mathcal{H}\times\mathcal{Z}\rightarrow\mathbb{R}_+$. In other words, it measures how well a specific hypothesis is doing.\\
    The expected risk is a measure of the average loss of a hypothesis, $h\in\mathcal{H}$ with respect to the probability distribution $\mathcal{D}$ over $\mathcal{Z}$ and can be defined as
    \begin{equation}
    L_{D}(h):=\mathbb{E}_{z\sim D}[\ell(h,z)]
    \end{equation}
    A learner should select a hypothesis with the lowest possible expected risk. However, the underlying probability distribution is unknown. Using $S$, the expected risk can be estimated using the empirical risk over the training data. This is defined by:
    \begin{equation}
    L_{S}(h):=\frac{1}{m}\sum_{i=1}^m\ell(h,z_i).
    \end{equation}
    Following this, learning can be formalized as solving an optimization problem of the form: 
    \begin{equation}
    \hat{h}=\arg\min_{h\in\mathcal{H}}L_{S}(h),
    \end{equation}
    which can be solved computationally. In parameterized models, this often involves the automated selection of those parameters $\theta\in\Theta$ of a chosen class of models that minimize $L_{S}(h_\theta)$. This optimization problem can then be solved by methods such as gradient descent or, e.g., analytically, using least squares estimation. The solution $\hat{h}$ is the learned model that can be used to solve the task at hand, e.g., predicting the label of new input data or uncovering patterns or structures in data. This is known as \gls{erm}.\\
    Upon \gls{erm}, more complex learning paradigms can be used to address common problems such as overfitting, in which the learned hypothesis too closely relies on the training data and therefore has low generalization performance, e.g., regularized risk minimization, which introduces regularization to \gls{erm} or structural risk minimization that penalizes complex models and encourages simplicity.\\
    \\
    Although most machine learning can be conceptualized within the framework of \gls{erm}, there are models that, instead of minimizing risk, assume that the underlying distribution over the data has a specific parametric form, and the goal is to estimate these parameters by using \gls{mle} which seeks to find the model parameters that maximize the likelihood of the observed data under the assumed parametric distribution, i.e., \\
    \begin{equation}
    \hat{\theta}_{\text{MLE}} = \arg\max_{\theta\in\Theta}\prod_{i=1}^{m}p_{\theta}(z_i),
    \end{equation}
    where $p_{\theta}(z)$ is the joint probability function of the assumed parametric distribution and $\hat{\theta}_{\text{MLE}}$ is the estimated value of the parameter vector $\theta$.
\end{tcolorbox}

\subsection{State-of-the-Art Approaches to Electroencephalographic Data}
\label{theory:ml:applications_eeg}
Various established supervised and unsupervised algorithms have been utilized in the analysis of \gls{eeg} data, and the selection is usually based on the goal of the analysis. Unsupervised learning aims to highlight specific information in the data, so the selection is made based on the information one aims to highlight \cite{Shalev2014}. This is to highlight group structure in \gls{eeg} data when using clustering or to highlight \gls{eeg} inherent characteristics in dimensionality reduction. In contrast, selecting a suitable supervised learning algorithm is more guided by its performance, i.e., its ability to derive generalizable rules that allow predictions from the available data. Typically, different classification algorithms and their parameters are selected, trained on one portion of the data, the so-called training data, and then tested for their performance on data not used for training, the so-called testing data \cite{Daumé2017}. The training data can further be divided into a training and validation portion to compare different model types or user-defined learning algorithm settings, so-called hyperparameters. Finally, the best model configuration is tested for its predictive performance on the test data and reported. However, this three-time division may drastically reduce the data size usable for training and may result in flawed generalization evaluation due to the randomness of the split \cite{Varoquaux2017}. Therefore several procedures can be applied. In simple k-fold cross-validation, for example, the training data is divided k-times. Thus each time, a different subset of the data is used for validation while the rest is used for training. Usually, this is repeated for a range of models and subsequent hyperparameters, and the model and hyperparameter performing best on average are selected for final testing. Building on this, 
nested cross-validation can be used to select the best model and test the generalization performance by adding a second cross-validation loop for the final model evaluation. In this way, an unbiased estimation of the generalization performance of a model can be obtained (see Figure \ref{fig:CV} for a visual representation of the procedure).

\begin{figure*}[ht]
\centering
  \input{figures/nested_cv.pdf_tex}
  \captionsetup{justification=justified}
  \caption[Exemplary nested cross-validation procedure.]{Exemplary nested cross-validation procedure. K-fold cross-validation is used in an outer loop for testing the best configuration tuned in an inner cross-validation loop. CV: cross-validation, Val: Validation}
  \label{fig:CV}
\end{figure*}

\noindent Recent work highlights deep neural networks that can be used for unsupervised and supervised machine learning applications to \gls{eeg} \cite{Roy2019}. However, their advantage comes into play with large data resources, which are often expensive to acquire in the case of \gls{eeg} \cite{Banville2021}. Traditional learning approaches can be more efficient with good performance and promise better interpretability, especially for comparatively smaller datasets and limited computational resources \cite{Gemein2020}. Due to the low signal-to-noise ratio and high complexity of \gls{eeg} data, the inputs in these approaches are often represented by well-known \gls{eeg} characteristics or features that are believed to be related to the problem being learned. Typical features include time, frequency, time-frequency, connectivity, and theoretical information parameters extracted for each sensor (see \citeauthor{Gemein2020} \cite{Gemein2020} for common choices). However, this approach may lead to less flexible and generalizable models with low spatial resolution and vulnerability to low signal-to-noise ratios \cite{Saeidi2021}.\\
Some approaches to address these problems compute the anatomical sources of the \gls{eeg} signals in the brain using biophysical models as a preprocessing step prior to feature extraction \cite{Khan2018, Westner2018}. However, they require a head model based on \gls{mri} often unavailable individually or merely estimated based on existing templates. Other approaches use supervised and unsupervised decomposition techniques belonging to the field of dimensionality reduction as a preprocessing step for further prediction tasks or provide information themselves in the sense of knowledge discovery. These methods aim at \textit{unmixing} the highly correlated sensor time series by assumptions about the underlying signal components. For example, \gls{ica} assumes statistical independence. In contrast, \gls{pca} assumes that the extracted components are maximally uncorrelated to each other, capturing the largest amount of variance in the data \cite{CohenX2017}. \Gls{dmd} is a method that explicitly considers the temporal structure of the signals, which requires that the extracted signal patterns (modes) are dynamically coherent, extracting spatiotemporal coherent structures and thus accounting for the network nature of the brain \cite{Brunton2016}. Additionally, supervised methods such as \gls{csp} \cite{Blankertz2008} or xDAWN \cite{Rivet2009} extract signal components that correlate with the labels to be predicted.\\
While the supervised and unsupervised dimensionality reduction methods mentioned so far offer ways of examining the complex \gls{eeg} signals in terms of components and patterns to generate knowledge, non-linear methods such as \gls{tsne} and \gls{umap} take into account the non-linear relationships between the data points and provide a lower-dimensional representation of the data that is often easier to interpret and visualize \cite{Mcinnes2018}. These methods can be beneficial for exploring the relationships between different \gls{eeg} features or identifying subgroups within a dataset.\\
It is important to note that these methods can be applied not only to the \gls{eeg} signals themselves but also to previously extracted \gls{eeg} parameters or in combination in terms of knowledge discovery. Thus, supervised and unsupervised dimensionality reduction provides data-driven insights into the complex underlying information but also serves as preprocessing for further tasks such as prediction.

\subsection{Applications in the Context of Aging Research}
\label{theory:ml:applications_aging}
Traditionally, the previously presented machine learning approaches have been the core building block for developing intelligent systems that can automate tasks or enhance and assist humans in performing their tasks. Such systems are critical in terms of assistive technology, for example, to support older adults with disabilities to live their daily lives, but are also relevant in the medical field. In the latter, the hope is to develop intelligent medical systems to inform clinical theory and support clinical decision-making, i.e., assist in diagnosis and risk management by predicting health status or forecasting treatment responses \cite{Woo2017}. In this context, supervised learning is often used to identify markers from \gls{eeg} by identifying signal features that are predictive of a particular disease or health condition, which is highly important in promoting a healthy aging trajectory \cite{Babiloni2021,Mei2021}. An application, known as brain age estimation, is to estimate biological age based on a regression model trained on neural data, e.g., \gls{eeg} data, recorded in extensive population studies \cite{Engemann2022}. The model can then be used to predict the age of an individual. If the brain appears older than it would chronologically, i.e., if the gap between predicted and actual age is large, this may be an early indication of an unfavorable state of health \cite{Gonneaud2021}.\\
Another highly relevant application in the context of aging is the development of devices to assist, augment or enhance humans' capabilities, such as \glspl{bci}. In \glspl{bci}, neural activity is decoded, using classification to generate control commands for various external devices such as computers or prosthetic limbs \cite{Saha2021, Anumanchipalli2019}. Decoding refers to learning a classification or regression model that predicts behavioral outcomes or cognitive states based on neural data. \\
Beyond the application in \glspl{bci}, decoding techniques are widely used in neuroscientific research to gain insights into the neural mechanisms underlying perception, cognition, and behavior. This type of analysis is often referred to as \gls{mvpa} because its goal is to detect multivariate patterns, e.g., a set of voxels in \gls{fmri} or an electrical pattern at a given time point in \gls{eeg}, associated with an experimental condition \cite{Holdgraf2017}. While the use has a long history in the field of \gls{fmri} analysis, it has only become more widespread in the field of \gls{eeg} in recent years. Therefore, decoding approaches to understanding age-related reorganization are mostly limited to \gls{fmri} studies. A common approach is to measure dedifferentiation at the individual level, i.e., the loss of neural specificity. Since dedifferentiation, by definition, results in more similar brain activation patterns for different tasks or stimuli, a poorer performance of classifiers trained to discriminate between them based on neural recordings is indicative of a less distinctive neural representation \cite{Koen2019, Park2010}. However, the literature on the application of this approach to \gls{eeg} data is minimal and restricted to single studies \cite{Chen2019}.\\
Classifying group membership or group-level regression can provide additional information about interesting relationships and their generalizability at the group level. Particularly for \gls{eeg} markers representing functional network characteristics can reveal insightful findings about the relationship to age-related changes \cite{Petti2016}.\\
In addition to typical statistically motivated analysis methods that calculate bivariate connectivity between sensors based on the phase difference or coherence of the \gls{eeg} signals, dimensionality reduction techniques, such as the aforementioned \gls{dmd}, provide a data-driven way to capture dynamic brain network characteristics. This approach has already been used to map age- or expertise-related changes related to brain networks \cite{Vieluf2018}. Further, unsupervised methods, such as nonlinear dimensionality reduction techniques, were frequently used to describe the structure of data sets with respect to age-related changes \cite{Banville2021,Kottlarz2020}.\\
\\
In summary, machine learning is very diverse and ranges from engineering applications to scientific knowledge discovery. Especially in the latter case, it offers the advantage of automated extraction of patterns from highly complex data that can contribute to studying age-related changes. While decoding approaches are particularly interesting for measuring age-related changes in the organization of neural systems, such as the level of differentiation, group analysis could provide new insights into datasets. Especially classification methods that predict a particular experimental condition or a group membership are particularly suitable. The combination with unsupervised learning algorithms, such as dimensionality reduction methods, could be particularly beneficial and used to visualize high-dimensional data.


 

    % Aims
    \chapter{Aims and scope}
    \label{chap:aims_scope}
    The main goal of this dissertation is to study the functional reorganization of the aging brain by applying established methods from supervised and unsupervised machine learning to \gls{eeg} signals.\\
\\
A prominent aspect of age-related functional reorganization discussed in the literature at the individual level is the loss of specificity of neural representations or dedifferentiation that accompanies the aging process and plays an essential role in the behavioral decline. This aspect is based on a computational model and has been confirmed in animal and human studies, as presented in \chapref{theory:aging:brain} showing more similar activation patterns of brain areas and networks during task execution \cite{Carb2011, Rieck2021, Antonenko2013, Geerligs2014}. According to \citeauthor{Koen2019}, considering \gls{fmri} studies, evidence for that is quite robust when dedifferentiation is operationalized based on the original model of \citeauthor{Li2001} \cite{Li2001,Li2002} as loss of neuronal selectivity for different stimuli. In other words, the assumption is that there is less difference in brain states between task conditions. In contrast, results based on \gls{eeg} studies are often ambiguous as markers, and clear operationalization of dedifferentiation is often missing. 
Due to the ability to extract multivariate patterns from high-dimensional data and test their differentiability, machine learning techniques could offer significant added value over classical statistical methods in testing the differentiation of brain systems at the individual level. Although this is already applied in studies utilizing \gls{fmri}, as presented in \chapref{theory:ml:applications_aging}, it has only been applied infrequently to \gls{eeg} data.\\
Consequently, the first approach of this work is to use machine learning to test the discriminability of task-related \gls{eeg} signals and draw conclusions about dedifferentiation. Compared with studies using \gls{fmri}, this would have the advantage of more directly capturing age-related reorganization and its dynamics at the individual level while offering several advantages in terms of practical availability, low cost, and ease of use.\\
As postulated by the reserve hypothesis, lifestyle factors contribute significantly to the development of individual age trajectories, and it can be assumed that individuals with a higher degree of reserve exhibit less dedifferentiation based on adaptive and flexible resources. This is based on comparative studies in which individuals with a low expression in a proxy parameter are contrasted with individuals with a high expression (see \chapref{theory:aging:brain}). Following this strategy, the second approach of this work is to compare the differentiability of tasks between individuals with low and high expression in known proxies, such as physical fitness or professional expertise.\\
Furthermore, detecting and better understanding global patterns of functional reorganization is critical to contextualize individual trajectories. As presented in \chapref{theory:ml:applications_aging}, group classifiers and unsupervised methods such as dimensionality reduction techniques could be used. The third approach is, therefore, to use these methods exploratively at the group level to extract patterns and relationships from complex and high-dimensional \gls{eeg} data.\\
\\
Taken together, this work focuses on studying age-related functional reorganization, such as dedifferentiation, and investigating the replicability of hypotheses, such as reserve. To achieve this, four empirical studies use datasets with participants from different life stages and lifestyles, including work experience and physical fitness. These datasets include experiments covering sensory, motor, and cognitive domains. Results from the analyses are presented in the following research articles that focus on specific sub-questions.\\
\\
In \textbf{\hyperref[results:paperI]{Research Article \uproman{1}}} we followed the first approach presented above. We investigated the difference in the performance of classifiers trained to discriminate \gls{eeg} derived brain network activation patterns during visuomotor tracking tasks between younger and older participants. The aim here was to draw conclusions about the reorganization of the motor system and extend the findings of a previous publication that found differences between younger and older adults in \gls{eeg} markers of sensorimotor processing during these tasks \cite{Vieluf2018}.\\
\\
Following the same approach, \textbf{\hyperref[results:paperII]{Research Article \uproman{2}}}, aimed to investigate whether the cortical representation of inhibitory control differs across different age groups ranging from children to older adults. Again, previously published findings, in which distinct mechanisms of selective attention in older adults and children were detected using classical \gls{erp} analyses, should be extended \cite{Reuter2019}. To this end, performance on the classification of two stimulus types of a flanker task, i.e., one with high demands on inhibitory control and one with low demands on inhibitory control, was compared between different age groups. Furthermore, following the third explorative approach, we investigated whether we can train a classifier to determine to which age group a participant belongs based on the \gls{eeg} data.\\
\\
\textbf{\hyperref[results:paperIII]{Research Article \uproman{3}}} aimed to examine the potential influence of cardiorespiratory fitness, a lifestyle factor, on brain network patterns of dedifferentiation extracted through dimensionality reduction applied to \gls{eeg}. This investigation followed the second approach and was motivated by the reserve hypothesis, which postulates that cardiorespiratory fitness could impact age-related brain reorganization and the observed patterns of dedifferentiation. While this has already been shown in \gls{fmri} studies mainly concerning resting-state brain networks \cite{Stillman2019}, it is not clear whether the differentiability of task-related information processing is affected as well and whether this is reflected in the \gls{eeg}.\\
\\
In addition to cardiorespiratory fitness, another significant lifestyle factor is professional expertise. Therefore, the subsequent \textbf{\hyperref[results:paperIV]{Research Article \uproman{4}}} aimed to characterize middle-aged experts using supervised and unsupervised machine learning techniques. In doing so, machine learning methods were applied as a complement to previous studies in which expertise-related differences were investigated utilizing classical statistical methods \cite{Vieluf2018, Goelz2018} in order to detect the influence of professional expertise on the dedifferentiation of brain network activation patterns during fine motor tasks and following the third approach to better understand the phenomenon of expertise employing group classifications.\\
\\
In summary, the application of machine learning followed three approaches presented in four research articles with the goal of better understanding individual trajectories and overarching patterns of age-related brain reorganization. The first two approaches followed established hypotheses of age-related reorganization (dedifferentiation and reserve), while the third approach aimed to provide exploratory insights and novel findings. \autoref{tab:approaches} summarizes the application of these approaches in each research article.

\begin{table}[ht]
\captionsetup{justification=raggedright,singlelinecheck=false}
\caption[Summary of the approaches followed in this dissertation]{Summary of the approaches followed in this dissertation.}
\label{tab:approaches}
\begin{tabular}{p{3.25cm} p{3.25cm} p{3.25cm} p{3.25cm}}
\toprule
                     & \multicolumn{1}{c}{Approach 1} & \multicolumn{1}{c}{Approach 2} & \multicolumn{1}{c}{Approach 3}\\ \cmidrule(l){2-4}
                     & \multicolumn{1}{c}{Dedifferentiation} & \multicolumn{1}{c}{Reserve} & \multicolumn{1}{c}{Overarching patterns} \\ \midrule
\hyperref[results:paperI]{Research Article \uproman{1}}   & \multicolumn{1}{c}{X} &                       &                       \\
\hyperref[results:paperII]{Research Article \uproman{2}}  & \multicolumn{1}{c}{X} &                       & \multicolumn{1}{c}{X} \\
\hyperref[results:paperIII]{Research Article \uproman{3}} & \multicolumn{1}{c}{X} & \multicolumn{1}{c}{X} &                       \\
\hyperref[results:paperIV]{Research Article \uproman{4}}  & \multicolumn{1}{c}{X} & \multicolumn{1}{c}{X} & \multicolumn{1}{c}{X} \\
\bottomrule
\end{tabular}
\end{table}


\noindent Applying machine learning methods on individual and group levels will allow concluding markers of functional brain reorganization and help identify the individual status and overreaching trajectories. The information gained from these tools could be used to determine and evaluate intervention programs, on-the-job-trainings, and support diagnosis. It may have applications in developing assistive technological systems by providing insights into decoding performance in different age groups and its relation to brain reorganization. 

    % Methods
    \chapter{General methodology}
    \label{chap:methods}
    This chapter introduces the overarching methods of the \hyperref[pub:papers]{published research articles} on which this thesis is based. The focus is rather on an overall description of the methods, which is crucial for the following summary of the main results than on details in terms of reproducibility. For this, reference is made to the original description as it can be found in the \hyperref[pub:papers]{published research articles}.
    As stated previously diverse datasets containing \gls{eeg} data from individuals spanning different life stages and lifestyle backgrounds were analyzed by using methods from the field of supervised and unsupervised machine learning. Unsupervised learning was used to generate a low-dimensional representation of the EEG that can be used to gain insight and as input to supervised learning. Different age groups and groups with different lifestyle factors were considered, and tested with different paradigms, so that both task-related and resting EEG were used in the analyses. Supervised learning consequently took place at the individual level as well as at the group level. The former means that one model was trained per subject, which for each subject individually the cortical representation of the task for each subject individually and finally allows conclusions about e.g. dedifferentiation. The latter means that one model was trained for the whole group to detect general overlapping patterns in the group structure. This approach is visualized in Figure XY

\section{Datasets}
The data sets were selected from experiments in published projects in which different study paradigms were used to investigate age-related differences between age groups and groups with different lifestyle backgrounds.

\subsection{Dataset \uproman{1}}
Dataset \uproman{1} was collected as part of the Bremen Hand Study@Jacobs, which investigated the influence of age and 
expertise on hand dexterity over the working life \cite{Voelcker-Rehage2013}. Several experiments were conducted to explore tactile perception, fine motor control, and their corresponding neurophysiological correlates, as well as their trainability. This dataset specifically contains recordings from one of those experiments that assessed fine motor control using force transducers in conjunction with \gls{eeg}. 

\subsubsection{Participants}
The dataset consists of recordings from 59 participants as a subset of the Bremen-Hands-Study@Jacobs, where individuals were recruited through flyers, newspaper articles, and phone calls. Prior to inclusion, all individuals gave their informed consent to participate and completed a questionnaire, in which they reported good health, no neurological disorders, and normal or corrected-to-normal hearing and vision. Participants were identified as right-handed using the Edinburgh Handedness Inventory \cite{Oldfield1971}. In compensation, participants were paid \euro{8} per hour. The ethical principles of the Declaration of Helsinki were followed and the study was approved by the Ethics Committee of the German Psychological Society.\\
Based on their age and occupation, participants were labeled as young novice (N=xY, age=XY), middle-aged novice (N=xY, age=XY), old novice (N=xY, age=XY), middle-aged expert (N=xY, age=XY) or old expert (N=xY, age=XY). Novices were defined as occupational profiles whose daily routine does not require fine motor control of the hands, such as service personnel, insurance agents, office workers, and students. Experts, on the other hand, referred to persons with more than 10 years of professional experience in a job with pronounced fine motor requirements for hand control such as opticians, goldsmiths, dentists, dental technicians, or hearing aid technicians. This criterion was selected in accordance with \cite{Ericsson1991}. 

\subsubsection{Experimental Procedures}
blavlalalalal\\
\\
\begin{figure}[h]
\def\svgwidth{\columnwidth}
\input{figures/dedifferentiation11.pdf_tex}
\caption[The computational model proposed by Li and colleagues \cite{Li2001,Li2002}.]{The authors used a feedforward backpropagation neural network model with logistic activation function $f(z)$ and simulated altered neuromodulation by varying the gain parameter $g$ in $f(z)$ of each neuron (A). Lower $g$ values represent deficient neuromodulation and responsiveness due to aging, resulting in a dampened neuron activation (B). Simulations showed that the activation pattern of simulated neurons differs less for different stimuli, i.e., the network's hidden layer shows a less distinctive representation of the stimulus (C). The activation of a single neuron is more variable in networks with lower $g$ value, i.e., older networks, for multiple stimulations with the same stimulus (D).}
\label{fig:dedifferentiation}
\end{figure}

% To approximate the performance of a predictive model a dataset is typically divided into a training and testing set. The training set is used for learning a model whereas the testing set is used to estimate the generalization performance to new unseen data, i.e. data which was not used during the process of training. The training data can further be divided into a training and validation portion in order to compare different model types or user defined settings of a learning algorithms, so called hyperparameters. However, this three time division may drastically reduce the data size usable for training and my result in flawed generalization evaluation due to the randomness of the split. Therefore several procedures can be applied. In a simple k-fold cross-validation, for example, the training data is divided k-times. Thus each time a different subset of the data is used for validation while the rest is used for training. Usually this is repeated for a range of models and subsequent hyperparamters and the model and hyperparameter performing best on average are selected for final testing. A more advanced method denoted nested cross-validation adds a second k-fold cross-validation loop for the final model evaluation (see Figure \ref{fig1:CV} for a visual representation of the procedures).    

% \begin{figure*}[h]
%   \dummyfig{Cross-validation procedures} 
%   \caption{Cross-validation procedures}
%   \label{fig1:CV}
% \end{figure*}

    
    % Results
    \chapter{Summary of the Main Results}
    \label{chap:results}
    In the following, the main results of the \hyperref[pub:papers]{published research articles} will be briefly presented. The style of the illustrations published in the respective articles has been adapted from the original publication. The Authors' contributions are indicated as published in the articles and reviewed and agreed to by all authors.
    
        \section{Research Article \uproman{1}}
        \label{results:paperI}
        \fullcite{Goelz2021a}\\
\\
This research article aimed To investigate whether age-related differences in the central processing of fine motor skills can be captured by machine learning based on the classification of different force control tasks in younger and older adults.\\
\\
To describe the classifier input, we used classical statistical methods first. We found significant differences in the expression of \gls{dmd} patterns between tasks and groups focusing mainly on central and posterior electrodes, most pronounced in the $\beta$ frequency bands. In addition, there were differences in the spatial distribution concerning a more bilateral and frontal expression of the patterns in the late middle-aged adults.\\
\\
We trained a classifier for each participant to output the appropriate task label based on the \gls{eeg} data recorded during a trial. That is, the classifier should be able to identify, given the \gls{eeg} data, whether the subject was tracking a sinusoidal target force with the left hand or right hand or whether the task was to track a steady target force with the right or left hand. The classifiers' performance was above chance level within all participants and yielded distinct misclassification patterns between younger and late middle-aged adults (see \autoref{fig:results1}).\\
The classification of the hand side in the late middle-aged participants performed significantly worse, i.e., the classifier misclassified trials performed with the right hand as left-handed trials and vice versa. On the other hand, the classification of which target force was followed, i.e., steady vs. sinus, worked significantly better in the late middle-aged adults. Consequently, fewer steady trials were classified as sinusoidal trials in the late middle-aged compared to the younger subjects or vice versa.

\begin{figure}[h]
\begin{center}
\input{figures/paper1_main_result.pdf_tex}
\caption[Main results of Research Article \uproman{1}.]{Main results of Research Article \uproman{1}. The misclassification rate of the hand side (left vs. right)  was higher and the misclassification rate of task type (sinus vs. steady) was lower in late middle-aged compared to young adults.}
\label{fig:results1}
\end{center}
\end{figure}

% \subsection*{CRediT: Author Contributions}
% \textbf{C. Goelz}: Conceptualization, Software, Formal analysis, Writing - Original Draft.
% \textbf{K. Mora}: Software, Formal analysis, Writing - Review \& Editing.
% \textbf{J. Rudisch}: Writing - Review \& Editing.
% \textbf{R. Gaidai}: Formal analysis, Writing - Review \& Editing.
% \textbf{E. Reuter}: Investigation, Writing - Review \& Editing.
% \textbf{B. Godde}: Conceptualization, Writing - Review \& Editing.
% \textbf{C. Reinsberger}: Writing - Review \& Editing.
% \textbf{C. Voelcker-Rehage}: Conceptualization, Writing - Review \& Editing.
% \textbf{S. Vieluf}: Conceptualization, Project administration, Supervision, Writing - Review \& Editing.\\
% All authors read and approved the final manuscript.

        
        \section{Research Article \uproman{2}}
        \label{results:paperII}
        \hyperref[pub:paperII]{\fullcite{Goelz2023}}\\
\\
Following the previous approach, the discriminability of inhibitory and non-inhibitory stimuli within a subject should be investigated using classification techniques. The intention was to study differences in the neural representation of inhibitory control across age groups and extend previous findings on the same data.\\
\\
In this study, we trained a classifier for each participant that could predict which stimulus, i.e., congruent (requires no inhibitory control) or incongruent (requires inhibitory control), was presented based on the \gls{eeg} data (see \chapref{methods:datasets:II:experiment}). We also examined how well classification worked over time after stimulus presentation, extracting the time points at which the most accurate classification was possible and the level of performance at that time. We compared the classification trajectories recorded in this manner between the different age groups using one-way ANOVAs, or Kruskal-Wallis tests, followed by t-tests or Dunn tests for post-hoc comparisons.\\
To further investigate the group structure, we also trained a group-level classifier to predict which age group the performing participant belongs to based on the \gls{eeg} recording of a trial.\\
\\
The maximum classification performance of the model trained to predict which stimulus was presented within a participant was above the chance level in over 95~\% of the participants (AUC: M~=~0.72, SD~=~0.06, analytical chance level: 0.61). The classification performance was further dependent on the group [F(5,206)~=~4.805, p~$<$~0.001], with classification performance lower in the children's group compared to the other groups (p~$<$~0.05). When comparing the trajectories, we found that this also differed between the groups (H(5)~=~35.575, p~$<$~0.001) with later performance peaks in the children and the two oldest age groups (p~$<$~0.05) (see \autoref{fig:results2} A).\\
\\
The classification of group membership was overall above chance level (accuracy: 0.55\%, empirical chance level: 0.17), and a characteristic pattern of misclassifications emerged (see \autoref{fig:results2} B).\\
The classification of children was the most accurate. An increasing number of misclassifications were observed in the other age groups, where the classifier incorrectly assigned trials to a neighboring age group. This resulted in clusters of adjacent age groups within which this misclassification accumulated. The first cluster ranged from young adults to late middle-aged adults, and the second cluster included the two oldest age groups. There were higher misclassification rates within these clusters but fewer misclassifications between clusters, especially between the two oldest groups and the adjacent group of late middle-aged adults.\\
We also studied the time points for which the classification performance of the group model was highest and found a 10~\% performance increase after stimulus onset compared to before, with a peak at 100~ms to 200~ms.

\begin{figure}[h]
\begin{center}
\input{figures/paper2_main_results.pdf_tex}
\caption[Main results of Published Research Article \uproman{2}.]{Main results of Published Research Article \uproman{2}. The evolution of classification performance of models trained to discriminate between congruent and incongruent trials differed between age groups. Mean trajectories per age group are shown here (A). Classifying between these age groups revealed clusters of groups in which misclassifications happened predominately (B). Acc.: accuracy, AUC: area under the receiver operating characteristic curve, ma: middle-aged.}
\label{fig:results2}
\end{center}
\end{figure}
\noindent The results of the task classification suggest that the distinctiveness of the cortical representation of inhibitory control does not differ with older age but that different time windows and, therefore, different processes are important for selective attention at different ages. The higher classification performance during the task than before stimulus onset underscores the added value of task-related EEG. The grouped structure of misclassifications,  especially the comparable fewer misclassifications between the oldest group and the late middle-aged groups, could reflect gross changes, e.g., after retirement. 


% \subsection*{Author Contributions}
% \textbf{C. Goelz}: Conceptualization, software, formal analysis, writing — original draft.
% \textbf{E. Reuter}: Conceptualization, investigation, data curation, writing — review, and editing.
% \textbf{S. Froehlich}: Writing — review and editing. 
% \textbf{J. Rudisch}: Writing — review and editing.
% \textbf{B. Godde}: Conceptualization, writing — review and editing.
% \textbf{S. Vieluf}: Conceptualization, investigation, supervision, writing — review, and editing.
% \textbf{C. Voelcker-Rehage}: Conceptualization, investigation, supervision, project administration, writing — review, and editing. All authors read and approved the final manuscript.

    
        \section{Research Article \uproman{3}}
        \label{results:paperIII}
        \hyperref[pub:paperIII]{\fullcite{Goelz2021b}}\\
\\
This study followed the assumption from the reserve hypothesis that life factors, such as physical fitness, influence age-related reorganization of the brain. Thus, the aim was to investigate the influence of cardiorespiratory fitness on the dedifferentiation of task-related brain network patterns at rest and during tasks representing the sensory, motor, and cognitive domains, respectively.\\
\\
We compared the dominant \gls{dmd} patterns derived by \gls{svd} per frequency band between the tasks using permutation t-tests to infer the differentiability of task-related \gls{dmd} modes in fit and less fit participants. For a statistical evaluation of group differences, we compared the multivariate distribution of obtained t-values with Cremér tests between the groups. We further compared the singular values associated with the dominant pattern to infer the prominence of that pattern throughout task execution between the groups using repeated measures ANOVAs.\\
\\
The comparison of t-distributions with Cramér tests between the groups revealed higher differentiability of dominant \gls{dmd} modes in the fit compared to the less fit participants in all frequency bands (all p~$<$~0.05, see Figure 3 and Table 2 in \hyperref[pub:paperIII]{Published Research Article \uproman{3}} for exact statistical values). However, the difference in the differentiability was most pronounced in the $\theta$ and $\beta_2$ frequency bands (see \autoref{fig:results3}). Furthermore, a significantly lower proportion of total variance could be explained by the dominant pattern in the $\beta_2$ frequency range for the less fit compared to the fit group [F(1,29) = 12.572, p = 0.001, partial $\eta²$ = 0.300] in the motor (fit: M~=~80.5~\%, SD~=~0.60~\%, less fit: M~=~79.95~\%, SD~=~0.52~\%), the sensory (fit: M~=~80.8~\%, SD~=~0.76~\%, less fit: M~=~80.05~\%, SD~=~0.53~\%) and the cognitive task (fit: M~=~81.07~\%, SD~=~0.78~\%, less fit: M~=~80.21~\%, SD~=~0.54~\%).\\

\begin{figure}[ht]
    \centering
    \input{figures/paper3_main_results.pdf_tex}
    \caption[Main results of Published Research Article \uproman{3}]{Main results of Published Research Article \uproman{3}. Statistical t-maps of significant differences of dominant \gls{dmd} mode patterns in the $\theta$ and $\beta_2$ frequency bands between the tasks. Fit participants (blue) showed higher specificity of task-related patterns than less fit participants (turquoise), indicated by higher pronounced differences between the tasks.}
    \label{fig:results3}
\end{figure}

\noindent The higher degree of task differentiability in the fit group compared to the unfit group supports the idea that physical fitness manifests in task-related brain activation patterns consistent with lower dedifferentiation in older adults. The higher proportion of explained variance in the fit participants can be interpreted as higher prominence of the patterns in this group due to lower noise levels, which is consistent with the predictions by the computational model of dedifferentiation. These findings are consistent with the predictions of the reserve hypothesis and support assumptions from the computational model of dedifferentiation.

% \subsection*{Author Contributions}
% \textbf{C. Goelz, J.K. Stroehlein, F.K. Haase, C. Reinsberger, and S. Vieluf} contributed to the study conception and design. \textbf{C. Goelz and F. K. Haase} set up the experiments. Data collection was performed by \textbf{C. Goelz, J. K. Stroehlein and F. K. Haase}. \textbf{C. Goelz and K. Moora} analyzed data. All authors interpreted results, drafted parts of the work, approved the final version of the manuscript, and agreed to be accountable for all aspects of the work.
        
        \section{Research Article \uproman{4}}
        \label{results:paperIV}
        \fullcite{Goelz2021b}\\

    % Discussion
    \chapter{General discussion}
    \label{chap:discussion}
    One of the greatest challenges of our society is to cope with an aging society. This requires targeted interventions, early detection of unfavorable trajectories, and assistive technology. To achieve this, an understanding of individual trajectories of age-related decline is essential. The approach of this work was to gain such an understanding by applying machine learning techniques. Following this the overall goal of this thesis was to better understand the reorganization of the aging brain by applying machine learning techniques. Three approaches were derived from the literature and taken up in four publications.\\

\section{Summary}
The first approach was based on a widely reported reorganization of the aging brain in the literature, namely dedifferentiation, which is based on a computational model with broad evidence from animal models and human studies. In this work, machine learning was used to quantify the selectivity of neural representations at the individual level. This approach provided insights into age-related dedifferentiation and allowed the formulation of new hypotheses, e.g., regarding the temporal aspect of this phenomenon.\\
The second approach of this research addressed the reserve hypothesis and examined the possible positive effects of certain lifestyle factors, such as physical fitness or occupational experience, on brain reorganization during aging. Using machine learning techniques, we compared individuals with different lifestyle backgrounds and the effects on age-related brain reorganization. This analysis provided new perspectives and helped confirm certain predictions of the reserve hypothesis.\\
In the third approach, dimensionality reduction, and group-level classification algorithms were used to identify overarching patterns in brain reorganization. This enabled the generation of new insights and hypotheses by these methods on high-dimensional data.\\
In the following, we discuss the main results of these approaches. However, for a detailed discussion of each specific result, please refer to the corresponding published journal articles.\\

\subsection{On the detection of dedifferentiated brain organization}
Motivated by applications from \gls{fmri} research, in which dedifferentiation, operationalized as loss of neural specificity, is quantified based on the performance of classification algorithms, we applied this to \gls{eeg} data to investigate the differentiability of signals in motor, cognitive, and sensory tasks in participants of different age groups.\\
In \hyperref[results:paperI]{Research Article \uproman{1}}, we found differences in the classification performance of fine motor tasks based on \gls{eeg} data. In particular, performance in classifying which hand was used to perform each task was reduced in late middle-aged compared with younger participants. Studying the classifier input, i.e. the spatiotemporal coherent activation patterns extracted with \gls{dmd}, we found that this corresponds to differences in the electrophysiological network activation patterns pointing to more bilateral and widespread activation in late middle-aged adults. Taken together these results correspond to a dedifferentiated motor network as reported in \gls{fmri} studies \cite{Carb2011,Cassedy2020} and are also consistent with other \gls{eeg} findings that have compared classification between age groups finding reduced performance in the classification of the body side of task execution in older adults \cite{Chen2019, Zich2015}. On the other hand, we found an increased classification performance in the late middle-aged participants when predicting whether a steady or sinusoidal target force was tracked by the participant. When considering the classifier input again, we found a stronger involvement of frontal electrodes in the network activation patterns of the late middle-aged participants and thus suspected that compensation processes play a role here. This could be present,  especially in the more complex sinusoidal force tracking and thus positively influence the differentiability of the two task classes and is consistent with the \gls{crunch} hypothesis which  postulated that during task performance, as task difficulty (or load) increases, more cortical resources will be activated \cite{Festini2018}.\\ 
In \hyperref[results:paperII]{Research Article \uproman{2}} we again utilized classification algorithms to investigate whether the cortical representation of selective attention differed between six different age groups ranging from children to older adults. Here we found that the discriminability between stimuli requiring and stimuli not requiring inhibitory control was reduced only in the children's group. This reduced discriminability in the group of children goes along with the assumption of inhibitory control in children that is not yet differentiated at the neural level \cite{Waszak2010,Reuter2019}. However, the results of similar good performance of the classifiability of the two stimulus categories contradict the assumption of a general dedifferentiation of all cortical systems. This is also known in \gls{fmri} studies, which, for example, found no dedifferentiation in the activation of the visual system in response to specific stimulus categories \cite{Voss2008}. In this article, we were interested in the temporal response to the stimulus and so the classifier input did not represent coherent activation patterns as in the previously described analysis, but rather single-trial \gls{eeg} traces filtered by xDAWN in response to the stimulus, through which it was possible to examine the performance of the classifier over time and thus gain insight into the temporal dynamics of information processing. In this way we found that the differentiability of the stimuli was delayed in older adults. Thus, dedifferentiation might also have a dynamic component related to the rise of neural noise, as postulated in the model of Li and colleagues \cite{Li2001,Li2000}. However, this only occurred in the two oldest age groups and not, as in the previous study, in the late middle-aged, so this effect might occur later in the aging process. 

\subsection{The role of lifestyle factors}
An important aspect to better understand the individuality of age-related alterations is to investigate how influencing factors affect the reorganization of the aging brain. From the literature, physical fitness can be deduced as a significant influencing factor that affects the brain's ability to resist degradation processes and maintain function, thus contributing to reserve. In this context \citeauthor{Stillman2019} \cite{Stillman2019} summarizes findings of \gls{fmri} studies proving that the network structure of the brain at rest of physically fit people show less signs of dedifferentiation. Based on this, in \hyperref[results:paperIII]{Research Article \uproman{3}} we investigated to what extent cardiorespiratory fitness affects the dedifferentiation of task-related network activation patterns based on \gls{eeg}. To do so, we used \gls{dmd} and extracted the dominant spatiotemporal \gls{eeg} patterns and their prominence over time of task performance in motor, sensory and cognitive tasks. In this way, we were able to estimate task-related brain network dynamic activation patterns in the main domains in which age-related changes are reported \cite{Baltes1997, Sala-Llonch2015, Park2009}. Comparison between a fit and less fit group revealed greater differences in activation patterns in the fit compared to the unfit group, suggesting lower levels of dedifferentiation. We further found that the extracted dominant modes of the fit group resolved more variance compared to the less fit group, indicating higher stability or prominence of the extracted patterns in the fit group. In this way, it was possible to investigate the dynamics of age-related changes described in the literature. From the perspective of the dedifferentiation model, this could indicate less neural noise and a lower rate of neural variability predicted by this model which points to higher efficiency of information processing. These differences were frequency dependent mainly to the $\theta$ and $\beta_2$ bands which could be carefully attributed to processes reflecting cognitive involvement and information integration \cite{Siegel2012}.\\
Another frequently studied factor of influence in the literature is that of professional expertise. Thus, we compared fine motor experts and novices in \hyperref[results:paperIV]{Research Article \uproman{4}} analogous to \hyperref[results:paperI]{Research Article \uproman{1}} with respect to the classification performance of the visuomotor tasks and found no differences in classifier performance. Consequently, we could not detect any differences in the differentiation of the neural representation of the task. At first glance, these results contradict previous studies from the Bremen Hand Study@Jacobs, which showed statistical differences in the expression of network characteristics of experts compared to novices in terms of neural efficiency \cite{Goelz2018, Vieluf2018}. However, in our study, we tested a very specific assumption of age-related reorganization, namely a reduced differentiation, and did not test neural efficiency as in the previous work. Also, the results seem to contradict the reserve concept, on the basis of which a lower level of dedifferentiation would have been expected. However, this is only one specific point and the concept of reserve is more multifaceted than this specific assumption including structural as well as functional aspects such as compensatory involvement and efficiency of brain networks \cite{Cabeza2018}. Therefore, our results do not necessarily contradict the reserve concept.

\subsection{Exploratory insights into age-related reorganization}
While the previous results refer to the application at the participant level, we also used the machine learning methods at the group level to gain explorative insights into the structure of the studied groups.\\
In \hyperref[results:paperII]{Research Article \uproman{2}} we found a grouped structure of misclassifications when using a classifier to distinguish between the six age groups, i.e. the classifier tended to misclassify only between certain age groups. This was especially noticeable in the two oldest groups, between which the misclassification rate was highest. Less frequently, however, older subjects were classified into the younger group of late middle-aged subjects and vice versa. Since the age cut was around 65 years, this could indicate changes that occur after retirement. Deterioration of cognitive abilities and cerebral blood flow after retirement has been reported in the literature \cite{Celidoni2017, Rohwedder2010, Rogers1990}. However, we could not further test this hypothesis because of a lack of data on the retirement age of the participants. In addition, no misclassification occurred between children and the oldest groups. Although differences based on \gls{erp} markers are often described as u-shaped \cite{Mueller2008, Reuter2019}, this result suggests different mechanisms, i.e., differentiation in children vs dedifferentiation in older adults. Overall, we found an increase in classification performance shortly after stimulus onset of about 10 \%, which highlights the added value of task-related \gls{eeg} recordings and suggests early processing of the stimulus to be affected by age-related reorganization.\\
In contrast, in \hyperref[results:paperIV]{Research Article \uproman{4}} we were unable to distinguish between experts and novices. This contrasts with studies in which this approach was successful. In contrast to these studies, the occupational background of the fine motor experts studied, such as opticians and watchmakers, was very diverse. Following, the visuomotor tracking tasks selected for analysis offered only a general approximation of the context of their expertise. Based on this, we visualized the underlying brain network characteristics using a nonlinear dimensionality reduction in two-dimensional space. However, differences in the data structure between experts and novices emerged. Compared to each other, the brain network patterns of the experts showed a higher degree of a clustered structure in which each cluster could be clearly assigned to a participant. This can be interpreted as a higher individuality of the brain network patterns in the experts. A high degree of individuality is also known in the movement patterns of flute players and in the muscle synergies of expert powerlifters \cite{Albrecht2014, Caramiaux2018, Kristiansen2015} and our results suggest that this is present at the network level of the brain as well.\\
\\
Overall, the results presented in the four research articles showed that machine learning techniques, including dimensionality reduction and classification of \gls{eeg} signals, were successful in detecting age-related brain reorganization. These results provided new starting points for understanding the reorganization process of the aging brain, particularly in relation to dedifferentiation, and partially supported the predictions of the reserve hypothesis.

\section{Methodological considerations}
Before discussing specific methodological aspects regarding the characteristics of the datasets used and the underlying study protocols, as well as the application of machine learning procedures, it should be noted that, noted that we have not presented changes but \textit{differences} as the study design of all studies from which the datasets were derived were cross-sectional. We also focused only on some well-documented specific aspects of the reorganization of the aging brain and explored overarching patterns in an exploratory manner. The results should be evaluated in light of the following methodological considerations. 

\subsection{Datasets}
With regard to the individual datasets and the underlying study protocols, there are some considerations that should be taken into account when evaluating the results presented.\\
Overall, different age groups were present in the datasets underlying the individual research articles. The older (late middle-aged) participants in the first dataset, which were examined in \hyperref[results:paperI]{Research Article \uproman{1}}, were comparatively young so for these results it is to be expected that possible age effects have been underestimated. In the second dataset, the minimum age difference between the six age groups varied from ten years (between children and young adults) to one year (within the two oldest groups). This may have significantly conditioned the classification results of the second research article. However, the minimum age difference between the oldest groups and the next younger group (the middle-aged adults) is also only two years, so the cutoff found between these groups is less likely to be explained by the age structure of the dataset. In regard to the age structure, it would be advantageous to have a continuous age structure for a fine-grained analysis of age-related reorganization.\\
Regarding the proxies of the reserve construct, it should be noted that the construct of the reserve hypothesis is more diverse than the dimesions depicted in \hyperref[results:paperI]{Research Articles \uproman{3}} and \hyperref[results:paperI]{\uproman{4}}. In our considerations, we followed the common scientific practice of comparing individuals with low expression with individuals with higher expression of a given proxy, i.e. cardiorespiratory fitness or expertise \cite{Koen2019}. Cardiorespiratory fitness was indirectly measured by a 6 min walk test, which is dependent on the motivation of the participants. However, a high correlation to the more objective VO$_2$max is reported in the literature \cite{Zhang2017}. The selection of the fine motor experts was based on the professional group of precision mechanics, whose daily routines and hand use may differ. On the other hand, it cannot be excluded that the novices also had a certain level of fine motor skill, which they acquired in their everyday life, so expertise effects may have been underestimated. By the criterion of a minimum of ten years of professional experience, however, the deliberate practice approach was followed, which was evaluated as a suitable criterion for professional expertise \cite{Ericsson1991, Voelcker-Rehage2013}. Furthermore, previous studies have shown that the selected fine motor task appropriately reflects the expertise context \cite{Vieluf2018, Goelz2018, Vieluf2012, Vieluf2013}. In general, a comparison to young subjects could have tested further assumptions of the reserve hypothesis. For example, whether different networks are activated and have higher task specificity or whether we can assume that networks are maintained. However, this was not possible during participant recruitment due to the selected criterion, i.e., at least ten years of work experience, in Dataset \uproman{1} or was not considered in the original intervention study on which Dataset \uproman{3} is based on.\\
In addition to the age structure and the proxies of the reserve concept, it should be noted that the tasks evaluated comprised very specific laboratory tasks, whose transferability to everyday life may be limited. Nevertheless, the tasks should map across relevant domains in which age-related changes are reported, i.e., sensory, cognitive, and motor domains. The task types were all based on well-tested and common paradigms in aging research and had the advantage that the results are comparable to other studies. One aspect that we cannot exclude is the possible effect of fatigue and motivation, which could have played a role, especially during longer sessions. Other aspects relate to the specific characteristics of the task designs. For example, the force control task in Dataset \uproman{1} was designed in a block design and the participants already attended several similar experiments in the context of the Bremen Hand Study, so that practice effects cannot be excluded. This could also have been advantageous since age differences in short-term adaptation effects could thus be eliminated. The flanker task in Dataset \uproman{2} deferred slightly between the individual studies on which this dataset is based on. This was present in the length of stimulus presentation, number of trials, and gender. However, because the subjects of the individual studies were present in several groups and we did not find increased rates of misclassification between these groups, we excluded that this had a major influence on the results. In addition, the influence of gender and the number of trials in previous or comparable studies was estimated to be low \cite{Reuter2019, Vahid2020}. In Dataset \uproman{3}, the task was quite long, which may have contributed to the aforementioned fatigue effects, but had the advantage of allowing the temporal dynamics to be better mapped using the machine learning algorithms of brain activation using the machine learning algorithms 

\subsection{Machine learning approaches}
The application of machine learning algorithms is strongly dependent on the dataset and utilized algorithm.\\
Especially different noise levels in the individual \gls{eeg} recordings could have influenced the results of the group comparisons. However, studies using comparable methods show that this approach is valid, provided that the noise level is taken into account. This is also confirmed by our results, which are consistent with common hypotheses from the \gls{fmri} literature. In the research articles, we used dimension reduction methods on the first level to improve the signal-to-noise ratio and to extract relevant features from the pre-processed \gls{eeg} data.\\
In \hyperref[results:paperI]{Research Article \uproman{1}}, \hyperref[results:paperIII]{\uproman{3}}, and \hyperref[results:paperIV]{\uproman{4}} we decided to use \gls{dmd}, which allows to extract coherent spatiotemporal patterns from the \gls{eeg} data and thus network characteristics. This approach allowed us to draw inferences about the dynamic properties of complex networks from coherent activation patterns \cite{Brunton2016} but differs from analyses of bivariate connectivity of voxel or (source) signals followed by graph analytic approaches to describe brain networks derived in this way. The decision to use \gls{dmd} in this study was driven by its ability to capture the complex dynamics of networks that include factors such as frequencies, growth, and decay. As a result, \gls{dmd} provides a physiologically plausible and low-dimensional representation of \gls{eeg} dynamics. Moreover, the effectiveness of \gls{dmd} in combination with machine learning has been demonstrated in previous studies. In particular, \gls{dmd} has been shown to be very successful in classifying fine motor tasks \cite{Shiraishi2020}. Additionally, we have already used \gls{dmd} in our own research to extract task-related sensorimotor network dynamics associated with age and expertise \cite{Vieluf2018,Goelz2018}. Also \gls{dmd} in this work focused on task-related network characteristics derived from \gls{eeg} signals and not on resting state networks. Thus, we cannot draw conclusions about the relationship to reorganization, i.e., less separated and modular resting networks. Although it is generally unclear how task-related dedifferentiation and the reorganization of resting networks are related, a recent study shows that these two levels are related and predict performance \cite{Cassedy2020netw_distinct}. Also, the analyses were performed in the signal space and the signals were not projected into the anatomical source space, so the anatomical interpretation was not possible with respect to known resting state networks. This decision was justified by the small number of electrodes in Dataset \uproman{1} and \uproman{2}. Studies show that a valid source reconstruction depends strongly on the number of electrodes \cite{Song2015,Lantz2003}. Only for Dataset \uproman{3} a sufficient number of electrodes was given and we visualized the found patterns in the source space but did not perform \gls{dmd} in the source space because validation studies for such an approach have not been performed so far and we only had individual \glspl{mri} for some of the subjects so that no individual head models were available.\\
To align our results with previous \gls{erp} analyses and enhance the signal-to-noise ratio, we employed the xDAWN algorithm in \hyperref[results:paperIV]{Research Article \uproman{4}}. This design choice allowed us to explore the temporal dynamics in response to stimuli, rather than inferring network characteristics as done in the other research articles. While calculating dynamic functional connectivity over time would have been an alternative approach, it is less extensively validated at this stage. Hence, we relied on the established xDAWN algorithm for our investigation.\\
The selection of classification algorithms and secondary dimensionality reduction algorithms may also have affected the results obtained. The process of picking suitable algorithms followed standard procedures in the field of machine learning \cite{Shalev2014}. That is the choice of the dimensionality reduction algorithms was based on the respective objectives of the analyses and of the classifiers was based on the performance of the respective algorithm within the training portions. In addition, we replicated the results obtained by the classifiers with different algorithms and provided them as supplementary material in the respective published research articles. The results of the classification algorithms were carefully evaluated using various metrics such as accuracy, precision, recall, and F1 scores. We used classical algorithms that fit better to the amount of data available, i.e. tens to hundreds of samples. Additionally, we used cross-validation procedures and hyperparameter tuning, e.g. to tune the regularization parameters, limit this, and estimate the generalization performance.\\
At this point, it should be mentioned that in \hyperref[results:paperIII]{Research Article \uproman{3}}, we did not use supervised classification to map the dedifferentiation, but instead extracted the patterns unsupervised via PCA and inferred the dedifferentiation using classical statistical methods. This was due to the relative amount of data available and the continuous design of the task. For this reason, it is possible that the effects of dedifferentiation were underestimated in both groups studied.\\
Although the classification results do not allow conclusions about the direction of the effects, these analyses should be seen as a supplement. We related the results to previous research and examined the input to the classification with methods that allow conclusions about directional effects. For this purpose, we partly used a combination of inferential statistical methods and dimension reduction methods. Overall, it was possible to bridge the gap between traditional and modern machine learning analysis.

\section{Outlook and practical implications}
In the research articles presented in this thesis, we used machine learning methods to gain valuable insights into the reorganization of the aging brain from \gls{eeg} data. Our results have provided new insights at the individual and group levels and offer perspectives for future research efforts.\\
To begin with, the analyses in \hyperref[results:paperI]{Research Article \uproman{1}} have shown that effects of dedifferentiation of the motor system can be detected at the \gls{eeg} level already in late middle-aged participants. It would therefore be exciting to investigate this in continuous age groups and whether there are signatures of different systems as reported e.g. in \gls{mri} findings \cite{Raz2006}. We found initial indications for this in \hyperref[results:paperII]{Research Article \uproman{2}}. In contrast to the motor system, we found signatures of dedifferentiation, here related to dynamics, only in the two oldest groups and not already in the late-middle-aged group. Moreover, especially with respect to temporal dynamics, dedifferentiation has not been explored, so this might represent a new dimension of dedifferentiation that could be investigated in further studies. Aiming at this dynamic, we also found in \hyperref[results:paperIII]{Research Article \uproman{3}} signs that the dynamic cortical representations could be a crucial point. Inspired by the computational model of \citeauthor{Li2002} \cite{Li2002,Li2000}, we proposed neural noise as an explanation for our findings. Although initial findings suggest that neural noise may indeed explain dedifferentiation at the \gls{eeg} level, this requires further investigation \cite{Pichot2022}.\\
Moreover, we only tested two dimensions of the reserve concept and tested varying results for dedifferentiation. It would be interesting to test other dimensions such as education. To understand the concept in relation to the reorganization of the brain, the development of a gold standard tool that captures all or as many as possible of the known domains of reserve would be necessary \cite{Nogueira2022}.\\
\indent Exploratory analysis at the group level additionally generated two hypotheses that could be tested in future research. With regard to expertise, the exploratory analysis in \hyperref[results:paperIII]{Research Article \uproman{3}} of the group structure allowed us to show that the factor of individuality, which was previously known only at the muscular or behavioral level, also exists at the neuronal level. Furthermore, \hyperref[results:paperII]{Research Article \uproman{2}} suggests significant alterations in the functional organization of the brain following retirement, as indicated by the classification of age groups.\\
\\
In addition to these points, there are several implications for practice. We could show with our analyses that the reorganization of the aging brain, more precisely dedifferentiation, can be mapped on an individual level. This could be useful for the development of potential markers that can be used for the development of preventive interventions. 
Although the clinical relevance of dedifferentiation is not fully understood, \citeauthor{Fornito2015} \cite{Fornito2015} presents it as a maladaptive mechanism of brain networks, so our results could also be useful for the early detection of unfavorable age trajectories and eventually for the development and evaluation of therapeutic concepts. The results of \hyperref[results:paperII]{Research Article \uproman{2}} showed that up to 10 \% more performance is possible in predicting the age group during the task, which underlines the added value of task-related \gls{eeg} data so that it could also be used in clinical practice to predict the cognitive status of patients. This could be of particular interest for \gls{eeg}-based detection of mild cognitive impairment, where resting state \gls{eeg} alone has limited informative value \cite{Froehlich2021, Farina2020}. An exciting starting point would be to integrate the methods used in this work into the framework of brain age prediction (see \chapref{theory:ml:applications_aging}). Perhaps this would help to overcome some limitations of this framework, such as the limited ability to predict cognitive decline \cite{Tetereva2023}. Interestingly, studies suggest that a dedifferentiated activation pattern of brain networks may be related to poorer behavioral performance independent of age, so this may have applications in other areas independent of aging research.\\
The decoding approaches we have used are applied in technical systems such as \glspl{bci} or systems used in therapy such as neurofeedback systems. Our results imply that age and reorganization of the brain are significant factors to be considered in the development of assistive systems.  Direct implications for the development of such systems include, for example, thje usage of age-appropriate features, paradigms and algorithms in the development of assistive systems for the elderly. This arises from the differences in decoding performance between age groups and could affect the selection and placement of electrodes and a possible delay in decoding attempts. Likewise, our findings suggest that fitness or expertise level could have an influence and thus have implications for the development of adaptive systems, i.e. systems that automatically adapt to the user. 

\section{Conclusion}
This thesis approached age-related reorganization with data-driven methods from the field of machine learning. Four resrearch articles were presented in which classification and dimensionality reduction methods were used to provide additional insights. 




Specifically in the firesst reserach article we used these methods and detecetd a 

Study 1
Based on age-related changes of brain networks, such as ad-
ditional recruitment of bilateral motor areas (Carp et al., 2011;
Ward & Frackowiak, 2003) and attentional resources (Berghuis
et al., 2019), we aimed to study differences in the classification
of EEG data recorded in active visuomotor tracking tasks.
In summary we found electrophysiological patterns associated
with an altered sensorimotor network in OA. Lower task speci-
ficity in combination with changes in symmetry of brain activity
point to bilateral and dedifferentiated, i.e., less task specific, brain
activity of the motor network and activation and interrelation of
several networks with age.
Most importantly these electrophysiological brain activity pat-
terns resulted in lower classification performance in the clas-
sification of body side of task execution in OA, indicating less
segregated brain network activation of the motor system. In
contrast, OA showed higher classification performance with re-
spect to the task characteristic. The study of the classifier input
indicates the relevance of markers of information integration for
classification performance in OA.
The current results confirm previous findings on age-related
reorganization of task-related brain networks and expand them
with reference to the characteristics of the task. Furthermore,
the findings may have practical implications for areas of applied
research such as BCI applications. Age-related differences should
be taken into account in the development of BCI and neurofeed-
back systems if they are designed for this target group. This could
include the selection of the appropriate positioning of electrodes,
e.g., the use of frontal and occipital electrodes, as well as the
choice of suitable features and algorithms.

Study 2
In summary, we were able to extend previous results
using machine learning techniques to detect age and
task differences in cognitive processing on a single-trial
level. This is especially crucial for a step behind classical
ERP components and a more direct link between behav-
ior and neural dynamics [55]. The data-driven approach
used in this research particularly highlights early atten-
tional processes in the classification of age groups and
suggests the benefit of task-related EEG data in the clas-
sification of different age groups, which could be used in
clinical contexts. With respect to information process-
ing in selective attention our analyses could confirm the
relevance of time windows corresponding to N1 and N2
components reported in ERP studies. Furthermore, the
time windows relevant for inhibitory control differed
between groups, i.e., later time windows were relevant
in older adults, suggesting that different processes are
important for selective attention at different ages. Over-
all, we showed that using machine learning compared
to a priori selected electrodes and timepoints, we were
able to obtain assumption-free insights into differences
in inhibitory control over the lifespan. Machine learning
thus represents an extension to classical methods that
can be used to test existing theories but also to extend
them.

Study 3 
In applying DMD to continuous EEG recordings during
rest and three different tasks, we considered both topo-
logical properties and the temporal dynamics of task-re-
lated
brain
networks.
Thus, we
identified electrophysiological signatures of age-related brain reor-
ganization processes in fit and less fit older adults. Fit
participants showed higher task specificity, i.e., more dif-
ferentiated brain activation patterns, as well as higher
prominence of these patterns, indicating less neural noise
throughout task execution. Our findings support the idea
that physical fitness manifests in task-related brain network
activation patterns that are in line with reduced dediffer-
entiation in older adults.

Study 4
In this study, we used machine learning to gain data-driven insights
into expressions of expertise. We found that classification of group
membership based on tasks that roughly reflect the expertise context is
not possible with high accuracy. In contrast to positive reports from the
literature, we assume a high importance of the expertise context in
classification. Furthermore, the classification of tasks is possible with
high accuracy within novices and experts at individual level. By exam­
ining a low-dimensional representation of the feature space, we found a
more pronounced individuality of EEG patterns in experts, suggesting
more specialized neural mechanisms in fine motor experts during task
performance. In addition to providing data-driven insights, these results
could be relevant to the application of machine learning in the context of
expertise classification as well as the generalizability of such algorithms
in the context of BCI research.

By using machine learning techniques, more precisely dimensionality reduction and the classification of tasks-related \gls{eeg} signals, it was consequently possible to detect age-related differences and confirm hypotheses about the aging brain such as dedifferentiation and compensation directly based on \gls{eeg} signals. In addition, new hypotheses such as the temporal component of dedifferentiation could be established.  Since classifiers are trained for each participant individually, the level of dedifferentiated and compensatory activity could thus be quantified on an individual level. 

    % References
    \printbibliography[heading=bibintoc,title={References}]

    % Publications
    \chapter*{\vspace*{\fill}\centering{Published Research Articles}\vspace*{\fill}}
    \label{pub:papers}
    \addcontentsline{toc}{chapter}{Published Research Articles}
        \newpage
        \section*{\vspace*{\fill}\centering{Published Research Article \uproman{1}}\vspace*{\fill}}
            \addcontentsline{toc}{section}{Published Research Article \uproman{1}}
            \label{pub:paperI}
            \includepdf[pages=-]{published_paper/ForceControl.pdf}
        \section*{\vspace*{\fill}\centering{Published Research Article \uproman{2}}\vspace*{\fill}}
            \addcontentsline{toc}{section}{Published Research Article \uproman{2}}
            \label{pub:paperII}
            \includepdf[pages=-]{published_paper/Attention.pdf}
        \section*{\vspace*{\fill}\centering{Published Research Article \uproman{3}}\vspace*{\fill}}
            \addcontentsline{toc}{section}{Published Research Article \uproman{3}}
            \label{pub:paperIII}
             \includepdf[pages=-]{published_paper/Fitness.pdf}
        \section*{\vspace*{\fill}\centering{Published Research Article \uproman{4}}\vspace*{\fill}}
            \addcontentsline{toc}{section}{Published Research Article \uproman{4}}
            \label{pub:paperIV}
            \includepdf[pages=-]{published_paper/Expertise.pdf}
            
\end{document}